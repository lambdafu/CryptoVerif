\documentclass[compsoc, conference, letterpaper, 10pt, times]{IEEEtran} 
\usepackage{
	color, url, amssymb, amsmath, amsthm, amsfonts, enumitem,
}
\newcommand{\Z}{\mathbb{Z}}
\newcommand{\Zp}{\mathbb{Z}_p}
\newcommand{\M}{\mathbb{M}}
\newcommand{\SIGN}{\mathsf{SIGN}}
\newcommand{\HMAC}{\mathsf{HMAC}}
\newcommand{\HKDF}{\mathsf{HKDF}}
\newcommand{\ENC}{\mathsf{ENC}}
\newcommand{\DEC}{\mathsf{DEC}}
\newcommand{\sk}[1]{\mathit{sk}_{#1}}
\newcommand{\vk}[1]{\mathit{vk}_{#1}}
\newcommand{\shared}{\mathit{shared}}
\newcommand{\const}{\mathit{const}}
\newcommand{\e}{\mathit{enc}}
\newcommand{\m}{\mathit{mac}}
\newcommand{\rk}{\mathit{rk}}
\newcommand{\ck}{\mathit{ck}}
\newcommand{\tg}{\mathit{tag}}
  
\newtheorem{lemma}{Lemma}
\newtheorem{definition}{Definition}
\makeatletter
\newcommand{\pname}[1]{\item[#1.]\def\@currentlabel{#1}}
\makeatother


\newcommand{\hkdftwo}{\HKDF_2}
\newcommand{\salt}{\mathit{salt}}
\newcommand{\key}{\mathit{key}}
\newcommand{\info}{\mathit{info}}
\newcommand{\prk}{\mathit{prk}}
\newcommand{\hmac}{\mathsf{HMAC}}
\newcommand{\Ssalt}{\mathcal{S}}
\newcommand{\Skey}{\mathcal{K}}
\newcommand{\Sinfo}{\mathcal{I}}
\newcommand{\Smac}{\mathcal{M}}
\newcommand{\truncate}{\mathsf{truncate}}

\newcommand{\ab}{\allowbreak}

\begin{document}
\title{Indifferentiability results for TextSecure}
\author{
{\rm Bruno Blanchet}\\
INRIA Paris
}

\maketitle
\thispagestyle{plain}
\pagestyle{plain}

\section{Definition}

Indifferentiabillity can be defined as follows (extension of~\cite{Coron05}
to several independent oracles).

\begin{definition}[Indifferentiability]
Functions $(F_i)_{1 \leq i \leq n}$ with oracle access to independent random oracles $(H_j)_{1 \leq j \leq m}$ are $(t_D, t_S, (q_{H_j})_{1 \leq j \leq m}, (q_{F_i})_{1 \leq i \leq n}, (q_{H'_i})_{1 \leq i \leq n}, \epsilon)$-indifferentiable from independent random oracles $(H'_i)_{1 \leq i \leq n}$ if there exists a simulator $S$ such that for any distinguisher $D$ 
\[|\Pr[D^{(F_i)_{1 \leq i \leq n},(H_j)_{1 \leq j \leq m}} =1] - \Pr[D^{(H'_i)_{1 \leq i \leq n},S}=1]| \leq \epsilon\]
The simulator $S$ has oracle access to
$(H'_i)_{1 \leq i \leq n}$, makes at most $q_{H'_i}$ queries to $H'_i$,
and runs in time $t_S$. 
The distinguisher $D$ runs in time $t_D$ and makes at most $q_{H_j}$ queries
to $H_j$ for $1 \leq j \leq m$ and $q_{F_i}$ queries to $F_i$ for
$1 \leq i \leq n$.
\end{definition}

In the game $G_0 = D^{(F_i)_{1 \leq i \leq n},(H_j)_{1 \leq j \leq m}}$,
the distinguisher interacts with the real functions $F_i$ and
the random oracles $H_j$ from which the functions $F_i$ are defined.
In the game $G_1 = D^{(H'_i)_{1 \leq i \leq n},S}$,
the distinguisher interacts with independent random oracles $H'_i$ instead of $F_i$, and with a simulator $S$, which simulates the behavior of the random oracles $H_j$ using calls to $H'_i$. Indifferentiability means that these two games are indistinguishable.

\section{Indifferentiability Lemmas}

\begin{lemma}\label{lem:disjdomain}
If $H$ is a random oracle, then the functions $H_1, \dots, H_n$ defined 
as $H$ on disjoint subsets $D_1, \dots, D_n$ of the domain $D$ of $H$ are $(t_D, \ab t_S, \ab q_H, \ab (q_{H_i})_{1 \leq i \leq n}, \ab (q'_{H_i})_{1 \leq i \leq n}, 0)$-indifferentiable
from independent random oracles, where $t_S = {\cal O}(q_H)$ assuming one can determine in constant time to which subset $D_i$ an element belongs, and $q'_{H_i}$ is the number of requests to $H$ in domain $D_i$ made by the distinguisher. Hence $q'_{H_1} + \dots + q'_{H_n} \leq q_H$, so in the worst case $q'_{H_i}$ is bounded by $q_H$. 
\end{lemma}
\begin{proof}
Consider
\begin{itemize}[leftmargin=*]
\item  the game $G_0$ in which $H$ is a random oracle, and $H_i(x) = H(x)$
for each $x \in D_i$ and $i \leq n$, and 
\item the game $G_1$ in which $H_1$, \dots $H_n$ 
are independent random oracles defined on $D_1, \dots, D_n$ respectively, and
$H(x) = H_i(x)$ if $x \in D_i$ for some $i \leq n$, and $H(x) = H_0(x)$ otherwise,
where $H_0$ is a random oracle of domain $D \setminus (D_1 \cup \dots \cup D_n)$.
\end{itemize}
It is easy to see that these two games are perfectly indistinguishable,
which proves indifferentiability.
\end{proof}

\begin{lemma}\label{lem:concatenation}
If $H_1$ and $H_2$ are independent random oracles with the same domain that return bitstrings of length $l_1$ and $l_2$ respectively, then 
the concatenation $H'$ of $H_1$ and $H_2$ is $(t_D, t_S, (q_{H_1}, q_{H_2}), q_{H'}, q_{H_1} + q_{H_2}, 0)$-indifferentiable from a random oracle, where $t_S = {\cal O}(q_{H_1} + q_{H_2})$.
\end{lemma}
\begin{proof}
Consider
\begin{itemize}[leftmargin=*]
\item  the game $G_0$ in which $H_1$ and $H_2$ are independent random oracles,
  and $H'(x) = H_1(x) \| H_2(x)$, and
\item the game $G_1$ in which $H'$ is a random oracle that returns bitstrings
  of length $l_1 + l_2$, $H_1(x)$ is the $l_1$ first bits of $H'(x)$
  and $H_2(x)$ is the $l_2$ last bits of $H'(x)$.
\end{itemize}
It is easy to see that these two games are perfectly indistinguishable,
which proves indifferentiability.
\end{proof}

\begin{lemma}\label{lem:splitting}
If $H$ is a random oracle that returns bitstrings of length $l$, then 
the function $H'_1$ returning the first $l_1$ bits of $H$
and the function $H'_2$ returning the last $l-l_1$ bits of $H$
are $(t_D, t_S, q_H, (q_{H'_1}, q_{H'_2}), (q_H, q_H), 0)$-indifferentiable from independent
random oracles, where $t_S = {\cal O}(q_H)$.
\end{lemma}
\begin{proof}
Consider
\begin{itemize}[leftmargin=*]
\item  the game $G_0$ in which $H$ is a random oracle,
  $H'_1(x)$ is the first $l_1$ bits of $H(x)$, and
  $H'_2(x)$ is the last $l - l_1$ bits of $H(x)$, and
\item the game $G_1$ in which $H'_1$ and $H'_2$ are independent random oracles that return bitstrings
  of length $l_1$ and $l - l_1$ respectively,
  and $H(x) = H'_1(x) \| H'_2(x)$.
\end{itemize}
It is easy to see that these two games are perfectly indistinguishable,
which proves indifferentiability. (It is the same indistinguishability
result as in Lemma~\ref{lem:concatenation}, swapping $G_0$ and $G_1$.)
\end{proof}


\begin{lemma}\label{lem:truncation}
If $H$ is a random oracle that returns bitstrings of length $l$, then 
the truncation $H'$ of $H$ to length $l' < l$ is $(t_D, t_S, q_H, q_{H'}, q_H, 0)$-indifferentiable from a random oracle, where $t_S = {\cal O}(q_H)$.
\end{lemma}
\begin{proof}
  This is a consequence of Lemma~\ref{lem:splitting}, by not giving
  access to oracle $H'_2$ to the distinguisher (so $q_{H'_2} = 0$).
  $H'_2$ is then included in the simulator. We assume that random
  oracles answer in constant time.
%% Consider
%% \begin{itemize}[leftmargin=*]
%% \item  the game $G_0$ in which $H$ is a random oracle, and $H'(x)$ is $H(x)$
%% truncated to length $l'$, and
%% \item the game $G_1$ in which $H'$ is a random oracle that returns bitstrings
%% of length $l'$ and $H(x) = H'(x) \| H''(x)$ where $H''$ is a random oracle that
%% returns bitstrings of length $l-l'$.
%% \end{itemize}
%% It is easy to see that these two games are perfectly indistinguishable,
%% which proves indifferentiability.
\end{proof}

\begin{lemma}\label{lem:hcomp1}
If $H_1 : S_1 \rightarrow S'_1$ and $H_2 : S'_1 \times S_2 \rightarrow S'_2$ 
are independent random oracles, then $H_3$ defined by $H_3(x,y) = H_2(H_1(x),y)$
is $(t_D, t_S, (q_{H_1}, q_{H_2}), q_{H_3}, q_{H_2}, \epsilon)$-indifferentiable from a random oracle, where $t_S = {\cal O}(q_{H_1} q_{H_2})$ and 
$\epsilon = {\cal O}((q_{H_2}q_{H_1} + q_{H_1}^2 + q_{H_2}q_{H_3} + q_{H_3}^2)/|S'_1|)$ and ${\cal O}$ just hides small constants.
%$\epsilon = (2q_{H_2}q_{H_1} + q_{H_1}^2 + q_{H_2}q_{H_3} + q_{H_3}^2)/|S'_1|$
\end{lemma}
\begin{proof}
Consider
\begin{itemize}[leftmargin=*]
\item  the game $G_0$ in which $H_1$ and $H_2$ are independent random oracles,
  and $H_3(x,y) = H_2(H_1(x),y)$, and
\item the game $G_1$ in which 
  $H_3$ is a random oracle;
  two lists $L_1$ and $L_2$ are initially empty;
  $H_1(x)$ returns $y$ if $(x,y) \in L_1$ for some $y$, 
  and otherwise chooses a fresh random $r$ in $S'_1$,
  adds $(x,r)$ to $L$ and returns $r$;
  $H_2(y,z)$ returns $H_3(x,z)$ if $(x,y) \in L_1$ for some $x$,
  otherwise returns $u$ if $((y,z),u) \in L_2$ for some $u$,
  and otherwise chooses a fresh random $r$ in $S'_2$,
  adds $((y,z),r)$ to $L_2$ and returns $r$.
\end{itemize}
CryptoVerif shows that these two games are indistinguishable,
up to probability $\epsilon$ (file \texttt{indiff\_HKDF\_1.ocv}).
\end{proof}

\begin{lemma}\label{lem:hcomp2}
If $H_1 : S_1 \rightarrow S'_1$ and $H_2 : S'_1 \times S_1 \rightarrow S'_2$ 
are independent random oracles, then $H_1' = H_1$ and $H'_2$ defined by 
$H'_2(x) = H_2(H_1(x),x)$ are $(t_D, t_S, (q_{H_1}, q_{H_2}), (q_{H'_1}, q_{H'_2}), (q_{H_1} + q_{H_2}, q_{H_2}), \epsilon)$-indifferentiable from independent random oracles, where $t_S = {\cal O}(q_{H_2})$ and 
$\epsilon = {\cal O}(q_{H_2}(q_{H_1}+q_{H'_1} + q_{H'_2})/|S'_1|)$ and ${\cal O}$ just hides small constants.
%$\epsilon = q_{H_2}(2q_{H_1}+2q_{H'_1} + q_{H'_2}+1)/|S'_1|$
\end{lemma}
\begin{proof}
Consider
\begin{itemize}[leftmargin=*]
\item  the game $G_0$ in which $H_1$ and $H_2$ are independent random oracles,
  $H'_1(x) = H_1(x)$, and $H'_2(x) = H_2(H_1(x),x)$, and
\item the game $G_1$ in which 
  $H'_1$ and $H'_2$ are independent random oracles;
  $H_1(x) = H'_1(x)$;
  $H_2(y,z)$ returns $H'_2(z)$ if $y = H'_1(z)$ and $H_3(y,z)$ otherwise,
  where $H_3$ is a random oracle (independent of $H'_1$ and $H'_2$).
\end{itemize}
CryptoVerif shows that these two games are indistinguishable,
up to probability $\epsilon$ (file \texttt{indiff\_HKDF\_2.ocv};
the oracles $H_1$ and $H'_1$ are considered as a single oracle,
which receives $q_{H_1}+q_{H'_1}$ queries in total).
\end{proof}

\section{Indifferentiability of HKDF}

We suppose that $\hmac$ is a random oracle and we define
\begin{align*}
\begin{split}
&\HKDF_n(\salt,\key,\info) = K_1 \| \dots \| K_n \text{ where}\\
&\quad \prk = \hmac(\salt,\key)\\
&\quad K_1 = \hmac(\prk, \info \| 0x00)\\
&\quad K_{i+1} = \hmac(\prk, K_i \| \info \| i) \text{ for }1 \leq i < n
\end{split}
\end{align*}
where $i$ is a byte.

Much like for HMAC in~\cite{Dodis12}, this function is not indifferentiable from a random oracle in general. Intuitively, the problem comes from a confusion between the first and the second (or third) call to $\hmac$, which makes it possible to generate $\prk$ by calling $\hkdftwo$ rather than $\hmac$. In more detail, let
\begin{align*}
&\prk \|\_ = \hkdftwo(s,k,\info)\\
&\salt = \hmac(s,k)\\
&x = \hmac(\prk,\info'\|0x00)\\
&x' \|\_ = \hkdftwo(\salt,\info\|0x00,\info')
\end{align*}
where the notation $x_1 \|x_2 = \hkdftwo(s,k,\info)$
denotes that $x_1$ consists of the first 256 bits of $\hkdftwo(s,k,\info)$ 
and $x_2$ its last 256 bits.

When $\hkdftwo$ is defined from $\hmac$ as above, we have 
$\prk = \hmac(\prk', \info\|0x00)$ where $\prk' = \hmac(s,k) = \salt$,
so $\prk = \hmac(\salt,\info\|0x00)$. Hence, $x' = \hmac(\prk,\info'\|0) = x$.
However, when $\hkdftwo$ is a random oracle and $\hmac$ is defined
from $\hkdftwo$, the simulator that computes $\hmac$ sees what seems
to be two unrelated calls to $\hmac$. (It is unable to see that $\prk$
is in fact related to the previous call $\salt = \hmac(s,k)$: we have 
$\prk \|\_ = \hkdftwo(s,k,\info)$ but the simulator does not know which value
of $\info$ it should use.) Therefore, the simulator can only return fresh random
values for $\salt$ and $x$, and $x \neq x'$ in general.

We can however recover the indifferentiability of $\hkdftwo$ under the 
additional assumption that the three calls to $\hmac$ use disjoint domains.
Let $\Ssalt$, $\Skey$, and $\Sinfo$ be the sets of possible values of $\salt$, 
$\key$, and $\info$ respectively, and $\Smac$ the output of $\hmac$.

\begin{lemma}\label{lem:hkdfindif}
If $\Skey \cap (\Sinfo \| 0x00 \cup \bigcup_{i = 1}^{n-1}\Smac \| \Sinfo \| i) = \emptyset$
then $\HKDF_n$ with domain $\Ssalt \times \Skey \times \Sinfo$ is
$(t_D, \ab t_S, \ab q_{\hmac}, \ab q_{\HKDF_n}, \ab 2q_{\hmac}, \ab \epsilon)$-indifferentiable from a random oracle,
where $t_S = {\cal O}(q_{\hmac}^2)$ and $\epsilon = {\cal O}((q_{\hmac} + q_{\HKDF_n})^2/|\Smac|)$,
and ${\cal O}$ just hides small constants.
\end{lemma}
\begin{proof}
Consider the game $G_0$ in which $\hmac$ is a random oracle and $\HKDF_n$
is defined as above.

By hypothesis, the different calls to $\hmac$ in the definition of 
$\HKDF_n$ use disjoint domains (the last byte differs among the last
$n$ calls to $\hmac$), so by Lemma~\ref{lem:disjdomain}, $G_0$ is
perfectly indistinguishable from $G_1$ in which
$\hmac = S_1[H_0, \ldots, H_n]$ and $\HKDF_n$ is defined by
\begin{align*}
\begin{split}
&\HKDF^1_n(\salt,\key,\info) = K_1 \| \dots \| K_n \text{ where}\\
&\quad \prk = H_0(\salt,\key)\\
&\quad K_1 = H_1(\prk, \info )\\
&\quad K_{i+1} = H_{i+1}(\prk, K_i \| \info) \text{ for }1 \leq i < n
\end{split}
\end{align*}
where $H_0, \dots, H_n$ are independent random oracles
and the simulator $S_1$ calls $H_i$ $q_{H_i}$ times, with
$q_{H_0} + \dots + q_{H_n} \leq q_{\hmac}$, and runs in time 
${\cal O}(q_{\hmac})$.

Slightly reorganizing the arguments of $H_i$, $G_1$ is
perfectly indistinguishable from $G_2$ in which
$\hmac = S_2[H_0, \ldots, H_n]$ and $\HKDF_n$ is defined by
\begin{align*}
\begin{split}
&\HKDF^2_n(\salt,\key,\info) = K_1 \| \dots \| K_n \text{ where}\\
&\quad \prk = H_0(\salt,\key)\\
&\quad K_1 = H_1(\prk, \info )\\
&\quad K_{i+1} = H_{i+1}(K_i, \prk, \info) \text{ for }1 \leq i < n
\end{split}
\end{align*}
where $H_0, \dots, H_n$ are independent random oracles
and the simulator $S_2$ calls $H_i$ $q_{H_i}$ times, with
$q_{H_0} + \dots + q_{H_n} \leq q_{\hmac}$, and runs in time 
${\cal O}(q_{\hmac})$.

By Lemma~\ref{lem:hcomp2}, $G_2 = G_{3,1}$ is
indistinguishable up to probability $\epsilon_2$ from $G_{3,2}$ in which
$\hmac = S_2[H_0, \ab H'_1, \ab S_{3,2}[H'_1, H'_2], \ab H_3, \ab \ldots, \ab H_n]$ and $\HKDF_n$ is defined by
\begin{align*}
\begin{split}
&\HKDF^{3,2}_n(\salt,\key,\info) = K_1 \| \dots \| K_n \text{ where}\\
&\quad \prk = H_0(\salt,\key)\\
&\quad K_1 = H'_1(\prk, \info )\\
&\quad K_2 = H'_2(\prk, \info )\\
&\quad K_{i+1} = H_{i+1}(K_i, \prk, \info) \text{ for }2 \leq i < n
\end{split}
\end{align*}
where $H_0, H'_1, H'_2, H_3, \dots, H_n$ are independent random oracles;
the simulator for $\hmac$ calls 
$H_0$ $q_{H_0}$ times,
$H'_1$ $q_{H_1} + q_{H_2}$ times,
$H'_2$ $q_{H_2}$ times, $H_i$ $q_{H_i}$ times for $i \geq 3$, with
$q_{H_0} + \dots + q_{H_n} \leq q_{\hmac}$, and runs in time 
${\cal O}(q_{\hmac} + q_{H_2})$; and
$\epsilon_2 = {\cal O}(q_{H_2}(q_{H_1} + 2 q_{\HKDF_n})/|\Smac|)$.

Repeating the same reasoning inductively, 
$G_{3,j-1}$ is
indistinguishable up to probability $\epsilon_{j}$ from $G_{3,j}$ in which
$\hmac = S_2[H_0, \ab H'_1, \ab S_{3,2}[H'_1, H'_2], \ab \dots, \ab S_{3,j}[H'_{j-1}, H'_{j}], \ab H_{j+1}, \ab \ldots, \ab H_n]$ and $\HKDF_n$ is defined by
\begin{align*}
\begin{split}
&\HKDF^{3,j}_n(\salt,\key,\info) = K_1 \| \dots \| K_n \text{ where}\\
&\quad \prk = H_0(\salt,\key)\\
&\quad K_i = H'_i(\prk, \info )\text{ for }1 \leq i \leq j\\
&\quad K_{i+1} = H_{i+1}(K_i, \prk, \info) \text{ for }j \leq i < n
\end{split}
\end{align*}
where $H_0, H'_1, \dots, H'_{j}, H_{j+1}, \dots, H_n$ are independent random oracles;
the simulator for $\hmac$ calls 
$H_0$ $q_{H_0}$ times,
$H'_i$ $q_{H_i} + q_{H_{i+1}}$ times for $1 \leq i < j$,
$H'_{j}$ $q_{H_{j}}$ times, $H_i$ $q_{H_i}$ times for $i > j$, with
$q_{H_0} + \dots + q_{H_n} \leq q_{\hmac}$, and runs in time 
${\cal O}(q_{\hmac} + q_{H_2} + \dots + q_{H_{j}})$; and
$\epsilon_{j} = {\cal O}(q_{H_{j}}(q_{H_{j-1}} + 2 q_{\HKDF_n})/|\Smac|)$.
%$\epsilon_{j} = (q_{H_{j}}(2q_{H_{j-1}} + 2q_{\HKDF_n}+q_{\HKDF_n}+1)/|\Smac|)$.

For $j=n$, we obtain a game $G_{3,n}$  in which
$\hmac = S_2[H_0, H'_1, S_{3,2}[H'_1, H'_2], \dots, S_{3,n}[H'_{n-1}, H'_{n}]]$ and $\HKDF_n$ is defined by
\begin{align*}
\begin{split}
&\HKDF^{3,n}_n(\salt,\key,\info) = K_1 \| \dots \| K_n \text{ where}\\
&\quad \prk = H_0(\salt,\key)\\
&\quad K_i = H'_i(\prk, \info )\text{ for }1 \leq i \leq n
\end{split}
\end{align*}
where $H_0, H'_1, \dots, H'_{n}$ are independent random oracles;
the simulator for $\hmac$ calls 
$H_0$ $q_{H_0}$ times,
$H'_i$ $q_{H_i} + q_{H_{i+1}}$ times for $1 \leq i < n$,
$H'_{n}$ $q_{H_{n}}$ times, with
$q_{H_0} + \dots + q_{H_n} \leq q_{\hmac}$, and runs in time 
${\cal O}(q_{\hmac} + q_{H_2} + \dots + q_{H_{n}}) = {\cal O}(q_{\hmac})$.

By Lemma~\ref{lem:concatenation}, $G_{3,n}$ is perfectly indistinguishable
from $G_4$ in which
$\hmac = S_2[H_0, \ab S_{4,1}[H], \ab S_{3,2}[S_{4,1}[H], S_{4,2}[H]], \ab \dots, \ab S_{3,n}[S_{4,n-1}[H], S_{4,n}[H]]]$ and $\HKDF_n$ is defined by
\begin{align*}
\begin{split}
&\HKDF^4_n(\salt,\key,\info) = H(\prk, \info) \text{ where}\\
&\quad \prk = H_0(\salt,\key)
\end{split}
\end{align*}
where $H_0$ and $H$ are independent random oracles;
the simulator for $\hmac$ calls 
$H_0$ $q_{H_0}$ times,
$H$ $q_H = \sum_{i = 1}^{n-1} (q_{H_i} + q_{H_{i+1}}) + q_{H_n} \leq 2 q_{\hmac}$ times,
with $q_{H_0} + \dots + q_{H_n} \leq q_{\hmac}$, and runs in time 
${\cal O}(q_{\hmac})$.

By Lemma~\ref{lem:hcomp1}, 
$G_4$ is indistinguishable up to probability $\epsilon'$
from $G_5$ in which
$\hmac = S_2[H_0, \ab S_{4,1}[H], \ab S_{3,2}[S_{4,1}[H], S_{4,2}[H]], \ab \dots, \ab S_{3,n}[S_{4,n-1}[H], S_{4,n}[H]]]$ with $H_0 = S_{5,1}[\HKDF^5_n]$ and $H = S_{5,2}[\HKDF^5_n]$
and $\HKDF_n = \HKDF^5_n$  is a random oracle, where
the simulator for $\hmac$ calls $\HKDF_n$ at most $q_H \leq 2 q_{\hmac}$ times,
%%NOTE: the factor 2 here is spurious: the same call to HKDF provides
%%the results of H_j and H_{j+1} that are needed to simulate one call to HMAC.
%%So in fact, the simulator for $\hmac$ calls $\HKDF_n$ at most $q_{\hmac}$ times.
runs in time ${\cal O}(q_{\hmac} + q_{H_0} q_H) = {\cal O}(q_{\hmac}^2)$,
and 
\begin{align*}
\epsilon' &= {\cal O}((q_H q_{H_0} + q_{H_0}^2 + q_H q_{\HKDF_n} + q_{\HKDF_n}^2)/|\Smac|)\\
& = {\cal O}((q_{\hmac}^2 + q_{\hmac}q_{\HKDF_n} + q_{\HKDF_n}^2)/|\Smac|) \\
&= {\cal O}((q_{\hmac} + q_{\HKDF_n})^2)/|\Smac|)\,.
\end{align*}
% \begin{align*}
% \epsilon' &= (2q_H q_{H_0} + q_{H_0}^2 + q_H q_{\HKDF_n} + q_{\HKDF_n}^2)/|\Smac|\\
% & = (2q_{\hmac}q_{H_0} + q_{H_0}^2 + q_{\hmac}q_{\HKDF_n} + q_{\HKDF_n}^2)/|\Smac| 
% \end{align*}

The probability of distinguishing $G_0$ from $G_5$ is then
at most 
\begin{align*}
\epsilon &= \sum_{j=2}^n \epsilon_j + \epsilon' \\
&= {\cal O}\left(\left(\sum_{j=2}^n \frac{q_{H_{j}}(q_{H_{j-1}} + 2q_{\HKDF_n})}{|\Smac|} \right) + \frac{(q_{\hmac} + q_{\HKDF_n})^2}{|\Smac|}\right)\\
&= {\cal O}\left(\left(\sum_{j=2}^n \frac{q_{H_{j}}(q_{\hmac} + 2q_{\HKDF_n})}{|\Smac|} \right) + \frac{(q_{\hmac} + q_{\HKDF_n})^2}{|\Smac|}\right)\\
&= {\cal O}\left(\left(\frac{(\sum_{j=2}^nq_{H_{j}})(q_{\hmac} + 2 q_{\HKDF_n})}{|\Smac|} \right) + \frac{(q_{\hmac} + q_{\HKDF_n})^2}{|\Smac|}\right)\\
&= {\cal O}\left(\left(\frac{q_{\hmac}(q_{\hmac} + 2 q_{\HKDF_n})}{|\Smac|} \right) + \frac{(q_{\hmac} + q_{\HKDF_n})^2}{|\Smac|}\right)\\
&= {\cal O}\left(\frac{(q_{\hmac} + q_{\HKDF_n})^2}{|\Smac|}\right)\\
\end{align*}
%MORE DETAILED PROBA WITHOUT O:
% \begin{align*}
% \epsilon &= \sum_{j=2}^n \epsilon_j + \epsilon' \\
% &= \left(\sum_{j=2}^n \frac{q_{H_{j}}(2q_{H_{j-1}} + 3q_{\HKDF_n}+1)}{|\Smac|} \right) + \frac{2q_{\hmac}q_{H_0} + q_{H_0}^2 + q_{\hmac}q_{\HKDF_n} + q_{\HKDF_n}^2}{|\Smac|}\\
% &= \frac{\left((\sum_{j=2}^n q_{H_{j}})(2q_{\hmac} + 3q_{\HKDF_n}+1) \right) + 2q_{\hmac}q_{H_0} + q_{H_0}^2 + q_{\hmac}q_{\HKDF_n} + q_{\HKDF_n}^2}{|\Smac|}\\
% &= \frac{2q_{\hmac}^2 + 3q_{\hmac}q_{\HKDF_n} + q_{\hmac} + q_{\hmac}^2 +  q_{\hmac}q_{\HKDF_n} + q_{\HKDF_n}^2}{|\Smac|}\\
% %because \sum_{j=2}^n q_{H_{j}} + q_{H_0} \leq q_{\hmac}
% &= \frac{3q_{\hmac}^2 + 4q_{\hmac}q_{\HKDF_n} + q_{\hmac} + q_{\HKDF_n}^2}{|\Smac|}\\
% &\leq \frac{(2q_{\hmac} + q_{\HKDF_n})^2}{|\Smac|} 
% \end{align*}
\end{proof}


\section{Application to TextSecure}

When there is a one-time prekey, the encryption key for 
the first message is derived as follows:
\[H_1(x_1, x_2, x_3, x_4, x_5) = F(c_0 \| x_1 \| x_2 \| x_3 \| x_4, x_5)\]
When there is no one-time prekey, $x_4$ is omitted, thus the key
is derived as follows:
\[H_2(x_1, x_2, x_3, x_5) = F(c_0 \| x_1 \| x_2 \| x_3, x_5)\]
where
\begin{align*}
&F(S, x_5) = k_{\e}\text{ where}\\
&\quad  (\rk_{ba}, \ck_{ba}) \Leftarrow \HKDF_2(c_1, S, c_2)\\
&\quad  (\rk_{ab}, \ck_{ab}) \Leftarrow \HKDF_2(\rk_{ba}, x_5, c_2)\\
&\quad  k_{\e} \Leftarrow \truncate_l(\HKDF_2(c_1, \HMAC(\ck_{ab}, c_3), c_1, c_4))
\end{align*}
$c_0$ is a bitstring of 256 bits set to 1,
%$0xff$ repeated 32 times,
%https://github.com/signalapp/libsignal-protocol-javascript/blob/f5a838f1ccc9bddb5e93b899a63de2dea9670e10/src/SessionBuilder.js lines 140--142
$c_1 = 0$ (256 bits), $c_2 = \text{"WhisperRatchet"}$,
$c_3 = 0x01$, and $c_4 = \text{"WhisperMessageKeys"}$.

We start from the assumption that the compression function underlying
SHA256 is a random oracle, and show that the functions $H_1$ and $H_2$
are indifferentiable from independent random oracles. In this proof,
we use a lighter style than in the proof of Lemma~\ref{lem:hkdfindif}:
we do not explicitly compute the runtime and probabilities.


Theorem~4.4 in~\cite{Dodis12} shows that HMAC-SHA256 is then indifferentiable from a random oracle, provided the MAC keys are less than the block size of the hash function minus one, which is true here: the block size of SHA256 is 512 bits and the MAC keys are 256-bit long.

The call $\HMAC(\ck_{ab}, c_3)$ uses a domain disjoint from the 
calls to $\HMAC$ inside $\HKDF_2$ because $c_3$ is one byte, while
$\key$ and $\info$ ($c_2$ or $c_4$) have greater length.
By Lemma~\ref{lem:disjdomain}, we can consider $\HMAC$ in $\HMAC(\ck_{ab}, c_3)$
as a random oracle independent of $\HMAC$ inside $\HKDF_2$. 

The hypothesis of Lemma~\ref{lem:hkdfindif} is satisfied in our case
because $\Skey$ consists of bitstrings of length 256 bits = 32 bytes or $3 \times 32 = 96$ bytes,
$\Sinfo \| 0x00$ consists of bitstrings of length at most 31 bytes
and $\Smac \| \Sinfo \| 0x01$ consists of bitstrings of length between 33 and 63 bytes. Hence, $\HKDF_2$ is indifferentiable from a random oracle.

By Lemma~\ref{lem:disjdomain}, we obtain that
\begin{align*}
&S \mapsto \hkdftwo(c_1, S, c_2)\\
&x,y \mapsto \hkdftwo(x,y,c_2)\\
&x \mapsto \hkdftwo(c_1,x,c_4)
\end{align*}
are indifferentiable from independent random oracles.
The domains are disjoint 
because different constants $c_2$ and $c_4$ are used and furthermore, the two cases that use $c_2$ differ by the length of their second argument (4 or 5 times 256 bits for
$S \mapsto \hkdftwo(c_1, S, c_2)$, and 256 bits for $x,y \mapsto \hkdftwo(x,y,c_2)$).
Hence, we can replace $F$ with:
\begin{align*}
&F(S, x_5) = k_{\e}\text{ where}\\
&\quad  (\rk_{ba}, \ck_{ba}) \Leftarrow H_3(S)\\
&\quad  (\rk_{ab}, \ck_{ab}) \Leftarrow H_4(\rk_{ba}, x_5)\\
&\quad  k_{\e} \Leftarrow \truncate_l(H_5(H_6(\ck_{ab})))
\end{align*}
where $H_3, H_4, H_5, H_6$ are independent random oracles.

By Lemmas~\ref{lem:splitting} and~\ref{lem:truncation},
we can replace it with
\begin{align*}
&F(S, x_5) = k_{\e}\text{ where}\\
&\quad  \rk_{ba} \Leftarrow H'_3(S)\\
&\quad  \ck_{ab} \Leftarrow H'_4(\rk_{ba}, x_5)\\
&\quad  k_{\e} \Leftarrow H'_5(H_6(\ck_{ab}))
\end{align*}
where $H'_3, H'_4, H'_5, H_6$ are independent random oracles.
that is,
\[F(S, x_5) = H'_5(H_6(H'_4(H'_3(S),x_5)))\]
By Lemma~\ref{lem:hcomp1} applied 3 times (in the last two applications,
the set $S_2$ consists of the empty bitstring), $F$ is indifferentiable
from a random oracle.

The calls to $F$ in $H_1$ and $H_2$ use disjoint domains 
(the length of $S$ is different: respectively, 5 and 4 times 256 bits),
so by Lemma~\ref{lem:disjdomain}, $H_1$ and $H_1$ are indifferentiable
from independent random oracles.

\bibliographystyle{IEEEtran}
\bibliography{biblio}


\end{document}

%%% Local Variables:
%%% mode: latex
%%% TeX-master: t
%%% End:
