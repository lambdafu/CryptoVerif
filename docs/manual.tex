\documentclass{article}

\usepackage[latin1]{inputenc}
\usepackage[T1]{fontenc} % correct font encoding to enable search in pdf
\usepackage{amsmath}
\usepackage{amsfonts}
\usepackage{url}
\usepackage[a4paper,hmargin=1in,vmargin=1.25in]{geometry}  
\usepackage[pdfborder={0 0 0.5},
            colorlinks=false]{hyperref}

\newcommand{\tttimes}{\mathop{\texttt{*}}}

\newcommand{\nonterm}[1]{\langle\textrm{#1}\rangle}
\newcommand{\seq}[1]{\textrm{seq}\nonterm{#1}}
\newcommand{\neseq}[1]{\textrm{seq}^+\nonterm{#1}}

% --------------------------------------------------------------------- %
% Typesetting definitions:              Sample output:                  %
%                                                                       %
% \begin{defn}                                                          %
% \categ{M,N}{terms}\\          M, N ::=        terms           %
% \entry{x}{variable}\\                   x               variable      %
% \entry{M\ N}{application}\\             M N             application   %
% \entry{\lambda x.\ M}{abstraction}      \x.M            abstraction   %
% \end{defn}                                                            %
%                                                                       %
% This is a tabbing environment; the last entry should have no \\.      %
% --------------------------------------------------------------------- %

\newenvironment{defn}{\begin{tabbing}
  \hspace{1.5em} \= \hspace{.27\linewidth - 1.5em} \= \hspace{1.5em} \= \kill
%  \hbra\\[-.8ex]
  }{
%\\[-.8ex]\hket
  \end{tabbing}}
\newcommand{\entry}[2]{\>$#1$\>\>#2}
\newcommand{\clause}[2]{$#1$\>\>#2}
\newcommand{\categ}[2]{\clause{#1::=}{#2}}
\newcommand{\subclause}[1]{\>\>\>#1}

\begin{document}

\title{CryptoVerif\\
Computationally Sound, Automatic\\
Cryptographic Protocol Verifier\\
User Manual}

\author{Bruno Blanchet and David Cad{\'e}\\
INRIA Paris-Rocquencourt, France}

\maketitle

\tableofcontents

\section{Introduction}

This manual describes the input syntax and output of our cryptographic
protocol verifier. It does not describe the internal algorithms used
in the system. These algorithms have been described in research
papers~\cite{Blanchet06,BlanchetEPrint05,Blanchet06b,BlanchetPointchevalEPrint06}
that can be downloaded at

\centerline{{\tt
http://prosecco.gforge.inria.fr/personal/bblanche/publications/index.html}.}

The goal of our protocol verifier is to prove security properties
of protocols in the computational model. The input file describes
the considered security protocol, the hypotheses on the cryptographic
primitives used in the protocol, and security properties to prove.

\section{Command Line}

The syntax of the command line is as follows:
\begin{quote}
$\texttt{./cryptoverif }[options]\ \nonterm{filename}$
\end{quote}
where $\nonterm{filename}$ is the name of the input file.
The options can be:
\begin{itemize}

\item $\texttt{-in }\nonterm{frontend}$: Chooses the frontend to use
by CryptoVerif. $\nonterm{frontend}$ can be either \texttt{channels}
(the default) or \texttt{oracles}. The \texttt{channels} frontend
uses a calculus inspired by the pi calculus, described 
in Section~\ref{sec:channels} and in~\cite{Blanchet06,BlanchetEPrint05}.
The \texttt{oracles} frontend uses a calculus closer to cryptographic
games, described in Section~\ref{sec:oracles} and 
in~\cite{Blanchet06b,BlanchetPointchevalEPrint06}.
By default, CryptoVerif uses the \texttt{oracles} frontend when the input
$\nonterm{filename}$ ends with \texttt{.ocv}, and otherwise it uses the
\texttt{channels} frontend.

\item $\texttt{-lib }\nonterm{filename}$: Sets the name of the library
  file (by default $\texttt{default}$) which is loaded by the system
  before reading the input file. In the \texttt{channels} front-end,
  the loaded file is $\nonterm{filename}\texttt{.cvl}$; in the
  \texttt{oracles} front-end, it is
  $\nonterm{filename}\texttt{.ocvl}$. The library file typically
  contains default declarations useful for all protocols.

\item $\texttt{-oproof }\nonterm{filename}$: Output the proof
in the given file name, instead of displaying it on the standard output.

\item $\texttt{-tex }\nonterm{filename}$: Activates TeX output, and sets
the output file name. In this mode, CryptoVerif outputs a TeX version
of the proof, in the given file.

\item $\texttt{-impl}$: Instead of proving the protocol, generate an
  implementation in OCaml corresponding to the modules defined in the input
  file.

\item $\texttt{-o }\nonterm{directory}$: If the $\texttt{-impl}$ option
  is given, outputs the implementation files in the given directory.

\end{itemize}

\newif\ifchannels

\section{\texttt{channels} Front-end}\label{sec:channels}

\channelstrue
%THIS FILE IS USED ONLY WITH \channelstrue

\ifchannels

\def\iprocess{\nonterm{iprocess}}
\def\oprocess{\nonterm{oprocess}}

\else

\def\iprocess{\nonterm{odef}}
\def\oprocess{\nonterm{obody}}

\fi

\def\fungroup{\nonterm{ogroup}}
\def\funmode{\nonterm{omode}}
\def\funbody{\nonterm{obody\_equiv}}
\def\Resavt{\texttt{new\ }\nonterm{vartype}}
\def\Resa#1#2{\texttt{new\ }{#1}\texttt{:}{#2}}
\def\Resbvt{\nonterm{ident}\texttt{ <-R }\nonterm{ident}}
\def\Resb#1#2{{#1}\texttt{ <-R }{#2}}
\def\yield{\texttt{yield}}

\def\epsrand#1{\texttt{eps\_rand(}#1\texttt{)}}

\newcommand{\eqt}{\mathrel{\texttt{=}}}
\newcommand{\leqt}{\mathrel{\texttt{<=}}}


Comments can be included in input files. Comments are surrounded by
{\tt (*} and {\tt *)}. Nested comments are not supported.

Identifiers begin with a letter (uppercase or lowercase) and contain
any number of letters, digits, the underscore character (\_),
the quote character ('), as well as accented letters of the ISO Latin 1
character set. Case is significant. 
Keywords cannot be used as ordinary identifiers. The keywords are:
{\tt builtin}, {\tt channel}, 
{\tt collision}, {\tt const}, {\tt def}, 
{\tt defined}, {\tt do}, {\tt else}, {\tt eps\_find}, 
{\tt eps\_rand}, {\tt equation}, {\tt equiv}, 
{\tt event}, {\tt event\string_abort}, {\tt expand}, {\tt find}, 
{\tt forall}, {\tt foreach}, {\tt fun}, 
{\tt get}, {\tt if}, {\tt implementation}, {\tt in}, {\tt inj-event}, 
{\tt insert}, {\tt length}, {\tt let}, {\tt letfun}, {\tt max}, {\tt maxlength},
{\tt new}, {\tt newChannel}, {\tt orfind}, 
{\tt out}, {\tt param}, {\tt Pcoll1rand}, {\tt Pcoll2rand}, 
{\tt proba}, {\tt process}, 
{\tt proof}, {\tt public\_vars}, {\tt query}, {\tt return},
{\tt secret}, {\tt set}, {\tt suchthat}, {\tt table}, 
{\tt then}, {\tt time}, {\tt type}, {\tt yield}%

In case of syntax error, the system indicates the character position
of the error (line and column numbers). Please use your text editor to
find the position of the error. (The error messages can be interpreted
by \texttt{emacs}.)

\begin{figure}
\begin{itemize}

\item $[M]$ means that $M$ is optional; $(M)^*$ means that $M$ occurs 0 or
any number of times.

\item
$\seq{X}$ is a sequence of $X$: $\seq{X} = [(\nonterm{X}\texttt{,})^*\nonterm{X}] = \nonterm{X}\texttt{,}\ldots\texttt{,}\nonterm{X}$. (The sequence can be empty, it can be one element $\nonterm{X}$, or it can be several elements $\nonterm{X}$ separated by commas.)

\item
$\neseq{X}$ is a non-empty sequence of $X$: $\neseq{X} = (\nonterm{X}\texttt{,})^*\nonterm{X} = \nonterm{X}\texttt{,}\ldots\texttt{,}\nonterm{X}$.
(It can be one or several elements of $\nonterm{X}$ separated by commas.)

\end{itemize}
\caption{Grammar notations}
\label{fig:syntax0}
\end{figure}

\begin{figure}
\def\phst{\phantom{\nonterm{simpleterm} = }\mid}
\def\pht{\phantom{\nonterm{term} = }\mid}
\def\phpat{\phantom{\nonterm{pattern} = }\mid}
\def\phq{\phantom{\nonterm{query} = }\mid}
\def\phqt{\phantom{\nonterm{queryterm} = }\mid}
\begin{align*}
&\nonterm{identbound} ::= [\nonterm{ident}\texttt{ =}]\ \nonterm{ident} \texttt{ <= }\nonterm{ident}\\
&\nonterm{vartype} ::= \nonterm{ident} \texttt{:}\nonterm{ident}\\
&\nonterm{vartypeb} ::= \nonterm{ident} \texttt{:}\nonterm{ident}\\
&\phantom{\nonterm{vartypeb} =}\mid \nonterm{ident} \texttt{ <= }\nonterm{ident}\\
&\nonterm{simpleterm} ::= \nonterm{ident}\\
&\phst \nonterm{ident}\texttt{(}\seq{simpleterm}\texttt{)}\\
&\phst \texttt{(}\seq{simpleterm}\texttt{)}\\
&\phst \nonterm{ident}\texttt{[}\seq{simpleterm}\texttt{]}\\
&\phst \nonterm{simpleterm} \texttt{ = } \nonterm{simpleterm} \\
&\phst \nonterm{simpleterm} \texttt{ <> } \nonterm{simpleterm} \\
&\phst \nonterm{simpleterm} \texttt{ || } \nonterm{simpleterm} \\
&\phst \nonterm{simpleterm} \texttt{ \&\& } \nonterm{simpleterm} \\
&\nonterm{term} ::= \ldots \text{(as in $\nonterm{simpleterm}$ with 
$\nonterm{term}$ instead of $\nonterm{simpleterm}$)}\\
&\pht \Resavt\texttt{;}\nonterm{term}\\
&\pht \Resbvt\texttt{;}\nonterm{term}\\
&\pht \nonterm{ident}[\texttt{:}\nonterm{ident}] \texttt{ <- } \nonterm{term}\\
&\pht \texttt{let } \nonterm{pattern} \texttt{ = }\nonterm{term}\texttt{ in }\nonterm{term}\ [\texttt{else }\nonterm{term}]\\
&\pht \texttt{if }\nonterm{cond}\texttt{ then } \nonterm{term} \texttt{ else }\nonterm{term}\\
&\pht \texttt{find}[\texttt{[unique]}]\ \nonterm{tfindbranch}\ (\texttt{orfind }\nonterm{tfindbranch})^*\texttt{ else }\nonterm{term}\\
&\pht \texttt{event }\nonterm{ident}[\texttt{(}\seq{term}\texttt{)}]\texttt{; }\nonterm{term}\\
&\pht \texttt{event\string_abort }\nonterm{ident}\\
&\pht \texttt{insert }\nonterm{ident}\texttt{(}\seq{term}\texttt{)}\texttt{; }\nonterm{term}\\
&\pht \texttt{get }\nonterm{ident}\texttt{(}\seq{pattern}\texttt{)}\ [\texttt{suchthat}\ \nonterm{term}]\texttt{ in }\nonterm{term}\texttt{ else }\nonterm{term}\\
&\nonterm{varref} ::= \nonterm{ident}\texttt{[}\seq{simpleterm}\texttt{]}\\
&\phantom{\nonterm{varref} =}\mid \nonterm{ident}\\
&\nonterm{cond} ::= \texttt{defined(} \neseq{varref} \texttt{) }\ [\texttt{\&\& }\nonterm{term}]\\
&\phantom{\nonterm{cond} = }\mid \nonterm{term}\\
&\nonterm{tfindbranch} ::= \seq{identbound} \texttt{ suchthat } \nonterm{cond} \texttt{ then }\nonterm{term}\\
&\nonterm{pattern} ::= \nonterm{ident}\\
&\phpat \nonterm{vartypeb}\\
&\phpat \nonterm{ident}\texttt{(} \seq{pattern}\texttt{)}\\
&\phpat \texttt{(} \seq{pattern}\texttt{)}\\
&\phpat \texttt{=} \nonterm{term}\\
&\nonterm{event} ::= \texttt{event(}\nonterm{ident}[\texttt{(}\seq{simpleterm}\texttt{))}]\\
&\phantom{\nonterm{event} =}\mid \texttt{inj-event(}\nonterm{ident}[\texttt{(}\seq{simpleterm}\texttt{))}]\\
&\nonterm{queryterm} ::= \nonterm{queryterm}\texttt{ \&\& }\nonterm{queryterm}\\
&\phqt \nonterm{queryterm}\texttt{ || }\nonterm{queryterm}\\
&\phqt \nonterm{event}\\
&\phqt \nonterm{simpleterm}\\
&\nonterm{query} ::= \texttt{secret }\nonterm{ident}\ [\texttt{public\_vars}\ \seq{ident}]\ [\texttt{[onesession]}]\\
&\phq \nonterm{event}\ (\texttt{\&\& }\nonterm{event})^* \texttt{ ==> }\nonterm{queryterm}\ [\texttt{public\_vars}\ \seq{ident}]\\
&\phq \nonterm{event}\ (\texttt{\&\& }\nonterm{event})^* \ [\texttt{public\_vars}\ \seq{ident}]
\end{align*}
\caption{Grammar for terms, patterns, and queries}
\label{fig:syntax1}
\end{figure}

\begin{figure}
\def\phf{\phantom{\fungroup = }\mid}
\def\phfm{\phantom{\funmode = }\mid}
\def\phpr{\phantom{\nonterm{proba} = }\mid}
\def\phfb{\phantom{\funbody = }\mid}
\begin{xxalignat}{2}
&\nonterm{proba} ::= \texttt{(} \nonterm{proba} \texttt{)}
&&\mid \texttt{time(}\nonterm{ident}[\texttt{, }\neseq{proba}]\texttt{)}\\
&\phpr \nonterm{proba} \texttt{ + } \nonterm{proba}
&&\mid\texttt{time(let }\nonterm{ident}[\texttt{, }\neseq{proba}]\texttt{)}\\
&\phpr \nonterm{proba} \texttt{ - } \nonterm{proba}
&&\mid \texttt{time((}\seq{ident}\texttt{)}[\texttt{, }\neseq{proba}]\texttt{)}\\
&\phpr \nonterm{proba} \texttt{ * } \nonterm{proba}
&&\mid \texttt{time(let (}\seq{ident}\texttt{)}[\texttt{, }\neseq{proba}]\texttt{)}\\
&\phpr \nonterm{proba} \texttt{ / } \nonterm{proba}
&&\mid \texttt{time(= }\nonterm{ident}[\texttt{, }\neseq{proba}]\texttt{)}\\
&\phpr \texttt{max(}\neseq{proba}\texttt{)}
&&\mid \texttt{time(!)}\\
&\phpr \nonterm{ident}[\texttt{(}\seq{proba}\texttt{)}]
&&\mid \texttt{time(foreach)}\\
&\phpr \texttt{|}\nonterm{ident}\texttt{|}
&&\mid \texttt{time([}n\texttt{])}\\
&\phpr \texttt{maxlength(}\nonterm{simpleterm}\texttt{)}
&&\mid \texttt{time(\&\&)}\\
&\phpr \texttt{length(}\nonterm{ident}[\texttt{, }\neseq{proba}]\texttt{)}
&&\mid \texttt{time(\string|\string|)}\\
&\phpr \texttt{length((}\seq{ident}\texttt{)}[\texttt{, }\neseq{proba}]\texttt{)}
&&\mid \texttt{time(new }\nonterm{ident}\texttt{)}\\
&\phpr n
&&\mid \texttt{time(<-R }\nonterm{ident}\texttt{)}\\
&\phpr \#\nonterm{ident}
&&\mid  \ifchannels\texttt{time(newChannel)}\else\texttt{time(newOracle)}\fi\qquad\qquad\qquad\qquad\\
&\phpr \texttt{eps\_find}
&&\mid  \texttt{time(if)}\\
&\phpr \texttt{eps\_rand(}T\texttt{)}
&&\mid \texttt{time(find }n\texttt{)}\\
&\phpr \texttt{Pcoll1rand(}T\texttt{)}
&&\ifchannels\mid \texttt{time(out }[\texttt{[}\neseq{ident}\texttt{]}]\nonterm{ident}[\texttt{, }\neseq{proba}]\texttt{)}\fi\\
&\phpr \texttt{Pcoll2rand(}T\texttt{)}
&&\ifchannels\mid \texttt{time(in }n\texttt{)}\fi\\
&\phpr \texttt{time}
\end{xxalignat}\vspace*{-8mm}%
\begin{align*}
&\nonterm{repl} ::= \texttt{!} [\nonterm{ident}\texttt{ <=}]\ \nonterm{ident}\\
&\phantom{\nonterm{repl} =}\mid \texttt{foreach } \nonterm{ident}\texttt{ <= }\nonterm{ident}\texttt{ do }\\
%
&\nonterm{res} ::= \Resavt\texttt{;}\\
&\phantom{\nonterm{res} =}\mid \Resbvt\texttt{;}\\
%
&\funbody ::= \texttt{(} \funbody \texttt{)}\\
&\phfb \texttt{event\string_abort }\nonterm{ident}\\
&\phfb \Resavt\texttt{; }\funbody\\
&\phfb \Resbvt\texttt{; }\funbody\\
&\phfb \nonterm{ident}[\texttt{:}\nonterm{ident}] \texttt{ <- }\nonterm{term}\texttt{; }\funbody\\
&\phfb \texttt{let }\nonterm{pattern} \texttt{ = }\nonterm{term}\ 
\texttt{in }\funbody\ \texttt{else }\funbody\\
&\phfb \texttt{if }\nonterm{cond}\texttt{ then }\funbody\ \texttt{else }\funbody\\
&\phfb \texttt{find}[\texttt{[unique]}]\ \nonterm{ffindbranch}\ (\texttt{orfind }\nonterm{ffindbranch})^* \ \texttt{else }\funbody\\
&\phfb \texttt{return(}\nonterm{term}\texttt{)}\\
&\nonterm{ffindbranch} ::= \seq{identbound} \texttt{ suchthat }\nonterm{cond}\texttt{ then }\funbody\\
%
&\fungroup ::= \nonterm{ident}\texttt{(}\seq{vartypeb}\texttt{) }[\texttt{[}n\texttt{]}]\ [\texttt{[useful\_change]}]\texttt{ := }\funbody\\
&\phf \nonterm{repl}\ \nonterm{res}^*\ 
\fungroup\\
&\phf \nonterm{repl}\ \nonterm{res}^*\ 
\texttt{(}\fungroup \texttt{ | }\ldots\texttt{ | }\fungroup\texttt{)}\\
&\funmode ::= \fungroup\ [\texttt{[exist]}]\\
&\phfm \fungroup \texttt{ [all]}
\end{align*}
\caption{Grammar for probabilities and equivalences}
\label{fig:syntax2}
\end{figure}

\begin{figure}[tp]
\def\phop{\phantom{\oprocess = }\mid}
\def\phip{\phantom{\iprocess = }\mid}
\begin{align*}
&\nonterm{channel} ::= \nonterm{ident}[\texttt{[}\seq{ident}\texttt{]}]\\
&\oprocess ::= \nonterm{ident}[\texttt{(}\seq{term}\texttt{)}]\\
&\phop \texttt{(} \oprocess \texttt{)}\\
&\phop \yield\\
&\phop \texttt{event }\nonterm{ident}[\texttt{(}\seq{term}\texttt{)}]\ [\texttt{; }\oprocess]\\
&\phop \texttt{event\string_abort }\nonterm{ident}\\
&\phop \Resavt[\texttt{; }\oprocess]\\
&\phop \Resbvt[\texttt{; }\oprocess]\\
&\phop \nonterm{ident}[\texttt{:}\nonterm{ident}] \texttt{ <- }\nonterm{term}[\texttt{; }\oprocess]\\
&\phop \texttt{let }\nonterm{pattern} \texttt{ = }\nonterm{term}\ 
[\texttt{in }\oprocess\ [\texttt{else }\oprocess]]\\
&\phop \texttt{if }\nonterm{cond}\texttt{ then }\oprocess\ [\texttt{else }\oprocess]\\
&\phop \texttt{find}[\texttt{[unique]}]\ \nonterm{findbranch}\ (\texttt{orfind }\nonterm{findbranch})^* \ [\texttt{else }\oprocess]\\
&\phop \texttt{insert }\nonterm{ident}\texttt{(}\seq{term}\texttt{)}\ [\texttt{; }\oprocess]\\
&\phop \texttt{get }\nonterm{ident}\texttt{(}\seq{pattern}\texttt{)}\ [\texttt{suchthat}\ \nonterm{term}]\texttt{ in }\oprocess\ [\texttt{else }\oprocess]\\
&\phop \ifchannels\texttt{out(}\nonterm{channel}\texttt{, }\nonterm{term}\texttt{)}\else \texttt{return(}\seq{term}\texttt{)}\fi[\texttt{; }\iprocess]\\
&\nonterm{findbranch} ::= \seq{identbound} \texttt{ suchthat }\nonterm{cond}\texttt{ then }\oprocess\\
&\iprocess ::= \nonterm{ident}[\texttt{(}\seq{term}\texttt{)}]\\
&\phip \texttt{(} \iprocess \texttt{)}\\
&\phip \texttt{0}\\
&\phip \iprocess \texttt{ | } \iprocess\\
&\phip \texttt{!} [\nonterm{ident}\texttt{ <=}]\ \nonterm{ident}\ \iprocess\\
&\phip \texttt{foreach }\nonterm{ident}\texttt{ <= } \nonterm{ident}\texttt{ do }\iprocess\\
&\phip \texttt{in(}\nonterm{channel}\texttt{,}\nonterm{pattern}\texttt{)}[\texttt{; }\oprocess]
\end{align*}
\caption{Grammar for processes (\texttt{channels} front-end)}
\label{fig:syntax3ch}
\end{figure}

The input file may consist of a list of declarations followed by  
a process:
\[\nonterm{declaration}^*\ {\tt process}\ \iprocess\]
The process
describes the considered security 
protocol; the declarations specify in particular hypotheses on the 
cryptographic primitives and the security properties to prove.

Alternatively, the input may also consist of a list of declarations followed
by an equivalence query:
\[\nonterm{declaration}^*\ {\tt equivalence}\ \iprocess\ \iprocess\ [\texttt{public\_vars}\ \seq{ident}]\]
The query ${\tt equivalence}\ Q_1\ Q_2$ tells CryptoVerif to show that
the processes (games) $Q_1$ and $Q_2$ are computationally indistinguishable.
When it is present, the indication $\texttt{public\_vars}\ x_1, \dots, x_n$
means that the adversary has read access to the variables $x_1, \dots, x_n$.

A library file (specified on the command-line by the
{\tt -lib} option) consists of a list of declarations.
Notations are summarized in Figure~\ref{fig:syntax0}
and various syntactic elements are described in 
Figures~\ref{fig:syntax1}, \ref{fig:syntax2}, and~\ref{fig:syntax3ch}.

Processes are described in a process calculus.
In this calculus, terms represent computations on bitstrings. 
Simple terms consist
of the following constructs:
\begin{itemize}

\item A term between parentheses $\texttt{(}M\texttt{)}$
allows to disambiguate syntactic expressions.

\item An identifier can be either a constant symbol $f$
(declared by \texttt{const} or \texttt{fun} without argument)
or a variable identifier.

\item The function application $f\texttt{(}M_1, \ldots, M_n\texttt{)}$
applies function $f$ to the result of $M_1, \ldots, M_n$.

\item The tuple application $\texttt{(}M_1, \ldots, M_n\texttt{)}$
builds a tuple from $M_1, \ldots, M_n$ (corresponds to the concatenation
of $M_1, \ldots, M_n$ with length and type indications so that 
$M_1, \ldots, M_n$ can be recovered without ambiguity).
This is allowed only for $n \neq 1$, so that it is distinguished
from parenthesing.

\item The array access $x\texttt{[}M_1, \ldots, M_n\texttt{]}$
returns the cell of indices $M_1, \ldots, M_n$ of array $x$.

\item \texttt{=}, \texttt{<>}, \texttt{||}, \texttt{\&\&}
are function symbols that represent equality and inequality tests, 
disjunction and conjunction. They use the infix notation, but
are otherwise considered as ordinary function symbols.

\end{itemize}
Terms contain further constructs \texttt{<-R}, \texttt{<-}, \texttt{event}, \texttt{event\string_abort}, \texttt{if}, \texttt{find},
\texttt{let}, \texttt{new}, \texttt{insert}, and \texttt{get} which are similar to the corresponding
constructs of output processes but return a bitstring instead of
executing a process. 
They are not allowed to occur in \texttt{defined} conditions of \texttt{find}.
% and in input channels.
The constructs \texttt{<-R}, \texttt{new}, \texttt{event}, \texttt{event\string_abort}, and \texttt{insert} are not allowed to 
occur in conditions of \texttt{find} or {\tt get}.
We refer the reader to the description of 
processes below for a fully detailed explanation.
\begin{itemize}

\item $\Resa{x}{T}\texttt{;}M$ chooses a new
random number in type $T$, stores it in $x$, and returns the result 
of $M$.

$\Resb{x}{T}\texttt{;}M$ is equivalent to $\Resa{x}{T}\texttt{;}M$.

\item $\texttt{let }p \texttt{ = }M\texttt{ in }M'\texttt{ else }M''$
tries to decompose the term $M$ according to pattern $p$.
In case of success, returns the result of $M'$, otherwise 
the result of $M''$. 

The pattern $p$ can be:
\begin{itemize}

\item $x[\texttt{:}T]$ variable, possibly with its type. Matches any bitstring
(in type $T$), and stores it in $x$.

\item $f\texttt{(}p_1, \ldots, p_n\texttt{)}$ 
where the function symbol $f$ is declared 
\texttt{[data]}. Matches bitstrings $M$ equal to $f(M_1, \ldots, M_n)$
for some $M_1, \ldots, M_n$ that match $p_1, \ldots, p_n$.
(The poly-injectivity of $f$ allows us to compute possible
values $M_1, \ldots, M_n$ of its arguments from the value of $M$, and to check
whether $M$ is equal to the resulting value of $f(M_1, \ldots, M_n)$.) 

\item $\texttt{(}p_1, \ldots, p_n\texttt{)}$ tuples, which are particular \texttt{[data]}
functions encoding unambiguously the values of $p_1, \ldots, p_n$
and their type.

\item $\texttt{=}M'$ matches a bitstring equal to $M'$.

\end{itemize}
When $p$ is a variable, the \texttt{else}
branch can be omitted (it cannot be executed).

\item $x[:T] \texttt{ <- }M\texttt{;}M'$
stores the result of $M$  in $x$
and returns the result of $M'$. This is equivalent
to the construct 
$\texttt{let }x[:T] \texttt{ = }M\texttt{ in }M'$.

\item \texttt{if} $\mathit{cond}$ \texttt{then} $M$ \texttt{else} $M'$ is 
syntactic sugar for $\texttt{find suchthat }\mathit{cond}$ \texttt{then} $M$ \texttt{else} $M'$.
It returns the result of $M$ if the condition $\mathit{cond}$ evaluates to \texttt{true} and of $M'$ if $\mathit{cond}$ evaluates to \texttt{false}.

\item 
$\texttt{find}\ FB_1 \texttt{ orfind }\ldots\texttt{ orfind }FB_m \texttt{ else } M$ where $FB_j = u_{j1} \eqt  i_{j1} \leqt  n_{j1}, \ldots, u_{jm_j} \eqt  i_{jm_j} \leqt  n_{jm_j}$ $\texttt{suchthat}$ $cond_j$ $\texttt{then}$ $M_j$
evaluates the conditions
$cond_j$ for each $j$ and
each value of $i_{j1}, \ldots, i_{jm_j}$ in $[1, n_{j1}] 
\times \ldots \times [1, n_{jm_j}]$.
If none of these conditions is \texttt{true}, it returns the result of $M$.
Otherwise, it chooses randomly with (almost) uniform probability
one $j$ and one value of $i_{j1}, \ldots, i_{jm_j}$
such that the corresponding condition is \texttt{true},
stores it in $u_{j1}, \ldots, u_{jm_j}$ and returns 
the result of $M_j$.
See the explanation of the {\tt find} process below for more details.

\item $\texttt{event }e\texttt{(}M_1, \ldots, M_n\texttt{);}P$ executes the
event $e\texttt{(}M_1, \ldots, M_n\texttt{)}$, then executes $P$.
Events serve in recording the execution of certain parts of the program
for using them in queries. The symbol $e$ must have been declared
by an \texttt{event} declaration.

\item \texttt{event\string_abort $e$} executes event $e$ and aborts the game.
It is intended to be used in the right-hand side
of the definitions of some cryptographic primitives. (See also
the \texttt{equiv} declaration; events in the right-hand side can be
used when the simulation of left-hand side by the right-hand side
fails. CryptoVerif is going to find a bound for the probability that the event is
executed and include it in the probability of success of an attack.)

\item \texttt{insert} $\mathit{tbl}\texttt{(}M_1, \ldots, M_n\texttt{)}; M$
inserts the tuples $(M_1, \ldots, M_n)$ in the table $\mathit{tbl}$, 
then returns the result of $M$.
The table $\mathit{tbl}$ must have been declared with the appropriate
types using the $\texttt{table}$ declaration.

\item \texttt{get} $\mathit{tbl}\texttt{(}p_1, \ldots, p_n\texttt{)}$ \texttt{suchthat} $M$ \texttt{in} $M'$ \texttt{else} $M''$ tries to find an element of the table $\mathit{tbl}$ that matches the patterns $p_1, \ldots, p_n$ and such that $M$ is true. If it succeeds, it returns the result of $M'$ with the variables of $p_1, \ldots, p_n$ bound to that element of the table. If several elements match, one of them is chosen randomly with (almost) uniform probability. If no element matches, it returns the result of $M''$. 

When \texttt{suchthat} $M$ is omitted, it is equivalent to \texttt{suchthat} $\mathit{true}$. Internally, \texttt{get} is converted into \texttt{find} by CryptoVerif.

\end{itemize}

\ifchannels
The calculus distinguishes two kinds of processes: input processes
$\iprocess$ are ready to receive a message on a channel; 
output processes $\oprocess$ 
output a message on a channel after executing some internal computations.
When an input or output process is an identifier, it is substituted with 
its value defined by a \texttt{let} declaration.
\else
The calculus distinguishes two kinds of processes: oracle definitions
$\iprocess$ define new oracles; oracle bodies $\oprocess$ return a
result after executing some internal computations.  When a process
(oracle definition or oracle body) is an identifier, it is substituted
with its value defined by a \texttt{let} declaration.
\fi
Processes allow parenthesing for disambiguation.

Let us first describe \ifchannels input processes\else oracle definitions\fi:
\begin{itemize}

\item $\mathit{proc}(M_1, \dots, M_n)$ is replaced with $P\{M_1/x_1, \dots, M_n/x_n\}$ when $\mathit{proc}$ is declared by $\texttt{let}\ \mathit{proc}(x_1:T_1, \dots, x_n:T_n)\texttt{ = }P\texttt{.}$
where $P$ is an input process.
The terms $M_1, \dots, M_n$ must contain only variables, replication indices, and function applications.

\item \texttt{0} does nothing.

\item $Q \texttt{ | } Q'$ is the parallel composition of $Q$ and $Q'$.

\item $\texttt{!}i\leqt N\ Q$ represents $N$ copies of $Q$ in
parallel each with a different value of $i \in [1,N]$.  The identifier
$N$ must have been declared by $\texttt{param }N$.  The identifier $i$
cannot be referred to explicitly in the process; it is used only
implicitly as array index of variables defined under the replication
$\texttt{!}i\leqt N$. The replication $\texttt{!}i\leqt N$ can be 
abbreviated $\texttt{!} N$.

When a program point is under replications $\texttt{!}i_1\leqt N_1$,
\ldots, $\texttt{!}i_n\leqt N_n$, the \emph{current replication
indices} at that point are $i_1, \ldots, i_n$.

$\texttt{foreach }i\leqt N\texttt{ do }Q$ is equivalent to
$\texttt{!}i\leqt N\ Q$.

\ifchannels
\item The semantics of the input 
%$\cinput{c[M_1, \ldots, M_l]}{x_1[i_1, \ldots, i_m]:T_1, \ldots, x_k[i_1, \ldots, i_m]:T_k};P$ 
$\texttt{in(}\nonterm{channel}\texttt{,}\nonterm{pattern}\texttt{);}\oprocess$
will be explained below together with the
semantics of the output. 

\else
\item $O\texttt{(}p_1, \ldots, p_n\texttt{) := }P$ defines an oracle
$O$ taking arguments $p_1, \ldots, p_n$, and returning the result of
the oracle body $P$. The patterns $p_1, \ldots, p_n$ are as in the
\texttt{let} construct above, except that variables in $p$ that are
not under a function symbol $f(\ldots)$ must be declared with their
type.

\fi

\end{itemize}
Note that the construct \ifchannels $\textbf{newChannel }c;Q$ \else
$\textbf{newOracle }c;Q$ \fi used in research papers
is absent from the implementation: this construct is useful in the proof
of soundness of CryptoVerif, but not essential for encoding games
that CryptoVerif manipulates.

Let us now describe output processes:
\begin{itemize}

\item $\mathit{proc}(M_1, \dots, M_n)$ is replaced with $\texttt{let}$ $x_1 = M_1$ $\texttt{in}$ \dots $\texttt{let}$ $x_n = M_n$ $\texttt{in}$ $P$ when $\mathit{proc}$ is declared by $\texttt{let}\ \mathit{proc}(x_1:T_1, \dots, x_n:T_n)\texttt{ = }P\texttt{.}$ where $P$ is an output process.

\ifchannels
\item {\yield} yields control to another process, by outputting
an empty message on channel \textit{yield}. It can be understood
as an abbreviation for $\texttt{out(}\textit{yield}\texttt{,());0}$.
\else
\item {\yield} terminates the oracle, returning control to the caller.
\fi

\item $\texttt{event }e\texttt{(}M_1, \ldots, M_n\texttt{);}P$ executes the
event $e\texttt{(}M_1, \ldots, M_n\texttt{)}$, then executes $P$.
Events serve in recording the execution of certain parts of the program
for using them in queries. The symbol $e$ must have been declared
by an \texttt{event} declaration.

\item {\tt event\string_abort} $e$ executes event $e$ and terminates the game. 
(Nothing can be executed after
this instruction, neither by the protocol nor by the adversary.)
The symbol $e$ must have been declared
by an \texttt{event} declaration, without any argument.

\item $\Resa{x}{T}\texttt{;}P$ or $\Resb{x}{T}\texttt{;}P$ chooses a new
random number in type $T$, stores it in $x$, and executes $P$.
$T$ must be declared with option {\tt fixed}, {\tt bounded}, or {\tt nonuniform}.
Each such type $T$ comes with an associated default probability distribution $D_T$;
the random number is chosen according to that distribution.
The time for generated random numbers in that distribution
is bounded by $\texttt{time(new }T\texttt{)}$ or equivalently
$\texttt{time(<-R }T\texttt{)}$.
\begin{itemize}

\item When the type $T$ is {\tt nonuniform}, the default probability 
distribution $D_T$ for type $T$ may be non-uniform. It is left unspecified.
(Notice that random bitstrings with non-uniform distributions can also
be obtained by applying a function to a random bitstring choosen 
uniformly among a finite set of bitstrings, chosen in another type.)

\item When the type $T$ is {\tt fixed}, it consists of the set of all 
bitstrings of a certain length $n$.  Probabilistic Turing machines can
return uniformly distributed random numbers in such types, in bounded
time.
If $T$ is not marked {\tt nonuniform}, the default probability 
distribution $D_T$ for $T$ is the uniform distribution.

\item For other {\tt bounded} types $T$, probabilistic bounded-time Turing 
machines can choose random numbers with a distribution as close as we
wish to uniform, but may not be able to produce exactly a uniform
distribution. If $T$ is not marked {\tt nonuniform}, 
the default probability distribution $D_T$ is such that its distance
to the uniform distribution is at most $\epsrand{T}$. The distance between
two probability distributions $D_1$ and $D_2$ for type $T$ is
\[d(D_1, D_2) = \sum_{a \in T} | \Pr[X_1 = a] - \Pr[X_2 = a] |\]
where $X_i$ ($i = 1, 2$) is a random variable of distribution $D_i$.

For example, a possible algorithm to obtain a random integer in $[0,
m-1]$ is to choose a random integer $x'$ uniformly among $[0, 2^k-1]$
for a certain $k$ large enough and return $x' \bmod m$. 
By euclidian division, we have $2^k = qm+r$ with $r \in [0,m-1]$.
With this algorithm
\[\Pr[x = a] = \begin{cases}
\frac{q+1}{2^k} &\text{if }a \in [0,r-1]\\
\frac{q}{2^k} &\text{if }a \in [r,m-1]
\end{cases}\]
so
\[\left|\Pr[x = a] -\frac{1}{m}\right| = \begin{cases}
\frac{q+1}{2^k} - \frac{1}{m}&\text{if }a \in [0,r-1]\\
\frac{1}{m} - \frac{q}{2^k}&\text{if }a \in [r,m-1]
\end{cases}\]
Therefore
\[\begin{split}
d(D_T, \mathit{uniform}) 
&=  \sum_{a \in T} \left| \Pr[x = a] - \frac{1}{m} \right|
= r\left(\frac{q+1}{2^k} - \frac{1}{m}\right) - (m-r)\left(\frac{1}{m} - \frac{q}{2^k}\right)\\
&=\frac{2r(m-r)}{m.2^k} \leq \frac{m}{2^k}
\end{split}\]
%If r <= m/2, we upper bound r(m-r) by m/2.m
%If r >= m/2, m-r <= m/2, so we upper bound r(m-r) by m.m/2 
so we can take $\epsrand{T} = \frac{m}{2^k}$. A given precision of $\epsrand{T} = \frac{1}{2^{k'}}$ can be obtained by choosing $k = k' + \text{number of bits of }m$ random bits.

When \texttt{ignoreSmallTimes} is set to a value greater than 0
(which is the default),
the time for random number generations and the probability
$\epsrand{T}$ are ignored, to make probability formulas 
more readable.

\end{itemize}

\item $\texttt{let }p \texttt{ = }M\texttt{ in }P\texttt{ else }P'$
tries to decompose the term $M$ according to pattern $p$.
In case of success, executes $P$, otherwise executes $P'$.

The pattern $p$ can be:
\begin{itemize}

\item $x[\texttt{:}T]$ variable, possibly with its type. Matches any bitstring
(in type $T$), and stores it in $x$.

\item $f\texttt{(}p_1, \ldots, p_n\texttt{)}$ 
where the function symbol $f$ is declared 
\texttt{[data]}. Matches bitstrings $M$ equal to $f(M_1, \ldots, M_n)$
for some $M_1, \ldots, M_n$ that match $p_1, \ldots, p_n$.
(The poly-injectivity of $f$ allows us to compute possible
values $M_1, \ldots, M_n$ of its arguments from the value of $M$, and to check
whether $M$ is equal to the resulting value of $f(M_1, \ldots, M_n)$.) 

\item $\texttt{(}p_1, \ldots, p_n\texttt{)}$ tuples, which are particular \texttt{[data]}
functions encoding unambiguously the values of $p_1, \ldots, p_n$
and their type.

\item $\texttt{=}M'$ matches a bitstring equal to $M'$.

\end{itemize}
The \texttt{else} clause is never executed when the pattern
is simply a variable.
When $\texttt{else }P'$ is omitted, it is equivalent to \texttt{else} \yield.
Similarly, when $\texttt{in }P$ is omitted, it is equivalent to 
\texttt{in} \yield.

\item $x[\texttt{:}T] \texttt{ <- }M\texttt{;}P$
stores the result of term $M$ in $x$ and executes $P$.
$M$ must be of type $T$ when $T$ is mentioned.
This is equivalent to the construct $\texttt{let } x[\texttt{:}T] 
\texttt{ = }M\texttt{ in }P$.

\item \texttt{if} $\mathit{cond}$ \texttt{then} $P$ \texttt{else} $P'$ is 
syntactic sugar for $\texttt{find suchthat }\mathit{cond}$ \texttt{then} $P$ \texttt{else} $P'$.
It executes $P$ if the condition $\mathit{cond}$ evaluates to \texttt{true} and $P'$ if $\mathit{cond}$ evaluates to \texttt{false}.
When the \texttt{else} clause is omitted, it is implicitly \texttt{else} \yield.
(\texttt{else 0} would not be syntactically correct.) 

\item 
Next, we explain the process 
$\texttt{find}\ FB_1 \texttt{ orfind }\ldots\texttt{ orfind }FB_m \texttt{ else } P$ where each branch $FB_j$ is $FB_j = u_{j1} \eqt  i_{j1} \leqt  n_{j1}, \ldots, u_{jm_j} \eqt  i_{jm_j} \leqt  n_{jm_j}$ $\texttt{suchthat}$ $cond_j$ $\texttt{then}$ $P_j$.

A simple example is the following:
$\texttt{find }u \eqt  i \leqt  n$ \texttt{suchthat} $\texttt{defined(}x[i]\texttt{) \&\& }x[i] = a$ \texttt{then} $P'$ \texttt{else} $P$
tries to find an index $i$ such that $x[i]$ is defined and
$x[i] = a$, and when such an $i$ is found,
it stores that $i$ in $u$ and executes $P'$;
otherwise, it executes $P$.
In other words, this $\texttt{find}$ construct looks for the value
$a$ in the array $x$, and when $a$ is found, it stores in
$u$ an index such that $x[u] = a$. Therefore, the $\texttt{find}$ construct
allows us to access arrays, which is key for our purpose.

More generally, $\texttt{find}\ u_{1} \eqt  i_1 \leqt  n_{1}, \ldots, u_{m} \eqt  i_m \leqt  n_{m}$ $\texttt{suchthat}$ $\texttt{defined(}M_{1}, \ldots, M_{l}\texttt{) \&\& } M$ \texttt{then} $P'$ \texttt{else} $P$ tries to find values of $i_1, \ldots, i_m$ for which
$M_1, \ldots, M_l$ are defined and $M$ is true. In case of success, it
stores the values of $i_1, \ldots, i_m$ in $u_1, \ldots, u_m$
executes $P'$. In case of failure, it executes $P$.

This is further generalized to $m$ branches: 
$\texttt{find}\ FB_1 \texttt{ orfind }\ldots\texttt{ orfind }FB_m \texttt{ else } P$
where $FB_j = u_{j1} \eqt  i_{j1} \leqt  n_{j1}, \ldots, u_{jm_j} \eqt  i_{jm_j} \leqt  n_{jm_j}$ $\texttt{suchthat}$ $\texttt{defined(}M_{j1}, \ldots, M_{jl_j}\texttt{) \&\& }M_j$ $\texttt{then}$ $P_j$
tries to find a branch $j$ in $[1,m]$ such that there are 
values of $i_{j1}, \ldots, i_{jm_j}$ for which 
$M_{j1}, \ldots, M_{jl_j}$ are defined and $M_j$ is true. In case of 
success, it stores the value of $i_{j1}, \ldots, i_{jm_j}$ in $u_{j1}, \ldots, u_{jm_j}$ and executes $P_j$.
In case of failure for all branches, it executes $P$. 
More formally, it evaluates the conditions
$cond_j = \texttt{defined(}M_{j1}, \ldots, M_{jl_j}\texttt{) \&\& }M_j$ for each $j$ and
each value of $i_{j1}, \ldots, i_{jm_j}$ in $[1, n_{j1}] 
\times \ldots \times [1, n_{jm_j}]$.
If none of these conditions is \texttt{true}, it executes $P$.
Otherwise, it chooses randomly with almost uniform 
probability\footnote{Precisely, the distance between the distribution actually
used for choosing $j, i_{j1}, \ldots, i_{jm_j}$ and the uniform
distribution is at most \texttt{eps\_find}. See the explanation of $\Resa{x}{T}$
for details on how to achieve this.}
one $j$ and one value of $i_{j1}, \ldots, i_{jm_j}$
such that the corresponding condition is \texttt{true},
stores that value in $u_{j1}, \ldots, iu_{jm_j}$ and executes $P_j$.

In the general case, the conditions $cond_j$ are of the form
$\texttt{defined(}M_1, \ldots, M_l\texttt{)}\ [\texttt{\&\& }M]$ or simply $M$.
The condition $\texttt{defined(}M_1, \ldots, M_l\texttt{)}$ means that
$M_1, \ldots, M_l$ are defined.
At least one of the two conditions $\texttt{defined}$ or $M$ must be
present. Omitted $\texttt{defined}$ conditions are considered empty;
when $M$ is omitted, it is considered \texttt{true}. 

The variables $i_{j1}, \ldots, i_{jm_j}$ are considered as replication indices, and are used in the $\texttt{defined}$ condition and in $M_j$: they are temporary variables that are used as loop indices to look for indices that satisfy the desired conditions. Once suitable indices are found, their value is stored in $u_{j1}, \ldots, u_{jm_j}$ and the \texttt{then} branch is executed using these variables. It is possible to make array accesses to $u_{j1}, \ldots, u_{jm_j}$ (such as $u_{j1}[M_1, \ldots, M_k]$) elsewhere in the game, which is not possible for $i_{j1}, \ldots, i_{jm_j}$.

As an abbreviation, one may write $FB_j = u_{j1} \leqt  n_{j1}, \ldots, u_{jm_j} \leqt  n_{jm_j}$ $\texttt{suchthat}$ $\texttt{defined(}M_{j1}, \allowbreak\ldots,\allowbreak M_{jl_j}\texttt{)}$ $\&\&$ $M_j$ $\texttt{then}$ $P_j$. In this case, the same identifier $u_{jk}$ is used for both the variable and the associated replication index $i_{jk}$.


A variant of {\tt find} is {\tt find[unique]}. 
Consider the process 
$\texttt{find[unique]}$ $FB_1$ \texttt{orfind} \ldots \texttt{orfind} $FB_m$ \texttt{else} $P$
where $FB_j = u_{j1} \eqt  i_{j1} \leqt  n_{j1}, \ldots, u_{jm_j} \eqt  i_{jm_j} \leqt  n_{jm_j}$ $\texttt{suchthat}$ $\texttt{defined(}M_{j1},\allowbreak \ldots, \allowbreak M_{jl_j}\texttt{)}$ $\texttt{\&\&}$ $M_j$ \texttt{then} $P_j$.
When there are several values of $j, i_{j1}, \ldots, i_{jm_j}$ for which 
$M_{j1}, \ldots, M_{jl_j}$ are defined and $M_j$ is true, this process executes an event $\mathsf{NonUnique}$ and aborts the game. In all other cases, it behaves as {\tt find}.
Intuitively, {\tt find[unique]} should be used when there is a negligible probability of finding several suitable values of $j, i_{j1}, \ldots, i_{jm_j}$. The construct {\tt find[unique]} is typically not used in the initial game. (One would have to prove manually that there is indeed a negligible probabibility of finding several suitable values of $j, i_{j1}, \ldots, i_{jm_j}$. CryptoVerif displays a warning if {\tt find[unique]} occurs in the initial game.) However, {\tt find[unique]} is used in the specification of cryptographic primitives, in the right-hand of equivalences specified by \texttt{equiv}. 

\item \texttt{insert} $\mathit{tbl}\texttt{(}M_1, \ldots, M_n\texttt{)}; P$
inserts the tuples $(M_1, \ldots, M_n)$ in the table $\mathit{tbl}$, 
then executes $P$.
The table $\mathit{tbl}$ must have been declared with the appropriate
types using the $\texttt{table}$ declaration.

\item \texttt{get} $\mathit{tbl}\texttt{(}p_1, \ldots, p_n\texttt{)}$ \texttt{suchthat} $M$ \texttt{in} $P$ \texttt{else} $P'$ tries to find an element of the table $\mathit{tbl}$ that matches the patterns $p_1, \ldots, p_n$ and such that $M$ is true. If it succeeds, it executes $P$ with the variables of $p_1, \ldots, p_n$ bound to that element of the table. If several elements match, one of them is chosen randomly with (almost) uniform probability. If no element matches, it executes $P'$. 

When \texttt{else} $P'$ is omitted, it is equivalent to \texttt{else} \yield. When \texttt{suchthat} $M$ is omitted, it is equivalent to \texttt{suchthat} $\mathit{true}$. Internally, \texttt{get} is converted into \texttt{find} by CryptoVerif.

\ifchannels
\item
Finally, let us explain the output $\texttt{out(}c\texttt{[}M_1,
\ldots, M_l\texttt{],}N\texttt{);}Q$.  A channel $c\texttt{[}M_1,
\ldots, M_l\texttt{]}$ consists of both a channel name $c$ (declared
by $\texttt{channel }c$) and a tuple of terms $M_1, \ldots, M_l$.  Terms
$M_1, \ldots, M_l$ are intuitively analogous to IP addresses and ports
which are numbers that the adversary may guess.  Two channels are
equal when they have the same channel name and terms that evaluate to
the same bitstrings.
%
A semantic configuration always consists of a single output process
(the process currently being executed) and several input processes.
When the output process executes $\texttt{out(}c\texttt{[}M_1, \ldots,
M_l\texttt{],}N\texttt{);}Q$, one looks for an input on the same
channel in the available input processes. If no such input process is
found, the process blocks.  Otherwise, one such input process
$\texttt{in(}c\texttt{[}M'_1, \ldots, M'_l\texttt{],}p\texttt{);}P $
is chosen randomly with (almost) uniform probability. The communication is then
executed: the output message $N$ is evaluated, its result is truncated
to the maximum length of bitstrings on channel $c$, the obtained
bitstring is matched against pattern $p$.  Finally, the output process
$P$ that follows the input is executed. The input process $Q$ that
follows the output is stored in the available input processes for
future execution.

Patterns $p$ are as in the \texttt{let} process, except that
variables in $p$ that are not under a function symbol $f(\ldots)$
must be declared with their type.

In the game as given to CryptoVerif, the channel can be either
$c[i_1, \ldots, i_n]$ where $i_1, \ldots, i_n$ are the current
replication indices at the considered input or output, or just a
channel name $c$, as an abbreviation for $c[i_1, \ldots, i_n]$.  It is
recommended to use as channel a different channel name for each input
and output. Then the adversary has full control over the network: it
can decide precisely from which copy of which input it receives a
message and to which copy of which output it sends a message, by using
the appropriate channel name and value of the replication indices.

Note that the syntax requires an output
to be followed by an input process, as in~\cite{Laud05}. If one
needs to output several messages consecutively, one can simply
insert fictitious inputs between the outputs. The adversary can
then schedule the outputs by sending messages to these inputs.

\else

\item 
$\texttt{return(}N_1, \ldots, N_l\texttt{);}Q$ terminates the oracle,
returning the result of the terms $N_1, \ldots, N_l$. Then, it makes
available the oracles defined in $Q$.

\fi

\end{itemize}

In this calculus, all variables are implicitly arrays.  When a
variable $x$ is defined (by \texttt{new}, 
\texttt{<-R}, \texttt{<-}, 
\texttt{let}, \texttt{find},
\ifchannels \texttt{in} \else and oracle definitions\fi) 
under replications 
$\texttt{!}i_1\leqt N_1$, \ldots, $\texttt{!}i_n\leqt N_n$, 
$x$ has implicitly indices $i_1,
\ldots, i_n$: $x$ stands for $x[i_1, \ldots, i_n]$. Arrays allow us to
have full access to the state of the process. Arrays can be read using
\texttt{find}.
%
Similarly, when $x$ is used with $k < n$ indices the missing $n-k$ indices are
implicit: $x[u_1, \ldots, u_k]$ stands for $x[i_1, \ldots, i_{n-k},
u_1, \ldots, u_k]$ where $i_1, \ldots, i_{n-k}$ must be the $n-k$
first replication indices both at the creation of $x$ and at the usage
$x[u_1, \ldots, u_k]$. (So the usage and creation of $x$ must
be under the same $n-k$ top-most replications.)
%
\ifchannels\else
When an oracle $O$ is defined under $\texttt{foreach }i_1\leqt N_1$, 
\ldots, $\texttt{foreach }i_n\leqt N_n$, it also implicitly
defines $O[i_1, \ldots, i_n]$.
\fi

In the initial game, several variables may be defined with the same
name, but they are immediately renamed to different names, so that
after renaming, each variable is defined once.  When several variables
are defined with the same name, they can be referenced only under
their definition without explicit array indices, because for other
references, we would not know which variable to reference after
renaming.

In subsequent games created by CryptoVerif, a variable may be defined
at several occurrences, but these occurrences must be in different
branches of \texttt{if}, \texttt{find}, or \texttt{let}, so that
they cannot be executed with the same value of the array indices.
This constraint guarantees that each array cell is defined at most once.

Each usage of $x$ must be either:
\begin{itemize}

\item $x$ without array index syntactically under its definition.
(Then $x$ is implicitly considered to have as indices the current
replication indices at its definition.)

\item $x$ possibly with array indices inside the \texttt{defined}
condition of a find.

\item $x[M_1, \ldots, M_n]$ in $M$  in a find branch
$\ldots\texttt{ suchthat defined(}M'_1, \ldots, M'_l\texttt{) \&\& }M
\texttt{ then }\ldots$, such that $x[M_1, \ldots, M_n]$
is a subterm of $M'_1, \ldots, M'_l$. 

\item $x[M_1, \ldots, M_n]$ in $P$  in a find branch
$u_1 \eqt  i_1 \leqt  n_1, \ldots, u_m \eqt  i_m \leqt  n_m$ $\texttt{suchthat}$ $\texttt{defined(}M'_1, \allowbreak \ldots, \allowbreak M'_l\texttt{)}$ $\texttt{\&\&}$ \ldots
$\texttt{then }P$, such that $x[M_1, \ldots, M_n] = M\{u_1/i_1,\dots,u_m/i_m\}$
and $M$ is a subterm of $M'_1, \ldots, M'_l$. 

\item $x[M_1, \ldots, M_n]$ in $M''$ in a find branch
$u_1 \eqt  i_1 \leqt  n_1, \ldots, u_m \eqt  i_m \leqt  n_m$ $\texttt{suchthat}$ $\texttt{defined(}M'_1, \allowbreak \ldots, \allowbreak M'_l\texttt{)}$ $\texttt{\&\&}$ \ldots
$\texttt{then }M''$, such that $x[M_1, \ldots, M_n] = M\{u_1/i_1,\dots,u_m/i_m\}$
and $M$ is a subterm of $M'_1, \ldots, M'_l$. 

\end{itemize}
These syntactic constraints guarantee that a variable is accessed
only when it is defined. Moreover, the variables defined in
conditions of {\tt find} or in patterns or conditions of {\tt get}
must not have array accesses (that is, accesses corresponding to the
last four cases above).

Finally, the calculus is equipped with a type system.
To be able to use variables outside their scope (by \texttt{find}),
the type checking algorithm works in two passes. 

In the first pass, 
it collects the type of each variable: when a variable $x$ is
defined with type $T$ under 
replications $\texttt{!}N_1$, \ldots, $\texttt{!}N_n$,
$x$ has type $[1, N_1] \times \ldots \times [1, N_n] \rightarrow T$.
When the type of $x$ is not explicitly given in its declaration
(\ifchannels in \texttt{<-} or in patterns in \texttt{let} or \texttt{in}\else in \texttt{<-} or 
in patterns in \texttt{let} or oracle definitions\fi), its type is left undefined
in this pass, and $x$ cannot be used outside its scope.

In the second pass, the type system checks the following requirements:
%
In $x\texttt{[}M_1, \ldots, M_m\texttt{]}$, $M_1, \ldots, M_m$ must be of the suitable 
interval type, that is, a suffix of the types of replication indices
at the definition of $x$.
%
In $f\texttt{(}M_1, \ldots, M_m\texttt{)}$, if $f$ has been declared
by $\texttt{fun }f\texttt{(}T_1, \ldots, T_m\texttt{):}T$, $M_j$ must
be of type $T_j$, and $f(M_1, \ldots, M_m)$ is then of type $T$.
%$T_1, \ldots, T_m$ must be bitstring types.
%
In $\texttt{(}M_1, \ldots, M_n\texttt{)}$, $M_j$ can be of any
bitstring type (that is, not an index type $[1, N]$), 
and the result is of type \texttt{bitstring}.
%
In $M_1 \texttt{ = }M_2$ and $M_1 \texttt{ <> }M_2$, $M_1$ and $M_2$
must be of the same type, and the result is of type
$\texttt{bool}$. In $M_1 \texttt{ || }M_2$ and $M_1 \texttt{ \&\&
}M_2$, $M_1$ and $M_2$ must be of type $\texttt{bool}$ and the result
is of type $\texttt{bool}$.
%
The type system requires each subterm to be well-typed. 
%Event
Furthermore, in $\texttt{event }e\texttt{(}M_1, \ldots,
M_n\texttt{)}$, if $e$ has been declared by $\texttt{event
}e\texttt{(}T_1, \ldots, T_n\texttt{)}$, $M_j$ must be of type $T_j$.
%New
In $\Resa{x}{T}$ or $\Resb{x}{T}$, $T$ must be declared with option {\tt bounded} (or {\tt fixed}).
%If
In $\texttt{if }M\texttt{ then }\ldots\texttt{ else }\ldots$, 
$M$ must be of type $\texttt{bool}$.
%Find
Similarly, for
\[\texttt{find }\ldots\texttt{ orfind } \ldots \texttt{ suchthat defined(}
\ldots\texttt{) \&\& }M\texttt{ then }\ldots\]
$M$ must be of type $\texttt{bool}$.
%Let
In $\texttt{let }p\texttt{ = }M\texttt{ in }\ldots$, $M$ and $p$ must
be of the same type. For function application and tuple patterns, the
typing rule is the same as for the corresponding terms.  The pattern
$x:T$ is of type $T$; the pattern $x$ can be of any bitstring type,
determined by the usage of $x$ (when the pattern $x$ is used as
argument of a tuple pattern, its type is \texttt{bitstring}); the
pattern $\texttt{=}M$ is of the type of $M$.
%Out
\ifchannels
In $\texttt{out(}c\texttt{[}M_1, \ldots, M_n\texttt{],}M\texttt{)}$,
$M$ must be of a bitstring type.
\else
In $\texttt{return(}M_1, \ldots, M_n\texttt{)}$,
$M_j$ must be of a bitstring type $T_j$ for all $j \leq n$
and that return instruction is said to be of type $T_1 \times \ldots 
\times T_n$.
All return instructions in an oracle body $P$ (excluding return
instructions that occur in oracle definitions $Q$ in processes of the form 
$\texttt{return(}M_1, \ldots, M_n\texttt{);}Q$) must be of the same
type, and that type is said to be the type of the oracle body $P$.
%
For each oracle definition $O\texttt{(}p_1, \ldots, p_m\texttt{) :=
}P$ under $\texttt{foreach }i_1\leqt N_1$, \ldots,
$\texttt{foreach }i_n\leqt N_n$, the oracle $O$ is said to be of
type $[1, N_1] \times \ldots \times [1, N_n] \rightarrow T'_1 \times
\ldots \times T'_m \rightarrow T_1 \times \ldots \times T_n$ where
$p_j$ is of type $T'_j$ for all $j \leq m$ and $P$ is of type $T_1
\times \ldots \times T_n$. When an oracle has several definitions,
it must be of the same type for all its definitions. Furthermore,
definitions of the same oracle $O$ must not occur on both sides
of a parallel composition $Q \texttt{|} Q'$ (so that several definitions
of the same oracle cannot be simultaneously available).
\fi


A declaration can be:
\begin{itemize}

\item ${\tt set\ } \nonterm{parameter} \texttt{ = } \nonterm{value}{\tt .}$

This declaration sets the value of configuration parameters.
The following parameters and values are supported:

\begin{itemize}

\item \texttt{set allowUndefinedVar = false.}\\
\texttt{set allowUndefinedVar = true.}

By default (\texttt{allowUndefinedVar = false}), variables in
\texttt{defined} conditions must be defined somewhere in the game.
The setting \texttt{allowUndefinedVar = true} allows
\texttt{defined} conditions with variables that are defined
nowhere. The corresponding branch of \texttt{find} is then
removed immediately, since the \texttt{defined} condition does not hold.
This setting is useful to parse intermediate games generated by
CryptoVerif, because such impossible \texttt{defined} conditions
may occur in these games.

\item \texttt{set diffConstants = true.}\\
\texttt{set diffConstants = false.}

When {\tt true}, different constant symbols are assumed to have a
different value. When {\tt false}, CryptoVerif does not make this
assumption.

\item \texttt{set constantsNotTuple = true.}\\
\texttt{set constantsNotTuple = false.}

When {\tt true}, constant symbols are assumed to be different from the
result of applying a tuple function to any argument. When {\tt false},
CryptoVerif does not make this assumption.

\item \texttt{set expandAssignXY = true.}\\
\texttt{set expandAssignXY = false.}

When {\tt true}, CryptoVerif automatically removes assignments 
{\tt let x = y} or {\tt x <- y}
where {\tt x} and {\tt y} are variables by substituting {\tt y} for {\tt x}
(in the transformation {\tt remove\string_assign useless})
When {\tt false}, this transformation is not performed as part of
{\tt remove\string_assign useless}.

\item \texttt{set minimalSimplifications = true.}\\
\texttt{set minimalSimplifications = false.}

When {\tt true}, simplification replaces a term with a rewritten term
only when the rewriting has used at least one rewriting rule given
by the user, not when only equalities that come from {\tt let} definitions
and other instructions in the game have been used.
When {\tt false}, a term is replaced with its rewritten form in
all cases. The latter configuration often leads to replacing
a term with a more complex one, in particular expanding {\tt let}
definitions, thus duplicating their contents.

\item \texttt{set mergeBranches = true.}\\
\texttt{set mergeBranches = false.}

When {\tt true}, the transformation {\tt merge\_branches} is applied
after simplification, to merge branches of {\tt if}, {\tt let},
and {\tt find} when all branches execute the same code.
This is useful in order to remove the test, which can remove
a use of a secret.
When {\tt false}, this transformation is not performed. 
This is useful in particular when the test has been
manually introduced in order to force CryptoVerif to
distinguish cases.

\item \texttt{set mergeArrays = true.}\\
\texttt{set mergeArrays = false.}

When {\tt true}, {\tt merge\_branches} advises {\tt merge\_arrays} commands
to make the merging of branches of {\tt if}, {\tt find}, {\tt let}
succeed more often. When {\tt false}, this advice is not
automatically given and the user should use the manual command
{\tt merge\_arrays} (defined in 
Section~\ref{sec:interact}) to perform the merging.

\item \texttt{set uniqueBranch = true.}\\
\texttt{set uniqueBranch = false.}

% We say that a {\tt find} is \emph{unique} when there is at most
% one branch and one value of the indices that we look up,
% for which the conditions are true.
When {\tt uniqueBranch = true}, the following transformation is 
enabled as part of {\tt simplify}:
if a branch of a {\tt find[unique]} is proved to succeed, 
then simplification removes all other branches of that {\tt find}.
When {\tt uniqueBranch = false}, this transformation is not performed. 

\item \texttt{set uniqueBranchReorganize = true.}\\
\texttt{set uniqueBranchReorganize = false.}

When {\tt uniqueBranchReorganize = true}, the following transformations are 
enabled as part of {\tt simplify}:
\begin{itemize}
\item
If a {\tt find[unique]} occurs in the {\tt then} branch 
of a {\tt find[unique]}, we reorganize them.

\item 
If a {\tt find[unique]} occurs in the condition of a {\tt find}, 
we reorganize them.

\end{itemize}
When {\tt uniqueBranchReorganize = false}, these transformations are not performed. 

\item \texttt{set inferUnique = false.}\\
\texttt{set inferUnique = true.}

When \texttt{inferUnique = true}, CryptoVerif tries to infer
that a \texttt{find} that is not explicitly tagged \texttt{[unique]}
is in fact unique, by showing that having several solutions
for this \texttt{find} leads to a contradiction.
When this proof succeeds, the \texttt{find} becomes {\tt find[unique]}.

When \texttt{inferUnique = false}, CryptoVerif does not try to
make such proofs and just exploits explicit \texttt{[unique]} tags.

\item \texttt{set autoSARename = true.}\\
\texttt{set autoSARename = false.}

When {\tt true}, and a variable is defined several times and
used only in the scope of its definition with the current
replication indices at that definition, each definition of
this variable is renamed to a different name, and the uses
are renamed accordingly, by the transformation {\tt remove\string_assign}.
When {\tt false}, such a renaming is not done automatically,
but in manual proofs, it can be requested specifically for each 
variable by {\tt SArename x}, where {\tt x} is the name of the variable.

\item \texttt{set autoRemoveAssignFindCond = true.}\\
\texttt{set autoRemoveAssignFindCond = false.}

When {\tt true}, the default removal of assignments performed by
CryptoVerif removes assignments on variables $x$ defined by
${\tt let}\ x = M\ {\tt in}\ ...$ inside a condition of {\tt find}.
When {\tt false}, the removal of this assignments is not
performed automatically, but in manual proofs, it can be requested 
by the command {\tt remove\string_assign\ findcond}.

\item \texttt{set autoRemoveIfFindCond = true.}\\
\texttt{set autoRemoveIfFindCond = false.}

When {\tt true}, simplification removes {\tt if} in defined conditions
of {\tt find} by transforming them into logical formulae.
When {\tt false}, this removal is not performed.

\item \texttt{set autoMove = true.}\\
\texttt{set autoMove = false.}

When {\tt true}, the transformation {\tt move all} is automatically
executed after each cryptographic transformation. This transformation
moves random number generations ({\tt new} or {\tt
<-R}) downwards as much as possible, duplicating them when crossing
a {\tt if}, {\tt let}, or {\tt find}.  (A future {\tt SArename}
transformation may then enable us to distinguish cases depending on
which of the duplicated random number generations a variable comes
from.)  It also moves assignments down in the syntax tree but without
duplicating them, when the assignment can be moved under a {\tt if},
{\tt let}, or {\tt find}, in which the assigned variable is used
only in one branch. (In this case, the assigned term is computed in
fewer cases thanks to this transformation.)

When {\tt false}, the transformation {\tt move all} is never
automatically executed.

\item \texttt{set optimizeVars = false.}\\
\texttt{set optimizeVars = true.}

When {\tt true}, CryptoVerif tries to reduce the number of different
intermediate variables introduced in cryptographic
transformations. This can lead to distinguishing fewer cases,
which unfortunately often leads to a failure of the proof.
When {\tt false}, different intermediate varaibles are used for
each occurrence of the transformed expression.

\item \texttt{set interactiveMode = false.}\\
\texttt{set interactiveMode = true.}

When {\tt false}, CryptoVerif runs automatically.
When {\tt true}, CryptoVerif waits for instructions of the user
on how to perform the proof. (See Section~\ref{sec:interact}
for details on these instructions.)
%
This setting is ignored when proof instructions are included
in the input file using the \texttt{proof} command.
In this case, the instructions given in the \texttt{proof} command
are executed, without user interaction.

\item \texttt{set autoAdvice = true.}\\
\texttt{set autoAdvice = false.}

In interactive mode, when \texttt{autoAdvice = true}, execute the
advised transformations automatically. When \texttt{autoAdvice = false},
display the advised transformations, but do not execute them.
The user may then give them as instructions if he wishes.

\item \texttt{set noAdviceCrypto = false.}\\
\texttt{set noAdviceCrypto = true.}

When \texttt{noAdviceCrypto = true}, prevents the cryptographic 
transformations from generating advice. Useful mainly for debugging
the proof strategy.

\item \texttt{set noAdviceGlobalDepAnal = false.}\\
\texttt{set noAdviceGlobalDepAnal = true.}

When \texttt{noAdviceGlobalDepAnal = true}, prevents the global
dependency analysis from generating advice. Useful when the global
dependency analysis generates bad advice.

\item \texttt{set simplifyAfterSARename = true.}\\
\texttt{set simplifyAfterSARename = false.}

When \texttt{simplifyAfterSARename = true}, apply simplification after
each execution of the SArename transformation. This slows down
the system, but enables it to succeed more often.

\item \texttt{set backtrackOnCrypto = false.}\\
\texttt{set backtrackOnCrypto = true.}

When \texttt{backtrackOnCrypto = true}, use backtracking when the proof
fails, to try other cryptographic transformations. This slows down
the system considerably (so it is false by default), but enables
it to succeed more often, in particular for public-key protocols
that mix several primitives. One usage is to try first with the default
setting and, if the proof fails although the property
is believed to hold, try again with backtracking.

\item \texttt{set useKnownEqualitiesInCryptoTransform = true.}\\
\texttt{set useKnownEqualitiesInCryptoTransform = false.}

When \texttt{useKnownEqualitiesInCryptoTransform = true}, CryptoVerif
relies on known equalities between terms to replace variables with
their values in the cryptographic transformations.
When it is false, CryptoVerif just uses the variables as their
appear in the game, and relies only on advice to replace variables
with their values. 

\item \texttt{set priorityEventUnchangedRand = $n$.} (default: 5)

During the cryptographic transformation, variables that occur in event
and are mapped to random variables marked \texttt{[unchanged]} in the
equivalence can be left unchanged.

Sometimes, it is also possible to transform the term that contains
them using one of the oracles of the equivalence.

This settings determines which option is chosen: CryptoVerif prefers
leaving the variable unchanged rather than using an oracle with
priority at least $n$. It prefers using an oracle with priority
less than $n$ rather than leaving the variable unchanged.

\item \texttt{set casesInCorresp = true.}\\
\texttt{set casesInCorresp = false.}

When \texttt{casesInCorresp = true}, CryptoVerif distinguishes
cases depending on the definition point of variables, to infer
more facts in order to prove correspondence properties.
However, this can be slow in complex cases. Using
\texttt{set casesInCorresp = false} disables this case
distinction and speeds up the proof of correspondences.

% \item \texttt{set detectIncompatibleDefined = true.}\\
% \texttt{set detectIncompatibleDefined = false.}
% 
% When true, the simplification detects when two \texttt{defined}
% conditions of \texttt{find} are incompatible because they require two
% variables to be simultaneously defined at the same indices, while
% this is in fact impossible in the considered game. Detecting this
% is rather costly, so it can be turned off.
%BB: made this setting undocumented, since now checking that is not so costly.

\item $\texttt{set ignoreSmallTimes = }\nonterm{n}{\tt .}$ (default 3)

When {\tt 0}, the evaluation of the runtime is very precise,
but the formulas are often too complicated to read.

When {\tt 1}, the system ignores many small values when computing
the runtime of the games. It considers only function applications
and pattern matching.

When {\tt 2}, the system ignores even more details, including
application of boolean operations (\texttt{\&\&},
\texttt{\string|\string|}, \texttt{not}), constants generated by the
system, \texttt{()} and matching on \texttt{()}. It ignores the
creation and decomposition of tuples in \ifchannels
inputs and outputs\else oracle calls and returns\fi.

When {\tt 3}, the system additionally ignores the time of equality
tests between values of bounded length, as well as the time of
all constants.

\item $\texttt{set maxIterSimplif = }\nonterm{n}{\tt .}$ (default 2)

Sets the maximum number of repetitions of the simplification transformation
for each {\tt simplify} instruction.
A greater value slows down the system but may enable it to obtain
simpler games, and therefore increase its chances of success.
When $n \leq 0$, repeats simplification until a fixpoint is reached.

\item $\texttt{set maxAddFactDepth = }\nonterm{n}{\tt .}$ (default 1000)

Sets the maximum depth of recursion in the addition and simplification
of known facts. 
When $n \leq 0$, puts no limit on this depth of recursion.
Putting a limit avoids an infinite loop in some rare cases.

\item $\texttt{set maxTryNoVarDepth = }\nonterm{n}{\tt .}$ (default 20)

Sets the maximum depth of recursion in the replacement of
variables with their values.
When $n \leq 0$, puts no limit on this depth of recursion.
Putting a limit avoids an infinite loop in some rare cases.

\item $\texttt{set maxIterRemoveUselessAssign = }\nonterm{n}{\tt .}$ (default 10)

Sets the maximum number of repetitions of the removal of useless assignments 
for each {\tt remove\string_assign useless} instruction.
A greater value slows down the system but may enable it to obtain
simpler games, and therefore increase its chances of success.
When $n \leq 0$, repeats removal of useless assignments until a fixpoint 
is reached.

\item $\texttt{set minAutoCollElim = }\nonterm{s}{\tt .}$ (default \texttt{size15})

Sets the minimum size of a type for which elimination of collisions is
possible automatically. The size argument $\nonterm{s}$ can be \texttt{large}, \texttt{password}, or \texttt{size}$n$ (see the \texttt{type} declaration for their meaning).

\item $\texttt{set maxAdvicePossibilitiesBeginning = }n_1{\tt .}$ (default \texttt{50})\\
$\texttt{set maxAdvicePossibilitiesEnd = }n_2{\tt .}$ (default \texttt{10})

In cryptographic transformations, when CryptoVerif can transform many terms in several ways of different priority, these various ways combine, yielding a very large number of advice possibilities. These two options allow to limit the number of considered advice possibilities by keeping the $n_1$ first possibilities (with highest priority) and the $n_2$ last possibilities (with lowest priority but fewer advised transformations). When $n_1$ or $n_2$ are not positive, all advice possibilities are kept, but that may yield a very slow execution. 

\item \texttt{set elsefindFactsInReplace = true.}\\
\texttt{set elsefindFactsInReplace = false.}

When \texttt{elsefindFactsInReplace = true}, CryptoVerif will try to
infer more facts when doing a \texttt{replace} operation: when it
encounters a \texttt{find} branch in the process, it considers a
variable $x[M_1, \ldots, M_l]$, which is guaranteed to be defined by this \texttt{find}.
If $x$ is defined in the \texttt{else} part of another \texttt{find}
construct, then at the definition of $x$, we know that the conditions
of the \texttt{then} branches of this \texttt{find} are not satisfied:
\[\forall u_1, \ldots, u_k, \texttt{not}(\texttt{defined}(y_1[M_{11}, \ldots, M_{1l_1}], \ldots, y_k[M_{k1}, \ldots, M_{kl_k}]) \wedge t)\]
We try to infer $\texttt{not}(t)$ from this fact.
\begin{itemize}
\item if each variable $y_j[M_{j1}, \ldots, M_{jl_j}]$ is defined before $x[M_1, \ldots, M_l]$,
then $\texttt{not}(t)$ indeed holds by the fact above;
\item for each $y_j[M_{j1}, \ldots, M_{jl_j}]$, 
we assume that $y_j[M_{j1}, \ldots, M_{jl_j}]$ is defined after or at the same time as $x[M_1, \ldots, M_l]$
and try to prove $\texttt{not}(t)$.

It this proof succeeds, we can infer that $\texttt{not}(t)$ holds
at the current program point.
\end{itemize}

\item \texttt{set elsefindFactsInSimplify = true.}\\
\texttt{set elsefindFactsInSimplify = false.}

Similar to \texttt{elsefindFactsInReplace}, but applies in
\texttt{simplify} operations. 

\item \texttt{set elsefindFactsInSuccess = true.}\\
\texttt{set elsefindFactsInSuccess = false.}

Similar to \texttt{elsefindFactsInReplace}, but applies in
\texttt{success} operations. 

\item \texttt{set elsefindAdditionalDisjunct = true.}\\
\texttt{set elsefindAdditionalDisjunct = false.}

When \texttt{elsefindAdditionalDisjunct = true}, the procedure that infers facts
from false conditions of \texttt{find} (see \texttt{set elsefindFactsInReplace})
is enriched: in case $y_j[M_{j1}, \ldots, M_{jl_j}]$ may be defined
at the same time as $x[M_1, \ldots, M_l]$, we additionally assume
that they have different indices, that is, $(M_{j1}, \ldots, M_{jl_j}) \neq (M_1, \ldots, M_l)$
to eliminate this case. Therefore, we infer 
$(M_{j1}, \ldots, M_{jl_j}) \neq (M_1, \ldots, M_l) \Rightarrow \texttt{not}(t)$
or equivalently $(M_{j1}, \ldots, M_{jl_j}) = (M_1, \ldots, M_l) \vee \texttt{not}(t)$.
This is typically more costly and more precise than the basic 
procedure that just infers $\texttt{not}(t)$ when possible.

\item \texttt{set improvedFactCollection = false.}\\
\texttt{set improvedFactCollection = true.}

When \texttt{improvedFactCollection = true}, and CryptoVerif collects
the facts that hold at each program point, it also takes into account
variables that cannot be defined at a certain program point, variables
that cannot be simultaneously defined, and elsefind facts, in order to
prove more facts.

It is a bit costly, so it is disabled by default
(\texttt{improvedFactCollection = false}).

\item \texttt{set useEqualitiesInSimplifyingFacts = false.}\\
\texttt{set useEqualitiesInSimplifyingFacts = true.}

When \texttt{useEqualitiesInSimplifyingFacts = true}, CryptoVerif
uses known equalities between terms to determine whether a fact
is equal to another fact.

It is a bit costly, so it is disabled by default
(\texttt{useEqualitiesInSimplifyingFacts = false}).

\item \texttt{set useKnownEqualitiesWithFunctionsInMatching = false.}\\
\texttt{set useKnownEqualitiesWithFunctionsInMatching = true.}

When \texttt{useKnownEqualitiesWithFunctionsInMatching = true}, CryptoVerif
uses known equalities $M_1 = M_2$ where the root of $M_1$ is a function 
application to normalize terms before testing whether they match
an equation or collision statement or an oracle in a cryptographic
transformation. That can allow to apply these statements or transformations
more often.

It is a bit costly, so it is disabled by default
(\texttt{useKnownEqualitiesWithFunctionsInMatching = false}).

\item \texttt{set maxReplaceDepth = $n$.} (default 20)

Sets the maximum number of rewriting steps that are allowed 
to prove that the new term is equal to the old one in a 
\texttt{replace} transformation. 

\item \texttt{set forgetOldGames = false.}\\
\texttt{set forgetOldGames = true.}

When \texttt{forgetOldGames = true}, old games are removed from memory after each
cryptographic transformation or each interactive command.  
That allows to save some memory, but prevents \texttt{undo}.
The display of the games is saved into a temporary file to allow
displaying the games at the end of the proof.

\end{itemize}
The default value is the first mentioned, except when explicitly specified.
In most cases, the default values should be left as they are, except
for {\tt interactiveMode}, which allows to perform 
interactive proofs.

\item $\texttt{param}\ \neseq{ident}\ [\texttt{[noninteractive]}\mid \texttt{[passive]} \mid \texttt{[size$n$]}]\texttt{.}$

$\texttt{param}\ n_1, \ldots, n_m\texttt{.}$ declares parameters $n_1, \ldots, n_m$.
Parameters are used to represent the number of copies of replicated processes
(that is, the maximum number of calls to each query).
In asymptotic analyses, they are polynomial in the security parameter.
In exact security analyses, they appear in the formulas that express the
probability of an attack.

The options \texttt{[noninteractive]}, \texttt{[passive]}, or \texttt{[size$n$]}
indicate to CryptoVerif an order of magnitude of the size of the parameter.
%
The option \texttt{[size$n$]} (where $n$ is a constant integer) indicates
that the considered parameter has ``size $n$'': the larger the $n$, the
larger the parameter is likely to be.
CryptoVerif uses this
information to optimize the computed probability bounds: when several
bounds are correct, it chooses the smallest one.

The option \texttt{[noninteractive]} means that
the queries bounded by the considered parameters can be made by the
adversary without interacting with the tested protocol, so the number
of such queries is likely to be large.
Parameters with option \texttt{[noninteractive]} are typically 
used for bounding the number of calls to random oracles.
\texttt{[noninteractive]} is equivalent to \texttt{[size20]}.

The option \texttt{[passive]} means that the queries bounded by the
considered parameters correspond to the adversary passively listening
to sessions of the protocol that run as expected. Therefore, for such
runs, the adversary is undetected. This number of runs is therefore
likely to be larger than runs in which the adversary actively 
interacts with the honest participants, when these participants stop 
after a certain number of failed attempts.
\texttt{[passive]} is equivalent to \texttt{[size10]}.

\item $\texttt{proba}\ \nonterm{ident}\texttt{.}$

$\texttt{proba}\ p\texttt{.}$ declares a probability $p$.
(Probabilities may be used as functions of other arguments,
without explicit checking of these arguments.)

\item $\texttt{type}\ \nonterm{ident}\ [\texttt{[}\neseq{option}\texttt{]}]\texttt{.}$

$\texttt{type}\ T\texttt{.}$ declares a type $T$. Types correspond to sets
of bitstrings or a special symbol $\bot$ (used for failed decryptions, 
for instance). Optionally, the declaration of a type may be followed by options
between brackets. These options can be:
\begin{itemize}

\item \texttt{bounded} means that the type is a set of bitstrings of
bounded length or perhaps $\bot$. In other words, the type is a finite
subset of bitstrings plus $\bot$.

\item \texttt{fixed} means that the type is the set of all bitstrings of 
a certain length $n$. In particular, the type is a finite set,
so \texttt{fixed} implies \texttt{bounded}. 

\item \texttt{nonuniform} means that random numbers may be chosen in
  the type with a non-uniform distribution. (When \texttt{nonuniform}
  is absent, random numbers are chosen using a uniform distribution
  for {\tt fixed} types, an almost uniform distribution for
  \texttt{bounded} types, and random values cannot be chosen among
  other types. Note that \texttt{fixed, nonuniform} and
  \texttt{bounded, nonuniform} are also allowed to have a non-uniform
  distribution on a \texttt{fixed} or \texttt{bounded} type.)

\item \texttt{large}, \texttt{password}, and \texttt{size$n$} indicate 
the order of magnitude of the probability of collision $\texttt{Pcoll1rand}(T)$
between a random element
chosen according to the default probability distribution $D_T$ for the considered type $T$,
and an independent element of type $T$. When the default distribution is uniform
or almost uniform ({\tt fixed} and {\tt bounded} types), $\texttt{Pcoll1rand}(T) = \frac{1}{|T|}$,
so these parameters give an order of magnitude of the cardinal of the type.

The option \texttt{size$n$} (where $n$ is a constant integer) indicates
that the considered type has ``size $n$'': the larger the $n$, the
smaller the probability of collision $\texttt{Pcoll1rand}(T)$.
When no size option is present, the type has size 0.
CryptoVerif uses this information to determine whether collisions 
with random elements of the considered type $T$ should be eliminated.
For collisions to be eliminated, two conditions must be satisfied:
\begin{enumerate}

\item the size of the type must be at least \texttt{minAutoCollElim}
(which is set by $\texttt{set minAutoCollElim = }n$; the default is 15),
or the size of the type must be at least 1 and elimination of collisions
on this data has been manually requested by the command 
$\texttt{simplify coll\string_elim }\ldots$.

\item the probability of collision is at most one of the formulas
specified by the command $\texttt{allowed\_collisions}$
(used inside a {\tt proof} environment).
By default, all collisions are eliminated for types of size at least 20,
and collisions are eliminated for types of size at least 10 when
the collision is repeated at most $N$ times, where $N$ is a parameter of size 0.
See the command $\texttt{allowed\_collisions}$ for more details.

\end{enumerate}

\texttt{large} means that the type $T$ is large enough so that
all collisions with random elements of $T$ can be eliminated. 
(In asymptotic analyses, $\texttt{Pcoll1rand}(T)$ is negligible. 
In exact security analyses, the
probability of a collision is correctly expressed by the system.)
\texttt{large} is equivalent to \texttt{size20}.

\texttt{password} is intended for passwords in password-based
security protocols. These passwords are taken in a dictionary whose
size is much smaller than the size of a nonce for instance,
so the probability of collisions among passwords is larger 
than among data of \texttt{large} types.
\texttt{password} is equivalent to \texttt{size10}.

\end{itemize}

\item $\texttt{fun}\ \nonterm{ident}\texttt{(}\seq{ident}\texttt{):}\nonterm{ident}\ [\texttt{[}\neseq{option}\texttt{]}]\texttt{.}$

$\texttt{fun}\ f\texttt{(}T_1, \ldots, T_n\texttt{):}T\texttt{.}$ 
declares a function that takes $n$ arguments, of types $T_1, \ldots, T_n$, 
and returns a result of type $T$.
Optionally, the declaration of a function may be followed by options
between brackets. These options can be:
\begin{itemize}

\item \texttt{[data]} means that $f$ is injective and that its
inverses can be computed in polynomial time: $f(x_1, \ldots, x_m) = y$
implies for $i \in \{1, \ldots, m\}$, $x_i = f_i^{-1}(y)$ for some 
functions $f_i^{-1}$. (In the vocabulary of~\cite{BlanchetEPrint05},
$f$ is poly-injective.) $f$ can then be used for pattern matching.

\item \texttt{[projection]} means that $f$ is an inverse of a poly-injective
function. $f$ must be unary. (Thanks to the pattern matching construct, one can
in general avoid completely the declaration of \texttt{projection} functions,
by just declaring the corresponding poly-injective function \texttt{data}.)

\item \texttt{[uniform]} means that $f$ maps the default distribution
of its argument into the default distribution of its result. $f$ must be unary;
the argument and the result of $f$ must be of types marked 
{\tt fixed}, {\tt bounded}, or {\tt nonuniform}.

\end{itemize}

\item $\texttt{letfun}\
  \nonterm{ident}[\texttt{(}\seq{vartypeb}\texttt{)}]\texttt{=}\nonterm{term}\texttt{.}$ 

  $\texttt{letfun}\ f\texttt{(} x_1\texttt{:} T_1, \ldots,
  x_n\texttt{:}T_n\texttt{)=} M\texttt{.}$
  declares a function $f$ that takes $n$ arguments named
  $x_1, \ldots, x_n$ of types $T_1, \ldots, T_n$, respectively. The
  subsequent calls to this function are replaced by the term $M$ in
  which we replace $x_1, \ldots, x_n$ with the arguments given by the
  caller. (We use $x_i \texttt{<=} N_i$ instead of
  $x_i\texttt{:} T_i$ when $x_i$ is of type $[1,N_i]$, where $N_i$ is
  a parameter, declared by $\texttt{param }N_i$.)

  Variables defined inside $\texttt{letfun}$ can be used in array references
  and in queries, provided the process after expansion of $\texttt{letfun}$
  satisfies the required conditions for that.

\item $\texttt{const}\ \neseq{ident}\texttt{:}\nonterm{ident}\texttt{.}$

$\texttt{const}\ c_1, \ldots, c_n\texttt{:}T\texttt{.}$ declares constants
$c_1, \ldots, c_n$ of type $T$.
Different constants are assumed to correspond to different bitstrings
(except when the instruction \texttt{set diffConstants = false.} is
given).

\item $\texttt{table}\ \nonterm{ident}\texttt{(}\neseq{ident}\texttt{).}$

$\texttt{table}\ \mathit{tbl}\texttt{(}T_1, \ldots, T_n\texttt{).}$
declares the table $\mathit{tbl}$, whose elements are tuples of
type $T_1, \ldots, T_n$. Types $T_i$ may be replaced with
parameters $N_i$, to declare a table that contains
a replication index of type $[1,N_i]$. Elements can be inserted in the table
by $\texttt{insert}\ \mathit{tbl}\texttt{(}M_1, \ldots, M_n\texttt{)}$
and the table can be read using $\texttt{get}$.

\ifchannels
\item $\texttt{channel}\ \neseq{ident}\texttt{.}$

$\texttt{channel}\ c_1, \ldots, c_n\texttt{.}$ declares communication channels
$c_1, \ldots, c_n$.
\fi

\item $\texttt{event}\ \nonterm{ident}[\texttt{(}\seq{ident}\texttt{)}]\texttt{.}$

$\texttt{event}\ e\texttt{(}T_1, \ldots, T_n\texttt{)}\texttt{.}$
declares an event $e$ that takes arguments of types $T_1, \ldots, T_n$.
When there are no arguments, we can simply declare 
$\texttt{event}\ e\texttt{.}$ Types $T_i$ may be replaced with
parameters $N_i$, to declare an event that takes as argument
a replication index of type $[1,N_i]$.


\item $\texttt{let}\ \nonterm{ident}[\texttt{(}\seq{vartypeb}\texttt{)}]\texttt{ = }\oprocess\texttt{.}$\\
$\texttt{let}\ \nonterm{ident}[\texttt{(}\seq{vartypeb}\texttt{)}]\texttt{ = }\iprocess\texttt{.}$

$\texttt{let}\ \mathit{proc}(x_1:T_1, \dots, x_n:T_n)\texttt{ = }P\texttt{.}$ says that $\mathit{proc}$ takes $n$ arguments, $x_1$ of type $T_1$, \dots, $x_n$ of type $T_n$, and is equal to the process
$P$. (We use $x_i \texttt{<=} N_i$ instead of
  $x_i\texttt{:} T_i$ when $x_i$ is of type $[1,N_i]$, where $N_i$ is
  a parameter, declared by $\texttt{param }N_i$.)
When parsing a process, $\mathit{proc}(M_1, \dots, M_n)$ will be replaced with $P\{M_1/x_1, \dots, M_n/x_n\}$ when $P$ is an input process. In this case, the terms $M_1, \dots, M_n$ must contain only variables, replication indices, and function applications and the variables $x_1, \dots, x_n$ cannot have array accesses.
The process $\mathit{proc}(M_1, \dots, M_n)$ will be replaced with $\texttt{let}$ $x_1 = M_1$ $\texttt{in}$ \dots $\texttt{let}$ $x_n = M_n$ $\texttt{in}$ $P$ when $P$ is an output process.

\item $\texttt{equation }[\texttt{forall }\seq{vartype}\texttt{;}]\nonterm{simpleterm}\ [\texttt{if}\ \nonterm{simpleterm}]\texttt{.}$

$\texttt{equation forall }x_1:T_1, \ldots, x_n:T_n\texttt{;}M\texttt{.}$ says
that for all values of $x_1, \ldots, x_n$ in types $T_1, \ldots, T_n$
respectively,
$M$ is true. The term $M$ must be a simple term without array accesses.
All bound variables $x_1, \dots, x_n$ must occur in $M$.
%
When $M$ is an equality $M_1 \eqt  M_2$, CryptoVerif uses this information
for rewriting $M_1$ into $M_2$, so one must be careful of the orientation
of the equality, in particular for termination. 
In this case, all bound variables $x_1, \dots, x_n$ must occur in $M_1$,
so that the target term $M_2$ is entirely determined knowing the instance of $M_1$.
%
When $M$ is an inequality, $M_1 \texttt{<>} M_2$, CryptoVerif rewrites
$M_1 \eqt  M_2$ to false and $M_1 \texttt{<>} M_2$ to true.
%
Otherwise, it rewrites $M$ to true.

$\texttt{equation forall }x_1:T_1, \ldots, x_n:T_n\texttt{;}M\texttt{ if }M'\texttt{.}$ says
that for all values of $x_1, \ldots, x_n$ in types $T_1, \ldots, T_n$
respectively such that $M'$ is true, we have that
$M$ is true. The terms $M$ and $M'$ must be simple terms without array accesses.
CryptoVerif tries to prove the precondition $M'$, and in case of success,
rewrites terms as explained above. 


\item $\texttt{equation builtin }\nonterm{eq\_name}\texttt{(}\neseq{ident}\texttt{).}$

This declaration declares the equational theories satisfied by function symbols.
The following equational theories are supported:
\begin{itemize}

\item \texttt{equation builtin commut($f$).} indicates that the function $f$ is commutative,
that is, $f(x,y) = f(y,x)$ for all $x,y$. In this case, the function
$f$ must be a binary function with both arguments of the same type.
(The equation $f(x,y) = f(y,x)$ cannot be given by the {\tt forall}
declaration because CryptoVerif interprets such declarations as rewrite rules,
and the rewrite rule $f(x,y) \rightarrow f(y,x)$ does not terminate.)

\item \texttt{equation builtin assoc($f$).} indicates that the function $f$ is associative, that is, $f(x,f(y,z)) = f(f(x,y),z)$ for all $x,y,z$. In this case, the function $f$ must be a binary function with both arguments and the result of
the same type.

\item \texttt{equation builtin AC($f$).} indicates that the function $f$ is associative and commutative. In this case, the function $f$ must be a binary function with both arguments and the result of
the same type.

\item \texttt{equation builtin assocU($f$, $n$).} indicates that the function $f$ is associative, and that $n$ is a neutral element for $f$, that $f(x,n) = f(n,x) = x$ for all $x$. In this case, the function $f$ must be a binary function with both arguments and the result of the same type as the type of the constant $n$.

\item \texttt{equation builtin ACU($f$, $n$).} indicates that the function $f$ is associative and commutative, and that $n$ is a neutral element for $f$. In this case, the function $f$ must be a binary function with both arguments and the result of the same type as the type of the constant $n$.

\item \texttt{equation builtin ACUN($f$, $n$).} indicates that the function $f$ is associative and commutative, that $n$ is a neutral element for $f$, and that $f$ satisfies the cancellation equation $f(x,x) = n$. In this case, the function $f$ must be a binary function with both arguments and the result of the same type as the type of the constant $n$.

\item \texttt{equation builtin group($f$, $inv$, $n$).} indicates that $f$ forms group with inverse $inv$ and neutral element $n$, that is, the function $f$ is associative, $n$ is a neutral element for $f$, and $inv(x)$ is the inverse of $x$, that is, $f(inv(x),x) = f(x,inv(x)) = n$. In this case, the function $f$ must be a binary function with both arguments and the result of the same type $T$, $inv$ must be a unary function that takes and returns a value of type $T$, and $n$ must be a constant of type $T$.

\item \texttt{equation builtin commut\_group($f$, $inv$, $n$).} indicates that $f$ forma commutative group with inverse $inv$ and neutral element $n$, that is, the function $f$ is associative and commutative, $n$ is a neutral element for $f$, and $inv(x)$ is the inverse of $x$. In this case, the function $f$ must be a binary function with both arguments and the result of the same type $T$, $inv$ must be a unary function that takes and returns a value of type $T$, and $n$ must be a constant of type $T$.

\end{itemize}

\item 
$\texttt{collision }\nonterm{res}^*
[\texttt{[restrictions\_may\_be\_equal]}]
[\texttt{forall }\seq{vartype}\texttt{;}]$\\
\null\qquad $\texttt{return(}\nonterm{simpleterm}\texttt{) <=(}\nonterm{proba}\texttt{)=> return(}\nonterm{simpleterm}\texttt{)}
[\ \texttt{if}\ \nonterm{cond}]\texttt{.}$\\
where
\begin{align*}
\nonterm{cond} &::= \nonterm{simpleterm}\\
&\ \ \mid \ \nonterm{ident}\texttt{ independent-of }\nonterm{ident}\\
&\ \ \mid \ \nonterm{cond}\texttt{ \&\& }\nonterm{cond}\\
&\ \ \mid \ \nonterm{cond}\texttt{ || }\nonterm{cond}
\end{align*}

$\texttt{collision }\Resa{x_1}{T_1}\texttt{;}\ldots 
\Resa{x_n}{T_n}\texttt{;}
\texttt{forall }y_1:T'_1, \ldots, y_m:T'_m\texttt{;}$\\
$\null\qquad \texttt{return(}M_1\texttt{) <=(}p\texttt{)=> return(}M_2\texttt{).}$\\
means that when
$x_1, \ldots, x_n$ are chosen randomly 
and independently in $T_1, \ldots, T_n$ respectively (with the default probability distributions for these types), a Turing machine running in
time $\texttt{time}$ has probability at most $p$ of finding
$y_1, \ldots, y_m$ in $T'_1, \ldots, T'_m$ such that $M_1 \neq M_2$.
%
The terms $M_1$ and $M_2$ must be simple terms without array accesses.
See below for the syntax of probability formulas.

This allows CryptoVerif to rewrite $M_1$ into $M_2$ with probability
loss $p$, when $x_1, \ldots, x_n$ are created by independent random
number generations of types $T_1, \ldots, T_n$ respectively. One
should be careful of the orientation of the equivalence, in particular
for termination.

$\texttt{collision }\Resa{x_1}{T_1}\texttt{;}\ldots 
\Resa{x_n}{T_n}\texttt{;}
\texttt{forall }y_1:T'_1, \ldots, y_m:T'_m\texttt{;}$\\
$\null\qquad \texttt{return(}M_1\texttt{) <=(}p\texttt{)=> return(}M_2\texttt{) if $c$.}$\\
means that the previous property holds when the condition $c$ is true, where
$c$ is built by conjunctions or disjunctions of simple terms and independence conditions ``$y_i$ \texttt{independent-of} $x_j$'',
where $y_i$ is bound by  \texttt{forall} and $x_j$ is bound by \texttt{new}. (However, disjunctions cannot
mix terms and independence conditions.)

The option \texttt{[restrictions\_may\_be\_equal]}, when it is present, allows several 
random number generations among $x_1, \ldots, x_n$ to be the same, instead of being
independent. One can then group, in a single \texttt{collision} statement, 
situations in which $x_1, \ldots, x_n$ are the same or they are independent.
The indices of the variables corresponding to $x_1, \ldots, x_n$
in the game are still made independent of $x_1, \ldots, x_n$. 
Hence, there are two cases: either $x_i$ is the same as $x_j$, or
$x_i$ and $x_j$ are independent of each other. With the option \texttt{[restrictions\_may\_be\_equal]}, 
the independence conditions can also be ``$x_i$ \texttt{independent-of} $x_j$'',
where $x_i$ and $x_j$ are both bound by \texttt{new}. This condition
then means $x_i$ and $x_j$ are different restrictions, so
$x_j$ is also independent of $x_i$. 

\ifchannels
\item $\texttt{equiv}[\texttt{(}\nonterm{ident}[\texttt{(}\nonterm{ident}\texttt{)}]\texttt{)}]$\\
$\nonterm{omode}\ [\texttt{|}\ \ldots\ \texttt{|}\nonterm{omode}]\texttt{ <=(}\nonterm{proba}\texttt{)=> }
[\texttt{[}n\texttt{]}]\ [\texttt{[}\neseq{option}\texttt{]}]\ \nonterm{ogroup}\ [\texttt{|}\ \ldots\ \texttt{|}\nonterm{ogroup}]\texttt{.}$

$\texttt{equiv(}\mathit{name}\texttt{)}\ L\texttt{ <=(}p\texttt{)=> }R\texttt{.}$ means that the
probability that a probabilistic Turing machine that runs in time
{\tt time} distinguishes $L$ from $R$ is at most $p$. The name $\mathit{name}$
is used to designate the equivalence in the \texttt{crypto} command used in manual proofs (see Section~\ref{sec:interact}). This name can be either an identifier $\mathit{id}$, or $\mathit{id}(f)$, where $\mathit{id}$ is an identifier and $f$ a second identifier. Names of the form $\mathit{id}(f)$ are most useful when the equivalence is defined inside a macro definition ($\texttt{def}$). In this case, the identifier $\mathit{id}$ is kept unchanged and the identifier $f$ is renamed during macro expansion; if $f$ is a parameter of the macro, it is then replaced with its value at macro expansion, so that one can always designate precisely the desired equivalence even when a macro is expanded several times.
The name may be omitted.

$L$ and $R$ define sets of oracles. (They can be translated into
processes as explained in~\cite{BlanchetEPrint05}.)
\begin{itemize}

\item $O\texttt{(}x_1:T_1, \ldots, x_n:T_n\texttt{) := }\mathit{FP}$ represents
an oracle $O$ that takes arguments $x_1, \ldots, x_n$ of types
$T_1, \ldots, T_n$ respectively, and returns the result computed by $\mathit{FP}$.
The oracle body $\mathit{FP}$ is similar to term, but terminates with
a $\texttt{return}$ as shown in the grammar of $\funbody$ 
(Figure~\ref{fig:syntax2}).

\item Optionally, in the left-hand side,
an integer between brackets $\texttt{[}n\texttt{]}$ ($n \geq 0$)
can be added in the definition of an oracle, which becomes 
$O\texttt{(}x_1:T_1, \ldots, x_n:T_n\texttt{) [}n\texttt{] := }\mathit{FP}$.
This integer does not change the semantics of the oracle, but is
used for the proof strategy: CryptoVerif uses preferably the oracles
with the smallest integers $n$ when several oracles can be used
for representing the same expression. When no integer is mentioned,
$n = 0$ is assumed, so the oracle has the highest priority.

\item Optionally, in the left-hand side, 
the indication \texttt{[useful\_change]} can also
be added in the definition of an oracle, which becomes 
$O\texttt{(}x_1:T_1, \ldots, x_n:T_n\texttt{) [useful\_change] := }\mathit{FP}$.
This indication is also used for the proof strategy: 
if at least one \texttt{[useful\_change]} indication is present,
CryptoVerif applies the transformation defined by the equivalence
only when at least one \texttt{[useful\_change]} function is called in the game.

\item $\texttt{!} i \leqt  N\ \Resa{y_1}{T'_1}\texttt{;}
\ldots \Resa{y_m}{T'_m} \texttt{;} \texttt{(}FG_1, \ldots,
FG_n\texttt{)}$ represents $N$ copies of a process that picks fresh
random numbers $y_1$, \ldots, $y_m$ of types $T'_1, \ldots, T'_m$
respectively, and makes available the functions described in $FG_1,
\ldots, FG_n$. Each copy has a different value of $i \in [1, N]$. The
identifier $i$ cannot be referred to explicitly in the process; it is
used only implicitly as array index of variables defined under
$\texttt{!} i \leqt  N$.  The replication $\texttt{!} i
\leqt  N$ can be abbreviated $\texttt{!} N$.

\end{itemize}
CryptoVerif uses such equivalences to transform processes that call
oracles of $L$ into processes that call oracles of $R$.

$L$ may contain mode indications to guide the rewriting: the mode
\texttt{[all]} means that all occurrences of the root function symbol
of oracles in the considered group must be transformed;
the mode \texttt{[exist]} means that at least one occurrence of an
oracle in this group must be transformed. (\texttt{[exist]} is the default;
there must be at most one oracle group with mode \texttt{[exist]};
when an oracle group contains no random number generation, it must be in mode 
\texttt{[all]}.)

Optionally, 
an integer between brackets $\texttt{[}n\texttt{]}$ ($n \geq 0$)
can be added in an equivalence.
This integer does not change the semantics of the equivalence, but is
used for the proof strategy: CryptoVerif uses preferably the equivalences
with the smallest integers $n$ when several equivalences can be used.
When no integer is mentioned,
$n = 0$ is assumed, so the equivalence has the highest priority.

Two options can specified for an equivalence, in
$\texttt{[}\neseq{option}\texttt{]}$:
\begin{itemize}

\item The \texttt{manual} option, when it is present in the equivalence,
prevents the automatic application of the transformation. The transformation
is then applied only using the manual \texttt{crypto} command.

\item The \texttt{computational} option, when it is present in the 
equivalence, means that the transformation relies on a computational assumption
(by opposition to decisional assumptions). This indication allows one to mark
some random number generations of the right-hand side of the equivalence with
\texttt{[unchanged]}, which means that the random value is preserved by 
the transformation. The transformation is then allowed even if the random 
value occurs as argument of events. (This argument will be unchanged.)
The mark \texttt{[unchanged]} is forbidden when the equivalence is
not marked \texttt{[computational]}. Indeed, decisional assumptions may
alter any random values.

\end{itemize}

$L$ and $R$ must satisfy certain syntactic constraints:
\begin{itemize}

\item %H0
$L$ and $R$ must be well-typed, satisfy the constraints on
array accesses (see the description of processes above), 
and the type of the results of 
corresponding oracles in $L$ and $R$ must be the same.

\item All oracle definitions in $L$ are of the form 
$O\texttt{(}\ldots\texttt{) := return(}M\texttt{)}$
where $M$ is a simple term. % without explicit array accesses.
Oracle definitions in $R$ are of the form 
$O\texttt{(}\ldots\texttt{) := }\funbody$.

\item $L$ and $R$ must have the same structure: same replications,
same number of oracles, same oracle names in the same order,
same number of arguments with the same types for each oracle.

\item Under a replication with no random number generation in $L$, 
one can have only a single oracle.

\item Replications in $L$ (resp. $R$) must have pairwise distinct
bounds. Oracles in $L$ (resp. $R$) must have pairwise distinct names.

\item %H7
\newcommand{\tup}[1]{\widetilde{#1}}

Finds in $R$ are of the form
\[\begin{split}
&\texttt{find}[\texttt{[unique]}]\ \ldots\\
&\texttt{orfind }u_1 \texttt{ <= } N_1, \ldots, u_m \texttt{ <= }N_m
\texttt{ suchthat defined(}z_1[\tup{u_1}], \ldots, z_l[\tup{u_l}]\texttt{) \&\& }M\texttt{ then }\mathit{FP}\\
&\ldots \texttt{ else }\mathit{FP}'
\end{split}\]
where $\tup{u_k}$ is a non-empty
prefix of $u_1, \ldots, u_m$, at least one $\tup{u_k}$ for $1 \leq
k \leq l$ is the whole sequence $u_1, \ldots, u_m$,
and the implicit prefix of the current array indices is the same
for all $z_1, \ldots, z_l$.
%
(When $z$ is defined under replications $\texttt{!}N_1$, \ldots,
$\texttt{!}N_n$, $z$ is always an array with $n$ dimensions, so it
expects $n$ indices, but the first $n'<n$ indices are left implicit
when they are equal to the current indices of the top-most $n'$ replications
above the usage of $z$---which must also be the top-most $n'$
replications above the definition of $z$. We require the implicit
indices to be the same for all variables $z_1, \ldots, z_l$.)
%TO DO is that clear?
Furthermore, there must exist $k \in \{ 1, \ldots, l_j\}$ such that
for all $k' \neq k$, $z_{k'}$ is defined syntactically above all
definitions of $z_k$ and $\tup{u_{k'}}$ is a prefix of $\tup{u_k}$. 
%
Finally, variables $z_k$ must not be defined by a $\texttt{find}$ in $R$.


\end{itemize}
\else
\item $\texttt{equiv }\nonterm{omode}\ [\texttt{|}\ \ldots\ \texttt{|}\nonterm{omode}]\texttt{ <=(}\nonterm{proba}\texttt{)=> }
[\texttt{[manual]}| \texttt{[computational]}]\ \nonterm{ogroup}\ [\texttt{|}\ \ldots\ \texttt{|}\nonterm{ogroup}]\texttt{.}$

$\texttt{equiv }\mathit{name}\ L\texttt{ <=(}p\texttt{)=> }R\texttt{.}$ means that the
probability that a probabilistic Turing machine that runs in time
{\tt time} distinguishes $L$ from $R$ is at most $p$. The name $\mathit{name}$
is used to designate the equivalence in the \texttt{crypto} command used in manual proofs (see Section~\ref{sec:interact}). This name can be either an identifier $\mathit{id}$, or $\mathit{id}(f)$, where $\mathit{id}$ is an identifier and $f$ a second identifier. Names of the form $\mathit{id}(f)$ are most useful when the equivalence is defined inside a macro definition ($\texttt{def}$). In this case, the identifier $\mathit{id}$ is kept unchanged and the identifier $f$ is renamed during macro expansion; if $f$ is a parameter of the macro, it is then replaced with its value at macro expansion, so that one can always designate precisely the desired equivalence even when a macro is expanded several times.

$L$ and $R$ define sets of oracles. (In these definitions, 
$\texttt{foreach }i\leqt N\texttt{ do }\Resb{x_1}{T_1}; \ldots
\Resb{x_m}{T_m};Q$ in fact stands for $\texttt{foreach }i\leqt N
\texttt{ do }O\texttt{() := } \Resb{x_1}{T_1}; \ldots \Resb{x_m}{T_m};
\texttt{return}; Q$, where $O$ is a fresh oracle name. The same oracle
names are used in both sides of the equivalence.)

In the left-hand side, an optional integer between brackets
$\texttt{[}n\texttt{]}$ ($n \geq 0$) can be added in the
definition of an oracle, which becomes 
$O\texttt{(}x_1:T_1, \ldots, x_n:T_n\texttt{) [}n\texttt{] := }P$.
This integer does not change the semantics of the oracle, but is
used for the proof strategy: CryptoVerif uses preferably the oracles
with the smallest integers $n$ when several oracles can be used
for representing the same expression. When no integer is mentioned,
$n = 0$ is assumed, so the oracle has the highest priority.

In the left-hand side, the optional indication \texttt{[useful\_change]} can also
be added in the definition of an oracle, which becomes 
$O\texttt{(}x_1:T_1, \ldots, x_n:T_n\texttt{) [useful\_change] := }P$.
This indication is also used for the proof strategy: 
if at least one \texttt{[useful\_change]} indication is present,
CryptoVerif applies the transformation defined by the equivalence
only when at least one \texttt{[useful\_change]} function is called in the game.

CryptoVerif uses such equivalences to transform processes that call
oracles of $L$ into processes that call oracles of $R$.

$L$ may contain mode indications to guide the rewriting: the mode
\texttt{[all]} means that all occurrences of the root function symbol
of oracles in the considered group must be transformed;
the mode \texttt{[exist]} means that at least one occurrence of an
oracle in this group must be transformed. (\texttt{[exist]} is the default;
there must be at most one oracle group with mode \texttt{[exist]};
when an oracle group contains no random number generation, it must be in mode 
\texttt{[all]}.)

The \texttt{[manual]} indication, when it is present in the equivalence,
prevents the automatic application of the transformation. The transformation
is then applied only using the manual \texttt{crypto} command.

The \texttt{[computational]} indication, when it is present in the 
equivalence, means that the transformation relies on a computational assumption
(by opposition to decisional assumptions). This indication allows one to mark
some random number generations of the right-hand side of the equivalence with
\texttt{[unchanged]}, which means that the random value is preserved by 
the transformation. The transformation is then allowed even if the random 
value occurs as argument of events. (This argument will be unchanged.)
The mark \texttt{[unchanged]} is forbidden when the equivalence is
not marked \texttt{[computational]}. Indeed, decisional assumptions may
alter any random values.

$L$ and $R$ must satisfy certain syntactic constraints:
\begin{itemize}

\item %H0
$L$ and $R$ must be well-typed, satisfy the constraints on
array accesses (see the description of processes above), 
and the type of the results of 
corresponding oracles in $L$ and $R$ must be the same.

\item All oracle definitions in $L$ are of the form 
$O\texttt{(}\ldots\texttt{) := return(}M\texttt{)}$
where $M$ is a simple term. % without explicit array accesses.
Oracle definitions in $R$ are of the form 
$O\texttt{(}\ldots\texttt{) := }\funbody$.

\item $L$ and $R$ must have the same structure: same replications,
same number of oracles, same oracle names in the same order,
same number of arguments with the same types for each oracle.

\item Under a replication with no random number generation in $L$, 
one can have only a single oracle.

\item Replications in $L$ (resp. $R$) must have pairwise distinct
bounds. Oracles in $L$ (resp. $R$) must have pairwise distinct names.

\item %H7
\newcommand{\tup}[1]{\widetilde{#1}}

Finds in $R$ are of the form
\[\begin{split}
&\texttt{find}[\texttt{[unique]}]\ \ldots\\
&\texttt{orfind }u_1 \texttt{ <= } N_1, \ldots, u_m \texttt{ <= }N_m
\texttt{ suchthat defined(}z_1[\tup{u_1}], \ldots, z_l[\tup{u_l}]\texttt{) \&\& }M\texttt{ then }\mathit{FP}\\
&\ldots \texttt{ else }\mathit{FP}'
\end{split}\]
where $\tup{u_k}$ is a non-empty
prefix of $u_1, \ldots, u_m$, at least one $\tup{u_k}$ for $1 \leq
k \leq l$ is the whole sequence $u_1, \ldots, u_m$,
and the implicit prefix of the current array indices is the same
for all $z_1, \ldots, z_l$.
%
(When $z$ is defined under replications $\texttt{!}N_1$, \ldots,
$\texttt{!}N_n$, $z$ is always an array with $n$ dimensions, so it
expects $n$ indices, but the first $n'<n$ indices are left implicit
when they are equal to the current indices of the top-most $n'$ replications
above the usage of $z$---which must also be the top-most $n'$
replications above the definition of $z$. We require the implicit
indices to be the same for all variables $z_1, \ldots, z_l$.)
%TO DO is that clear?
Furthermore, there must exist $k \in \{ 1, \ldots, l_j\}$ such that
for all $k' \neq k$, $z_{k'}$ is defined syntactically above all
definitions of $z_k$ and $\tup{u_{k'}}$ is a prefix of $\tup{u_k}$. 
%
Finally, variables $z_k$ must not be defined by a $\texttt{find}$ in $R$.


\end{itemize}


\fi


This is the key declaration for defining the security properties of
cryptographic primitives. Since such declarations are delicate to
design, we recommend using predefined primitives listed in
Section~\ref{sect:prim}, or copy-pasting declarations from examples.

\item $\texttt{query }[\seq{vartypeb}\texttt{;}]\nonterm{query}\texttt{;} (\nonterm{query}\texttt{;})^* \texttt{.}$

The {\tt query} declaration indicates which security properties we 
would like to prove. It is of the form $\texttt{query }x_1\texttt{:}T_1\texttt{,} \ldots\texttt{,} x_n\texttt{:}T_n\texttt{;} Q_1\texttt{;} \dots\texttt{;}Q_n$. First, we declare the types of all variables $x_1, \ldots, x_n$
that occur in correspondence queries that follow. (We use $x_i \texttt{<=} N_i$ instead of
  $x_i\texttt{:} T_i$ when $x_i$ is of type $[1,N_i]$, where $N_i$ is
  a parameter, declared by $\texttt{param }N_i$.) Second, we give the queries themselves. The available queries $Q_i$ are as follows:
\begin{itemize}

\item ${\tt secret}\ x\ [\texttt{public\_vars}\ l]$: show that the array $x$ is indistinguishable
from an array of independent random numbers (by several test queries),
even when the variables in $l$ are public. The list $l$ is considered empty when it is omitted.
In the vocabulary of~\cite{BlanchetEPrint05}, this is secrecy.

\item ${\tt secret}\ x\ [\texttt{public\_vars}\ l]\ \texttt{[cv\_onesession]}$: 
show that any element of the array $x$ 
cannot be distinguished from a random number (by a single test query),
even when the variables in $l$ are public. The list $l$ is considered empty when it is omitted.
In the vocabulary of~\cite{BlanchetEPrint05}, this is one-session
secrecy.

In addition to the option \texttt{cv\_onesession}, the options \texttt{real\_or\_random},
\texttt{cv\_real\_or\_random} and all options starting with \texttt{pv\_} are also allowed,
but ignored. Real-or-random secrecy is the default for CryptoVerif and
the options starting with \texttt{pv\_} are for ProVerif.

\item $M \texttt{ ==> } M'$.
The system shows that, for all values of variables that occur in $M$,
if $M$ is true then there exist values of variables of $M'$ that do not
occur in $M$ such that $M'$ is true.

$M$ must be a conjunction of terms $\texttt{event(}e\texttt{)}$, $\texttt{inj-event(}e\texttt{)}$, 
$\texttt{event(}e\texttt{(}M_1, \ldots, M_n\texttt{))}$, or 
$\texttt{inj-event(}e\texttt{(}M_1, \ldots, M_n\texttt{))}$
where $e$ is an event declared by ${\tt event}$ and
the $M_i$ are simple terms without array accesses (not containing
events). 

$M'$ must be formed by conjunctions and disjunctions of terms 
$\texttt{event(}e\texttt{)}$, $\texttt{inj-event(}e\texttt{)}$, 
$\texttt{event(}e\texttt{(}M_1, \ldots, M_n\texttt{))}$, 
$\texttt{inj-event(}e\texttt{(}M_1, \ldots, M_n\texttt{))}$, or
simple terms without array accesses
(not containing events).

When $\texttt{inj-event}$ is present, the system proves an injective
correspondence, that is, it shows that several different events marked
$\texttt{inj-event}$ before $\texttt{==>}$ imply the execution of several
different events marked $\texttt{inj-event}$ after $\texttt{==>}$.
%
More precisely, $\texttt{inj-event(}e_1\texttt{(}M_{11}, \ldots, M_{1m_1}\texttt{))}$
$\texttt{\&\&}$ $\ldots$ $\texttt{\&\&}$ $\texttt{inj-event(}e_n\texttt{(}M_{n1}, \ldots, \allowbreak
M_{nm_n} \texttt{))}$ $\texttt{\&\&}$ $\ldots$ $\texttt{==>}$ $M'$ means that for each
tuple of executed events $e_1(M_{11}, \allowbreak \ldots, M_{1m_1})$
(executed $N_1$ times), \ldots, $e_n(M_{n1}, \ldots, M_{nm_n})$
(executed $N_n$ times), $M'$ holds, considering that an event
$\texttt{inj-event(}e'\texttt{(}M_1, \ldots, M_m\texttt{))}$ in $M'$ holds when it has been
executed at least $N_1 \times \ldots \times N_n$ times.
%
The $\texttt{inj-event}$ marker must
occur either both before and after $\texttt{==>}$ or not at all. (Otherwise,
the query would be equivalent to a non-injective correspondence.)

\item $M$. This query is an abbreviation for $M \texttt{ ==> false}$.

\end{itemize}

\item $\texttt{proof \{}\nonterm{command}\texttt{;}\ldots \texttt{;}\nonterm{command} \texttt{\}}$

Allows the user to include in the CryptoVerif input file the commands
that must be executed by CryptoVerif in order to prove the protocol.
The allowed commands are those described in Section~\ref{sec:interact},
except that \texttt{help} and \texttt{?} are not allowed and that
the \texttt{crypto} command must be fully specified (so that no user 
interaction is required). If the command contains a string that
is not a valid identifier, \texttt{*}, or \texttt{.}, then this string
must be put between quotes \texttt{"}. This is useful in particular for
variable names introduced internally by CryptoVerif and that contain
\texttt{\string@} (so that they cannot be confused with variables introduced
by the user), for example \texttt{"\string@2\_r1"}.

\item ${\tt def\ }\nonterm{ident}\texttt{(}\seq{ident}\texttt{) \{}
\seq{decl}\texttt{\}}$ 

${\tt def\ }m\texttt{(}x_1, \ldots, x_n\texttt{) \{}
d_1, \ldots, d_k\texttt{\}}$ defines a macro named $m$, with arguments
$x_1, \ldots, x_n$. This macro expands to the declarations
$d_1, \ldots, d_k$, which can be any of the declarations listed in
this manual, except $\texttt{def}$ itself.
The macro is expanded by the \texttt{expand} declaration described below.
When the \texttt{expand} declaration appears inside a \texttt{def}
declaration, the expanded macro must have been defined before the
\texttt{def} declaration (which prevents recursive macros, whose
expansion would loop).
Macros are used in particular to define a library of standard
cryptographic primitives that can be reused by the user without
entering their full definition. These primitives are presented
in Section~\ref{sect:prim}.

\item ${\tt expand\ }\nonterm{ident}\texttt{(}\seq{ident}\texttt{).}$

${\tt expand\ }m\texttt{(}y_1, \ldots, y_n\texttt{).}$ expands the macro
$m$ by applying it to the arguments $y_1, \ldots, y_n$. If the definition
of the macro $m$ is ${\tt def\ }m\texttt{(}x_1, \ldots, x_n\texttt{) \{}
d_1, \ldots, d_k\texttt{\}}$, then it generates $d_1, \ldots, d_k$ in which
$y_1, \ldots, y_n$ are substituted for $x_1, \ldots, x_n$ and the other
identifiers that were not already defined at the $\texttt{def}$ declaration
are renamed to fresh identifiers.

\end{itemize}

The following identifiers are predefined:
\begin{itemize}

\item The type {\tt bitstring} is the type of all bitstrings.
It is large.

\item The type {\tt bitstringbot} is the type that contains
all bitstrings and $\bot$. It is also large.

\item The type {\tt bool} is the type of boolean values, which consists
of two constant bitstrings {\tt true} and {\tt false}.
It is declared {\tt fixed}.

\item The function {\tt not} is the boolean negation, from
{\tt bool} to {\tt bool}.

\item The constant {\tt bottom} represents $\bot$. (The special
element of {\tt bitstringbot} that is not a bitstring.)

\end{itemize}

The syntax of probability formulas allows parenthesing and the usual
algebraic operations \texttt{+}, \texttt{-}, \texttt{*}, \texttt{/}.
(\texttt{*} and \texttt{/} have higher priority than \texttt{+} and
\texttt{-}, as usual.), as well as the maximum, denoted 
$\texttt{max(}p_1\texttt{,}\ldots\texttt{,}p_n\texttt{)}$. 
They may also contain 
\begin{itemize}

\item $P$ or $P(p_1, \ldots,
p_n)$ where $P$ has been declared by $\texttt{proba }P$ and $p_1,
\ldots, p_n$ are probability formulas; this formula represents an
unspecified probability depending on $p_1, \ldots, p_n$. 

\item $N$, where $N$ has been declared by $\texttt{param }N$,
designates the number of copies of a replication.

\item $\#O$, where $O$ is an oracle,
designates the number of different calls to the oracle $O$. 

\item $|T|$, where
$T$ has been declared by $\texttt{type }T$ and is \texttt{fixed}
or \texttt{bounded}, designates the cardinal of $T$.

\item $\texttt{maxlength(}M\texttt{)}$ is the maximum
length of term $M$ ($M$ must be a simple term without array access, 
and must be of a non-bounded type).

\item $\texttt{length(}f, p_1, \ldots, p_n\texttt{)}$ designates the maximal
length of the result of a call to $f$, where $p_1, \ldots, p_n$
represent the maximum length of the non-bounded arguments of $f$
($p_i$ must be built from $\texttt{max}$,
$\texttt{maxlength(}M\texttt{)}$, and $\texttt{length(}f', \ldots
\texttt{)}$, where $M$ is a term of the type of the corresponding
argument of $f$ and the result of $f'$ is of the type of the
corresponding argument of $f$).

\item $\texttt{length}(T)$ designates the maximal 
length of a bitstring of type $T$, where $T$ is a bounded type.

\item $\texttt{length((}T_1, \ldots, T_n\texttt{)}, p_1, \ldots,
p_n\texttt{)}$ designates the maximal length of the result of the
tuple function from $T_1 \times \ldots \times T_m$ to
\texttt{bitstring}, where $p_1, \ldots, p_n$ represent the maximum
length of the non-bounded arguments of this function.

\item $n$ is an integer constant.

\item \texttt{eps\_find} is the maximum distance between the uniform probability
distribution and the probability distribution used for choosing elements
in {\tt find}.

\item $\texttt{eps\_rand(}T\texttt{)}$ is the maximum distance between the 
uniform probability distribution and the default probability distribution 
$D_T$ for type $T$ (when $T$ is \texttt{bounded}).

\item 
$\texttt{Pcoll1rand(}T\texttt{)}$ is the maximum probability of
collision between a random value $X$ of type $T$ chosen according
to the default distribution $D_T$ for type $T$ and an element of type $T$
that does not depend on it (when $T$ is \texttt{nonuniform}).
This is also the maximum probability of choosing any given element of 
$T$ in the default distribution for that type:
\[\texttt{Pcoll1rand(}T\texttt{)} = \max_{a \in T} \Pr[X = a]\]
where $X$ is chosen according to distribution $D_T$.

\item $\texttt{Pcoll2rand(}T\texttt{)}$ is the maximum probability of
collision between two independent random values of type $T$  
chosen according to the default distribution $D_T$ for type $T$
(when $T$ is \texttt{nonuniform}). We have
\[\texttt{Pcoll2rand(}T\texttt{)} = \sum_{a \in T} \Pr[X = a]^2 \leq \texttt{Pcoll1rand(}T\texttt{)}\]
where $X$ is chosen according to the default distribution $D_T$.

\item $\texttt{time}$ designates the runtime of the environment (attacker).

\end{itemize}
Finally, $\texttt{time(}\ldots\texttt{)}$ designates the runtime time of each
elementary action of a game:
\begin{itemize}
\item
$\texttt{time(}f, p_1, \ldots, p_n\texttt{)}$ designates the maximal runtime of
one call to function symbol $f$, where $p_1, \ldots, \allowbreak p_n$ represent
the maximum length of the non-bounded arguments of $f$.
\item
$\texttt{time(let }f, p_1, \ldots, p_n\texttt{)}$ designates the
maximal runtime of one pattern matching operation with function symbol
$f$, where $p_1, \ldots, p_n$ represent the maximum length of the
non-bounded arguments of $f$.
\item
$\texttt{time((}T_1, \ldots, T_m\texttt{)}, p_1, \ldots, p_n\texttt{)}$ designates the
maximal runtime of one call to the tuple function from $T_1 \times
\ldots \times T_m$ to \texttt{bitstring}, where $p_1, \ldots, p_n$
represent the maximum length of the non-bounded arguments of this
function.
\item
$\texttt{time(let(}T_1, \ldots, T_m\texttt{)}, p_1, \ldots, p_n\texttt{)}$ designates the
maximal runtime of one pattern matching with the tuple function from
$T_1 \times \ldots \times T_m$ to \texttt{bitstring}, where $p_1, \ldots, p_n$
represent the maximum length of the non-bounded arguments of this function.
\item
$\texttt{time(=}T[, p_1, p_2]\texttt{)}$ designates the
maximal runtime of one call to bitstring comparison function
for bitstrings of type $T$, where $p_1, p_2$ represent the
maximum length of the arguments of this function when $T$ is non-bounded.
\item
$\texttt{time(!)}$ or $\texttt{time(foreach)}$ is the maximum time of an access to a replication index.
\item
$\texttt{time([}n\texttt{])}$ is the maximum time of an array access 
with $n$ indices.
\item
$\texttt{time(\&\&)}$ is the maximum time of a boolean and.
\item
$\texttt{time(\string|\string|)}$ is the maximum time of a boolean or.
\item
$\texttt{time(new }T\texttt{)}$ or $\texttt{time(<-R }T\texttt{)}$ 
is the maximum time needed to choose
a random number of type $T$ according to the default distribution for type $T$.
\item
\ifchannels
$\texttt{time(newChannel)}$ is the maximum time to create a new
private channel.
\else
$\texttt{time(newOracle)}$ is the maximum time to create a new
private oracle.
\fi
\item
$\texttt{time(if)}$ is the maximum time to perform a boolean test.
\item
$\texttt{time(find }n\texttt{)}$ is the maximum time to perform 
one condition test of a find with $n$ indices to choose.
(Essentially, the time to store the values of the indices in a 
list and part of the time needed to randomly choose an element
of that list.)
\ifchannels
\item
$\texttt{time(out [}T_1, \ldots, T_m\texttt{]}T, p_1, \ldots, p_n\texttt{)}$
represents the time of an output in which the channel indices are
of types $T_1, \ldots, T_m$, the output bitstring is of type $T$,
and the maximum length of the channel indices and the output bitstring
is represented by $p_1, \ldots, p_n$ when they are non-bounded.
\item
$\texttt{time(in }n\texttt{)}$ is the maximum time to store an
input in which the channel has $n$ indices in the list of
available inputs.
\fi
\end{itemize}
CryptoVerif checks the dimension of probability formulas.
%Actually, the check is stricter than usual dimension checking,
%since it distinguishes probabilities from other data without dimension.



\section{\texttt{oracles} Front-end}\label{sec:oracles}

\channelsfalse
\def\iprocess{\nonterm{odef}}
\def\oprocess{\nonterm{obody}}

The \texttt{oracles} front-end is similar to the  \texttt{channels}
with the following differences.
The keyword \texttt{newChannel} is replaced with \texttt{newOracle},
\texttt{run} is a keyword,
and \texttt{channel} and \texttt{out} are not keywords.

\begin{figure}[tp]
\def\phop{\phantom{\oprocess = }\mid}
\def\phip{\phantom{\iprocess = }\mid}
\begin{align*}
&\oprocess ::= \texttt{run}\ \nonterm{ident}[\texttt{(}\seq{term}\texttt{)}]\\
&\phop \texttt{(} \oprocess \texttt{)}\\
&\phop \yield\\
&\phop \texttt{event }\nonterm{ident}[\texttt{(}\seq{term}\texttt{)}]\ [\texttt{; }\oprocess]\\
&\phop \texttt{event\string_abort }\nonterm{ident}\\
&\phop \Resavt[\texttt{; }\oprocess]\\
&\phop \Resbvt[\texttt{; }\oprocess]\\
&\phop \nonterm{ident}[\texttt{:}\nonterm{ident}] \texttt{ <- }\nonterm{term}[\texttt{; }\oprocess]\\
&\phop \texttt{let }\nonterm{pattern} \texttt{ = }\nonterm{term}\ 
[\texttt{in }\oprocess\ [\texttt{else }\oprocess]]\\
&\phop \texttt{if }\nonterm{cond}\texttt{ then }\oprocess\ [\texttt{else }\oprocess]\\
&\phop \texttt{find}[\texttt{[unique]}]\ \nonterm{findbranch}\ (\texttt{orfind }\nonterm{findbranch})^* \ [\texttt{else }\oprocess]\\
&\phop \texttt{insert }\nonterm{ident}\texttt{(}\seq{term}\texttt{)}\ [\texttt{; }\oprocess]\\
&\phop \texttt{get }\nonterm{ident}\texttt{(}\seq{pattern}\texttt{)}\ [\texttt{suchthat}\ \nonterm{term}]\texttt{ in }\oprocess\ [\texttt{else }\oprocess]\\
&\phop \texttt{return(}\seq{term}\texttt{)}[\texttt{; }\iprocess]\\
&\nonterm{findbranch} ::= \seq{identbound} \texttt{ suchthat }\nonterm{cond}\texttt{ then }\oprocess\\
&\iprocess ::= \texttt{run}\ \nonterm{ident}[\texttt{(}\seq{term}\texttt{)}]\\
&\phip \texttt{(} \iprocess \texttt{)}\\
&\phip \texttt{0}\\
&\phip \iprocess \texttt{ | } \iprocess\\
&\phip \texttt{!} [\nonterm{ident}\texttt{ <=}]\ \nonterm{ident}\ \iprocess\\
&\phip \texttt{foreach }\nonterm{ident}\texttt{ <= } \nonterm{ident}\texttt{ do }\iprocess\\
&\phip \nonterm{ident}\texttt{(}\seq{pattern}\texttt{) := }\oprocess
\end{align*}
\caption{Grammar for processes (\texttt{oracles} front-end)}
\label{fig:syntax3or}
\end{figure}


The input file consists of a list of declarations followed by 
an oracle definition:
\[\nonterm{declaration}^*\ {\tt process}\ \iprocess\]
The syntax of processes is given in Figure~\ref{fig:syntax3or}.
The calculus distinguishes two kinds of processes: oracle definitions
$\iprocess$ define new oracles; oracle bodies $\oprocess$ return a
result after executing some internal computations.  When a process
(oracle definition or oracle body) is an identifier, it is substituted
with its value defined by a \texttt{let} declaration.

The oracle definition $\texttt{run }\mathit{proc}(M_1, \dots, M_n)$ is replaced with $P\{M_1/x_1, \dots, M_n/x_n\}$ when $\mathit{proc}$ is declared by $\texttt{let}\ \mathit{proc}(x_1:T_1, \dots, x_n:T_n)\texttt{ = }P\texttt{.}$ where $P$ is an oracle definition.
The terms $M_1, \dots, M_n$ must contain only variables, replication indices, and function applications.

The oracle definition $O\texttt{(}p_1, \ldots, p_n\texttt{) := }P$ defines an oracle
$O$ taking arguments $p_1, \ldots, p_n$, and returning the result of
the oracle body $P$. The patterns $p_1, \ldots, p_n$ are as in the
\texttt{let} construct above, except that variables in $p$ that are
not under a function symbol $f(\ldots)$ must be declared with their
type. The other oracle definitions are similar to input processes
in the \texttt{channels} front-end.

When an oracle $O$ is defined under $\texttt{foreach }i_1\texttt{<=}N_1$, 
\ldots, $\texttt{foreach }i_n\texttt{<=}N_n$, it also implicitly
defines $O[i_1, \ldots, i_n]$.

Note that the construct 
$\textbf{newOracle }c;Q$ used in research papers
is absent from the implementation: this construct is useful in the proof
of soundness of CryptoVerif, but not essential for encoding games
that CryptoVerif manipulates.

Let us now describe oracle bodies:
\begin{itemize}

\item $\texttt{run }\mathit{proc}(M_1, \dots, M_n)$ is replaced with $\texttt{let}$ $x_1 = M_1$ $\texttt{in}$ \dots $\texttt{let}$ $x_n = M_n$ $\texttt{in}$ $P$ when $\mathit{proc}$ is declared by $\texttt{let}\ \mathit{proc}(x_1:T_1, \dots, x_n:T_n)\texttt{ = }P\texttt{.}$ where $P$ is an oracle body.

\item {\yield} terminates the oracle, returning control to the caller.

\item 
$\texttt{return(}N_1, \ldots, N_l\texttt{);}Q$ terminates the oracle,
returning the result of the terms $N_1, \ldots, N_l$. Then, it makes
available the oracles defined in $Q$.

\item The other oracle bodies are similar to output processes
in the \texttt{channels} front-end.

\end{itemize}

In $\texttt{return(}M_1, \ldots, M_n\texttt{)}$,
$M_j$ must be of a bitstring type $T_j$ for all $j \leq n$
and that return instruction is said to be of type $T_1 \times \ldots 
\times T_n$.
All return instructions in an oracle body $P$ (excluding return
instructions that occur in oracle definitions $Q$ in processes of the form 
$\texttt{return(}M_1, \ldots, M_n\texttt{);}Q$) must be of the same
type, and that type is said to be the type of the oracle body $P$.
%
For each oracle definition $O\texttt{(}p_1, \ldots, p_m\texttt{) :=
}P$ under $\texttt{foreach }i_1\texttt{<=}N_1$, \ldots,
$\texttt{foreach }i_n\texttt{<=}N_n$, the oracle $O$ is said to be of
type $[1, N_1] \times \ldots \times [1, N_n] \rightarrow T'_1 \times
\ldots \times T'_m \rightarrow T_1 \times \ldots \times T_n$ where
$p_j$ is of type $T'_j$ for all $j \leq m$ and $P$ is of type $T_1
\times \ldots \times T_n$. When an oracle has several definitions,
it must be of the same type for all its definitions. Furthermore,
definitions of the same oracle $O$ must not occur on both sides
of a parallel composition $Q \texttt{|} Q'$ (so that several definitions
of the same oracle cannot be simultaneously available).
%
The other constructs are typed as in the \texttt{channels} front-end.

The $\texttt{channel}\ \neseq{ident}\texttt{.}$ declaration is removed,
since channels do not exist in the \texttt{oracles} front-end.

In probability formulas (Figure~\ref{fig:syntax2}), 
\texttt{time(out $\dots$)} and \texttt{time(in $n$)} are removed and
\texttt{time(newChannel)} is replaced with \texttt{time(newOracle)}.
$\texttt{time(newOracle)}$ is the maximum time to create a new
private oracle.


\section{Summary of the Main Differences between the two Front-ends}

The main difference between the two front-ends is that the \texttt{oracles}
front-end uses oracles while the \texttt{channels} front-end uses channels.
So we have essentially the following correspondence:
\begin{center}
\begin{tabular}{l|l}
\texttt{channels}&\texttt{oracles}\\
\hline
input process& oracle definition\\
output process& oracle body\\
$\texttt{newChannel }c$& $\texttt{newOracle }O$\\
$\texttt{in(}c\texttt{, (}x_1:T_1, \ldots, x_l:T_l\texttt{));}P$&$O\texttt{(}x_1:T_1, \ldots, x_l:T_l\texttt{) := }P$\\
$\texttt{out(}c\texttt{, (}M_1, \ldots, M_l\texttt{));}Q$&$\texttt{return(}M_1, \ldots, M_l\texttt{);}Q$\\
%$\texttt{yield}$&$\texttt{end}$\\
\end{tabular}
\end{center}
The \texttt{newChannel} or \texttt{newOracle} instruction does not appear
in processes, but appears in the evaluation time of contexts.
In the \texttt{channels} front-end, channels must be declared by a
\texttt{channel} declaration. There is no such declaration in the 
\texttt{oracles} front-end.

Finally, both front-ends accept two syntaxes for replication,
generation of random numbers, and assignments. However, 
the default syntax for the display differs:
\begin{center}
\begin{tabular}{l|l}
display in \texttt{channels}&display in \texttt{oracles}\\
\hline
$\texttt{!}i\texttt{<=}N\ Q$& $\texttt{foreach }i\texttt{<=}N\texttt{ do }Q$\\
$\texttt{new }x\texttt{:}T\texttt{; }P$&$x\texttt{ <-R }T\texttt{; }P$\\
$\texttt{let }x\texttt{:}T\texttt{ = }M\texttt{ in }P$&$x\texttt{:}T\texttt{ <- }M\texttt{; }P$\\
\end{tabular}
\end{center}
The assignment $x\texttt{:}T\texttt{ <- }M$ can be
used only for assigning a variable; when a pattern occurs instead
of the variable $x$, one has to use the \texttt{let} instruction.

\section{Predefined cryptographic primitives}\label{sect:prim}

\newcommand{\vn}[1]{\mathit{#1}}
\newcommand{\ab}{\allowbreak}

\sloppy

A number of standard cryptographic primitives are predefined in
CryptoVerif.  The definitions of these primitives are given as macros
in the library file \texttt{default.cvl} (or \texttt{default.ocvl} for
the \texttt{oracles} front-end) that is automatically loaded at
startup.  The user does not need to redefine these primitives, he can
just expand the corresponding macro. The examples contained in the library
can be used as a basis in order to build definitions of new primitives, by copying
and modifying them as desired. Here is a list of the predefined primitives.
\begin{itemize}

\item $\texttt{expand IND\_CPA\_sym\_enc(}\vn{key}$,
$  \vn{cleartext}$, $\vn{ciphertext}$, $\vn{enc}$$,
$  $\vn{dec}$, $\vn{injbot}$, $\vn{Z}$, $\vn{Penc}\texttt{).}$ defines a
  IND-CPA (indistinguishable under chosen plaintext attacks)
  probabilistic symmetric encryption scheme.

   $\vn{key}$ is the type of keys, must be \texttt{bounded} (to be able to generate random numbers from it, and to talk about the runtime of $\vn{enc}$ without mentioning the length of the key), typically \texttt{fixed} and \texttt{large}.

   $\vn{cleartext}$ is the type of cleartexts.

   $\vn{ciphertext}$ is the type of ciphertexts.

   $\vn{enc}(\vn{cleartext}, \vn{key}): \vn{ciphertext}$ is the encryption function. Internally, it generates random coins, so that it is probabilistic.

   $\vn{dec}(\vn{ciphertext}, \vn{key}): \texttt{bitstringbot}$ is the
  decryption function; it returns \texttt{bottom} when decryption
  fails.

   $\vn{injbot}(\vn{cleartext}): \texttt{bitstringbot}$ is the natural
  injection from $\vn{cleartext}$ to \texttt{bitstringbot}.

   $\vn{Z}(\vn{cleartext}): \vn{cleartext}$ is the function that
  returns for each cleartext a cleartext of the same length consisting
  only of zeroes.

  $\vn{Penc}(t, N, l)$ is the probability of breaking the IND-CPA
  property in time $t$ for one key and $N$ encryption queries with
  cleartexts of length at most $l$.

   The types $\vn{key}$, $\vn{cleartext}$,
   $\vn{ciphertext}$ and the probability $\vn{Penc}$ must
   be declared before this macro is expanded. The functions
   $\vn{enc}$, $\vn{dec}$, $\vn{injbot}$, and $\vn{Z}$ are declared by this
   macro. They must not be declared elsewhere, and they can be used
   only after expanding the macro.

   This macro defines the equivalence named $\texttt{ind\_cpa}(\vn{enc})$
   for use in the \texttt{crypto} command in interactive proofs
   (see Section~\ref{sec:interact}).

\item $\texttt{expand IND\_CPA\_sym\_enc\_all\_args(}\vn{key}$,
  $\vn{cleartext}$, $\vn{ciphertext}$, $\vn{enc\_seed}$, $\vn{enc}$, $\vn{enc\_r}$, $\vn{enc\_r}'$,
  $\vn{dec}$, $\vn{injbot}$, $\vn{Z}$, $\vn{Penc}\texttt{).}$ is similar to the above,
  with three additional arguments. 

  $\vn{enc\_seed}$ is the type of random coins for encryption, must be \texttt{bounded}.

  $\vn{enc\_r}(\vn{cleartext}, \vn{key}, \vn{enc\_seed}): \vn{ciphertext}$ is the encryption function that takes coins as argument (instead of generating them internally).

  $\vn{enc\_r}'$ is the symbol that replaces $\vn{enc\_r}$ after game transformation.

\item $\texttt{expand IND\_CPA\_INT\_CTXT\_sym\_enc(}\vn{key}$,
  $\vn{cleartext}$, $\vn{ciphertext}$, $\vn{enc}$,
  $\vn{dec}$, $\vn{injbot}$, $\vn{Z}$, $\vn{Penc}$, $\vn{Pencctxt}\texttt{).}$ defines a
  IND-CPA (indistinguishable under chosen plaintext attacks) and INT-CTXT (ciphertext integrity)
  probabilistic symmetric encryption scheme.

   $\vn{key}$ is the type of keys, must be \texttt{bounded} (to be able to generate random numbers from it, and to talk about the runtime of $\vn{enc}$ without mentioning the length of the key), typically \texttt{fixed} and \texttt{large}.

   $\vn{cleartext}$ is the type of cleartexts.

   $\vn{ciphertext}$ is the type of ciphertexts.

   $\vn{enc}(\vn{cleartext}, \vn{key}): \vn{ciphertext}$ is the encryption function. Internally, it generates random coins, so that it is probabilistic.

   $\vn{dec}(\vn{ciphertext}, \vn{key}): \texttt{bitstringbot}$ is the
  decryption function; it returns \texttt{bottom} when decryption
  fails.

   $\vn{injbot}(\vn{cleartext}): \texttt{bitstringbot}$ is the natural
  injection from $\vn{cleartext}$ to \texttt{bitstringbot}.

   $\vn{Z}(\vn{cleartext}): \vn{cleartext}$ is the function that
  returns for each cleartext a cleartext of the same length consisting
  only of zeroes.

  $\vn{Penc}(t, N, l)$ is the probability of breaking the IND-CPA
  property in time $t$ for one key and $N$ encryption queries with
  cleartexts of length at most $l$.

  $\vn{Pencctxt}(t, N, N', l, l')$ is the probability of breaking the
  INT-CTXT property in time $t$ for one key, $N$ encryption queries,
  $N'$ decryption queries with cleartexts of length at most $l$ and
  ciphertexts of length at most $l'$.

   The types $\vn{key}$, $\vn{cleartext}$,
   $\vn{ciphertext}$ and the probabilities $\vn{Penc}$ and $\vn{Pencctxt}$ must
   be declared before this macro is expanded. The functions
   $\vn{enc}$, $\vn{dec}$, $\vn{injbot}$, and $\vn{Z}$ are declared by this
   macro. They must not be declared elsewhere, and they can be used
   only after expanding the macro.

   This macro defines the equivalences named $\texttt{ind\_cpa}(\vn{enc})$,
   $\texttt{int\_ctxt}(\vn{enc})$, and $\texttt{int\_ctxt\_corrupt}(\vn{enc})$ 
   for use in the \texttt{crypto} command 
   (see Section~\ref{sec:interact}). 
   The first equivalence corresponds to the
   IND-CPA property, the last two to the INT-CTXT property.
   The equivalence $\texttt{int\_ctxt\_corrupt}(\vn{enc})$ is used when the
   key may be corrupted. It is applied only manually.
   The equivalence $\texttt{int\_ctxt}(\vn{enc})$
   should generally be applied before $\texttt{ind\_cpa}(\vn{enc})$,
   because $\texttt{int\_ctxt}(\vn{enc})$ eliminates the decryption oracle.

\item $\texttt{expand IND\_CPA\_INT\_CTXT\_sym\_enc\_all\_args(}\vn{key}$,
$  \vn{cleartext}$, $\vn{ciphertext}$, $\vn{enc\_seed}$, $\vn{enc}$, $\vn{enc\_r}$, $\vn{enc\_r}'$,
$  \vn{dec}$, $\vn{injbot}$, $\vn{Z}$, $\vn{Penc}$, $\vn{Pencctxt}\texttt{).}$  is similar to the above,
  with three additional arguments. 

  $\vn{enc\_seed}$ is the type of random coins for encryption, must be \texttt{bounded}.

  $\vn{enc\_r}(\vn{cleartext}, \vn{key}, \vn{enc\_seed}): \vn{ciphertext}$ is the encryption function that takes coins as argument (instead of generating them internally).

  $\vn{enc\_r}'$ is the symbol that replaces $\vn{enc\_r}$ after game transformation.

\item $\texttt{expand AEAD(}\vn{key}$,
$  \vn{cleartext}$, $\vn{ciphertext}$, $\vn{add\_data}$, $\vn{enc}$,
$  \vn{dec}$, $\vn{injbot}$, $\vn{Z}$, $\vn{Penc}$, $\vn{Pencctxt}\texttt{).}$ defines an
authenticated encryption scheme with additional data.

   $\vn{key}$ is the type of keys, must be \texttt{bounded} (to be able to generate random numbers from it, and to talk about the runtime of $\vn{enc}$ without mentioning the length of the key), typically \texttt{fixed} and \texttt{large}.

   $\vn{cleartext}$ is the type of cleartexts.

   $\vn{ciphertext}$ is the type of ciphertexts.

   $\vn{add\_data}$ is the type of additional data.

   $\vn{enc}(\vn{cleartext}, \vn{add\_data}, \vn{key}): \vn{ciphertext}$ is the encryption function. Internally, it generates random coins, so that it is probabilistic.

   $\vn{dec}(\vn{ciphertext}, \vn{add\_data}, \vn{key}): \texttt{bitstringbot}$ is the
  decryption function; it returns \texttt{bottom} when decryption
  fails.

   $\vn{injbot}(\vn{cleartext}): \texttt{bitstringbot}$ is the natural
  injection from $\vn{cleartext}$ to \texttt{bitstringbot}.

   $\vn{Z}(\vn{cleartext}): \vn{cleartext}$ is the function that
  returns for each cleartext a cleartext of the same length consisting
  only of zeroes.

  $\vn{Penc}(t, N, l)$ is the probability of breaking the IND-CPA
  property in time $t$ for one key and $N$ encryption queries with
  cleartexts of length at most $l$.

  $\vn{Pencctxt}(t, N, N', l, l', \vn{ld}, \vn{ld}')$ is the probability of
  breaking the INT-CTXT property in time $t$ for one key, $N$
  encryption queries, $N'$ decryption queries with cleartexts of
  length at most $l$ and ciphertexts of length at most $l'$,
  additional data for encryption of length at most $\vn{ld}$, and
  additional data for decryption of length at most $\vn{ld}'$.

   The types $\vn{key}$, $\vn{cleartext}$,
   $\vn{ciphertext}$, $\vn{add\_data}$ and the probabilities $\vn{Penc}$ and $\vn{Pencctxt}$ must
   be declared before this macro is expanded. The functions
   $\vn{enc}$, $\vn{dec}$, $\vn{injbot}$, and $\vn{Z}$ are declared by this
   macro. They must not be declared elsewhere, and they can be used
   only after expanding the macro.

   This macro defines the equivalences named $\texttt{ind\_cpa}(\vn{enc})$,
   $\texttt{int\_ctxt}(\vn{enc})$, and $\texttt{int\_ctxt\_corrupt}(\vn{enc})$ 
   for use in the \texttt{crypto} command 
   (see Section~\ref{sec:interact}). 
   The first equivalence corresponds to the
   IND-CPA property, the last two to the INT-CTXT property.
   The equivalence $\texttt{int\_ctxt\_corrupt}(\vn{enc})$ is used when the
   key may be corrupted. It is applied only manually.
   The equivalence $\texttt{int\_ctxt}(\vn{enc})$
   should generally be applied before $\texttt{ind\_cpa}(\vn{enc})$,
   because $\texttt{int\_ctxt}(\vn{enc})$ eliminates the decryption oracle.

\item $\texttt{expand AEAD\_all\_args(}\vn{key}$,
$  \vn{cleartext}$, $\vn{ciphertext}$, $\vn{add\_data}$, $\vn{enc\_seed}$, $\vn{enc}$, $\vn{enc\_r}$, $\vn{enc\_r}'$,
$  \vn{dec}$, $\vn{injbot}$, $\vn{Z}$, $\vn{Penc}$, $\vn{Pencctxt}\texttt{).}$ is similar to the above,
  with three additional arguments. 

  $\vn{enc\_seed}$ is the type of random coins for encryption, must be \texttt{bounded}.

  $\vn{enc\_r}(\vn{cleartext}, \vn{add\_data}, \vn{key}, \vn{enc\_seed}): \vn{ciphertext}$ is the encryption function that takes coins as argument (instead of generating them internally).

  $\vn{enc\_r}'$ is the symbol that replaces $\vn{enc\_r}$ after game transformation.


\item $\texttt{expand AEAD\_nonce(}\vn{key}$,
$  \vn{cleartext}$, $\vn{ciphertext}$, $\vn{add\_data}$, $\vn{nonce}$, $\vn{enc}$,
$  \vn{dec}$, $\vn{injbot}$, $\vn{Z}$, $\vn{Penc}$, $\vn{Pencctxt}\texttt{).}$ defines an
authenticated encryption scheme with additional data, using a nonce that must have a different
value in each call to encryption. A typical example is AES-GCM.

   $\vn{key}$ is the type of keys, must be \texttt{bounded} (to be able to generate random numbers from it, and to talk about the runtime of $\vn{enc}$ without mentioning the length of the key), typically \texttt{fixed} and \texttt{large}.

   $\vn{cleartext}$ is the type of cleartexts.

   $\vn{ciphertext}$ is the type of ciphertexts.

   $\vn{add\_data}$ is the type of additional data.

   $\vn{nonce}$ is the type of nonces.

   $\vn{enc}(\vn{cleartext}, \vn{add\_data}, \vn{key}, \vn{nonce}): \vn{ciphertext}$ is the encryption function. 

   $\vn{dec}(\vn{ciphertext}, \vn{add\_data}, \vn{key}, \vn{nonce}): \texttt{bitstringbot}$ is the
  decryption function; it returns \texttt{bottom} when decryption
  fails.

   $\vn{injbot}(\vn{cleartext}): \texttt{bitstringbot}$ is the natural
  injection from $\vn{cleartext}$ to \texttt{bitstringbot}.

   $\vn{Z}(\vn{cleartext}): \vn{cleartext}$ is the function that
  returns for each cleartext a cleartext of the same length consisting
  only of zeroes.

  $\vn{Penc}(t, N, l)$ is the probability of breaking the IND-CPA
  property in time $t$ for one key and $N$ encryption queries with
  cleartexts of length at most $l$.

  $\vn{Pencctxt}(t, N, N', l, l', \vn{ld}, \vn{ld}')$ is the probability of
  breaking the INT-CTXT property in time $t$ for one key, $N$
  encryption queries, $N'$ decryption queries with cleartexts of
  length at most $l$ and ciphertexts of length at most $l'$,
  additional data for encryption of length at most $\vn{ld}$, and
  additional data for decryption of length at most $\vn{ld}'$.

   The types $\vn{key}$, $\vn{cleartext}$,
   $\vn{ciphertext}$, $\vn{add\_data}$, $\vn{nonce}$ and the probabilities $\vn{Penc}$ and $\vn{Pencctxt}$ must
   be declared before this macro is expanded. The functions
   $\vn{enc}$, $\vn{dec}$, $\vn{injbot}$, and $\vn{Z}$ are declared by this
   macro. They must not be declared elsewhere, and they can be used
   only after expanding the macro.

   This macro defines the equivalences named $\texttt{ind\_cpa}(\vn{enc})$,
   $\texttt{int\_ctxt}(\vn{enc})$, and $\texttt{int\_ctxt\_corrupt}(\vn{enc})$ 
   for use in the \texttt{crypto} command 
   (see Section~\ref{sec:interact}). 
   The first equivalence corresponds to the
   IND-CPA property, the last two to the INT-CTXT property.
   The equivalence $\texttt{int\_ctxt\_corrupt}(\vn{enc})$ is used when the
   key may be corrupted. It is applied only manually.
   The equivalence $\texttt{int\_ctxt}(\vn{enc})$
   should generally be applied before $\texttt{ind\_cpa}(\vn{enc})$,
   because $\texttt{int\_ctxt}(\vn{enc})$ eliminates the decryption oracle.

\item $\texttt{expand AEAD\_nonce\_all\_args(}\vn{key}$,
$  \vn{cleartext}$, $\vn{ciphertext}$, $\vn{add\_data}$, $\vn{nonce}$, $\vn{enc}$, $\vn{enc}'$,
$  \vn{dec}$, $\vn{injbot}$, $\vn{Z}$, $\vn{Penc}$, $\vn{Pencctxt}\texttt{).}$ is similar to the above with one additional argument.

  $\vn{enc}'$ is the symbol that replaces $\vn{enc}$ after game transformation.


\item $\texttt{expand IND\_CCA2\_sym\_enc(}\vn{key}$,
$  \vn{cleartext}$, $\vn{ciphertext}$, $\vn{enc}$,
$  \vn{dec}$, $\vn{injbot}$, $\vn{Z}$, $\vn{Penc}\texttt{).}$ defines a
  IND-CCA2 (indistinguishable under adaptive chosen ciphertext attacks)
  probabilistic symmetric encryption scheme.

   $\vn{key}$ is the type of keys, must be \texttt{bounded} (to be able to generate random numbers from it, and to talk about the runtime of $\vn{enc}$ without mentioning the length of the key), typically \texttt{fixed} and \texttt{large}.

   $\vn{cleartext}$ is the type of cleartexts.

   $\vn{ciphertext}$ is the type of ciphertexts.

   $\vn{enc}(\vn{cleartext}, \vn{key}): \vn{ciphertext}$ is the encryption function. Internally, it generates random coins, so that it is probabilistic.

   $\vn{dec}(\vn{ciphertext}, \vn{key}): \texttt{bitstringbot}$ is the
  decryption function; it returns \texttt{bottom} when decryption
  fails.

   $\vn{injbot}(\vn{cleartext}): \texttt{bitstringbot}$ is the natural
  injection from $\vn{cleartext}$ to \texttt{bitstringbot}.

   $\vn{Z}(\vn{cleartext}): \vn{cleartext}$ is the function that
  returns for each cleartext a cleartext of the same length consisting
  only of zeroes.

  $\vn{Penc}(t, N, \vn{Nu}, N', l, l')$ is the probability of breaking the
  IND-CCA2 property in time $t$ for one key, $N$ encryption queries that are 
  different in both sides of the IND-CCA2 equivalence, 
  $\vn{Nu}$ encryption queries that are the same in both side of the IND-CCA2 equivalence, $N'$
  decryption queries with cleartexts of length at most $l$ and
  ciphertexts of length at most $l'$.

   The types $\vn{key}$, $\vn{cleartext}$,
   $\vn{ciphertext}$ and the probability $\vn{Penc}$ must
   be declared before this macro is expanded. The functions
   $\vn{enc}$, $\vn{dec}$, $\vn{injbot}$, and $\vn{Z}$ are declared by this
   macro. They must not be declared elsewhere, and they can be used
   only after expanding the macro.

   This macro defines the equivalences named
   $\texttt{ind\_cca2}(\vn{enc})$ and
   $\texttt{ind\_cca2\_partial}(\vn{enc})$, for use in the
   \texttt{crypto} command (see Section~\ref{sec:interact}). While the
   equivalence $\texttt{ind\_cca2}(\vn{enc})$ replaces all cleartexts
   with zeroes, the equivalence
   $\texttt{ind\_cca2\_partial}(\vn{enc})$ replaces only some of them
   with zeroes. The latter equivalence can be applied only manually.
   The user should map the occurrences of encryption that he wants to
   transform to oracle $\vn{Oenc}$, the ones he wants to leave unchanged to
   oracle $\vn{Oenc\_unchanged}$, and the ones that have already been transformed
   by a previous application of this equivalence to oracle $\vn{Oenc\_unchanged}'$.

\item $\texttt{expand IND\_CCA2\_sym\_enc\_all\_args(}\vn{key}$,
$  \vn{cleartext}$, $\vn{ciphertext}$, $\vn{enc\_seed}$, $\vn{enc}$, $\vn{enc\_r}$, $\vn{enc\_r}'$,
$  \vn{dec}$, $\vn{dec}'$, $\vn{injbot}$, $\vn{Z}$, $\vn{Penc}\texttt{).}$ is similar to the above,
  with four additional arguments. 

  $\vn{enc\_seed}$ is the type of random coins for encryption, must be \texttt{bounded}.

  $\vn{enc\_r}(\vn{cleartext}, \vn{key}, \vn{enc\_seed}): \vn{ciphertext}$ is the encryption function that takes coins as argument (instead of generating them internally).

  $\vn{enc\_r}'$ and $\vn{dec}'$ are the symbols that replace $\vn{enc\_r}$ and $\vn{dec}$ respectively after game transformation.


\item $\texttt{expand INT\_PTXT\_sym\_enc(}\vn{key}$,
$  \vn{cleartext}$, $\vn{ciphertext}$, $\vn{enc}$,
$  \vn{dec}$, $\vn{injbot}$, $\vn{Pencptxt}\texttt{).}$ defines an INT-PTXT (plaintext integrity)
  probabilistic symmetric encryption scheme.

   $\vn{key}$ is the type of keys, must be \texttt{bounded} (to be able to generate random numbers from it, and to talk about the runtime of $\vn{enc}$ without mentioning the length of the key), typically \texttt{fixed} and \texttt{large}.

   $\vn{cleartext}$ is the type of cleartexts.

   $\vn{ciphertext}$ is the type of ciphertexts.

   $\vn{enc}(\vn{cleartext}, \vn{key}): \vn{ciphertext}$ is the encryption function. Internally, it generates random coins, so that it is probabilistic.

   $\vn{dec}(\vn{ciphertext}, \vn{key}): \texttt{bitstringbot}$ is the
  decryption function; it returns \texttt{bottom} when decryption
  fails.

   $\vn{injbot}(\vn{cleartext}): \texttt{bitstringbot}$ is the natural
  injection from $\vn{cleartext}$ to \texttt{bitstringbot}.

  $\vn{Pencptxt}(t, N, N', \vn{Nu}', l, l')$ is the probability of breaking the
  INT-PTXT property in time $t$ for one key, $N$ encryption queries, $N'$
  decryption queries that are modified by the transformation, and 
  $\vn{Nu}'$ decryption queries that are left unchanged by the transformation,
  with cleartexts of length at most $l$ and
  ciphertexts of length at most $l'$.

   The types $\vn{key}$, $\vn{cleartext}$,
   $\vn{ciphertext}$ and the probability $\vn{Pencptxt}$ must
   be declared before this macro is expanded. The functions
   $\vn{enc}$, $\vn{dec}$, and $\vn{injbot}$ are declared by this
   macro. They must not be declared elsewhere, and they can be used
   only after expanding the macro.

   This macro defines the equivalences named
   $\texttt{int\_ptxt}(\vn{enc})$ and
   $\texttt{int\_ptxt\_corrupt\_partial}(\vn{enc})$, for use in the
   \texttt{crypto} command (see Section~\ref{sec:interact}).  While
   the equivalence $\texttt{ind\_ptxt}(\vn{enc})$ replaces all
   decryption with lookups in encryption queries, the equivalence
   $\texttt{ind\_ptxt\_corrupt\_partial}(\vn{enc})$ may replace only some of them
   and supports corruption of the key. 
   The latter equivalence can be applied only manually.
   To transform only some occurrences of decryption, 
   the user should map the occurrences of decryption that he wants to
   transform to oracle $\vn{Odec}$, the ones he wants to leave
   unchanged to oracle $\vn{Odec\_unchanged}$, and the ones that have
   already been transformed by a previous application of this
   equivalence to oracle $\vn{Odec\_unchanged}'$.

\item $\texttt{expand INT\_PTXT\_sym\_enc\_all\_args(}\vn{key}$,
$  \vn{cleartext}$, $\vn{ciphertext}$, $\vn{enc\_seed}$, $\vn{enc}$, $\vn{enc\_r}$,
$  \vn{dec}$, $\vn{dec}'$, $\vn{injbot}$, $\vn{Pencptxt}\texttt{).}$ is similar to the above,
  with three additional arguments. 

  $\vn{enc\_seed}$ is the type of random coins for encryption, must be \texttt{bounded}.

  $\vn{enc\_r}(\vn{cleartext}, \vn{key}, \vn{enc\_seed}): \vn{ciphertext}$ is the encryption function that takes coins as argument (instead of generating them internally).

  $\vn{dec}'$ is the symbol that replaces $\vn{dec}$ after game transformation.

\item $\texttt{expand IND\_CCA2\_INT\_PTXT\_sym\_enc(}\vn{key}$,
$  \vn{cleartext}$, $\vn{ciphertext}$, $\vn{enc}$,
$  \vn{dec}$, $\vn{injbot}$, $\vn{Z}$, $\vn{Penc}$, $\vn{Pencptxt}\texttt{).}$ defines a
  IND-CCA2 (indistinguishable under adaptive chosen ciphertext attacks) and INT-PTXT (plaintext integrity)
  probabilistic symmetric encryption scheme.

   $\vn{key}$ is the type of keys, must be \texttt{bounded} (to be able to generate random numbers from it, and to talk about the runtime of $\vn{enc}$ without mentioning the length of the key), typically \texttt{fixed} and \texttt{large}.

   $\vn{cleartext}$ is the type of cleartexts.

   $\vn{ciphertext}$ is the type of ciphertexts.

   $\vn{enc}(\vn{cleartext}, \vn{key}): \vn{ciphertext}$ is the encryption function. Internally, it generates random coins, so that it is probabilistic.

   $\vn{dec}(\vn{ciphertext}, \vn{key}): \texttt{bitstringbot}$ is the
  decryption function; it returns \texttt{bottom} when decryption
  fails.

   $\vn{injbot}(\vn{cleartext}): \texttt{bitstringbot}$ is the natural
  injection from $\vn{cleartext}$ to \texttt{bitstringbot}.

   $\vn{Z}(\vn{cleartext}): \vn{cleartext}$ is the function that
  returns for each cleartext a cleartext of the same length consisting
  only of zeroes.

  $\vn{Penc}(t, N, \vn{Nu}, N', l, l')$ is the probability of breaking the
  IND-CCA2 property in time $t$ for one key, $N$ encryption queries that are 
  different in both sides of the IND-CCA2 equivalence, 
  $\vn{Nu}$ encryption queries that are the same in both side of the IND-CCA2 equivalence, $N'$
  decryption queries with cleartexts of length at most $l$ and
  ciphertexts of length at most $l'$.

  $\vn{Pencptxt}(t, N, N', \vn{Nu}', l, l')$ is the probability of breaking the
  INT-PTXT property in time $t$ for one key, $N$ encryption queries, $N'$
  decryption queries that are modified by the transformation, and 
  $\vn{Nu}'$ decryption queries that are left unchanged by the transformation,
  with cleartexts of length at most $l$ and
  ciphertexts of length at most $l'$.

   The types $\vn{key}$, $\vn{cleartext}$,
   $\vn{ciphertext}$ and the probabilities $\vn{Penc}$ and $\vn{Pencptxt}$ must
   be declared before this macro is expanded. The functions
   $\vn{enc}$, $\vn{dec}$, $\vn{injbot}$, and $\vn{Z}$ are declared by this
   macro. They must not be declared elsewhere, and they can be used
   only after expanding the macro.

   This macro defines the equivalences named
   $\texttt{ind\_cca2}(\vn{enc})$,
   $\texttt{ind\_cca2\_after\_int\_ptxt}(\vn{enc})$,
   $\texttt{ind\_cca2\_partial}(\vn{enc})$,
   $\texttt{int\_ptxt}(\vn{enc})$,
   $\texttt{int\_ptxt\_after\_ind\_cca2}(\vn{enc})$, and
   $\texttt{int\_ptxt\_corrupt\_partial}(\vn{enc})$, for use in the
   \texttt{crypto} command (see Section~\ref{sec:interact}). 
   %
   The first three correspond to the IND-CCA2 property, the last three
   to the INT-PTXT property. The equivalence
   $\texttt{ind\_cca2}(\vn{enc})$ can be applied before applying the
   INT-PTXT property, while
   $\texttt{ind\_cca2\_after\_int\_ptxt}(\vn{enc})$ can be applied
   after applying the INT-PTXT property. Similarly, the equivalence
   $\texttt{int\_ptxt}(\vn{enc})$ can be applied before applying the
   IND-CCA2 property, while
   $\texttt{int\_ptxt\_after\_ind\_cca2}(\vn{enc})$ can be applied
   after applying the IND-CCA2 property.
   %
   The equivalences $\texttt{ind\_cca2\_partial}(\vn{enc})$ and
   $\texttt{int\_ptxt\_corrupt\_partial}(\vn{enc})$ may transform only some
   occurrences of encryption and/or decryption, and
   $\texttt{int\_ptxt\_corrupt\_partial}(\vn{enc})$ supports corruption
   of the key.  They can be applied only manually, in any order.
   %
   For $\texttt{ind\_cca2\_partial}(\vn{enc})$, the user should map
   the occurrences of encryption that he wants to transform to oracle
   $\vn{Oenc}$, the ones he wants to leave unchanged to oracle
   $\vn{Oenc\_unchanged}$.
   %
   For $\texttt{int\_ptxt\_partial}(\vn{enc})$, the user should map
   the occurrences of decryption that he wants to transform to oracle
   $\vn{Odec}$, the ones he wants to leave unchanged to oracle
   $\vn{Odec\_unchanged}$.

   CryptoVerif often needs manual guidance with this property,
   because it does not know which property (IND-CCA2 or INT-PTXT)
   to apply first. Moreover, when empty plaintexts are not allowed,
   IND-CCA2 and INT-PTXT is equivalent to IND-CPA and INT-CTXT, 
   which is much easier to use for CryptoVerif, so we recommend
   using the latter property when possible.

\item $\texttt{expand IND\_CCA2\_INT\_PTXT\_sym\_enc\_all\_args(}\vn{key}$,
$  \vn{cleartext}$, $\vn{ciphertext}$, $\vn{enc\_seed}$, $\vn{enc}$, $\vn{enc\_r}$, $\vn{enc\_r}'$,
$  \vn{dec}$, $ \vn{dec}'$, $\vn{injbot}$, $\vn{Z}$, $\vn{Penc}$, $\vn{Pencptxt}\texttt{).}$ is similar to the above,
  with four additional arguments. 

  $\vn{enc\_seed}$ is the type of random coins for encryption, must be \texttt{bounded}.

  $\vn{enc\_r}(\vn{cleartext}, \vn{key}, \vn{enc\_seed}): \vn{ciphertext}$ is the encryption function that takes coins as argument (instead of generating them internally).

  $\vn{enc\_r}'$ and $\vn{dec}'$ are the symbols that replace $\vn{enc\_r}$ and $\vn{dec}$ respectively after game transformation.


\item $\texttt{expand SPRP\_cipher(}\vn{key}$,
$  \vn{blocksize}$, $\vn{enc}$, $\vn{dec}$, $\vn{Penc}\texttt{).}$
  defines a SPRP (super-pseudo-random permutation) deterministic
  symmetric encryption scheme.

   $\vn{key}$ is the type of keys, must be \texttt{bounded} (to be able to generate random numbers from it, and to talk about the runtime of $\vn{enc}$ without mentioning the length of the key), typically \texttt{fixed} and \texttt{large}.

   $\vn{blocksize}$ is the type of cleartexts and ciphertexts, 
   must be \texttt{fixed} and \texttt{large}.
   (The modeling of SPRP block ciphers is not perfect in that, in
   order to encrypt a new message, one chooses a fresh random number,
   not necessarily different from previously generated random
   numbers. Then CryptoVerif needs to eliminate collisions between
   those random numbers, so $\vn{blocksize}$ must really be
   \texttt{large}.)

   $\vn{enc}(\vn{blocksize}, \vn{key}): \vn{blocksize}$ is the encryption function.

   $\vn{dec}(\vn{blocksize}, \vn{key}): \vn{blocksize}$ is the
  decryption function.

  $\vn{Penc}(t, N, N')$ is the probability of breaking the SPRP
  property in time $t$ for one key, $N$ encryption queries, and $N'$
  decryption queries.

  The types $\vn{key}$, $\vn{blocksize}$ and the
  probability $\vn{Penc}$ must be declared before this macro is
  expanded. The functions $\vn{enc}$ and $\vn{dec}$ are
  declared by this macro. They must not be declared elsewhere, and
  they can be used only after expanding the macro.

   This macro defines the equivalence named $\texttt{sprp}(\vn{enc})$
   for use in the \texttt{crypto} command 
   (see Section~\ref{sec:interact}).

\item $\texttt{expand PRP\_cipher(}\vn{key}$,
$  \vn{blocksize}$, $\vn{enc}$, $\vn{dec}$, $\vn{Penc}\texttt{).}$
  defines a PRP (pseudo-random permutation) deterministic
  symmetric encryption scheme.

   $\vn{key}$ is the type of keys, must be \texttt{bounded} (to be able to generate random numbers from it, and to talk about the runtime of $\vn{enc}$ without mentioning the length of the key), typically \texttt{fixed} and \texttt{large}.

   $\vn{blocksize}$ is the type of cleartexts and ciphertexts, must be
   \texttt{fixed} and \texttt{large}.
   (The modeling of PRP block ciphers is not perfect in that, in order
   to encrypt a new message, one chooses a fresh random number, not
   necessarily different from previously generated random numbers. In
   other words, we model a PRF rather than a PRP, and apply the
   PRF/PRP switching lemma to make sure that this is sound. Then
   CryptoVerif needs to eliminate collisions between those random
   numbers, so $\vn{blocksize}$ must really be \texttt{large}.)

   $\vn{enc}(\vn{blocksize}, \vn{key}): \vn{blocksize}$ is the encryption function.

   $\vn{dec}(\vn{blocksize}, \vn{key}): \vn{blocksize}$ is the
  decryption function.

   $\vn{Penc}(t, N)$ is the probability of breaking the PRP property
   in time $t$ for one key and $N$ encryption queries.

   The types $\vn{key}$, $\vn{blocksize}$ and the probability $\vn{Penc}$ must
   be declared before this macro is expanded. The functions
   $\vn{enc}$ and $\vn{dec}$ are declared by this
   macro. They must not be declared elsewhere, and they can be used
   only after expanding the macro.

   This macro defines the equivalence named $\texttt{prp}(\vn{enc})$
   for use in the \texttt{crypto} command 
   (see Section~\ref{sec:interact}).

\item $\texttt{expand ICM\_cipher(}\vn{cipherkey}$, $\vn{key}$, $\vn{blocksize}$, $\vn{enc}$,
$   \vn{dec}$, $\vn{enc\_dec\_oracle}$, $\vn{qE}$, $\vn{qD}\texttt{).}$
   defines a block cipher in the ideal cipher model.

   $\vn{cipherkey}$ is the type of keys that correspond to the choice of the scheme, must be \texttt{bounded} or \texttt{nonuniform}, typically \texttt{fixed}.

   $\vn{key}$ is the type of keys (typically \texttt{large}).

   $\vn{blocksize}$ is type of the input and output of the cipher, 
   must be \texttt{bounded} or \texttt{nonuniform} (to be able to 
   generate random numbers from it; typically \texttt{fixed}), and \texttt{large}.
   (The modeling of the ideal cipher model is not perfect in that, in
   order to encrypt a new message, one chooses a fresh random number,
   not necessarily different from previously generated random
   numbers. Then CryptoVerif needs to eliminate collisions between
   those random numbers, so blocksize must really be \texttt{large}.)

   $\vn{enc}(\vn{cipherkey}, \vn{blocksize}, \vn{key}): \vn{blocksize}$ is the encryption function.

   $\vn{dec}(\vn{cipherkey}, \vn{blocksize}, \vn{key}): \vn{blocksize}$ is the decryption function.

   $\vn{enc\_dec\_oracle}$ is a parametric process that allows the adversary to
   call the encryption and decryption functions.
   WARNING: the encryption and decryption functions take 2 keys as
   input: the key of type cipherkey that corresponds to the choice of
   the scheme, and the normal encryption/decryption key. The cipherkey
   must be chosen once and for all at the beginning of the game and
   the encryption and decryption oracles must be made available to the
   adversary, by including the process $\vn{enc\_dec\_oracle}(\vn{ck})$
   where $\vn{ck}$ is the cipherkey.

   $\vn{qE}$ is the number of queries to the encryption oracle.

   $\vn{qD}$ is the number of queries to the decryption oracle.
 
   The types $\vn{cipherkey}$, $\vn{key}$, $\vn{blocksize}$ must be
   declared before this macro is expanded. The functions $\vn{enc}$,
   $\vn{dec}$, the process $\vn{enc\_dec\_oracle}$, and the paramters 
   $\vn{qE}$ and $\vn{qD}$ are declared by this macro. They must not be declared
   elsewhere, and they can be used only after expanding the macro.

   This macro defines the equivalence named $\texttt{icm}(\vn{enc})$
   for use in the \texttt{crypto} command 
   (see Section~\ref{sec:interact}).

\item $\texttt{expand SUF\_CMA\_det\_mac(}\vn{mkey}$,
$  \vn{macinput}$, $\vn{macres}$, $\vn{mac}$, $\vn{check}$,
$  \vn{Pmac}\texttt{).}$ defines an SUF-CMA (strongly unforgeable under chosen
  message attacks) deterministic MAC (message authentication code).

  The difference between a UF-CMA (unforgeable under chosen message
  attacks) MAC and a SUF-CMA MAC is that, for a UF-CMA MAC, the
  adversary may easily forge a new MAC for a message for which he has
  already seen a MAC. Such a forgery is guaranteed to be hard for a
  SUF-CMA MAC. For deterministic MACs, the verification can be done by
  recomputing the MAC, and in this case, an UF-CMA MAC is always
  SUF-CMA, so we model only SUF-CMA deterministic MACs. This macro
  transforms tests $\vn{mac}(k,m) = m'$ into $\vn{check}(k, m, m')$,
  so that the MAC verification can also be written
  $\vn{mac}(k,m) = m'$.

  $\vn{mkey}$ is the type of keys, must be \texttt{bounded} (to be
  able to generate random numbers from it, and to talk about the
  runtime of $\vn{mac}$ without mentioning the length of the key),
  typically \texttt{fixed} and \texttt{large}.

   $\vn{macinput}$ is the type of inputs of MACs

   $\vn{macres}$ is the type of MACs.

   $\vn{mac}(\vn{macinput}, \vn{mkey}): \vn{macres}$ is the MAC function.

   $\vn{check}(\vn{macinput}, \vn{mkey}, \vn{macres}): \texttt{bool}$ is the verification function.

   $\vn{Pmac}(t, N, N', \vn{Nu}', l)$ is the probability of breaking the SUF-CMA
   property in time $t$ for one key, $N$ MAC queries, $N'$ verification
   queries modified by the transformation and $\vn{Nu}$ verification
   queries left unchanged by the transformation for messages of length at most $l$.

   The types $\vn{mkey}$, $\vn{macinput}$,
   $\vn{macres}$ and the probability $\vn{Pmac}$ must be declared
   before this macro is expanded. The functions 
   $\vn{mac}$, $\vn{check}$ are declared by this macro. They must not
   be declared elsewhere, and they can be used only after expanding
   the macro.

   This macro defines the equivalences named
   $\texttt{suf\_cma}(\vn{mac})$,
   $\texttt{suf\_cma\_corrupt}(\vn{mac})$, and
   $\texttt{suf\_cma\_corrupt\_partial}(\vn{mac})$, for use in the
   \texttt{crypto} command (see Section~\ref{sec:interact}).
   All equivalences correspond to the SUF-CMA property, but the first one 
   does not allow corruption of the secret keys while last two allow it.
   The last two equivalences are applied only manually, in particular because their automatic
   application can sometimes be done too early, when other transformations
   should first be done in order to eliminate uses of the secret keys.
   The equivalence $\texttt{suf\_cma\_corrupt\_partial}(\vn{mac})$ allows the
   user to transform only some occurrences of the MAC verification into a 
   lookup in the MACed messages. The user should map the occurrences 
   he wants to transform to the oracle $\vn{Ocheck}$ and the ones
   he does not want to transform to the oracle $\vn{Ocheck\_unchanged}$.

\item $\texttt{expand SUF\_CMA\_det\_mac\_all\_args(}\vn{mkey}$,
$  \vn{macinput}$, $\vn{macres}$, $\vn{mac}$, $\vn{mac}'$, $\vn{check}$,
$  \vn{Pmac}\texttt{).}$ is similar to the above, with one additional argument.

  $\vn{mac}'$ is the symbol that replaces $\vn{mac}$ after game transformation.

\item $\texttt{expand UF\_CMA\_proba\_mac(}\vn{mkey}$,
$  \vn{macinput}$, $\vn{macres}$, $\vn{mac}$, $\vn{check}$,
$  \vn{Pmac}\texttt{).}$ defines a UF-CMA (unforgeable under chosen
  message attacks) probabilistic MAC (message authentication code).
  The arguments are the same as for $\texttt{SUF\_CMA\_det\_mac}$, but the $\vn{mac}$ function chooses random
  coins internally so that it is probabilistic, and the verification is not done by recomputing the MAC.
   This macro defines the equivalences named
   $\texttt{uf\_cma}(\vn{mac})$,
   $\texttt{uf\_cma\_corrupt}(\vn{mac})$, and
   $\texttt{uf\_cma\_corrupt\_partial}(\vn{mac})$ for use in the
   \texttt{crypto} command (see Section~\ref{sec:interact}),
   similarly to $\texttt{SUF\_CMA\_det\_mac}$.

\item $\texttt{expand UF\_CMA\_proba\_mac\_all\_args(}\vn{mkey}$,
$  \vn{macinput}$, $\vn{macres}$, $\vn{mac\_seed}$, $\vn{mac}$, $\vn{mac\_r}$, $\vn{mac\_r}'$, $\vn{check}$, $\vn{check}'$,
$  \vn{Pmac}\texttt{).}$ is similar to the above,
  with four additional arguments. 

  $\vn{mac\_seed}$ is the type of random coins for MAC, must be \texttt{bounded}.

  $\vn{mac\_r}(\vn{macinput}, \vn{mkey}, \vn{mac\_seed}): \vn{macres}$ is the MAC function that takes coins as argument (instead of generating them internally).

  $\vn{mac\_r}'$ and $\vn{check}'$ are the symbols that replace $\vn{mac\_r}$ and $\vn{check}$ respectively after game transformation.

\item $\texttt{expand SUF\_CMA\_proba\_mac(}\vn{mkey}$,
$  \vn{macinput}$, $\vn{macres}$, $\vn{mac}$, $\vn{check}$,
$  \vn{Pmac}\texttt{).}$ defines a SUF-CMA (strongly unforgeable under
  chosen message attacks) probabilistic MAC (message authentication code). 
  The arguments are the same as for $\texttt{SUF\_CMA\_det\_mac}$, but the $\vn{mac}$ function chooses random
  coins internally so that it is probabilistic, and the verification is not done by recomputing the MAC.
  This macro defines the equivalences named
   $\texttt{suf\_cma}(\vn{mac})$,
   $\texttt{suf\_cma\_corrupt}(\vn{mac})$, and
   $\texttt{suf\_cma\_corrupt\_partial}(\vn{mac})$, for use in the
   \texttt{crypto} command (see Section~\ref{sec:interact}),
   similarly to $\texttt{SUF\_CMA\_det\_mac}$.

\item $\texttt{expand SUF\_CMA\_proba\_mac\_all\_args(}\vn{mkey}$,
$  \vn{macinput}$, $\vn{macres}$, $\vn{mac\_seed}$, $\vn{mac}$, $\vn{mac\_r}$, $\vn{mac\_r}'$, $\vn{check}$, 
  $\vn{Pmac}\texttt{).}$ is similar to the above,
  with three additional arguments. 

  $\vn{mac\_seed}$ is the type of random coins for MAC, must be \texttt{bounded}.

  $\vn{mac\_r}(\vn{macinput}, \vn{mkey}, \vn{mac\_seed}): \vn{macres}$ is the MAC function that takes coins as argument (instead of generating them internally).

  $\vn{mac\_r}'$ is the symbol that replaces $\vn{mac\_r}$ after game transformation.

\item $\texttt{expand IND\_CCA2\_public\_key\_enc(}\vn{keyseed}$, $\vn{pkey}$, $\vn{skey}$,
$  \vn{cleartext}$, $\vn{ciphertext}$, $\vn{skgen}$, $\vn{pkgen}$, $\vn{enc}$, $\vn{dec}$, $\vn{injbot}$, $\vn{Z}$, $\vn{Penc}$, $\vn{Penccoll}\texttt{).}$ defines a
  IND-CCA2 (indistinguishable under adaptive chosen ciphertext attacks)
  probabilistic public-key encryption scheme.

  $\vn{keyseed}$ is the type of key seeds, must be \texttt{bounded}
  (to be able to generate random numbers from it, and to talk about
  the runtime of $\vn{pkgen}$ without mentioning the length of the key),
  typically \texttt{fixed} and \texttt{large}.

   $\vn{pkey}$ is the type of public keys, must be \texttt{bounded}.

   $\vn{skey}$ is the type of secret keys, must be \texttt{bounded}.

   $\vn{cleartext}$ is the type of cleartexts.

   $\vn{ciphertext}$ is the type of ciphertexts.

   $\vn{skgen}(\vn{keyseed}): \vn{skey}$ is the secret key generation function.

   $\vn{pkgen}(\vn{keyseed}): \vn{pkey}$ is the public key generation function.

   $\vn{enc}(\vn{cleartext}, \vn{pkey}): \vn{ciphertext}$ is the encryption function. Internally, it generates random coins, so that it is probabilistic.

   $\vn{dec}(\vn{ciphertext}, \vn{skey}): \texttt{bitstringbot}$ is the
  decryption function; it returns \texttt{bottom} when decryption
  fails.

   $\vn{injbot}(\vn{cleartext}): \texttt{bitstringbot}$ is the natural
  injection from $\vn{cleartext}$ to \texttt{bitstringbot}.

   $\vn{Z}(\vn{cleartext}): \vn{cleartext}$ is the function that
  returns for each cleartext a cleartext of the same length consisting
  only of zeroes.

   $\vn{Penc}(t, N)$ is the probability of breaking the IND-CCA2 property
   in time $t$ for one key and $N$ decryption queries.

   $\vn{Penccoll}$ is the probability of collision between independently generated keys.

   The types $\vn{keyseed}$, $\vn{pkey}$, $\vn{skey}$,
   $\vn{cleartext}$, $\vn{ciphertext}$, and the
   probabilities $\vn{Penc}$, $\vn{Penccoll}$ must be declared before
   this macro is expanded. The functions $\vn{skgen}$, $\vn{pkgen}$,
   $\vn{enc}$, $\vn{dec}$, $\vn{injbot}$, and $\vn{Z}$ are declared by
   this macro. They must not be declared elsewhere, and they can be
   used only after expanding the macro.

   This macro defines the equivalences named $\texttt{ind\_cca2}(\vn{enc})$
   and $\texttt{ind\_cca2\_partial}(\vn{enc})$
   for use in the \texttt{crypto} command 
   (see Section~\ref{sec:interact}). The equivalence $\texttt{ind\_cca2\_partial}(\vn{enc})$
   can be applied only manually and allows the user to replace the encryption 
   of a message with the encryption of zeroes for only some occurrences of
   encryption under the considered key, the ones in which the public key appears explicitly.

\item $\texttt{expand IND\_CCA2\_public\_key\_enc\_all\_args(}\vn{keyseed}$, $\vn{pkey}$, $\vn{skey}$,
$  \vn{cleartext}$, $\vn{ciphertext}$, $\vn{enc\_seed}$, $\vn{skgen}$, $\vn{skgen}'$, $\vn{pkgen}$, $\vn{pkgen}'$, $\vn{enc}$, $\vn{enc\_r}$, $\vn{enc\_r}'$, $\vn{dec}$, $\vn{dec}'$, $\vn{injbot}$, $\vn{Z}$, $\vn{Penc}$, $\vn{Penccoll}\texttt{).}$ is similar to the above,
  with six additional arguments. 

  $\vn{enc\_seed}$ is the type of random coins for encryption, must be \texttt{bounded}.

  $\vn{enc\_r}(\vn{cleartext}, \vn{pkey}, \vn{enc\_seed}): \vn{ciphertext}$ is the encryption function that takes coins as argument (instead of generating them internally).

  $\vn{pkgen}'$, $\vn{skgen}'$, $\vn{enc\_r}'$, and $\vn{dec}'$ are the symbols that replace $\vn{pkgen}$, $\vn{skgen}$, $\vn{enc\_r}$ and $\vn{dec}$ respectively after game transformation.

\item $\texttt{expand UF\_CMA\_det\_signature(}\vn{keyseed}$, $\vn{pkey}$, $\vn{skey}$,
$  \vn{signinput}$, $\vn{signature}$, $\vn{skgen}$, $\vn{pkgen}$, $\vn{sign}$, $
  \vn{check}$, $\vn{Psign}$, $\vn{Psigncoll}\texttt{).}$ defines a
  UF-CMA (unforgeable under chosen message attacks)
  deterministic signature scheme.

   $\vn{keyseed}$ is the type of key seeds, must be \texttt{bounded} (to be able to generate random numbers from it, and to talk about
  the runtime of $\vn{pkgen}$ without mentioning the length of the key), typically \texttt{fixed} and \texttt{large}.

   $\vn{pkey}$ is the type of public keys, must be \texttt{bounded}.

   $\vn{skey}$ is the type of secret keys, must be \texttt{bounded}.

  $\vn{signinput}$ is the type of signature inputs.

   $\vn{signature}$ is the type of signatures.

   $\vn{skgen}(\vn{keyseed}): \vn{skey}$ is the secret key generation function.

   $\vn{pkgen}(\vn{keyseed}): \vn{pkey}$ is the public key generation function.

   $\vn{sign}(\vn{signinput}, \vn{skey}): \vn{signature}$ is the signature function.

   $\vn{check}(\vn{signinput}, \vn{pkey}, \vn{signature}): \texttt{bool}$ is the
  verification function.

   $\vn{Psign}(t, N, l)$ is the probability of breaking the UF-CMA property
   in time $t$, for one key, $N$ signature queries with messages of length
   at most $l$.

   $\vn{Psigncoll}$ is the probability of collision between independently generated keys.

   The types $\vn{keyseed}$, $\vn{pkey}$, $\vn{skey}$, $\vn{signinput}$,
   $\vn{signature}$ and the probabilities $\vn{Psign}$, $\vn{Psigncoll}$ must
   be declared before this macro is expanded. The functions
   $\vn{skgen}$, $\vn{pkgen}$, $\vn{sign}$, and $\vn{check}$ are declared by this
   macro. They must not be declared elsewhere, and they can be used
   only after expanding the macro.

   This macro defines the equivalences named
   $\texttt{uf\_cma}(\vn{sign})$,
   $\texttt{uf\_cma\_corrupt}(\vn{sign})$, and
   $\texttt{uf\_cma\_corrupt\_partial}(\vn{sign})$, for use in the
   \texttt{crypto} command (see Section~\ref{sec:interact}).
   All three equivalences correspond to the UF-CMA property, but the first one 
   does not allow corruption of the secret keys while last two allow it.
   The last two equivalences are applied only manually, in particular because their automatic
   application can sometimes be done too early, when other transformations
   should first be done in order to eliminate uses of the secret keys.
   The equivalence $\texttt{uf\_cma\_corrupt\_partial}(\vn{sign})$ allows the
   user to transform only some occurrences of the signature verification into a 
   lookup in the signed messages, the ones in which the public key appears explicitly.  

\item $\texttt{expand UF\_CMA\_det\_signature\_all\_args(}\vn{keyseed}$, $\vn{pkey}$, $\vn{skey}$,
$  \vn{signinput}$, $\vn{signature}$, $\vn{skgen}$, $\vn{skgen}'$, $\vn{pkgen}$, $\vn{pkgen}'$, $\vn{sign}$, $\vn{sign}'$, $
  \vn{check}$, $\vn{check}'$, $\vn{Psign}$, $\vn{Psigncoll}\texttt{).}$ is similar to the above with four additional arguments.

  $\vn{pkgen}'$, $\vn{skgen}'$, $\vn{sign}'$, and $\vn{check}'$ are the symbols that replace $\vn{pkgen}$, $\vn{skgen}$, $\vn{sign}$ and $\vn{check}$ respectively after game transformation.

\item $\texttt{expand SUF\_CMA\_det\_signature(}\vn{keyseed}$, $\vn{pkey}$, $\vn{skey}$,
$  \vn{signinput}$, $\vn{signature}$, $\vn{skgen}$, $\vn{pkgen}$, $\vn{sign}$, $
  \vn{check}$, $\vn{Psign}$, $\vn{Psigncoll}\texttt{).}$ defines a
  SUF-CMA (strongly unforgeable under chosen message attacks)
  deterministic signature scheme.
  The difference between a UF-CMA signature and a SUF-CMA MAsignature
  is that, for a UF-CMA signature, the adversary may easily forge a
  new signature for a message for which he has already seen a
  signature. Such a forgery is guaranteed to be hard for a SUF-CMA
  signature. The arguments are the same as for $\texttt{UF\_CMA\_det\_signature}$.
   This macro defines the equivalences named
   $\texttt{suf\_cma}(\vn{sign})$,
   $\texttt{suf\_cma\_corrupt}(\vn{sign})$, and
   $\texttt{suf\_cma\_corrupt\_partial}(\vn{sign})$, for use in the
   \texttt{crypto} command (see Section~\ref{sec:interact}).

\item $\texttt{expand SUF\_CMA\_det\_signature\_all\_args(}\vn{keyseed}$, $\vn{pkey}$, $\vn{skey}$,
$  \vn{signinput}$, $\vn{signature}$, $\vn{skgen}$, $\vn{skgen}'$, $\vn{pkgen}$, $\vn{pkgen}'$, $\vn{sign}$, $\vn{sign}'$, $\vn{check}$, $\vn{check}'$, $\vn{Psign}$, $\vn{Psigncoll}\texttt{).}$ is similar to the above with four additional arguments.

  $\vn{pkgen}'$, $\vn{skgen}'$, $\vn{sign}'$, and $\vn{check}'$ are the symbols that replace $\vn{pkgen}$, $\vn{skgen}$, $\vn{sign}$ and $\vn{check}$ respectively after game transformation.

\item $\texttt{expand UF\_CMA\_proba\_signature(}\vn{keyseed}$, $\vn{pkey}$, $\vn{skey}$,
$  \vn{signinput}$, $\vn{signature}$, $\vn{skgen}$, $\vn{pkgen}$, $\vn{sign}$, $
  \vn{check}$, $\vn{Psign}$, $\vn{Psigncoll}\texttt{).}$ defines a
  UF-CMA (strongly unforgeable under chosen message attacks)
  probabilistic signature scheme.
  The arguments are the same as for $\texttt{UF\_CMA\_det\_signature}$,
  but the signature function internally generated random coins,
  so that it is probabilistic.
   This macro defines the equivalences named
   $\texttt{uf\_cma}(\vn{sign})$,
   $\texttt{uf\_cma\_corrupt}(\vn{sign})$, and
   $\texttt{uf\_cma\_corrupt\_partial}(\vn{sign})$, for use in the
   \texttt{crypto} command (see Section~\ref{sec:interact}).

\item $\texttt{expand UF\_CMA\_proba\_signature\_all\_args(}\vn{keyseed}$, $\vn{pkey}$, $\vn{skey}$,
$  \vn{signinput}$, $\vn{signature}$, $\vn{sign\_seed}$, $\vn{skgen}$, $\vn{skgen}'$, $\vn{pkgen}$, $\vn{pkgen}'$, $\vn{sign}$, $\vn{sign\_r}$, $\vn{sign\_r}'$, $\vn{check}$, $\vn{check}'$, $\vn{Psign}$, $\vn{Psigncoll}\texttt{).}$ is similar to the above,
  with six additional arguments. 

  $\vn{sign\_seed}$ is the type of random coins for signature, must be \texttt{bounded}.

  $\vn{sign\_r}(\vn{signinput}, \vn{skey}, \vn{sign\_seed}): \vn{signature}$ is the signature function that takes coins as argument (instead of generating them internally).

  $\vn{pkgen}'$, $\vn{skgen}'$, $\vn{sign\_r}'$, and $\vn{check}'$ are the symbols that replace $\vn{pkgen}$, $\vn{skgen}$, $\vn{sign\_r}$ and $\vn{check}$ respectively after game transformation.

\item $\texttt{expand SUF\_CMA\_proba\_signature(}\vn{keyseed}$, $\vn{pkey}$, $\vn{skey}$,
$  \vn{signinput}$, $\vn{signature}$, $\vn{skgen}$, $\vn{pkgen}$, $\vn{sign}$, $
  \vn{check}$, $\vn{Psign}$, $\vn{Psigncoll}\texttt{).}$ defines a
  SUF-CMA (strongly unforgeable under chosen message attacks)
  probabilistic signature scheme.
  The arguments are the same as for $\texttt{UF\_CMA\_det\_signature}$,
  but the signature function internally generated random coins,
  so that it is probabilistic.
   This macro defines the equivalences named
   $\texttt{suf\_cma}(\vn{sign})$,
   $\texttt{suf\_cma\_corrupt}(\vn{sign})$, and
   $\texttt{suf\_cma\_corrupt\_partial}(\vn{sign})$, for use in the
   \texttt{crypto} command (see Section~\ref{sec:interact}).

\item $\texttt{expand SUF\_CMA\_proba\_signature\_all\_args(}\vn{keyseed}$, $\vn{pkey}$, $\vn{skey}$,
  $\vn{signinput}$, $\vn{signature}$, $\vn{sign\_seed}$, $\vn{skgen}$, $\vn{pkgen}$, $\vn{sign}$, $\vn{sign\_r}$, $
  \vn{check}$, $\vn{Psign}$, $\vn{Psigncoll}\texttt{).}$ is similar to the above,
  with six additional arguments. 

  $\vn{sign\_seed}$ is the type of random coins for signature, must be \texttt{bounded}.

  $\vn{sign\_r}(\vn{signinput}, \vn{skey}, \vn{sign\_seed}): \vn{signature}$ is the signature function that takes coins as argument (instead of generating them internally).

  $\vn{pkgen}'$, $\vn{skgen}'$, $\vn{sign\_r}'$, and $\vn{check}'$ are the symbols that replace $\vn{pkgen}$, $\vn{skgen}$, $\vn{sign\_r}$ and $\vn{check}$ respectively after game transformation.

\item $\texttt{expand ROM\_hash(}\vn{key}, \vn{hashinput}, \vn{hashoutput}, \vn{hash}, \vn{hashoracle}, \vn{qH}\texttt{).}$
defines a hash function in the random oracle model.

$\vn{key}$ is the type of the key of the hash function, which models
the choice of the hash function, must be \texttt{bounded}, typically 
\texttt{fixed}.

   $\vn{hashinput}$ is the type of the input of the hash function.

   $\vn{hashoutput}$ is the type of the output of the hash function, must be \texttt{bounded} or \texttt{nonuniform} (typically \texttt{fixed}), and \texttt{large}.

   $\vn{hash}(\vn{key}, \vn{hashinput}): \vn{hashoutput}$ is the hash function.

    $\vn{hashoracle}$ is a process that allows the adversary to call the hash function.
    WARNING: The key must be generated once and for all at the beginning of the game 
   and the hash oracle must be made available to the adversary,
    by including $\vn{hashoracle}(\vn{hk})$ in the executed process,
    where $\vn{hk}$ is the key.

    $\vn{qH}$ is the number of queries to the hash oracle.

    The types $\vn{key}$, $\vn{hashinput}$, and $\vn{hashoutput}$ must
    be declared before this macro.  The function $\vn{hash}$, the
    process $\vn{hashoracle}$, and the parameter $\vn{qH}$ are defined
    by this macro. They must not be declared elsewhere, and they can
    be used only after expanding the macro.

   This macro defines the equivalence named $\texttt{rom}(\vn{hash})$
   for use in the \texttt{crypto} command 
   (see Section~\ref{sec:interact}).

 \item Similarly, for $N$ from 1 to 10, the macros\\
   $\texttt{expand ROM\_hash\_$N$(}\vn{key}, \vn{hashinput1}, \dots, \vn{hashinput}N, \vn{hashoutput}, \vn{hash}, \vn{hashoracle}, \vn{qH}\texttt{).}$\\
   define random oracles with $N$ arguments, similarly to \texttt{ROM\_hash} above.
   $\vn{hashinput1}$, \dots, $\vn{hashinput}N$ are the types of the inputs of the hash function and
   $\vn{hash}(\vn{key}$, $\vn{hashinput1}$, \dots, $\vn{hashinput}N):$ $\vn{hashoutput}$ is the hash function.

\item $\texttt{expand CollisionResistant\_hash(}\vn{key}, \vn{hashinput}, \vn{hashoutput}, \vn{hash}, \vn{hashoracle}, \vn{Phash}\texttt{).}$
defines a collision-resistant hash function.

   $\vn{key}$ is the type of the key of the hash function, must be \texttt{bounded} or \texttt{nonuniform}, typically \texttt{fixed}.

   $\vn{hashinput}$ is the type of the input of the hash function.

   $\vn{hashoutput}$ is the type of the output of the hash function.

   $\vn{hash}(\vn{key}, \vn{hashinput}): \vn{hashoutput}$ is the hash function.

   $\vn{hashoracle}$ is a process that leaks the key that it receives as argument.
   WARNING: A collision resistant hash function is a keyed hash function.
   The key must be generated once and for all at the beginning of the game,
   and immediately made available to the adversary, for instance
   by including the process $\vn{hashoracle}(\vn{hk})$,
   where $\vn{hk}$ is the key.

   $\vn{Phash}$ is the probability of breaking collision resistance.

   The types $\vn{key}$, $\vn{hashinput}$, and $\vn{hashoutput}$ and
   the probability $\vn{Phash}$ must be declared before this macro.
   The function $\vn{hash}$ and the process $\vn{hashoracle}$ are
   defined by this macro. They must not be declared elsewhere, and
   they can be used only after expanding the macro.

 \item Similarly, for $N$ from 1 to 10, the macros\\
   $\texttt{expand CollisionResistant\_hash\_$N$(}\vn{key}$, $\vn{hashinput1}$, \dots, $\vn{hashinput}N$, $\vn{hashoutput}$, $\vn{hash}$, $\vn{hashoracle}$, $\vn{Phash}\texttt{).}$\\
   define collision-resistant hash functions with $N$ arguments, similarly to \texttt{CollisionResistant\_hash} above.
   $\vn{hashinput1}$, \dots, $\vn{hashinput}N$ are the types of the inputs of the hash function and
   $\vn{hash}(\vn{key}$, $\vn{hashinput1}$, \dots, $\vn{hashinput}N):$ $\vn{hashoutput}$ is the hash function.

\item $\texttt{expand OW\_trapdoor\_perm(}\vn{seed}, \vn{pkey}, \vn{skey}, \vn{D}, \vn{pkgen}, \vn{skgen}, \vn{f}, \vn{invf}, \vn{POW}\texttt{).}$ defines a one-way trapdoor permutation.

   $\vn{seed}$ is the type of key seeds, must be \texttt{bounded} (to be able to generate random numbers from it, and to talk about
  the runtime of $\vn{pkgen}$ without mentioning the length of the key), typically \texttt{fixed} and \texttt{large}.

   $\vn{pkey}$ is the type of public keys, must be \texttt{bounded}.

   $\vn{skey}$ is the type of secret keys, must be \texttt{bounded}.

   $\vn{D}$ is the type of the input and output of the permutation, must be \texttt{bounded}, typically \texttt{fixed}.

   $\vn{pkgen}(\vn{seed}): \vn{pkey}$ is the public key generation function.

   $\vn{skgen}(\vn{seed}): \vn{skey}$ is the secret key generation function.

   $\vn{f}(\vn{pkey}, \vn{D}):\vn{D}$ is the permutation (taking as argument the public key)

   $\vn{invf}(\vn{skey}, \vn{D}):\vn{D}$ is the inverse permutation of f (taking as argument the secret key,
         i.e. the trapdoor)

   $\vn{POW}(t)$ is the probability of breaking the one-wayness property
   in time $t$, for one key and one permuted value.

   The types $\vn{seed}$, $\vn{pkey}$, $\vn{skey}$, $\vn{D}$, and the probability $\vn{POW}$ must be
   declared before this macro. The functions $\vn{pkgen}$, $\vn{skgen}$, $\vn{f}$, $\vn{invf}$
   are defined by this macro. They must not be declared elsewhere, and
   they can be used only after expanding the macro. 

   This macro defines the equivalences $\texttt{remove\_invf}(f)$,
   which expresses that, for $y$ chosen randomly in $D$, $y$ and
   $\vn{invf}(\vn{skey}, y)$ are distributed like for $x$ chosen
   randomly in $D$, $\vn{f}(\vn{pkey}, x)$ and $x$, and
   $\texttt{ow}(f)$, which corresponds to one-wayness, for use in the
   \texttt{crypto} command (see Section~\ref{sec:interact}).

\item $\texttt{expand OW\_trapdoor\_perm\_RSR(}\vn{seed}, \vn{pkey}, \vn{skey}, \vn{D}, \vn{pkgen}, \vn{skgen}, \vn{f}, \vn{invf}, \vn{POW}\texttt{).}$ defines a one-way trapdoor permutation, with random self-reducibility. The arguments are the same as for $\texttt{OW\_trapdoor\_perm}$, but the probability of breaking one-wayness is bounded more precisely. This macro defines the equivalences $\texttt{remove\_invf}(f)$ as above and $\texttt{ow\_rsr}(f)$.

\item $\texttt{expand set\_PD\_OW\_trapdoor\_perm(}\vn{seed}$, $\vn{pkey}$, $\vn{skey}$, $\vn{D}$, $\vn{Dow}$, $\vn{Dr}$, $\vn{pkgen}$, $\vn{skgen}$, $\vn{f}$, $\vn{invf}$, $\vn{concat}$, $\vn{P\_PD\_OW}\texttt{).}$ defines a set partial-domain one-way trapdoor permutation.

   $\vn{seed}$ is the type of key seeds, must be \texttt{bounded} (to be able to generate random numbers from it, and to talk about
  the runtime of $\vn{pkgen}$ without mentioning the length of the key), typically \texttt{fixed} and \texttt{large}.

   $\vn{pkey}$ is the type of public keys, must be \texttt{bounded}.

   $\vn{skey}$ is the type of secret keys, must be \texttt{bounded}.

   $\vn{D}$ is the type of the input and output of the permutation, must be \texttt{bounded}, typically \texttt{fixed}.
   The domain $\vn{D}$ consists of the concatenation of bitstrings in $\vn{Dow}$ and $\vn{Dr}$.
   $\vn{Dow}$ is the set of sub-bitstrings of $\vn{D}$ on which one-wayness holds (it is difficult to compute the
   random element $x$ of $\vn{Dow}$ knowing $f(\vn{pk}, \vn{concat}(x,y))$ where $y$ is a random element of $\vn{Dr}$).
   $\vn{Dow}$ and $\vn{Dr}$ must be \texttt{bounded}, typically \texttt{fixed}.
  
   $\vn{pkgen}(\vn{seed}): \vn{pkey}$ is the public key generation function.

   $\vn{skgen}(\vn{seed}): \vn{skey}$ is the secret key generation function.

   $\vn{f}(\vn{pkey}, \vn{D}):\vn{D}$ is the permutation (taking as argument the public key)

   $\vn{invf}(\vn{skey}, \vn{D}):\vn{D}$ is the inverse permutation of f (taking as argument the secret key,
         i.e. the trapdoor)

   $\vn{concat}(\vn{Dow}, \vn{Dr}):\vn{D}$ is bitstring concatenation.

   $\vn{P\_PD\_OW}(t,l)$ is the probability of breaking the set partial-domain one-wayness property
   in time $t$, for one key, one permuted value, and $l$ tries.

   The types $\vn{seed}$, $\vn{pkey}$, $\vn{skey}$, $\vn{D}$, $\vn{Dow}$, $\vn{Dr}$ 
   and the probability $\vn{P\_PD\_OW}$ must be
   declared before this macro. The functions $\vn{pkgen}$, $\vn{skgen}$, $\vn{f}$, $\vn{invf}$, $\vn{concat}$
   are defined by this macro. They must not be declared elsewhere, and
   they can be used only after expanding the macro. 

   This macro defines the equivalences $\texttt{remove\_invf}(f)$,
   which expresses that, for $y$ chosen randomly in $D$, $y$ and
   $\vn{invf}(\vn{skey}, y)$ are distributed like for $x$ chosen
   randomly in $D$, $\vn{f}(\vn{pkey}, x)$ and $x$, and
   $\texttt{pd\_ow}(f)$, which corresponds to set partial-domain one-wayness, for use in the
   \texttt{crypto} command (see Section~\ref{sec:interact}).

\item $\texttt{expand OW\_trapdoor\_perm\_all\_args(}\vn{seed}, \vn{pkey}, \vn{skey}, \vn{D}, \vn{pkgen}, \vn{pkgen}', \vn{skgen}, \vn{f}, \vn{f}', \vn{invf}, \vn{POW}\texttt{).}$\\
$\texttt{expand OW\_trapdoor\_perm\_RSR\_all\_args(}\vn{seed}, \vn{pkey}, \vn{skey}, \vn{D}, \vn{pkgen}, \vn{pkgen}', \vn{skgen}, \vn{f}, \vn{f}', \vn{invf}, \vn{POW}\texttt{).}$\\
$\texttt{expand set\_PD\_OW\_trapdoor\_perm\_all\_args(}\vn{seed}$, $\vn{pkey}$, $\vn{skey}$, $\vn{D}$, $\vn{Dow}$, $\vn{Dr}$, $\vn{pkgen}$, $\vn{pkgen}'$, $\vn{skgen}$, $\vn{f}$, $\vn{f}'$, $\vn{invf}$, $\vn{concat}$, $\vn{P\_PD\_OW}\texttt{).}$ are similar to $\texttt{OW\_trapdoor\_perm}$, $\texttt{OW\_trapdoor\_perm\_RSR}$, and
$\texttt{set\_PD\_OW\_trapdoor\_perm\_all\_args}$ respectively, with two additional arguments.

$\vn{pkgen}'$ and $\vn{f}'$ are the symbols that replace $\vn{pkgen}$ and $\vn{f}$ respectively after game transformation.

\item $\texttt{expand PRF(}\vn{key}, \vn{input}, \vn{output}, \vn{f}, \vn{Pprf}\texttt{).}$ 
defines a pseudo-random function.

   $\vn{key}$ is the type of keys, must be \texttt{bounded} (to be able to generate random numbers from it, and to talk about the runtime of $\vn{f}$ without mentioned the length of the key), typically \texttt{fixed} and \texttt{large}.

   $\vn{input}$ is the type of the input of the PRF.

   $\vn{output}$ is the type of the output of the PRF, must be \texttt{bounded}, typically \texttt{fixed}.

   $\vn{f}(\vn{key}, \vn{input}):\vn{output}$ is the PRF function.

   $\vn{Pprf}(t, N, l)$ is the probability of breaking the PRF property
   in time $t$, for one key, $N$ queries to the PRF of length at most $l$.

   The types $\vn{key}$, $\vn{input}$, $\vn{output}$
   and the probability $\vn{Pprf}$ must be declared before this macro
   is expanded. The function $\vn{f}$ is declared by
   this macro. It must not be declared elsewhere, and it can be
   used only after expanding the macro.

   This macro defines the equivalence named $\texttt{prf}(\vn{f})$
   for use in the \texttt{crypto} command 
   (see Section~\ref{sec:interact}).

 \item Similarly, for $N$ from 1 to 10, the macros\\
$\texttt{expand PRF\_$N$(}\vn{key}, \vn{input1}, \dots, \vn{input}N, \vn{output}, \vn{f}, \vn{Pprf}\texttt{).}$ \\
define pseudo-random functions with $N$ arguments, similarly to \texttt{PRF} above.
$\vn{input1}$, \dots, $\vn{input}N$ are the types of the inputs of the PRF and
$\vn{f}(\vn{key}, \vn{input1}, \dots, \vn{input}N):\vn{output}$ is the PRF.

\item $\texttt{expand PRF\_large(}\vn{key}, \vn{input}, \vn{output}, \vn{f}, \vn{Pprf}\texttt{).}$ 
defines a pseudo-random function with a large output, that is, it optimizes the 
definition by eliminating collisions between random output elements.
Its interface is the same as the one of \texttt{PRF} above.

 \item Similarly, for $N$ from 1 to 10, the macros\\
$\texttt{expand PRF\_large\_$N$(}\vn{key}, \vn{input1}, \dots, \vn{input}N, \vn{output}, \vn{f}, \vn{Pprf}\texttt{).}$ \\
define pseudo-random functions with $N$ arguments and a large output, 
similarly to \texttt{PRF\_large} above.
$\vn{input1}$, \dots, $\vn{input}N$ are the types of the inputs of the PRF and
$\vn{f}(\vn{key}, \vn{input1}, \dots, \vn{input}N):\vn{output}$ is the PRF.

 \item The specification of Diffie-Hellman key agreements is typically composed of two or three macro expansions:

   \begin{itemize}
   \item One from the following set of macros, which defines properties of group:
     \begin{itemize}
     \item $\texttt{expand\ DH\_basic(}G, Z, g, \vn{exp}, \vn{exp}', \vn{mult}\texttt{).}$ defines a group $G$.

       $G$: type of group elements (must be \texttt{bounded} and \texttt{large}).

       $Z$: type of exponents (must be \texttt{bounded} and
       \texttt{large}). 
       
       $g$: an element of the group $G$.

       $\vn{exp}(G, Z): G$: the exponentiation function.  
       
       $\vn{exp}'(G, Z): G$: symbol used to replace $\vn{exp}$ after game transformations.

       $\vn{mult}(Z, Z): Z$: the multiplication function for exponents, commutative.

       The equation $\vn{exp}(\vn{exp}(a,x), y) = \vn{exp}(a,
       \vn{mult}(x,y))$ must be satisfied.

       The private Diffie-Hellman keys are generated by choosing an
       element randomly in $Z$, according to its default distribution
       (which is not necessarily uniform). The public Diffie-Hellman
       keys are generated as $X = \vn{exp}(g,x)$, where $x$ is a
       private Diffie-Hellman key, and similarly $Y =
       \vn{exp}(g,y)$. The Diffie-Hellman shared secret is
       $\vn{exp}(X,y) = \vn{exp}(Y,x) = \vn{exp}(g,\vn{mult}(x,y))$.
       This macro makes no other assumption. In particular, it allows
       $G$ to contain elements other than those generated by $g$.

       The types $G$ and $Z$ must be declared before this macro.  The
       functions $g$, $\vn{exp}$, and $\vn{mult}$ are defined by this
       macro. They must not be declared elsewhere, and they can be used
       only after expanding the macro.

     \item $\texttt{expand\ DH\_proba\_collision(}G, Z, g, \vn{exp},
       \vn{exp}', \vn{mult}, \vn{PCollKey1},
       \vn{PCollKey2}\texttt{).}$ defines a group $G$ like
       \texttt{DH\_basic}, with the following additional properties:
       the probability that $\vn{exp}(g, x) = Y$ with random $x$ and
       with $Y$ independent of $x$ is at most $\vn{PCollKey1}$, and
       the probability that $\vn{exp}(g, \vn{mult}(x,y)) = Y$ with
       random $x$ and with $y$ and $Y$ independent of $x$ is at most
       $\vn{PCollKey2}$. These probabilities are negligible in most
       Diffie-Hellman groups, but need to be evaluated more precisely
       for using this property.
       
       The types $G$ and $Z$ and the probabilities $\vn{PCollKey1}$
       and $\vn{PCollKey2}$ must be declared before this macro.  The
       functions $g$, $\vn{exp}$, and $\vn{mult}$ are defined by this
       macro. They must not be declared elsewhere, and they can be
       used only after expanding the macro.

     \item $\texttt{expand\ square\_DH\_proba\_collision(}G, Z, g, \vn{exp},
       \vn{exp}', \vn{mult}, \vn{PCollKey1},
       \vn{PCollKey2}$, $\vn{PCollKey3}\texttt{).}$ is similar to \texttt{DH\_proba\_collision}, but additionally
   says that the probability that $\vn{exp}(g, \vn{mult}(x,x)) = Y$ with
   random $x$ and with $Y$ independent of $x$ is at most $\vn{PCollKey3}$, with $\vn{PCollKey3} \geq \vn{PCollKey2}$.
       
       The types $G$ and $Z$ and the probabilities $\vn{PCollKey1}$, $\vn{PCollKey2}$, 
       and $\vn{PCollKey3}$ must be declared before this macro.  The
       functions $g$, $\vn{exp}$, and $\vn{mult}$ are defined by this
       macro. They must not be declared elsewhere, and they can be
       used only after expanding the macro.

     \item $\texttt{expand\ DH\_good\_group(}G, Z, g, \vn{exp}, \vn{exp}', \vn{mult}\texttt{).}$ defines a group $G$ like
       \texttt{DH\_basic}, with the following additional properties:
       $G$ is a group of prime order $q$, with the neutral element excluded, 
       the set of exponents $Z$ is $\{1, \dots, q-1\}$,
       $g$ is a generator of $G$, 
       $\vn{mult}$ is the product modulo $q$ in $\{1, \dots, q-1\}$, i.e. in the group $(\mathbb{Z}/q\mathbb{Z})*$,
       the distributions of random choices in $Z$ and $G$ are uniform.

       It may not be obvious when an element is received on the network
       whether it really belongs to the group $G$ generated by $g$. That should 
       be checked for the properties assumed in this macro to hold.

       This macro defines the following equivalences for use in the
       \texttt{crypto} command (see Section~\ref{sec:interact}):
       \begin{itemize}
       \item $\texttt{group\_to\_exp\_strict}(\vn{exp})$ which allows to replace
         a random $X \in G$ with $\vn{exp}(g,x)$ for a random $x \in Z$, provided
         $\vn{exp}(X,\_)$ occurs in the game.
       \item $\texttt{group\_to\_exp}(\vn{exp})$ which allows to replace
         a random $X \in G$ with $\vn{exp}(g,x)$ for a random $x \in Z$ in any case.
         (This transformation is applied only manually.)
       \item $\texttt{exp\_to\_group}(\vn{exp})$ which allows to replace
         $\vn{exp}(g,x)$ for a random $x \in Z$ with a random $X \in G$.
       \item $\texttt{exp'\_to\_group}(\vn{exp})$ which allows to replace
         $\vn{exp'}(g,x)$ for a random $x \in Z$ with a random $X \in G$.
       \end{itemize}

\newcommand{\F}{\mathbb{F}}%
\newcommand{\red}{\mathrm{red}}%
\newcommand{\repr}{\mathrm{repr}}%
\newcommand{\modop}{\mathbin{\mathrm{mod}}}%

     \item $\texttt{expand\ DH\_single\_coord\_ladder}(G$, $Z$, $g$, $\vn{exp}$, $\vn{mult}$, $\vn{subG}$, $\vn{Znw}$, $\vn{ZnwtoZ}$, $\vn{g\_k}$, $\vn{exp\_div\_k}$, $\vn{exp\_div\_k}'$, $\vn{pow\_k}$, $\vn{subGtoG}$, $\vn{zero}$, $\vn{sub\_zero}\texttt{).}$ models an elliptic curve defined by the equation
   $Y^2 = X^3 + A X^2 + X$ in the field $\F_p$ of non-zero integers modulo the 
   large prime $p$, where $A^2 - 4$ is not a square modulo $p$.
   This curve must form a commutative group of order $kq$ where $k$ is a 
   small integer and $q$ is a large prime.
   Its quadratic twist must form a commutative group of order $k'q'$ where $k'$
   is a small integer and $q'$ is a large prime.
   $k$ must be a multiple of $k'$.
   We must use a single coordinate ladder defined as follows: we
   consider the elliptic curve $E(\F_{p^2})$ defined by the equation $Y^2 =
   X^3 + A X^2 + X$ in a quadratic extension $\F_{p^2}$ of $\F_p$, we define 
   $X_0 : E(\F_{p^2}) \rightarrow \F_{p^2}$ by $X_0(\infty) = 0$ and $X_0(X,Y) = X$, and 
   for $X \in \F_p$ and $y$ an integer, we define $X^y \in \F_p$ as $X^y = X_0(yQ)$
   for all $Q \in E(\F_{p^2})$ such that $X_0(Q) = X$.
   The value $g = X_0(g_0)$ represents the base point $g_0$, which must have order $q$.
   The public keys (bitstrings) are mapped to elements of $\F_p$ by the function 
   $\red$ and conversely, elements of $\F_p$ are mapped to public keys by
   the function $\repr$, such that $\red \circ \repr$ is the identity.
   The Diffie-Hellman ``exponentiation'' is defined by 
      \[\exp(X,y) = \repr((\red(X))^y)\]
   The secret keys are chosen uniformly in $\{ kn \mid n \in [n_{min},n_{max}] \}$
   where $n_{min} < n_{max}$, $n_{max} - n_{min} < q$ and $n_{max} - n_{min} < q'$.
   Therefore the set of secret keys may contain a multiple of $q$ (resp. $q'$).
   Such keys are weak, in the sense that they yield 0 for all public
   keys on the curve (resp. on the twist). We exclude them as a first step
   in the proof, by applying the equivalence $\texttt{exclude\_weak\_keys}(Z)$
   defined by this macro, automatically or with the
   \texttt{crypto} command (see Section~\ref{sec:interact}).

This model is justified in~\cite{LippBlanchetBharagavanInria19}.

       $G$: type of public keys (must be \texttt{bounded} and \texttt{large}).

       $\vn{subG}$: type of $\{ X^k \mid X \in F_p \}$  (must be \texttt{bounded} and \texttt{large}). 

       $Z$, $\vn{Znw}$: type of exponents (must be \texttt{bounded}, \texttt{nonuniform}, and \texttt{large}). 
   $\vn{Znw}$ is the set of integers multiple of $k$, prime to $qq'$ modulo $kqq'$, that is, exponents without weak keys.
     Random choices in $\vn{Znw}$ are done by choosing uniformly in 
     $\{ kn \mid n \in [n_{min},n_{max}], n$ prime to $qq' \}$.
       $Z$ is the set of integers multiple of $k$ modulo $kqq'$, that is, exponents with weak keys. Random choices in $Z$ are done by choosing uniformly in 
   $\{ kn \mid n \in [n_{min},n_{max}] \}$, hence $\texttt{Pcoll1rand}(Z) = 1/(n_{max}-n_{min}+1)$. 
       
       $\vn{ZnwtoZ}(\vn{Znw}):Z$: injection from $\vn{Znw}$ to $Z$.

       $g: G$: represents the base point.

       $\vn{exp}(G, Z): G$: the exponentiation function.  
       
       $\vn{mult}(\vn{Znw}, \vn{Znw}): \vn{Znw}$: the multiplication function for exponents, defined as
       $\vn{mult}(x,y) = x.y \mod kqq'$. (It remains in $\vn{Znw}$.)

       $\vn{g\_k} = \red(g)^k$. It is an element of $\vn{subG}$.

       $\vn{exp\_div\_k}(\vn{subG},\vn{Znw}): \vn{subG}$ is defined by $\vn{exp\_div\_k}(X,y) = X^{y/k}$.

       $\vn{exp\_div\_k}'$: symbol that replaces $\vn{exp\_div\_k}$ after game transformation,
       with the same definition as $\vn{exp\_div\_k}$.

       $\vn{pow\_k}(G):\vn{subG}$, defined by $\vn{pow\_k}(x) = \red(x)^k$.

       $\vn{subGtoG}(\vn{subG}):G$ is $\repr$ restricted to $\vn{subG}$.

       $\vn{zero}: G$ is the public key 0.
       
       $\vn{sub\_zero}: \vn{subG}$ is 0, as an element of $\vn{subG}$.

       The types $G$, $\vn{subG}$, $Z$, and $\vn{Znw}$ must be declared before this macro.  The
       functions $g$, $\vn{exp}$, $\vn{mult}$, $\vn{ZnwtoZ}$, $\vn{g\_k}$, $\vn{exp\_div\_k}$, $\vn{exp\_div\_k}'$, $\vn{pow\_k}$, $\vn{subGtoG}$,
       $\vn{zero}$, $\vn{sub\_zero}$ are defined by this macro. They must not be declared
       elsewhere, and they can be used only after expanding the macro.

       When this macro is used, the Diffie-Hellman assumptions (detailed below)
       should be applied to the subgroup, that is,
       $\texttt{expand\ }\vn{assumption}\texttt{(}\vn{subG}$, $\vn{Znw}$, $\vn{g\_k}$, $\vn{exp\_div\_k}$, $\vn{exp\_div\_k}'$, $\vn{mult}$, $\dots\texttt{).}$


     \item $\texttt{expand\ DH\_X25519}(G$, $Z$, $g$, $\vn{exp}$, $\vn{mult}$, $\vn{subG}$, $\vn{g\_k}$, $\vn{exp\_div\_k}$, $\vn{exp\_div\_k}'$, $\vn{pow\_k}$, $\vn{subGtoG}$, $\vn{zero}$, $\vn{sub\_zero}\texttt{).}$ models Curve25519 as defined in RFC 7748 (\url{https://tools.ietf.org/html/rfc7748}). It is justified in detail in~\cite{LippBlanchetBharagavanInria19}.
   More generally, it supports the same curves as \texttt{DH\_single\_coord\_ladder}
   with the additional assumption that all secret keys are prime to $qq'$. 
   Therefore, we do not need to exclude weak secret keys, so the
   parameters $\vn{Znw}$ and $\vn{ZnwtoZ}$ are removed, and we use $Z$ instead of $\vn{Znw}$.

   Curve25519 satisfies these assumptions with
   $p = 2^{255}-19$,
   $k = 8$, $k' = 4$, $q = 2^{252} + \delta$ with $0 < \delta < 2^{128}$,
      $q' = 2^{253} - 9 - 2\delta$,
    $\red(X) = (X \modop 2^{255})\modop p$, $\repr(X)$ is the representation 
      of $X$ as an element of $\{0, \dots, p-1\}$,
    $n_{min} = 2^{251}$, and $n_{max} = 2^{252}-1$, so $\texttt{Pcoll1rand}(Z) = 2^{-251}$.
       (For simple examples that use Curve25519, using the macro 
       \texttt{DH\_proba\_collision} may also work.)

     \item $\texttt{expand\ DH\_X448}(G$, $Z$, $g$, $\vn{exp}$, $\vn{mult}$, $\vn{subG}$, $\vn{Znw}$, $\vn{ZnwtoZ}$, $\vn{g\_k}$, $\vn{exp\_div\_k}$, $\vn{exp\_div\_k}'$, $\vn{pow\_k}$, $\vn{subGtoG}$, $\vn{zero}$, $\vn{sub\_zero}\texttt{).}$ models Curve448 as defined in RFC 7748
(\url{https://tools.ietf.org/html/rfc7748}).
   More generally, it supports the same curves as \texttt{DH\_single\_coord\_ladder}
   with the additional assumptions that there is at most one secret key
   multiple of $q$ or $q'$, and that $q = -1 \mod 4$, so $-1$ is not a 
   square modulo $q$. That allows to reduce some probabilities.
   This model is justified in~\cite{LippBlanchetBharagavanInria19}.

     \end{itemize}

   \item Optionally, 
     $\texttt{expand\ DH\_dist\_random\_group\_element\_vs\_exponent(}G,
     Z, g, \vn{exp}, \vn{exp}', \vn{mult}$, $\vn{PDist}\texttt{).}$ This macro says
     that the probability of distinguishing a random group element
     from an exponentiation $\vn{exp}(g,x)$ with a random exponent $x$
     is at most $\vn{PDist}$.  The other arguments are as in
     \texttt{DH\_basic} and all arguments must be defined before 
     expanding the macro.

   This macro defines the following equivalences for use in the
   \texttt{crypto} command (see Section~\ref{sec:interact}):
   \begin{itemize}
   \item $\texttt{group\_to\_exp\_strict}(\vn{exp})$ which allows to replace
     a random $X \in G$ with $\vn{exp}(g,x)$ for a random $x \in Z$, provided
     $\vn{exp}(X,\_)$ occurs in the game.
   \item $\texttt{group\_to\_exp}(\vn{exp})$ which allows to replace
     a random $X \in G$ with $\vn{exp}(g,x)$ for a random $x \in Z$ in any case.
     (This transformation is applied only manually.)
   \item $\texttt{exp\_to\_group}(\vn{exp})$ which allows to replace
     $\vn{exp}(g,x)$ for a random $x \in Z$ with a random $X \in G$.
   \item $\texttt{exp'\_to\_group}(\vn{exp})$ which allows to replace
     $\vn{exp'}(g,x)$ for a random $x \in Z$ with a random $X \in G$.
   \end{itemize}

     This macro can be used with any of the previous macros, except
     that it is useless with the macro \texttt{DH\_good\_group}, because this macro
     already includes these properties with $\vn{PDist} = 0$. When the macro 
$\texttt{DH\_single\_coord\_ladder}$, $\texttt{DH\_X25519}$, or $\texttt{DH\_X448}$ is used, this macro should be applied to the subgroup. For instance, with
$\texttt{expand\ DH\_single\_coord\_ladder}(G$, $Z$, $g$, $\vn{exp}$, $\vn{mult}$, $\vn{subG}$, $\vn{Znw}$, $\vn{ZnwtoZ}$, $\vn{g\_k}$, $\vn{exp\_div\_k}$, $\vn{exp\_div\_k}'$, $\vn{pow\_k}$, $\vn{subGtoG}$, $\vn{zero}$, $\vn{sub\_zero}\texttt{).}$, it should be
     $\texttt{expand\ DH\_dist\_random\_group\_element\_vs\_exponent(}\vn{subG}$, $\vn{Znw}$, $\vn{g\_k}$, $\vn{exp\_div\_k}$, $\vn{exp\_div\_k}'$, $\vn{mult}$, $\vn{Pdist}\texttt{).}$
     
   \item One from the following set of macros, which defines the Diffie-Hellman assumption itself:
     \begin{itemize}
     \item $\texttt{expand\ CDH(}G, Z, g, \vn{exp}, \vn{exp}', \vn{mult}, p\texttt{).}$ says that the group $G$ satisfies the computational
  Diffie-Hellman assumption; $p(t)$ is the probability of breaking the CDH assumption, for one pair of exponents, in time $t$.
   This macro defines the equivalence $\texttt{cdh}(\vn{exp})$, whichs corresponds to the CDH property, for use in the
   \texttt{crypto} command (see Section~\ref{sec:interact}).

     \item $\texttt{expand\ CDH\_RSR(}G, Z, g, \vn{exp}, \vn{exp}', \vn{mult}, p\texttt{).}$ 
       is similar to \texttt{CDH}, but uses random self reducibility. It may yield lower probabilities but requires the exponents 
       to be chosen uniformly in $(\mathbb{Z}/q\mathbb{Z})^*$ or in $\mathbb{Z}/q\mathbb{Z}$, where $q$ is the order of $g$, 
       so it is not correct for usual implementations of Curve25519 for instance. (The proof is done using uniform choices in $\mathbb{Z}/q\mathbb{Z}$; we add the probability of distinguishing these choices from choices in $(\mathbb{Z}/q\mathbb{Z})^*$ so that the result also applies to choices in $(\mathbb{Z}/q\mathbb{Z})^*$, as assumed in \texttt{DH\_good\_group} and as often done in Diffie-Hellman implementations. Indeed, 0 is a bad secret key, since with a 0 secret key, the Diffie-Hellman shared secret is always the neutral element of the group, independently of the other public key.)

     \item $\texttt{expand\ DDH(}G, Z, g, \vn{exp}, \vn{exp}', \vn{mult}, p\texttt{).}$
       says that the group $G$ satisfies the decisional Diffie-Hellman
       assumption; $p(t)$ is the probability of breaking the DDH
       assumption, for one pair of exponents, in time $t$.
   This macro defines the equivalence $\texttt{ddh}(\vn{exp})$, whichs corresponds to the DDH property, for use in the
   \texttt{crypto} command (see Section~\ref{sec:interact}).

     \item $\texttt{expand\ GDH(}G, Z, g, \vn{exp}, \vn{exp}', \vn{mult}, p\texttt{).}$
       says that the group $G$ satisfies the gap Diffie-Hellman
       assumption (GDH). The probability $p(t,n)$ is the probability of breaking
       the GDH assumption for one pair of exponents in time $t$ with at most $n$ 
       calls to the decisional Diffie-Hellman oracle. This macro defines
       the equivalence $\texttt{gdh}(\vn{exp})$, whichs corresponds to the GDH property, for use in the
       \texttt{crypto} command (see Section~\ref{sec:interact}).

     \item $\texttt{expand\ GDH\_RSR(}G, Z, g, \vn{exp}, \vn{exp}', \vn{mult}, p\texttt{).}$ 
       is similar to \texttt{GDH}, but uses random self reducibility. It may yield lower probabilities but requires the exponents 
       to be chosen uniformly in $(\mathbb{Z}/q\mathbb{Z})^*$ or in $\mathbb{Z}/q\mathbb{Z}$, where $q$ is the order of $g$, 
       so it is not correct for usual implementations of Curve25519 for instance.

     \item
       $\texttt{expand\ square\_CDH(}G, Z, g, \vn{exp}, \vn{exp}',
       \vn{mult}, p, \vn{sqp}\texttt{).}$
       says that the group $G$ satisfies the computational
       Diffie-Hellman assumption and the square computational
       Diffie-Hellman assumption; $p(t)$ is the probability of
       breaking the CDH assumption, for one pair of exponents, in time
       $t$ and $\vn{sqp}(t)$ is the probability of breaking the square
       CDH assumption, for one pair of exponents, in time $t$.  This
       macro defines the equivalence $\texttt{cdh}(\vn{exp})$, whichs
       corresponds to the (square) CDH property, for use in the
       \texttt{crypto} command (see Section~\ref{sec:interact}). When
       the group has prime order, the computational Diffie-Hellman
       assumption is equivalent to the square variant, but CryptoVerif
       can do more proofs using the square variant. (It allows
       transforming $\vn{exp}(g, \vn{mult}(x,x))$.)

     \item
       $\texttt{expand\ square\_CDH\_RSR(}G, Z, g, \vn{exp},
       \vn{exp}', \vn{mult}, \vn{sqp}\texttt{).}$
       says that the group $G$ satisfies the square computational
       Diffie-Hellman assumption; $\vn{sqp}(t)$ is the probability of
       breaking the square CDH assumption, for one pair of exponents,
       in time $t$.  This macro defines the equivalence
       $\texttt{cdh}(\vn{exp})$, whichs corresponds to the square CDH
       property, for use in the \texttt{crypto} command (see
       Section~\ref{sec:interact}). It uses random self
       reducibility. It may yield lower probabilities than
       $\texttt{square\_CDH}$ but requires the exponents to be chosen
       uniformly in $(\mathbb{Z}/q\mathbb{Z})^*$ or in $\mathbb{Z}/q\mathbb{Z}$, where $q$ is the order
       of $g$, so it is not correct for usual implementations of Curve25519 for instance.

     \item
       $\texttt{expand\ square\_DDH(}G, Z, g, \vn{exp}, \vn{exp}',
       \vn{mult}, p, \vn{sqp}\texttt{).}$
       says that the group $G$ satisfies the decisional Diffie-Hellman
       assumption and the square decisional Diffie-Hellman assumption;
       $p(t)$ is the probability of breaking the DDH assumption, for
       one pair of exponents, in time $t$ and $\vn{sqp}(t)$ is the
       probability of breaking the square DDH assumption, for one pair
       of exponents, in time $t$.  This macro defines the equivalence
       $\texttt{ddh}(\vn{exp})$, whichs corresponds to the (square)
       DDH property, for use in the \texttt{crypto} command (see
       Section~\ref{sec:interact}).

     \item
       $\texttt{expand\ square\_GDH(}G, Z, g, \vn{exp}, \vn{exp}',
       \vn{mult}, p, \vn{sqp}\texttt{).}$
       says that the group $G$ satisfies the gap Diffie-Hellman (GDH)
       assumption and the square gap Diffie-Hellman assumption;
       $p(t)$ is the probability of breaking the GDH assumption, for
       one pair of exponents, in time $t$ with at most $n$ 
       calls to the decisional Diffie-Hellman oracle and $\vn{sqp}(t)$ is the
       probability of breaking the square GDH assumption, for one pair
       of exponents, in time $t$ with at most $n$ 
       calls to the decisional Diffie-Hellman oracle.  This macro defines the equivalence
       $\texttt{gdh}(\vn{exp})$, whichs corresponds to the (square)
       GDH property, for use in the \texttt{crypto} command (see
       Section~\ref{sec:interact}).

     \item $\texttt{expand\ square\_GDH\_RSR(}G, Z, g, \vn{exp}, \vn{exp}',
       \vn{mult}, \vn{sqp}\texttt{).}$
       says that the group $G$ satisfies the square gap Diffie-Hellman assumption;
       $\vn{sqp}(t)$ is the
       probability of breaking the square GDH assumption, for one pair
       of exponents, in time $t$ with at most $n$ 
       calls to the decisional Diffie-Hellman oracle.  This macro defines the equivalence
       $\texttt{gdh}(\vn{exp})$, whichs corresponds to the square
       GDH property, for use in the \texttt{crypto} command (see
       Section~\ref{sec:interact}). It uses random self
       reducibility. It may yield lower probabilities than
       $\texttt{square\_CDH}$ but requires the exponents to be chosen
       uniformly in $(\mathbb{Z}/q\mathbb{Z})^*$ or in $\mathbb{Z}/q\mathbb{Z}$, where $q$ is the order
       of $g$, so it is not correct for usual implementations of Curve25519 for instance.

     \end{itemize}
     All arguments of these macros must be defined before expanding the macro.
     
   \end{itemize}

\iffalse
 \item Another possible set of macros for Diffie-Hellman assumptions is the following.
   \begin{itemize}
   \item $\texttt{expand\ DH\_prime\_subgroup\_secret\_keys\_not\_multiple\_k(}G$, $Z$, $g$, $\vn{exp}$, $\vn{exp}'$, $\vn{expblock}$, $\vn{mult}$, $\vn{subG}$, $\vn{g\_k}$, $\vn{expsub}$, $\vn{expsub}'$, $\vn{pow\_k}$, $\vn{PCollKey1}$, $\vn{PCollKey2}$, $\vn{PCollKey3}$, $\vn{PCollKey4}\texttt{).}$ defines a Diffie-Hellman group with the following properties.

       $G$: type of group elements (must be \texttt{bounded} and \texttt{large}).   
       $G$ is a group of cardinal $kq$, where $q$ is a large prime and $k$ is a small integer.

       $\vn{subG}$: type of subgroup elements (must be \texttt{bounded} and \texttt{large}). 
       $\vn{subG}$ is a subgroup of $G$ of cardinal $q$.

       $Z$: type of exponents (must be \texttt{bounded} and \texttt{large}). 
       $Z$ is the set of integers modulo $kq$.
       
       $g$: an element of the group $G$, which generates the subgroup $\vn{subG}$.

       $\vn{exp}(G, Z): G$: the exponentiation function.  

       $\vn{exp}'$ is a symbol that replaces $\vn{exp}$ after the GDH game transformation.

       $\vn{expblock}$ is a symbol that replaces $\vn{exp}$ after the $\texttt{use\_subgroup}(\vn{exp})$ transformation, explained below.

       $\vn{mult}(Z,Z): Z$ is the product in $Z$.

       $\vn{g\_k} = g^k$. It is an element of $\vn{subG}$, which generates $\vn{subG}$.

       $\vn{expsub}(\vn{subG},Z): \vn{subG}$ is defined by $\vn{expsub}(X,y) = X^y$.
       
       $\vn{expsub}'$ is a symbol that replaces $\vn{expsub}$ after GDH game transformation,
       defined like $\vn{expsub}$.

       $\vn{pow\_k}(G):\vn{subG}$ is defined by $\vn{pow\_k}(x) = x^k$.
   
       $\vn{PCollKey1}$ is the maximum probability that $g^{kx} = Y$ with $x$ random in $Z$ and with $Y$ independent of $x$.

       $\vn{PCollKey2}$ is the maximum probability that $g^{kxy} = Y$ with $x$ random in $Z$ and with $y$ and $Y$ independent of $x$.

       $\vn{PCollKey3}$ is the maximum of $\vn{PCollKey2}$ and 
                the maximum probability that $g^{kxx} = Y$ with $x$ random in $Z$ and with $Y$ independent of $x$.

       $\vn{PCollKey4}$ is the probability that $y = 0$ modulo $q$ for $y$ random in $Z$.


       $\texttt{use\_subgroup}(\vn{exp})$ transforms
       $\vn{exp}(X,a) = Y$ into
       \[(\vn{expblock}(X,a) = Y) \ \&\&\ (\vn{expsub}(\vn{pow\_k}(X),a) =
       \vn{pow\_k}(Y))\]
       $X^a = Y$ implies $(X^a)^k = Y^k$ that is $(X^k)^a = Y^k$, and
       in the latter equality $X^k$ and $Y^k$ are in the subgroup
       generated by $g$, so we can apply more equalities.  Since
       $X^a = Y$ is not equivalent to $(X^k)^a = Y^k$, we keep the
       original equality $X^a = Y$ as well, but replace $\vn{exp}$
       with $\vn{expblock}$ to avoid a loop. The transformation
       $\texttt{use\_subgroup}(\vn{exp})$ should be applied before
       GDH, using \texttt{crypto} command (see
       Section~\ref{sec:interact}).
       
       The types $G$, $\vn{subG}$, and $Z$ and the probabilities $\vn{PCollKey1}$, $\vn{PCollKey2}$, 
       $\vn{PCollKey3}$, $\vn{PCollKey4}$ must be declared before this macro.  The
       functions $g$, $\vn{exp}$, $\vn{exp}'$, $\vn{expblock}$, $\vn{mult}$, $\vn{g\_k}$, $\vn{expsub}$, $\vn{expsub}'$, $\vn{pow\_k}$
       are defined by this macro. They must not be declared
       elsewhere, and they can be used only after expanding the macro.

       For instance, these properties are typically satisfied by Curve25519 when
       the secret keys are not normalized (that is, they may not be multiple of 8).

   \item One of the following two macros:
     \begin{itemize}
     \item $\texttt{expand\ GDH\_subgroup(}G$, $Z$, $g$, $\vn{exp}$, $\vn{exp}'$, $\vn{expblock}$, $\vn{mult}$, $\vn{subG}$, $\vn{g\_k}$, $\vn{expsub}$, $\vn{expsub}'$, $\vn{pow\_k}$, $p\texttt{).}$ says that the group $G$ satisfies the gap Diffie-Hellman
       assumption (GDH). It applies the computational Diffie-Hellman assumption in the subgroup $\vn{subG}$, but allows decisional Diffie-Hellman oracle queries in the whole group $G$. The probability $p(t,n)$ is the probability of breaking
       the GDH assumption for one pair of exponents in time $t$ with at most $n$ 
       calls to the decisional Diffie-Hellman oracle. This macro defines
       the equivalence $\texttt{gdh}(\vn{exp})$, whichs corresponds to the GDH property, for use in the
       \texttt{crypto} command (see Section~\ref{sec:interact}).

     \item $\texttt{expand\ square\_GDH\_subgroup(}G$, $Z$, $g$, $\vn{exp}$, $\vn{exp}'$, $\vn{expblock}$, $\vn{mult}$, $\vn{subG}$, $\vn{g\_k}$, $\vn{expsub}$, $\vn{expsub}'$, $\vn{pow\_k}$, $p$, $\vn{sqp}\texttt{).}$ 
       says that the group $G$ satisfies the gap Diffie-Hellman (GDH)
       assumption and the square gap Diffie-Hellman assumption. It applies the (square) computational Diffie-Hellman assumption in the subgroup $\vn{subG}$, but allows decisional Diffie-Hellman oracle queries in the whole group $G$.
       The probability $p(t)$ is the probability of breaking the GDH assumption, for
       one pair of exponents, in time $t$ with at most $n$ 
       calls to the decisional Diffie-Hellman oracle and $\vn{sqp}(t)$ is the
       probability of breaking the square GDH assumption, for one pair
       of exponents, in time $t$ with at most $n$ 
       calls to the decisional Diffie-Hellman oracle.  This macro defines the equivalence
       $\texttt{gdh}(\vn{exp})$, whichs corresponds to the (square)
       GDH property, for use in the \texttt{crypto} command (see
       Section~\ref{sec:interact}).
     \end{itemize}
     We do not define other properties in this case, because we
     believe that for simpler examples that require only CDH or DDH,
     it should be possible to use \texttt{DH\_proba\_collision}
     together with CDH or DDH, which also applies to this kind of
     groups.

   \end{itemize}
\fi  
\item $\texttt{expand\ Xor(}D, \vn{xor}, \vn{zero}\texttt{).}$ defines the
  function symbol $\vn{xor}$ to be exclusive or on the set of
  bitstrings $D$, where $\vn{zero}$ is the bitstring consisting only
  of zeroes in $D$. $D$ should be \texttt{fixed}.

  The type $D$ must be declared before this macro is expanded. The
  function $\vn{xor}$ and the constant $\vn{zero}$ are declared by
  this macro.  They must not be declared elsewhere, and they can be
  used only after expanding the macro.

   This macro defines the equivalence named $\texttt{remove\_xor}(\vn{xor})$
   for use in the \texttt{crypto} command 
   (see Section~\ref{sec:interact}).

\item $\texttt{expand\ keygen(}\vn{keyseed}, \vn{key}, \vn{kgen}\texttt{).}$
defines a key generation function $\vn{kgen}$. It can be used to add a key
generation function to symmetric cryptographic primitives, if needed.

$\vn{keyseed}$ is the type of key seeds, must be \texttt{bounded} or \texttt{nonuniform} (to be able to generate random numbers from it), typically \texttt{fixed}, and \texttt{large}.

$\vn{key}$ type of keys, must be \texttt{bounded}.

$\vn{kgen}(\vn{keyseed}): \vn{key}$ is the key generation function.

The types $\vn{keyseed}$ and $\vn{key}$ must be declared before this
macro is expanded. The function $\vn{kgen}$ is declared by this
macro. It must not be declared elsewhere, and it can be used only
after expanding the macro.

This macro defines the equivalence named $\texttt{keygen}(\vn{kgen})$
   for use in the \texttt{crypto} command 
   (see Section~\ref{sec:interact}).

\item
  $\texttt{expand\ Auth\_Enc\_from\_Enc\_then\_MAC(}\vn{key}, \vn{cleartext}, \vn{ciphertext}, \vn{enc}, \vn{dec}, \vn{injbot}, \vn{Z}, \vn{Penc}, \vn{Pmac}\texttt{).}$ defines an authenticated encryption scheme, built by encrypt-then-MAC from an IND-CPA encryption scheme and an SUF-CMA deterministic MAC.

  The arguments are the same as for \texttt{IND\_CPA\_INT\_CTXT\_sym\_enc} except that $\vn{Penc}(t, N, l)$ is the probability of breaking the IND-CPA
  property of the underlying encryption scheme in time $t$ for one key and $N$ encryption queries with
  cleartexts of length at most $l$, and
  $\vn{Pmac}(t, N, N', \vn{Nu}', l)$ is the probability of breaking the SUF-CMA
   property of the underlying MAC scheme in time $t$ for one key, $N$ MAC queries, $N'$ verification
   queries modified by the transformation and $\vn{Nu}$ verification
   queries left unchanged by the transformation for messages of length at most $l$.

 \item 
   $\texttt{expand\ Auth\_Enc\_from\_AEAD(}\vn{key}, \vn{cleartext},
   \vn{ciphertext}, \vn{enc}, \vn{dec}, \vn{injbot}, \vn{Z},
   \vn{Penc}, \vn{Pencctxt}\texttt{).}$ defines an authenticated
   encryption scheme, built from an AEAD scheme using empty additional
   data.

   The arguments are the same as for
   \texttt{IND\_CPA\_INT\_CTXT\_sym\_enc} except that $\vn{Penc}(t, N,
   l)$ is the probability of breaking the IND-CPA property of the
   underlying AEAD scheme in time $t$ for one key and $N$ encryption
   queries with cleartexts of length at most $l$, and
   $\vn{Pencctxt}(t, N, N', l, l', \vn{ld}, \vn{ld}')$ is the
   probability of breaking the INT-CTXT property of the underlying
   AEAD scheme in time $t$ for one key, $N$ encryption queries, $N'$
   decryption queries with cleartexts of length at most $l$ and
   ciphertexts of length at most $l'$, additional data for encryption
   of length at most $\vn{ld}$, and additional data for decryption of
   length at most $\vn{ld}'$.
   
 \item 
   $\texttt{expand\ Auth\_Enc\_from\_AEAD\_nonce(}\vn{key}, \vn{cleartext},
   \vn{ciphertext}, \vn{enc}, \vn{dec}, \vn{injbot}, \vn{Z},
   \vn{Penc}, \vn{Pencctxt}\texttt{).}$ defines an authenticated
   encryption scheme, built from an AEAD scheme with a nonce
   by choosing the nonce randomly at each encryption and
   using empty additional data.

   The arguments are the same as for
   \texttt{IND\_CPA\_INT\_CTXT\_sym\_enc} except that $\vn{Penc}(t, N,
   l)$ is the probability of breaking the IND-CPA property of the
   underlying AEAD scheme in time $t$ for one key and $N$ encryption
   queries with cleartexts of length at most $l$, and
   $\vn{Pencctxt}(t, N, N', l, l', \vn{ld}, \vn{ld}')$ is the
   probability of breaking the INT-CTXT property of the underlying
   AEAD scheme in time $t$ for one key, $N$ encryption queries, $N'$
   decryption queries with cleartexts of length at most $l$ and
   ciphertexts of length at most $l'$, additional data for encryption
   of length at most $\vn{ld}$, and additional data for decryption of
   length at most $\vn{ld}'$.

\item $\texttt{expand AEAD\_from\_Enc\_then\_MAC(}\vn{key},
  \vn{cleartext}$, $\vn{ciphertext}$, $\vn{add\_data}$, $\vn{enc},
  \vn{dec}$, $\vn{injbot}$, $\vn{Z}$, $\vn{Penc}$, $\vn{Pmac}\texttt{).}$ defines an
authenticated encryption scheme with additional data built by encrypt-then-MAC from an IND-CPA encryption scheme and an SUF-CMA deterministic MAC.

The arguments are the same as for \texttt{AEAD} except that $\vn{Penc}(t, N, l)$ is the probability of breaking the IND-CPA
  property of the underlying encryption scheme in time $t$ for one key and $N$ encryption queries with
  cleartexts of length at most $l$, and
  $\vn{Pmac}(t, N, N', \vn{Nu}', l)$ is the probability of breaking the SUF-CMA
   property of the underlying MAC scheme in time $t$ for one key, $N$ MAC queries, $N'$ verification
   queries modified by the transformation and $\vn{Nu}$ verification
   queries left unchanged by the transformation for messages of length at most $l$.

\item 
   $\texttt{expand\ AEAD\_from\_AEAD\_nonce(}\vn{key}$, $\vn{cleartext},
   \vn{ciphertext}$, $\vn{add\_data}$, $\vn{enc}$, $\vn{dec}$, $\vn{injbot}$, $\vn{Z},
   \vn{Penc}$, $\vn{Pencctxt}\texttt{).}$ defines an authenticated
   encryption scheme with additional data, built from an AEAD scheme with a nonce
   by choosing the nonce randomly at each encryption.

   The arguments are the same as for
   \texttt{AEAD} except that $\vn{Penc}(t, N,
   l)$ is the probability of breaking the IND-CPA property of the
   underlying AEAD scheme in time $t$ for one key and $N$ encryption
   queries with cleartexts of length at most $l$, and
   $\vn{Pencctxt}(t, N, N', l, l', \vn{ld}, \vn{ld}')$ is the
   probability of breaking the INT-CTXT property of the underlying
   AEAD scheme in time $t$ for one key, $N$ encryption queries, $N'$
   decryption queries with cleartexts of length at most $l$ and
   ciphertexts of length at most $l'$, additional data for encryption
   of length at most $\vn{ld}$, and additional data for decryption of
   length at most $\vn{ld}'$.

\item $\texttt{expand\ random\_split\_$N$(}\vn{input\_t}, \vn{part1\_t}, \dots,
\vn{partN\_t}, \vn{tuple\_t}, \vn{tuple}, \vn{split}\texttt{).}$ defines allows to split a random 
value into $N$ values, for $N \leq 10$.
   
  $\vn{input\_t}$: type of the input value

  $\vn{part1\_t}, \dots, \vn{partN\_t}$: types of the output parts.

  $\vn{tuple\_t}$: type of a tuple of the output parts

  $\vn{tuple}(\vn{part1\_t}, \dots, \vn{partN\_t}): \vn{tuple\_t}$: builds a tuple from $N$ parts.

  $\vn{split}(\vn{input\_t}): \vn{tuple\_t}$ splits the input into $N$ parts and returns a tuple of these parts. 
  The macro says that if $y$ is a random value in $\vn{input\_t}$, then
$\vn{split}(y)$ is a tuple $\vn{tuple}(x1, \dots, xN)$ of $N$
independent random values in $\vn{part1\_t}, \dots, \vn{partN\_t}$.


  To split a value $y$ of type $\vn{input\_t}$ into $N$ parts 
  of types $\vn{part1\_t}, \dots, \vn{partN\_t}$, write:
  \[\texttt{let}\ \vn{tuple}(x1, \dots, xN) = \vn{split}(y)\ \texttt{in}\ \dots\]
  Note that a priori, CryptoVerif thinks that the pattern-matching
  with $\vn{tuple}(x1, \dots, xN)$ may fail, and thus requires an
  \texttt{else} branch when the \texttt{let} occurs in a term.
  CryptoVerif realizes that the pattern-matching never fails when
  it expands the definition of $\vn{split}$.
  
This macro defines the equivalence named
$\texttt{splitter}(\vn{split})$ which replaces the splitting of a
random number generation in $\vn{input\_t}$ with $N$ independent
random number generations in $\vn{part1\_t}, \dots, \vn{partN\_t}$.

$\vn{input\_t}$, $\vn{part1\_t}, \dots, \vn{partN\_t}$,
$\vn{tuple\_t}$ must be defined before expanding this macro.
$\vn{tuple}$ and $\vn{split}$ are defined by this macro. They must not
be declared elsewhere, and they can be used only after expanding the
macro.

\end{itemize}

\fussy

\section{Interactive Mode}\label{sec:interact}

In interactive mode, the user specifies transformations to perform.
Some of the instructions take a program point (or occurrence)
in the current game as argument.
One should use the command \texttt{show\string_game occ}
or \texttt{out\string_game $f$ occ} (mentioned below) to
display the game with an occurrence number at each program point.
The program points can then be specified as follows:
\begin{itemize}
\item an integer designates the program point labeled by that integer in
  the displayed game.
\item $\texttt{before "}\mathit{regexp}\texttt{"}$ designates the program
  point at the beginning of the line that matches the regular expression
  $\mathit{regexp}$. Regular expressions follow the syntax of regular expressions in the OCaml Str module, see
  \url{https://caml.inria.fr/pub/docs/manual-ocaml-4.07/libref/Str.html}.
  In regular expressions, blackslash ($\backslash$) must be escaped as $\backslash\backslash$, as in OCaml string literals.
  There must be a single line that matches this
  regular expression, otherwise CryptoVerif shows an error message.
  
\item $\texttt{after "}\mathit{regexp}\texttt{"}$ designates the program
  point at the beginning of the first line that has an occurrence number after 
  the line that matches the regular expression
  $\mathit{regexp}$. 
  There must be a single line that matches this
  regular expression, otherwise CryptoVerif shows an error message.

\item $\texttt{before\_nth }n\texttt{ "}\mathit{regexp}\texttt{"}$ designates the program
  point at the beginning of the $n$-th line that matches the regular expression
  $\mathit{regexp}$.  
  
\item $\texttt{after\_nth }n\texttt{ "}\mathit{regexp}\texttt{"}$
  designates the program point at the beginning of the first line
  that has an occurrence number after 
  the $n$-th line that matches the regular expression
  $\mathit{regexp}$.

\item $\texttt{at }n'\texttt{ "}\mathit{regexp}\texttt{"}$
  designates the program point at the $n'$-th occurrence number
  that occurs inside the string that matches the regular expression
  $\mathit{regexp}$ in the displayed game.
  There must be a single match for this
  regular expression, otherwise CryptoVerif shows an error message.
  (With \texttt{at}, if the same line matches the regular expression
  several times, it counts as several matches.)

\item $\texttt{at\_nth }n\ n'\texttt{ "}\mathit{regexp}\texttt{"}$
  designates the program point at the $n'$-th occurrence number
  that occurs inside the string corresponding to the
  $n$-th match of the regular expression
  $\mathit{regexp}$ in the displayed game.
  (With \texttt{at\_nth}, if the same line matches the regular expression
  several times, it counts as several matches.)

\end{itemize}
Using an explicit integer to designate a program point is very unstable:
it changes if the verified protocol is slightly modified, or
if a new version of CryptoVerif itself is used, which may transform
games in a slightly different way.
The other ways of designating program points are therefore preferable
when possible.

When an identifier is expected in an instruction, it is possible
to put it between quotes. This is useful in particular for identifiers
that clash with proof keywords.

Here is a list of available instructions:
\begin{itemize}

\item \texttt{help} or \texttt{?}: display a list of available commands.

\item \texttt{remove\string_assign useless}: remove useless assignments,
that is, assignments to $x$ when $x$ is unused and assignments
between variables.

\item \texttt{remove\string_assign findcond}: removes useless assignments,
as above, as well as assignments ${\tt let}\ x = M\ {\tt in}$ inside
conditions of {\tt find}. 

\item \texttt{remove\string_assign all}: remove all assignments,
by replacing variables with their values. This is not recommended:
you should try to specify which assignments to remove more precisely.

\item $\texttt{remove\string_assign binder}\ x_1\ \dots\ x_n$: remove assignments
to $x_1, \ldots, x_n$ by replacing $x_i$ with its value. When $x_i$ becomes unused,
its definition is removed. When $x_i$ is used only in \texttt{defined} 
tests after transformation, its definition is replaced with 
a constant. The variables $x_i$ may also be regular expressions, 
following the syntax of regular expressions in the OCaml Str module, see
  \url{https://caml.inria.fr/pub/docs/manual-ocaml-4.07/libref/Str.html}.
In this case, they designate all variables that match the regular expression.
This is particularly helpful to designate all variables that come
from the same initial name but have different numbers:
\texttt{"$\mathit{name}$\_[0-9]*"}. Regular expressions 
need to be put between quotes because they use characters that do
not belong to ordinary identifiers. Blackslash ($\backslash$) must then be escaped 
as $\backslash\backslash$, as in OCaml string literals.

\item $\texttt{move }m$: Try to move instructions as follows:
\begin{itemize}

\item Move random number generations down in the syntax tree as much
  as possible, in order to delay the choice of random numbers. This is
  especially useful when the random number generations can be moved
  under a test {\tt if}, {\tt let}, or {\tt find}, so that we can
  distinguish in which branch of the test the random number is created
  by a subsequent \texttt{SArename} instruction.

\item Move assignments down in the syntax tree but without duplicating
  them. This is especially useful when the assignment can be moved
  under a test, in which the assigned variable is used only in one
  branch. In this case, the assigned term is computed in fewer cases
  thanks to this transformation.
  (Note that only assignments without array accesses can be moved,
  because in the presence of array accesses, the computation would have
  to be kept in all branches of the test, yielding a duplication that 
  we want to avoid.)

\end{itemize}
The argument $m$ specifies which instructions should be moved:
\begin{itemize}
\item \texttt{all}: move random number generations and assignments,
when this is beneficial, that is, when they can be moved under a test.
\item \texttt{noarrayref}: move random number generations and assignments
without array accesses, when this is beneficial.
\item \texttt{random}: move random number generations, when this is beneficial.
\item \texttt{random\string_noarrayref}: move random number generations 
without array accesses, when this is beneficial.
\item \texttt{assign}: move assignments, when this is beneficial.
\item $\texttt{binder }x_1\ \dots\ x_n$: move random number generations and assignments
that define $x_1, \ldots, x_n$ (even when this is not beneficial).
The variables $x_i$ may also be regular expressions.
\item $\texttt{array }x$: move random number generations that define $x$
when $x$ is of a {\tt large} and {\tt bounded} type and $x$ is not used in
the process that defines it, until the next output after the definition of $x$.
$x$ is then chosen at the point where it is really first used. 
(Since this point may depend on the trace, the uses of $x$ are often
transformed into {\tt find} instructions that test whether $x$ has been 
chosen before, and reuse the previously chosen value if this is true.)
%Internally, this is implemented by calling a cryptographic transformation
%defined in the library by the macro "move_array_internal_macro".
The variables $x$ may also be a regular expression. 
In this case, it designates all variables that match the regular expression;
all these variables must have the same type.
\end{itemize}

\item \texttt{simplify}: simplify the game.

\item $\texttt{simplify coll\_elim(variables:}\ x_1, \ldots, x_n\texttt{; types:}t_1, \ldots, t_{n'}\texttt{; terms:}occ_1, \ldots, occ_{n''}\texttt{)}$: simplify the game,
additionally allowing elimination of collisions on data of types with
option \texttt{password}, at all occurrences of variables
$x_1, \ldots, x_n$, at all data of types $t_1, \ldots, t_{n'}$,
and at the program points $occ_1, \ldots, occ_{n''}$.
See above for how to specify the program points $occ_i$. 
Some of the lists of variables, types, or terms may be omitted.
In this case, the separating semi-colon \texttt{;} is obviously
omitted as well. It is also possible to reorder or repeat these lists; the lists add up.

\item $\texttt{global\_dep\_anal}\ x$ performs global dependency
  analysis on $x$: it computes all variables that depend on $x$, and
  when possible, shows that all output messages are independent of $x$
  and that all tests are independent of $x$ after eliminating
  collisions. The tests are then simplified by eliminating these
  collisions, so that all dependencies on $x$ can be removed.

$\texttt{global\_dep\_anal}\ x\ \texttt{coll\_elim(variables:}\ x_1, \ldots, x_n\texttt{; types:}t_1, \ldots, t_{n'}\texttt{; terms:}occ_1, \ldots, occ_{n''}\texttt{)}$ 
performs global dependency analysis on $x$, 
additionally allowing elimination of collisions on data of types with
option \texttt{password}, at all occurrences of variables
$x_1, \ldots, x_n$, at all data of types $t_1, \ldots, t_{n'}$,
and at the program points $occ_1, \ldots, occ_{n''}$.
See above for how to specify the program points $occ_i$. 
Some of the lists of variables, types, or terms may be omitted.
In this case, the separating semi-colon \texttt{;} is obviously
omitted as well. It is also possible to reorder or repeat these lists; the lists add up.

One must allow elimination on $x$ independently of the program point, 
so if $x$ is not large, $x$ must be mentioned in $x_1, \ldots, x_n$
or its type must be mentioned in $t_1, \ldots, t_n$;
mentioning the occurrences of $x$ in $occ_1, \ldots, occ_{n''}$
is not sufficient.

The variable $x$ may also be a regular expression.
In this case, it designates all variables that match the regular expression,
and the command \texttt{global\_dep\_anal} is executed for each of these variables in
turn.

\item $\texttt{SArename }x$: When $x$ is defined at several places,
rename $x$ to a different name for each definition. This is useful for
distinguishing cases depending on which definition of $x$ is used.
The variable $x$ may also be a regular expression.
In this case, it designates all variables that match the regular expression,
and the command \texttt{SArename} is executed for each of these variables in
turn.

\item \texttt{all\string_simplify}: perform several simplifications 
on the game, as if \texttt{simplify}, \texttt{move all}
if \texttt{autoMove = true}, and \texttt{remove\string_assign useless}
had been called.

\item \texttt{crypto} \ldots: applies a cryptographic transformation
that comes from a statement \texttt{equiv}. This command can
have several forms:
\begin{itemize}

\item \texttt{crypto}: list all available \texttt{equiv} statements,
and ask the user to choose which one should be applied, 
with which variables of the game corresponding to random number generations
of the left-hand side of the equivalence.

\item $\texttt{crypto }\nonterm{name}\ \texttt{*}$: apply a cryptographic
transformation determined by the name $\nonterm{name}$. 
This name can be either an identifier $\mathit{id}$ or $\mathit{id}(f)$, 
and corresponds to the name given at the declaration of the 
cryptographic transformation by \texttt{equiv}.
In case the name is not found, CryptoVerif reverts to the
old way of designating cryptographic transformations, in which
$\nonterm{name}$ can be either a
function symbol that occurs in the terms in the left-hand side of the
\texttt{equiv} statement, or a probability function that occurs in the
probability formula of the \texttt{equiv} statement.
When several equivalences correspond, the user is prompted for choice.
The transformation is applied as many times as possible. (In this
case, the advised transformations are applied automatically even when
\texttt{set autoAdvice = false}.)

\item $\texttt{crypto }\nonterm{name}\ \texttt{**}$: similar to 
$\texttt{crypto }\nonterm{name}\ \texttt{*}$, but the game is simplified
only after the last cryptographic transformation instead of simplifying it
after each transformation, for faster execution. This is recommended
only for very simple cryptographic transformations.

\item $\texttt{crypto }\nonterm{name}\ x_1\ \ldots\ x_n$: apply a cryptographic
transformation chosen as above, where $x_1, \ldots, x_n$ are variable
names of the game corresponding to random number generations 
in the left-hand side of the
equivalence. (CryptoVerif may automatically add variables to the list
$x_1, \ldots, x_n$ if needed, except when a dot is added at the end of
the list $x_1, \ldots, x_n$. The transformation is applied only once.
If several disjoint lists $x_1, \ldots, x_n$ are possible and no variable
name is mentioned, CryptoVerif makes a choice. It is better to mention
at least one variable name when the left-hand side of the equivalence
contains a random number generation, to make explicit which transformation 
should be applied.)

In case the command ends with a dot ($\texttt{.}$), CryptoVerif never adds 
other variable names to those already listed. If the dot is absent,
CryptoVerif may add other variable names if that seems necessary to perform
the transformation.

The variables $x_i$ may also be regular expressions.
In this case, they designate all variables that match the regular expression.

\item $\texttt{crypto }\nonterm{name}\ 
\texttt{[variables: }x_1\texttt{->}y_1,\ldots, x_n\texttt{->}y_n
\texttt{; terms: }occ_1\texttt{->}O_1,\ldots,occ_m\texttt{->}O_m\texttt{]}$:
apply a cryptographic transformation chosen as above, where
\begin{itemize}

\item $x_1, \ldots, x_n$ are variable names of the game which
  correspond to random number generations $y_1, \ldots, y_n$
  respectively in the left-hand side of the equivalence. (CryptoVerif
  may automatically add variables to the list
  $x_1\texttt{->}y_1,\ldots, x_n\texttt{->}y_n$ if needed, except when
  a dot is added at the end of this list.)

The variables $x_i$ may also be regular expressions.
In this case, they designate all variables that match the regular expression,
and they are mapped to the same variable $y_i$ in the equivalence.

\item $occ_1, \ldots, occ_m$ are program points at which terms 
  will be transformed using oracles $O_1, \ldots, O_m$ respectively of
  the equivalence. See above for how to specify the program points $occ_i$. 
  (CryptoVerif may automatically add elements to the
  list $occ_1\texttt{->}O_1,\ldots,occ_m\texttt{->}O_m$ if needed, except
  when a dot is added at the end of this list.)

\end{itemize}
When the considered equivalence is defined inside a macro,
macro expansion may add an integer suffix \texttt{\_$k$} to the variable
and oracle names of the equivalence (or may modify that suffix if
they already have one). This suffix must be included in
the variable and oracle names used in this command.
This happens in particular for primitives defined in the
library of primitives of CryptoVerif. The right value of $k$ in the
suffix can be determined by issuing a command \texttt{crypto} without
further indication. This command will display the equivalences
as they are stored by CryptoVerif after macro expansion.

One of the lists of variables or terms may be omitted.
In this case, the separating semi-colon \texttt{;} is obviously
omitted as well. It is also possible to reorder or repeat the
\texttt{variables} and /or \texttt{terms} lists; the lists add up.

\end{itemize}


\item $\texttt{insert\string_event }e\ occ$ replaces the subprocess at
program point $occ$ with the event $\texttt{event }e$. The games may
be distinguished if and only if the event $e$ is executed, and CryptoVerif
then tries to find a bound for the probability of executing that event.
See above for how to specify the program point $occ$.
The program point $occ$ must
correspond to an output process (resp. oracle body in the oracles
front-end).

\item $\texttt{insert}\ occ\ \texttt{"}ins\texttt{"}$ inserts instruction $ins$ at
program point $occ$. The instruction $ins$ can be 
\begin{align*}
&\texttt{new\ }\nonterm{vartype}\\
&\texttt{if }\nonterm{cond}\texttt{ then}\\
&\texttt{event }\nonterm{ident}[\texttt{(}\seq{term}\texttt{)}]\\
&\texttt{let }\nonterm{pattern} \texttt{ = }\nonterm{term}\ \texttt{in}\\
&\texttt{find }\nonterm{findbranch}\ (\texttt{orfind }\nonterm{findbranch})^*
\end{align*}
or in the oracles front-end
\begin{align*}
&\nonterm{ident}\texttt{ <-R }\nonterm{ident}\\
&\texttt{if }\nonterm{cond}\texttt{ then}\\
&\texttt{event }\nonterm{ident}[\texttt{(}\seq{term}\texttt{)}]\\
&\nonterm{ident}[\texttt{:}\nonterm{ident}] \texttt{ <- }\nonterm{term}\\
&\texttt{let }\nonterm{pattern} \texttt{ = }\nonterm{term}\ \texttt{in}\\
&\texttt{find }\nonterm{findbranch}\ (\texttt{orfind }\nonterm{findbranch})^*
\end{align*}
where $\nonterm{findbranch} ::= \seq{identbound} \texttt{ suchthat }\nonterm{cond}\texttt{ then}$

In contrast to the initial game, the terms {\tt new}, {\tt if}, {\tt
  find}, or {\tt let} are not expanded, so {\tt if}, {\tt find}, {\tt
  let} can occur only in conditions of {\tt find} and {\tt new} must
not occur as a term.
%
The variables of the inserted instruction are not renamed, so one must
be careful when redefining variables with the same name. In
particular, one is not allowed to add a new definition for a variable
on which array accesses are done (because it could change the result
of these array accesses).
%
The obtained game must satisfy the required invariants (each variable
is defined at most once in each branch of {\tt if}, {\tt find}, or {\tt let};
each usage of a variable $x$ must be either $x$ without array index syntactically
under its definition, inside a {\tt defined} condition of a find, or
$x[M_1, \dots, M_n]$ under a {\tt defined} condition that contains $x[M_1, \dots, M_n]$ 
as a subterm). 
%
In case the inserted instruction is not appropriate, an error
message explaining the problem is displayed.

The obtained game is indistinguishable from the initial game.
The main practical usage of this command is to introduce case
distinctions ({\tt if}, {\tt find}, or {\tt let} with a pattern that
is not a variable). In this situation, the process that follows the
insertion point is duplicated in each branch of {\tt if}, {\tt find},
or {\tt let}, and can subsequently be transformed in different ways in
each branch. 
%
It may be useful to disable the merging of branches in simplification
by \texttt{set mergeBranches = false} when a case distinction is
inserted, so that the operation is not immediately undone at the next
simplification.

See above for how to specify the program point $occ$. 
The program point $occ$ must
correspond to an output process (resp. oracle body in the oracles
front-end).

\item $\texttt{replace}\ occ\ \texttt{"}term\texttt{"}$ replaces the term at program point
  $occ$ with the term $term$. Obviously, CryptoVerif must be able to
  prove that these two terms are equal.
  These terms must not contain \texttt{if}, \texttt{let},
  \texttt{find}, \texttt{new}.
See above for how to specify the program point $occ$. 
The program point $occ$ must correspond to a term.

\item \texttt{merge\_branches} merges the branches of {\tt if}, {\tt
    find}, and {\tt let} when they execute equivalent code. Such a
  merging is already done in simplification, but
  \texttt{merge\_branches} goes further. It performs several merges
  simultaneously and takes into account that merges may remove array
  accesses in conditions of {\tt find} and thus allow further
  merges. Moreover, it advises {\tt merge\_arrays} when variables with
  different names and with array accesses are used in the branches
  that we may want to merge.

\item $\texttt{merge\_arrays } x_{11}\ \ldots\ x_{1n} \texttt{ , }
  \ldots \texttt{ , } x_{k1}\ \ldots\ x_{kn}$ takes as argument $k$
  lists of $n$ variables separated by commas. It merges the variables
  $x_{i1}, \ldots, x_{in}$ into $x_{i1}$. This is useful when these
  variables play the same role in different branches of {\tt if}, {\tt
    find}, {\tt let}: merging them into a single variable may allow to
  merge the branches of {\tt if}, {\tt find}, {\tt let} by
  \texttt{merge\_branches}. 

  The $k$ lists to merge must contain the same number of variables $n$
  (at least 2). Variables $x_{ij}$ and $x_{i'j'}$ for $i \neq i'$ must
  never be simultaneously defined for the same value of their array
  indices.  Variables $x_{ij}$ must have the same type and the same
  array indices for all $j$. Each variable $x_{ij}$ must have a single
  definition, and must not be used in queries.

  In general, the variables $x_{i1}$ should preferably belong to the
  \texttt{else} branch of the {\tt if}, {\tt find}, {\tt let} that we
  want to merge later. Indeed, the code of the {\tt else} branch is
  often more general than the code of the other branches (which may
  exploit the conditions that are tested), so merging towards the code
  of the {\tt else} branch works more often.

  The variables $x_{1j}$ should preferably be defined above the
  variables $x_{ij}$ for any $i > 1$. If this is true, we can
  introduce special variables $y_j$ at the definition site of $x_{1j}$
  which are used only for testing that branch $j$ has been executed.
  This allows the merge to succeed more often.

\item \texttt{start\_from\_other\_end}: for proofs of indistinguishability
  only (\texttt{equivalence}), instruct CryptoVerif to start transforming
  from the other game. When your input file contains
  $\texttt{equivalence}\ Q_1\ Q_2$, CryptoVerif initally works on the first  
  process $Q_1$. When you issue the command \texttt{start\_from\_other\_end},
  CryptoVerif stores your current state, and starts working from $Q_2$.
  If you issue \texttt{start\_from\_other\_end} again, it will store
  what you did from $Q_2$, and will restart working from the end of the
  sequence that you built from $Q_1$. This command allows you to
  alternate between the sequence that starts from $Q_1$ and the one that
  starts from $Q_2$. The property is proved when both sequences end with the
  same game (which you can check with the command \texttt{success}, as usual).
  
\item \texttt{quit}: terminate execution.

\item \texttt{success}: test whether the desired properties are
proved in the current game. If yes, display the proof and stop.
Otherwise, wait for further instructions.

\item \texttt{show\string_game}: display the current game.

\item \texttt{show\string_game occ}: display the current game with
  occurrence numbers. Useful for some commands that require specifying a
  program point; see above for how program point are specified.

\item \texttt{show\string_state}: display the whole sequence
of games until the current game.

\item $\texttt{show\string_facts}\ occ$: show the facts that are proved
by CryptoVerif in the current game, at the program point $occ$. 
See above for how to specify the program point $occ$.
This command is mainly helpful for debugging.

\item \texttt{out\string_game $f$}: output the current game to file $f$.
  (Be careful: file $f$ will be overwritten if it already exists.)

\item \texttt{out\string_game $f$ occ}: output the current game with
  occurrence numbers to file $f$. Useful for some commands that require specifying a
  program point; see above for how occurrences are specified.
  (Be careful: file $f$ will be overwritten if it already exists.)

\item \texttt{out\string_state $f$}: output the whole sequence
of games until the current game to file $f$.
  (Be careful: file $f$ will be overwritten if it already exists.)

\item $\texttt{out\string_facts}\ f\ occ$: output the facts that are proved
  by CryptoVerif in the current game, at the program point $occ$,
  to file $f$. 
See above for how to specify the program point $occ$.
This command is mainly helpful for debugging.
  (Be careful: file $f$ will be overwritten if it already exists.)

\item \texttt{auto}: switch to automatic mode; try to
terminate the proof automatically from the current game.

\item $\texttt{set }\nonterm{parameter}\texttt{ = }\nonterm{value}$:
sets parameters, as the \texttt{set} instruction in input files.

\item \texttt{allowed\string_collisions} $\nonterm{formulas}$: 
determine when to eliminate collisions. $\nonterm{formulas}$ is 
a comma-separated list of formulas of the form
$\nonterm{psize}_1\texttt{\string^}{n_1} \tttimes \dots \tttimes \nonterm{psize}_k\texttt{\string^}{n_k} / \nonterm{tsize}$,
where the exponents $n_i$ can be omitted when equal to 1;
$\nonterm{psize}_i$ is an identifier that determines the size of a
parameter: \texttt{size$n$} for parameters of size $n$,
\texttt{small} for size 0, \texttt{passive} for size 10,
\texttt{noninteractive} for size 20;
$\nonterm{tsize}$ is an identifier that determines the size of a type:
\texttt{size$n$} for types of size $n$,
\texttt{small} for size 0, \texttt{password} for size 10,
\texttt{large} for size 20. (See also the declarations \texttt{param}
and \texttt{type} for explanations of sizes.)

Collisions are eliminated when the probability that they generate
is at most of the form  $\textit{constant} \times p_1^{n_1} \times \dots \times p_k^{n_k} \times \texttt{Pcoll1rand}(T)$,
where $p_i$ is a parameter of size at most $\nonterm{psize}_i$
and $T$ is a type of size at least $\nonterm{tsize}$.
By default, collisions are eliminated for $\textit{anything} \times \texttt{Pcoll1rand}(T)$ when $T$ is a \texttt{large} type,
and for $p \times \texttt{Pcoll1rand}(T)$ when $p$ is \texttt{small} and $T$ is a \texttt{password} type.

Additionally, $\nonterm{formulas}$ may also contain elements of the form
$ \texttt{collision} \tttimes \nonterm{psize}_1\texttt{\string^}{n_1} \tttimes \dots \tttimes \nonterm{psize}_k\texttt{\string^}{n_k}$.
These formulas allow the transformation of terms by \texttt{collision} statements, provided 
the number of times the collision statement is applied is at most 
$\textit{constant} \times p_1^{n_1} \times \dots \times p_k^{n_k}$
where $p_i$ is a parameter of size at most $\nonterm{psize}_i$.
By default, \texttt{collision} statements can always be applied.

\item $\texttt{focus}\ \texttt{"}\nonterm{querydecl}\texttt{"}, \dots, \texttt{"}\nonterm{querydecl}\texttt{"}$ where
  $\nonterm{querydecl} ::= \texttt{query }[\seq{vartypeb}\texttt{;}]\nonterm{query} (\texttt{;}\nonterm{query})^*$
  follows the syntax of query declarations given in Section~\ref{sec:channels} without the final dot:
  tell CryptoVerif to try to prove only the mentioned queries, ignoring all other queries.
  That sometimes allows to simplify the game further (e.g. remove events that are not used in the queries on which we focus),
  and may allow to prove the mentioned queries.
  The queries are considered equal modulo renaming of variables declared in $\seq{vartypeb}$.
  When the queries on which we focus are all proved,
  CryptoVerif goes back to the state before the last \texttt{focus} command, to try
  to prove the other queries.
  \texttt{undo focus}
  also goes back to the state before the last \texttt{focus} command, to try
  to prove remaining queries.

\item $\texttt{undo}$: undo the last transformation.

\item $\texttt{undo}\ n$: undo the last $n$ transformations.

\item $\texttt{undo\ focus}$: go back to the state before the last focus command.
  
\item \texttt{restart}: restart the proof from the beginning.
(Still simplify automatically the first game.)

\item \texttt{interactive}: starts interactive mode.
Allowed in \texttt{proof} environments, but not when one is
already in interactive mode. Useful to start interactive mode
after some proof steps.

\item \texttt{forget\_old\_games}: removes games before the current one
from memory. That allows to save some memory, but prevents \texttt{undo}.
The display of the games is saved into a temporary file to allow
displaying the games at the end of the proof.
You can save more memory by applying this command systematically with
the setting \texttt{set forgetOldGames = true}.

\end{itemize}
Ctrl-C allows to interrupt a command in interactive mode,
and go back to the state before the beginning of this command.
This feature can be helpful when a command is very slow,
to be able to try another command without waiting for the current command 
to terminate. It may not work under Windows.

The following indications can help finding a proof:
\begin{itemize}

\item When a message contains several nested cryptographic primitives,
it is in general better to apply first the security definition of the
outermost primitive.

\item In order to distinguish more cases, one can start by applying
the security of primitives used in the first messages, before applying
the security of primitives used in later messages.

\end{itemize}
Using a text editor such as \texttt{emacs} to look at games output
by \texttt{out\_game} can be
helpful, in order to use the search function to look for definitions
or usages of variables in large games.  For example, when trying to
prove secrecy of $x$, one may look for usages of $x$, for
definitions of $x$, and for usages of other variables used in those
definitions.

\section{Output of the system}

The system outputs the executed transformations when it performs
them. At the end, it outputs the sequence of games that leads to the
proof of the desired properties. Between consecutive games, it prints
the name of the performed transformation and details of what it
actually did, and the formula giving the difference of probability
between these games (if it is not 0).
The description of the transformation between game may refer to 
program points in the previous game. These program points may not be
completely accurate for the following reasons:
\begin{itemize}
\item When a step of the transformations transforms the same part of 
the game as a previous step, the program point in the second step actually
refers to the code generated by the previous step, so it is not found
in the previously displayed game.
\item When a step transforms part of the game that was duplicated by
a previous step of the transformation, the displayed program point
is in fact ambiguous: one does not know which of the copies is actually
transformed.
\end{itemize}
One can usually clarify the ambiguities by looking at the previous and
next games. 

Lines that begin with \texttt{RESULT} give the proved results.
They may indicate that a property is proved and give 
an upper bound of the probability that the adversary breaks
the property. 
%
In the end, they may also list the properties that could not be
proved, if any.

When the \texttt{-tex} command-line option is specified,
CryptoVerif also outputs a {\LaTeX} file containing the 
sequence of games and the proved properties.

\paragraph{Correspondence between ACSII and {\LaTeX} outputs}

To use nicer and more conventional notations, the {\LaTeX} output sometimes
differs from the ASCII output. Here is a table of correspondence:
\begin{center}
\begin{tabular}{l|l}
ASCII&{\LaTeX}\\
\hline
$\texttt{<=(}p\texttt{)=>}$& $\approx_p$\\
\texttt{\&\&}&$\wedge$\\
\texttt{||}&$\vee$\\
\texttt{<>}&$\neq$\\
\texttt{<=}&$\leq$\\
\texttt{orfind}&$\oplus$\\
\texttt{==>}&$\Longrightarrow$\\
\hline
\multicolumn{2}{l}{\textbf{For the \texttt{channels} front-end}}\\
\hline
$\texttt{in(}c\texttt{,}p\texttt{)}$& $c(p)$\\
$\texttt{in(}c\texttt{,(}p_1, \ldots, p_n\texttt{))}$& $c(p_1, \ldots, p_n)$\\
$\texttt{out(}c\texttt{,}M\texttt{)}$& $\overline{c}\langle M\rangle$\\
$\texttt{out(}c\texttt{,(}M_1, \ldots, M_n\texttt{))}$& $\overline{c}\langle M_1, \ldots, M_n\rangle$\\
$\texttt{!}N$&$!^N$\\
\texttt{yield}& $\overline{0}$\\
\texttt{->}& $\rightarrow$\\
\hline
\multicolumn{2}{l}{\textbf{For the \texttt{oracles} front-end}}\\
\hline
\texttt{<-}&$\leftarrow$\\
\texttt{<-R}&$\stackrel{R}{\leftarrow}$\\
\end{tabular}
\end{center}

\section{Implementation}
CryptoVerif can generate an OCaml implementation of the protocol
from the CryptoVerif specification, using the option \texttt{-impl}.

CryptoVerif generates the code for the protocol itself, but the code
for the cryptographic primitives and for interacting with the network
and the application has to be manually written in OCaml. 
\begin{itemize}

\item For the cryptographic primitives, one can specify which OCaml 
function corresponds to which CryptoVerif function as explained in
Section~\ref{sec:implopt} below. For the security guarantees to hold, the OCaml
implementation must satisfy the security assumptions mentioned
in the CryptoVerif specification. The subdirectory \texttt{implementation}
provides a basic implementation for some cryptographic primitives,
in the module \texttt{Crypto}. This module has two implementations:
\begin{itemize}

\item \texttt{crypto\_real.ml} corresponds to real cryptographic primitives,
implemented by relying on the OCaml cryptographic library \texttt{Cryptokit}
(\url{http://forge.ocamlcore.org/projects/cryptokit/}). You need to
install this library in order to run the protocol implementations
generated by CryptoVerif. (It is used at least for random number generation
even if you implement the cryptographic primitives by other means.)

\item \texttt{crypto\_dbg.ml} is a debugging implementation, which 
constructs terms instead of applying the real cryptographic primitives.

\end{itemize}
You can choose which implementation to use by linking \texttt{crypto.ml}
to the desired implementation. If you implement your own protocol,
you will probably need to define your own cryptographic primitives.

The module \texttt{Base} contains functions used by code
generated by CryptoVerif. It should not be modified.

\item The network and application code calls the code generated
by CryptoVerif. From the point of view of security, this code can
be considered as part of the adversary. We require that this code
does not use unsafe OCaml functions (such as \texttt{Obj.magic}
or marshalling/unmarshalling with different types) to bypass the
typesystem (in particular to access the environment of closures
and send it on the network).

We also require that this code does not mutate the values received
from or passed to functions generated by CryptoVerif. This can be
guaranteed by using unmutable types, with the previous requirement.
However, OCaml typically uses \texttt{string} for cryptographic
functions and for network input/output, and the type \texttt{string}
is mutable in OCaml. For simplicity and efficiency, the generated 
code uses the type \texttt{string}, with the requirement mentioned above.

We also require that all data structures manipulated by the generated
code are non-circular. This is necessary because we use OCaml
structural equality to compare values, and this equality may not
terminate in the presence of circular data structures. This can easily
be guaranteed by requiring that all OCaml types declared in the
CryptoVerif input file are non-recursive.

We also require that this code does not fork after obtaining but
before calling an oracle that can be called only once (because it is
not under a replication in the CryptoVerif specification). Indeed,
forking at this point would allow the oracle to be called several times.
In practice, forking generally occurs only at the very beginning of the
protocol, when the server starts a new session, so this requirement
should be easily fulfilled.

Finally, we require that the programs do not perform several
simultaneous writes to the same file and do not simultaneously read
and write in the same file. This requirement could be enforced using
locks, but in practice, it is generally obtained for free if the
programs are run as intended. More precisely, we have two categories
of files:
\begin{itemize}

\item Files that are created to store variables defined in a program
and used in another program, for example, long-term keys generated by
a key generation program, then used by the protocol. These files are
written in one program, and read at the beginning of another
program. These two programs should not be run concurrently, and the
program that writes the file should be run once on each machine,
not several times.

\item Files that store tables of keys. The programs that insert elements 
in the table should be run one at a time. The insertion in the table
is actually appending the file, so the system should support reading the
table while inserting elements in it.  (Elements not yet completely
inserted are ignored.)

\end{itemize}

\end{itemize} 
The subdirectories \texttt{implementation/nspk} and
\texttt{implementation/wlsk} provide two complete examples,
with the CryptoVerif specification and the OCaml network and application
code.


\subsection{Restrictions on the processes for implementation}

The following two constraints must be satisfied:
\begin{itemize}
%%% The next restriction may help us in the proof, but is not really necessary
% \item One cannot have two replications directly under an oracle:
%   $\texttt{!} N\ \texttt{!}N'\ \texttt{in(}\ldots$ is forbidden.

\item \texttt{find} must not be used. You can obtain a similar
result using \texttt{insert} and \texttt{get}, which are supported.

\item 
Let us name ``oracles'' the parts of the process that are between an
$\texttt{in}$/$\nonterm{ident}\texttt{(}\seq{pattern}\texttt{) :=
}\ldots$ and an $\texttt{out}$/$\texttt{return}$ statement,
because in the oracle frontend, they correspond exactly to that.

Let us define the signature of an oracle as the pair containing
\begin{itemize}
\item the type
  $T_1\times\ldots\times T_k \rightarrow T'_1\times\ldots\times T'_n$, 
where $T_1\times\ldots\times T_k$ are the types of the
  arguments expected in the
  $\texttt{in}$/$\nonterm{ident}\texttt{(}\seq{pattern}\texttt{) :=}$
  statement, and $T'_1\times\ldots\times T'_n$ are the types of the result
  given in the $\texttt{out}$/$\texttt{return}$ statements, and

\item the list containing for each of the following oracles,
its name and whether it is under a replication or not.

\end{itemize}
  An oracle can have multiple $\texttt{out}$/$\texttt{return}$
  statements.  To be able to implement it, we must be able to define
  the signature above for each oracle, that is, all
  $\texttt{out}$/$\texttt{return}$ must return the same type of
  elements, and the oracles present after each
  $\texttt{out}$/$\texttt{return}$ statement must be the same.
  Moreover, if an oracle with the same name is defined at several
  places, all its definitions must have the same signature.
\end{itemize}

\subsection{Defining modules}

The syntax of the processes is extended to add annotations, described in
Figure~\ref{fig:syntaxext}. The symbol $::+=$ means that we add the rule at
the right-hand side to the non-terminal symbol at the left-hand side.

\begin{figure}
\begin{align*}
&\nonterm{mod\_opt} ::= \nonterm{ident}
(\texttt{<}\mid\texttt{>})\nonterm{string}\\
&\iprocess ::+= \nonterm{ident} [\texttt{[ }\neseq{mod\_opt}\texttt{ ]}]\texttt{ \{ }\iprocess{}\\
&\text{If channel frontend, }\oprocess ::+= \texttt{out(}\nonterm{channel}\texttt{,
}\nonterm{term}\texttt{)}[\texttt{\}}][\texttt{; }\iprocess]\\
&\text{If oracle frontend, }\oprocess ::+= \texttt{return(}\seq{term}\texttt{)}[\texttt{\}}][\texttt{; }\iprocess]
\end{align*}
\caption{Extensions to the syntax}
\label{fig:syntaxext}
\end{figure}

The terminals $\texttt{\{}$ and $\texttt{\}}$ are used to mark the boundary
of a module. Different modules typically correspond to different
programs, for instance, key generation, client, and server of a protocol.
More precisely, the following two constructs define respectively the beginning and the end of a
module:
\begin{itemize}
\item $\mu\texttt{[}x_1\texttt{>"}\mathit{filex}_1\texttt{",} \ldots\texttt{,} x_n\texttt{>"}\mathit{filex}_n\texttt{",} y_1\texttt{<"}\mathit{filey}_1\texttt{",} \ldots\texttt{,} y_m\texttt{<"}\mathit{filey}_m\texttt{"] \{ } Q$:
The module $\mu$ will contain the oracles defined in $Q$.
The implementation of the module $\mu$ will write the contents of the
variables $x_1, \ldots, x_n$ upon instanciation in the files $\mathit{filex}_1$, \ldots, $\mathit{filex}_n$ respectively. The variables
$x_1, \ldots, x_n$ must be defined under no replication inside module $\mu$. 
These variables can then be used in other modules defined after the end of $\mu$;
these modules will read them automatically from the files $\mathit{filex}_1$, \ldots, $\mathit{filex}_n$ respectively.
The module $\mu$ will read at initialization the value
of the variables $y_1, \ldots, y_m$ from the files $\mathit{filey}_1$, \ldots, $\mathit{filey}_m$ respectively.
The variables $y_1, \ldots, y_m$ must be free in $\mu$. (They are defined before the
beginning of $\mu$.)

\item In the channel frontend, $\texttt{out(}c\texttt{,
  }t\texttt{)}\texttt{\}}\texttt{; }Q$, or in the oracle frontend
  $\texttt{return(}t_1,\ldots,t_n\texttt{)}\texttt{\}}\texttt{; }Q$:
The module being defined will not contain $Q$.
\end{itemize}
We transform the oracles present in the module into functions taking the
arguments given to the oracle, and returning a tuple containing 
the result of the oracle and closures corresponding
to the oracles following the current oracle that are in the same module. 
A module implementation
contains only one function: the function $\texttt{init}$, which returns
closures corresponding to the oracles accessible at the beginning of the module.



\subsection{Implementation options}
\label{sec:implopt}

The implementation options declares how the implementation should translate
functions, tables and types, and one must declare them after the
declaration of the element it modifies and before use.  
The syntax is described in 
Figure~\ref{fig:syntaximpl}.

\newcommand{\neseqsemi}[1]{\textrm{seq;}^+\nonterm{#1}}
\begin{figure}
\def\phio{\phantom{\nonterm{impl\_opt} = }\mid}
\begin{align*}
&\neseqsemi{N} ::= N \mid N \texttt{;} \neseqsemi{N}\\
&\nonterm{impl\_block} ::= \texttt{implementation
}\nonterm{impl\_opt}(\texttt{;}\nonterm{impl\_opt})^*\texttt{.}\\
&\nonterm{type\_opt} ::= \nonterm{ident}\texttt{=}\neseq{string}\\
&\nonterm{fun\_opt} ::= \nonterm{ident}\texttt{=}\nonterm{string}\\
&\nonterm{impl\_opt} ::= \texttt{type }\nonterm{ident}\texttt{=}\nonterm{string}
\ [\texttt{[}\neseqsemi{type\_opt}\texttt{]}]\\
&\phio\texttt{type }\nonterm{ident}\texttt{=}
\nonterm{integer}\ [\texttt{[}\neseqsemi{type\_opt}\texttt{]}]\\
&\phio\texttt{table }\nonterm{ident}\texttt{=}\nonterm{string}\\
&\phio\texttt{fun }\nonterm{ident}\texttt{=}\nonterm{string}\ 
[\texttt{[}\neseqsemi{fun\_opt}\texttt{]}]\\
&\phio\texttt{const }\nonterm{ident}\texttt{=}\nonterm{string}
\end{align*}
\caption{Grammar for implementation options}
\label{fig:syntaximpl}
\end{figure}

The available implementation options are described hereafter:
\begin{itemize}
\item $\texttt{type }T\texttt{="ty"}$: Sets the OCaml type $\texttt{ty}$ to
  be the type corresponding to the type $T$. 
%  The type $\texttt{ty}$ must be
%  non recursive, because we may use operators like $\texttt{(=)}$ on
%  elements of this type.

  This also can be followed by options between brackets and separated by
  semicolons. These options are:
  \begin{itemize}
  \item $\texttt{serial="s","d"}$: 
    Sets the serialization/deserialization of the type. There is no
    default, and this is required when a variable of type $T$ is written or
    read to a file/table, or when it is contained in a tuple.
    The serialization function \texttt{s} must be of type $\texttt{ty} \rightarrow {\tt string}$,
    the deserialization function \texttt{d} must be of type ${\tt string} \rightarrow \texttt{ty}$.
    When deserialization fails, it must raise exception ${\tt Match\_fail}$.

  \item $\texttt{pred="p"}$:
    Sets the predicate function, this function must be an OCaml function of
    type $\texttt{ty} \rightarrow \texttt{bool}$. It returns whether an
    element is of type $T$ or not. The default predicate function is a
    function that accepts every element.

  \item $\texttt{random="f"}$:
    Sets the random generation function. This function must be an OCaml
    function of type $\texttt{unit} \rightarrow \texttt{ty}$, and must
    return uniformly a random element of type $\texttt{ty}$. In particular,
    if a predicate function has been defined, the predicate function must
    return $\texttt{true}$ on every element returned by the random
    generation function.
  \end{itemize}
\item $\texttt{type }T\texttt{=}n$: Sets the size of the $\texttt{fixed}$
  type $T$. The size must be a multiple of 8 and then will be represented
  by a string or 1 and then by a boolean.
  This can be followed by options between brackets and separated
  by semicolons. The only allowed option is:
  \begin{itemize}
  \item $\texttt{serial="s","d"}$:
    Modifies the default serialization/deserialization of the type (used
    when a variable of this type is read/written to a file/table).
  \end{itemize}

\item $\texttt{table }tbl\texttt{="file"}$: Sets the file in which the table
  $tbl$ is written.
\item $\texttt{fun }f\texttt{="s"}$: 
  Sets the implementation of the function $f$ to the OCaml function
  $\texttt{s}$.  If the function $f$ takes arguments of type $T_1\times
  \ldots \times T_n$ and returns a result of type $T$, the type of
  $\texttt{s}$ must be $st_1\rightarrow st_2\rightarrow\ldots\rightarrow
  st_n\rightarrow st$, where for all $i$ between $1$ and $n$, $st_i$ must
  be the corresponding type declared using the $\texttt{type}$ declaration
  for the type $T_i$, and $st$ is the corresponding type for $T$. For
  functions $f$ with no arguments, the type of the function $\texttt{s}$
  must be $\texttt{unit}\rightarrow st$, with $st$ the type corresponding
  to $T$.  This can take the following options:
  \begin{itemize}
  \item $\texttt{inverse="s\_inv"}$:
    If $f$ has the $\texttt{compos}$ attribute, this declares
    $\texttt{s\_inv}$ as the inverse function.
    With the previous notations, this function must be of type
    $st \rightarrow st_1\times st_2\times\ldots\times st_n$.
    $\texttt{s\_inv\ }x$ must return a tuple $(x_1, \ldots, x_n)$
    such that $\texttt{s}\ x_1\ \ldots\ x_n = x$. If there is no such element,
    \texttt{s\_inv} must raise \texttt{Match\_fail}.
  \end{itemize}
  CryptoVerif allows one to define macros by \texttt{letfun}. Specifying
  an OCaml implementation for these macros is optional. When the OCaml
  implementation is not specified, CryptoVerif generates code according to 
  the \texttt{letfun} macro. When the OCaml
  implementation is specified, it is used when generating the OCaml code,
  while the CryptoVerif macro defined by \texttt{letfun} is used for
  proving the protocol. This feature can be used, for instance, to define
  probabilistic functions: the OCaml implementation generates the
  random choices inside the function, while the CryptoVerif definition
  by \texttt{letfun} first makes the random choices, then calls a
  deterministic function.

\item $\texttt{const }f\texttt{="s"}$: Sets the implementation of the
  function $f$ that has no arguments to an OCaml constant. If the constant
  is a string, one can write, for example, $\texttt{const }f\texttt{="\char`\\"constant\char`\\""}$.
\end{itemize}




\section*{Acknowledgments}

CryptoVerif was partly developed while the authors were at \'Ecole Normale Sup\'erieure, Paris.

\bibliography{../../dev/allbib/biblio}
\bibliographystyle{abbrv} %alpha

\end{document}
