\def\iprocess{\nonterm{odef}}
\def\oprocess{\nonterm{obody}}

The \texttt{oracles} front-end is similar to the  \texttt{channels}
with the following differences.
The keyword \texttt{newChannel} is replaced with \texttt{newOracle},
\texttt{run} is a keyword,
and \texttt{channel} and \texttt{out} are not keywords.

\begin{figure}[tp]
\def\phop{\phantom{\oprocess = }\mid}
\def\phip{\phantom{\iprocess = }\mid}
\begin{align*}
&\oprocess ::= \texttt{run}\ \nonterm{ident}[\texttt{(}\seq{term}\texttt{)}]\\
&\phop \texttt{(} \oprocess \texttt{)}\\
&\phop \yield\\
&\phop \texttt{event }\nonterm{ident}[\texttt{(}\seq{term}\texttt{)}]\ [\texttt{; }\oprocess]\\
&\phop \texttt{event\string_abort }\nonterm{ident}\\
&\phop \Resavt[\texttt{; }\oprocess]\\
&\phop \Resbvt[\texttt{; }\oprocess]\\
&\phop \nonterm{ident}[\texttt{:}\nonterm{ident}] \texttt{ <- }\nonterm{term}[\texttt{; }\oprocess]\\
&\phop \texttt{let }\nonterm{pattern} \texttt{ = }\nonterm{term}\ 
[\texttt{in }\oprocess\ [\texttt{else }\oprocess]]\\
&\phop \texttt{if }\nonterm{cond}\texttt{ then }\oprocess\ [\texttt{else }\oprocess]\\
&\phop \texttt{find}[\texttt{[unique]}]\ \nonterm{findbranch}\ (\texttt{orfind }\nonterm{findbranch})^* \ [\texttt{else }\oprocess]\\
&\phop \texttt{insert }\nonterm{ident}\texttt{(}\seq{term}\texttt{)}\ [\texttt{; }\oprocess]\\
&\phop \texttt{get }\nonterm{ident}\texttt{(}\seq{pattern}\texttt{)}\ [\texttt{suchthat}\ \nonterm{term}]\texttt{ in }\oprocess\ [\texttt{else }\oprocess]\\
&\phop \texttt{return(}\seq{term}\texttt{)}[\texttt{; }\iprocess]\\
&\nonterm{findbranch} ::= \seq{identbound} \texttt{ suchthat }\nonterm{cond}\texttt{ then }\oprocess\\
&\iprocess ::= \texttt{run}\ \nonterm{ident}[\texttt{(}\seq{term}\texttt{)}]\\
&\phip \texttt{(} \iprocess \texttt{)}\\
&\phip \texttt{0}\\
&\phip \iprocess \texttt{ | } \iprocess\\
&\phip \texttt{!} [\nonterm{ident}\texttt{ <=}]\ \nonterm{ident}\ \iprocess\\
&\phip \texttt{foreach }\nonterm{ident}\texttt{ <= } \nonterm{ident}\texttt{ do }\iprocess\\
&\phip \nonterm{ident}\texttt{(}\seq{pattern}\texttt{) := }\oprocess
\end{align*}
\caption{Grammar for processes (\texttt{oracles} front-end)}
\label{fig:syntax3or}
\end{figure}


The input file consists of a list of declarations followed by 
an oracle definition or an equivalence query:
\begin{align*}
&\nonterm{declaration}^*\ {\tt process}\ \iprocess\\[2mm]
&\nonterm{declaration}^*\ {\tt equivalence}\ \iprocess\ \iprocess\ [\texttt{public\_vars}\ \seq{ident}]\\[2mm]
&\nonterm{declaration}^*\ {\tt query\_equiv}[\texttt{(}\nonterm{ident}[\texttt{(}\nonterm{ident}\texttt{)}]\texttt{)}]\\
&\qquad \nonterm{omode}\ [\texttt{|}\ \ldots\ \texttt{|}\nonterm{omode}]\texttt{ <=(?)=> }
[\texttt{[}n\texttt{]}]\ [\texttt{[}\neseq{option}\texttt{]}]\ \nonterm{ogroup}\ [\texttt{|}\ \ldots\ \texttt{|}\nonterm{ogroup}]
\end{align*}

The syntax of processes is given in Figure~\ref{fig:syntax3or}.
The calculus distinguishes two kinds of processes: oracle definitions
$\iprocess$ define new oracles; oracle bodies $\oprocess$ return a
result after executing some internal computations.  When a process
(oracle definition or oracle body) is an identifier, it is substituted
with its value defined by a \texttt{let} declaration.

The oracle definition $\texttt{run }\mathit{proc}(M_1, \dots, M_n)$ is replaced with $P\{M_1/x_1, \dots, M_n/x_n\}$ when $\mathit{proc}$ is declared by $\texttt{let}\ \mathit{proc}(x_1:T_1, \dots, x_n:T_n)\texttt{ = }P\texttt{.}$ where $P$ is an oracle definition.
The terms $M_1, \dots, M_n$ must contain only variables, replication indices, and function applications.

The oracle definition $O\texttt{(}p_1, \ldots, p_n\texttt{) := }P$ defines an oracle
$O$ taking arguments $p_1, \ldots, p_n$, and returning the result of
the oracle body $P$. The patterns $p_1, \ldots, p_n$ are as in the
\texttt{let} construct above, except that variables in $p$ that are
not under a function symbol $f(\ldots)$ must be declared with their
type. The other oracle definitions are similar to input processes
in the \texttt{channels} front-end.

When an oracle $O$ is defined under $\texttt{foreach }i_1\texttt{<=}N_1$, 
\ldots, $\texttt{foreach }i_n\texttt{<=}N_n$, it also implicitly
defines $O[i_1, \ldots, i_n]$.

Note that the construct 
$\textbf{newOracle }c;Q$ used in research papers
is absent from the implementation: this construct is useful in the proof
of soundness of CryptoVerif, but not essential for encoding games
that CryptoVerif manipulates.

Let us now describe oracle bodies:
\begin{itemize}

\item $\texttt{run }\mathit{proc}(M_1, \dots, M_n)$ is replaced with $\texttt{let}$ $x_1 = M_1$ $\texttt{in}$ \dots $\texttt{let}$ $x_n = M_n$ $\texttt{in}$ $P$ when $\mathit{proc}$ is declared by $\texttt{let}\ \mathit{proc}(x_1:T_1, \dots, x_n:T_n)\texttt{ = }P\texttt{.}$ where $P$ is an oracle body.

\item {\yield} terminates the oracle, returning control to the caller.

\item 
$\texttt{return(}N_1, \ldots, N_l\texttt{);}Q$ terminates the oracle,
returning the result of the terms $N_1, \ldots, N_l$. Then, it makes
available the oracles defined in $Q$.

\item The other oracle bodies are similar to output processes
in the \texttt{channels} front-end.

\end{itemize}

In $\texttt{return(}M_1, \ldots, M_n\texttt{)}$,
$M_j$ must be of a bitstring type $T_j$ for all $j \leq n$
and that return instruction is said to be of type $T_1 \times \ldots 
\times T_n$.
All return instructions in an oracle body $P$ (excluding return
instructions that occur in oracle definitions $Q$ in processes of the form 
$\texttt{return(}M_1, \ldots, M_n\texttt{);}Q$) must be of the same
type, and that type is said to be the type of the oracle body $P$.
%
For each oracle definition $O\texttt{(}p_1, \ldots, p_m\texttt{) :=
}P$ under $\texttt{foreach }i_1\texttt{<=}N_1$, \ldots,
$\texttt{foreach }i_n\texttt{<=}N_n$, the oracle $O$ is said to be of
type $[1, N_1] \times \ldots \times [1, N_n] \rightarrow T'_1 \times
\ldots \times T'_m \rightarrow T_1 \times \ldots \times T_n$ where
$p_j$ is of type $T'_j$ for all $j \leq m$ and $P$ is of type $T_1
\times \ldots \times T_n$. When an oracle has several definitions,
it must be of the same type for all its definitions. Furthermore,
definitions of the same oracle $O$ must not occur on both sides
of a parallel composition $Q \texttt{|} Q'$ (so that several definitions
of the same oracle cannot be simultaneously available).
%
The other constructs are typed as in the \texttt{channels} front-end.

The $\texttt{channel}\ \neseq{ident}\texttt{.}$ declaration is removed,
since channels do not exist in the \texttt{oracles} front-end.

In probability formulas (Figure~\ref{fig:syntax2}), 
\texttt{time(out $\dots$)} and \texttt{time(in $n$)} are removed and
\texttt{time(newChannel)} is replaced with \texttt{time(newOracle)}.
$\texttt{time(newOracle)}$ is the maximum time to create a new
private oracle.
