%THIS FILE IS USED ONLY WITH \channelstrue

\ifchannels

\def\iprocess{\nonterm{iprocess}}
\def\oprocess{\nonterm{oprocess}}

\else

\def\iprocess{\nonterm{odef}}
\def\oprocess{\nonterm{obody}}

\fi

\def\fungroup{\nonterm{ogroup}}
\def\funmode{\nonterm{omode}}
\def\funbody{\nonterm{obody\_equiv}}
\def\Resavt{\texttt{new\ }\nonterm{ident} \texttt{:}\nonterm{ident}}
\def\Resa#1#2{\texttt{new\ }{#1}\texttt{:}{#2}}
\def\Resbvt{\nonterm{ident}\texttt{ <-R }\nonterm{ident}}
\def\Resb#1#2{{#1}\texttt{ <-R }{#2}}
\def\yield{\texttt{yield}}

\def\epsrand#1{\texttt{eps\_rand(}#1\texttt{)}}

\newcommand{\eqt}{\mathrel{\texttt{=}}}
\newcommand{\leqt}{\mathrel{\texttt{<=}}}


Comments can be included in input files. Comments are surrounded by
{\tt (*} and {\tt *)}. Nested comments are not supported.

Identifiers ($\nonterm{ident}$) begin with a letter (uppercase or lowercase) and contain
any number of letters, digits, the underscore character (\_),
the quote character ('), as well as accented letters of the ISO Latin 1
character set. Case is significant. 
Keywords cannot be used as ordinary identifiers. The keywords are:
{\tt builtin}, {\tt channel}, 
{\tt collision}, {\tt const}, {\tt def}, 
{\tt defined}, {\tt do}, {\tt else}, {\tt eps\_find}, 
{\tt eps\_rand}, {\tt equation}, {\tt equiv}, {\tt equivalence}, 
{\tt event}, {\tt event\string_abort}, {\tt expand}, {\tt find}, 
{\tt forall}, {\tt foreach}, {\tt fun}, 
{\tt get}, {\tt if}, {\tt implementation}, {\tt in}, 
{\tt inf}, %used for float infinity
{\tt inj-event}, 
{\tt insert}, {\tt is-cst}, {\tt length}, {\tt let}, {\tt letfun}, {\tt letproba}, {\tt max}, {\tt maxlength},
{\tt min}, {\tt new}, {\tt newChannel}, {\tt number}, {\tt optim-if},
{\tt orfind}, 
{\tt out}, {\tt param}, {\tt Pcoll1rand}, {\tt Pcoll2rand}, 
{\tt proba}, {\tt process}, 
{\tt proof}, {\tt public\_vars}, {\tt query}, {\tt query\_equiv}, {\tt return},
{\tt secret}, {\tt set}, {\tt special}, {\tt suchthat}, {\tt table}, 
{\tt then}, {\tt time}, {\tt type}, {\tt yield}.

\newcommand{\bs}{\textbackslash}

Strings ($\nonterm{string}$) start and end with \texttt{"}. 
Inside the string, \texttt{\bs "} stands for \texttt{"},
\texttt{\bs '} stands for \texttt{'},
\texttt{\bs n} for linefeed, \texttt{\bs t} for tab,
\texttt{\bs b} for backspace, \texttt{\bs r} for carriage return, and 
\texttt{\bs \bs} for \texttt{\bs}.
Other combinations with \texttt{\bs} are not allowed.
Characters other than \texttt{"} and \texttt{\bs} stand for themselves.

In case of syntax error, the system indicates the character position
of the error (line and column numbers). Please use your text editor to
find the position of the error. (The error messages can be interpreted
by \texttt{emacs}.)

\begin{figure}
\begin{itemize}

\item $[M]$ means that $M$ is optional; $(M)^*$ means that $M$ occurs 0 or
any number of times.

\item
$\seq{X}$ is a sequence of $X$: $\seq{X} = [(\nonterm{X}\texttt{,})^*\nonterm{X}] = \nonterm{X}\texttt{,}\ldots\texttt{,}\nonterm{X}$. (The sequence can be empty, it can be one element $\nonterm{X}$, or it can be several elements $\nonterm{X}$ separated by commas.)

\item
$\neseq{X}$ is a non-empty sequence of $X$: $\neseq{X} = (\nonterm{X}\texttt{,})^*\nonterm{X} = \nonterm{X}\texttt{,}\ldots\texttt{,}\nonterm{X}$.
(It can be one or several elements of $\nonterm{X}$ separated by commas.)

\end{itemize}
\caption{Grammar notations}
\label{fig:syntax0}
\end{figure}

\begin{figure}
\def\phst{\phantom{\nonterm{simpleterm} = }\mid}
\def\pht{\phantom{\nonterm{term} = }\mid}
\def\phpat{\phantom{\nonterm{pattern} = }\mid}
\def\phq{\phantom{\nonterm{query} = }\mid}
\def\phqt{\phantom{\nonterm{queryterm} = }\mid}
\begin{minipage}{8.5cm}
\begin{align*}
&\nonterm{identbound} ::= [\nonterm{ident}\texttt{ =}]\ \nonterm{ident} \texttt{ <= }\nonterm{ident}\\
&\nonterm{vartype} ::= \neseq{ident} \texttt{:}\nonterm{ident}\\
&\nonterm{vartypeb} ::= \neseq{ident} \texttt{:}\nonterm{ident}\\
&\phantom{\nonterm{vartypeb} =}\mid \neseq{ident} \texttt{ <= }\nonterm{ident}\\
&\nonterm{basicpat} ::= \nonterm{ident}\\
&\phantom{\nonterm{basicpat} =}\mid \nonterm{ident}\texttt{:}\nonterm{ident}\\
&\phantom{\nonterm{basicpat} =}\mid \nonterm{ident}\texttt{ <= }\nonterm{ident}\\[-1mm]
\end{align*}
\end{minipage}
\begin{minipage}{7cm}
\begin{align*}
&\nonterm{simpleterm} ::= \nonterm{ident}\\
&\phst \nonterm{ident}\texttt{(}\seq{simpleterm}\texttt{)}\\
&\phst \texttt{(}\seq{simpleterm}\texttt{)}\\
&\phst \nonterm{ident}\texttt{[}\seq{simpleterm}\texttt{]}\\
&\phst \nonterm{simpleterm} \texttt{ = } \nonterm{simpleterm} \\
&\phst \nonterm{simpleterm} \texttt{ <> } \nonterm{simpleterm} \\
&\phst \nonterm{simpleterm} \texttt{ || } \nonterm{simpleterm} \\
&\phst \nonterm{simpleterm} \texttt{ \&\& } \nonterm{simpleterm} 
\end{align*}
\end{minipage}\\
\begin{align*}
&\nonterm{term} ::= \ldots \text{(as in $\nonterm{simpleterm}$ with 
$\nonterm{term}$ instead of $\nonterm{simpleterm}$)}\\
&\pht \Resavt\texttt{;}\nonterm{term}\\
&\pht \Resbvt\texttt{;}\nonterm{term}\\
&\pht \nonterm{basicpat} \texttt{ <- } \nonterm{term}\\
&\pht \texttt{let } \nonterm{pattern} \texttt{ = }\nonterm{term}\texttt{ in }\nonterm{term}\ [\texttt{else }\nonterm{term}]\\
&\pht \texttt{if }\nonterm{cond}\texttt{ then } \nonterm{term} \texttt{ else }\nonterm{term}\\
&\pht \texttt{find}[\texttt{[unique]}]\ \nonterm{tfindbranch}\ (\texttt{orfind }\nonterm{tfindbranch})^*\texttt{ else }\nonterm{term}\\
&\pht \texttt{event }\nonterm{ident}[\texttt{(}\seq{term}\texttt{)}]\texttt{; }\nonterm{term}\\
&\pht \texttt{event\string_abort }\nonterm{ident}\\
&\pht \texttt{insert }\nonterm{ident}\texttt{(}\seq{term}\texttt{)}\texttt{; }\nonterm{term}\\
&\pht \texttt{get}[\texttt{[unique]}]\ \nonterm{ident}\texttt{(}\seq{pattern}\texttt{)}\ [\texttt{suchthat}\ \nonterm{term}]\texttt{ in }\nonterm{term}\texttt{ else }\nonterm{term}\\
&\nonterm{varref} ::= \nonterm{ident}\texttt{[}\seq{simpleterm}\texttt{]}\\
&\phantom{\nonterm{varref} =}\mid \nonterm{ident}\\
&\nonterm{cond} ::= \texttt{defined(} \neseq{varref} \texttt{) }\ [\texttt{\&\& }\nonterm{term}]\\
&\phantom{\nonterm{cond} = }\mid \nonterm{term}\\
&\nonterm{tfindbranch} ::= \seq{identbound} \texttt{ suchthat } \nonterm{cond} \texttt{ then }\nonterm{term}\\
&\nonterm{pattern} ::= \nonterm{basicpat}\\
&\phpat \nonterm{ident}\texttt{(} \seq{pattern}\texttt{)}\\
&\phpat \texttt{(} \seq{pattern}\texttt{)}\\
&\phpat \texttt{=} \nonterm{term}\\
&\nonterm{event} ::= \texttt{event(}\nonterm{ident}[\texttt{(}\seq{simpleterm}\texttt{))}]\\
&\phantom{\nonterm{event} =}\mid \texttt{inj-event(}\nonterm{ident}[\texttt{(}\seq{simpleterm}\texttt{))}]\\
&\nonterm{queryterm} ::= \nonterm{queryterm}\texttt{ \&\& }\nonterm{queryterm}\\
&\phqt \nonterm{queryterm}\texttt{ || }\nonterm{queryterm}\\
&\phqt \nonterm{event}\\
&\phqt \nonterm{simpleterm}\\
&\nonterm{query} ::= \texttt{secret }\nonterm{ident}\ [\texttt{public\_vars}\ \seq{ident}]\ [\texttt{[onesession]}]\\
&\phq \nonterm{event}\ (\texttt{\&\& }\nonterm{event})^* \texttt{ ==> }\nonterm{queryterm}\ [\texttt{public\_vars}\ \seq{ident}]\\
&\phq \nonterm{event}\ (\texttt{\&\& }\nonterm{event})^* \ [\texttt{public\_vars}\ \seq{ident}]
\end{align*}
\caption{Grammar for terms, patterns, and queries}
\label{fig:syntax1}
\end{figure}

\begin{figure}
\def\phpr{\phantom{\nonterm{proba} = }\mid}
\def\phcond{\phantom{\nonterm{optimcond} = }\mid}
\begin{xxalignat}{2}
&\nonterm{proba} ::= \texttt{(} \nonterm{proba} \texttt{)}
&&\mid \texttt{time}\\
&\phpr \nonterm{proba} \texttt{ + } \nonterm{proba}
&&\mid\texttt{time(}\nonterm{ident}[\texttt{, }\neseq{proba}]\texttt{)}\\
&\phpr \nonterm{proba} \texttt{ - } \nonterm{proba}
&&\mid \texttt{time(let }\nonterm{ident}[\texttt{, }\neseq{proba}]\texttt{)}\\
&\phpr \nonterm{proba} \texttt{ * } \nonterm{proba}
&&\mid \texttt{time((}\seq{ident}\texttt{)}[\texttt{, }\neseq{proba}]\texttt{)}\\
&\phpr \nonterm{proba} \texttt{ / } \nonterm{proba}
&&\mid \texttt{time(let (}\seq{ident}\texttt{)}[\texttt{, }\neseq{proba}]\texttt{)}\\
&\phpr \nonterm{proba} \texttt{\^{ }}\nonterm{int}
&&\mid \texttt{time(= }\nonterm{ident}[\texttt{, }\neseq{proba}]\texttt{)}\\
&\phpr \texttt{max(}\neseq{proba}\texttt{)}
&&\mid \texttt{time(!)}\\
&\phpr \texttt{min(}\neseq{proba}\texttt{)}
&&\mid \texttt{time(foreach)}\\
&\phpr \nonterm{ident}[\texttt{(}\seq{proba}\texttt{)}]
&&\mid \texttt{time([}n\texttt{])}\\
&\phpr \texttt{|}\nonterm{ident}\texttt{|}
&&\mid \texttt{time(\&\&)}\\
&\phpr \texttt{maxlength(}\nonterm{simpleterm}\texttt{)}
&&\mid \texttt{time(\string|\string|)}\\
&\phpr \texttt{length(}\nonterm{ident}[\texttt{, }\neseq{proba}]\texttt{)}
&&\mid \texttt{time(new }\nonterm{ident}\texttt{)}\\
&\phpr \texttt{length((}\seq{ident}\texttt{)}[\texttt{, }\neseq{proba}]\texttt{)}
&&\mid \texttt{time(<-R }\nonterm{ident}\texttt{)}\\
&\phpr n
&&\mid  \ifchannels\texttt{time(newChannel)}\else\texttt{time(newOracle)}\fi\qquad\qquad\qquad\qquad\\
&\phpr \#\nonterm{ident}
&&\mid  \texttt{time(if)}\\
&\phpr \texttt{eps\_find}
&&\mid \texttt{time(find }n\texttt{)}\\
&\phpr \texttt{eps\_rand(}T\texttt{)}
&&\ifchannels\mid \texttt{time(out }[\texttt{[}\neseq{ident}\texttt{]}]\nonterm{ident}[\texttt{, }\neseq{proba}]\texttt{)}\fi\\
&\phpr \texttt{Pcoll1rand(}T\texttt{)}
&&\ifchannels\mid \texttt{time(in }n\texttt{)}\fi\\
&\phpr \texttt{Pcoll2rand(}T\texttt{)}
&&\mid \texttt{optim-if}\ \nonterm{optimcond}\ \texttt{then}\ \nonterm{proba}\ \texttt{else}\ \nonterm{proba}
\end{xxalignat}\vspace*{-8mm}%
\begin{align*}
&\nonterm{optimcond} ::= \texttt{(}\nonterm{optimcond}\texttt{)}\\
&\phcond \texttt{is-cst(}\nonterm{proba}\texttt{)}\hspace*{11cm}\\
&\phcond \nonterm{proba}\texttt{ = }\nonterm{proba}\\
&\phcond \nonterm{proba}\texttt{ <= }\nonterm{proba}\\
&\phcond \nonterm{proba}\texttt{ >= }\nonterm{proba}\\
&\phcond \nonterm{proba}\texttt{ < }\nonterm{proba}\\
&\phcond \nonterm{proba}\texttt{ > }\nonterm{proba}\\
&\phcond \nonterm{optimcond}\texttt{ \&\& }\nonterm{optimcond}\\
&\phcond \nonterm{optimcond}\texttt{ || }\nonterm{optimcond}
\end{align*}
\caption{Grammar for probabilities}
\label{fig:syntax2}
\end{figure}


\begin{figure}
\def\phf{\phantom{\fungroup = }\mid}
\def\phfm{\phantom{\funmode = }\mid}
\def\phfb{\phantom{\funbody = }\mid}
\def\phsa{\phantom{\nonterm{specialarg} = }\mid}
\begin{align*}
&\nonterm{repl} ::= \texttt{!} [\nonterm{ident}\texttt{ <=}]\ \nonterm{ident}\\
&\phantom{\nonterm{repl} =}\mid \texttt{foreach } \nonterm{ident}\texttt{ <= }\nonterm{ident}\texttt{ do }\\
%
&\nonterm{res} ::= \Resavt\texttt{;}\\
&\phantom{\nonterm{res} =}\mid \Resbvt\texttt{;}\\
%
&\funbody ::= \texttt{(} \funbody \texttt{)}\\
&\phfb \texttt{event\string_abort }\nonterm{ident}\\
&\phfb \nonterm{res}\ \funbody\\
&\phfb \nonterm{basicpat} \texttt{ <- }\nonterm{term}\texttt{; }\funbody\\
&\phfb \texttt{let }\nonterm{pattern} \texttt{ = }\nonterm{term}\ 
\texttt{in }\funbody\ [\texttt{else }\funbody]\\
&\phfb \texttt{if }\nonterm{cond}\texttt{ then }\funbody\ \texttt{else }\funbody\\
&\phfb \texttt{find}[\texttt{[unique]}]\ \nonterm{ffindbranch}\ (\texttt{orfind }\nonterm{ffindbranch})^* \ \texttt{else }\funbody\\
&\phfb \texttt{insert }\nonterm{ident}\texttt{(}\seq{term}\texttt{)}\texttt{; }\funbody\\
&\phfb \texttt{get}[\texttt{[unique]}]\ \nonterm{ident}\texttt{(}\seq{pattern}\texttt{)}\ [\texttt{suchthat}\ \nonterm{term}]\texttt{ in }\funbody\texttt{ else }\funbody\\
&\phfb \texttt{return(}\nonterm{term}\texttt{)}\\
&\nonterm{ffindbranch} ::= \seq{identbound} \texttt{ suchthat }\nonterm{cond}\texttt{ then }\funbody\\
%
&\fungroup ::= \nonterm{ident}\texttt{(}\seq{vartypeb}\texttt{) }[\texttt{[}n\texttt{]}]\ [\texttt{[useful\_change]}]\texttt{ := }\funbody\\
&\phf [\nonterm{repl}]\ \nonterm{res}^*\ 
\fungroup\\
&\phf [\nonterm{repl}]\ \nonterm{res}^*\ 
\texttt{(}\fungroup \texttt{ | }\ldots\texttt{ | }\fungroup\texttt{)}\\
&\funmode ::= \fungroup\ [\texttt{[exist]}]\\
&\phfm \fungroup \texttt{ [all]}\\
&\nonterm{specialarg} ::= \nonterm{ident}\\
&\phsa \nonterm{string}\\
&\phsa \texttt{(} \seq{specialarg} \texttt{)}
\end{align*}
\caption{Grammar for equivalences}
\label{fig:syntax3}
\end{figure}

\begin{figure}[tp]
\def\phd{\phantom{\nonterm{dim} = }\mid}
\begin{align*}
&\nonterm{dim} ::= \texttt{time}[\texttt{\^{ }}\nonterm{int}]\\
&\phd \texttt{length}[\texttt{\^{ }}\nonterm{int}]\\
&\phd \texttt{number}\\
&\phd \nonterm{dim}\texttt{ * }\nonterm{dim}\\
&\phd \nonterm{dim}\texttt{ / }\nonterm{dim}\\[2mm]
&\nonterm{vardim} ::= \neseq{ident}\texttt{:}\nonterm{dim}
\end{align*}
\caption{Grammar for dimensions}
\label{fig:syntaxdim}
\end{figure}

\begin{figure}[tp]
\def\phop{\phantom{\oprocess = }\mid}
\def\phip{\phantom{\iprocess = }\mid}
\begin{align*}
&\nonterm{channel} ::= \nonterm{ident}[\texttt{[}\seq{ident}\texttt{]}]\\
&\oprocess ::= \nonterm{ident}[\texttt{(}\seq{term}\texttt{)}]\\
&\phop \texttt{(} \oprocess \texttt{)}\\
&\phop \yield\\
&\phop \texttt{event }\nonterm{ident}[\texttt{(}\seq{term}\texttt{)}]\ [\texttt{; }\oprocess]\\
&\phop \texttt{event\string_abort }\nonterm{ident}\\
&\phop \Resavt[\texttt{; }\oprocess]\\
&\phop \Resbvt[\texttt{; }\oprocess]\\
&\phop \nonterm{basicpat} \texttt{ <- }\nonterm{term}[\texttt{; }\oprocess]\\
&\phop \texttt{let }\nonterm{pattern} \texttt{ = }\nonterm{term}\ 
[\texttt{in }\oprocess\ [\texttt{else }\oprocess]]\\
&\phop \texttt{if }\nonterm{cond}\texttt{ then }\oprocess\ [\texttt{else }\oprocess]\\
&\phop \texttt{find}[\texttt{[unique]}]\ \nonterm{findbranch}\ (\texttt{orfind }\nonterm{findbranch})^* \ [\texttt{else }\oprocess]\\
&\phop \texttt{insert }\nonterm{ident}\texttt{(}\seq{term}\texttt{)}\ [\texttt{; }\oprocess]\\
&\phop \texttt{get}[\texttt{[unique]}]\ \nonterm{ident}\texttt{(}\seq{pattern}\texttt{)}\ [\texttt{suchthat}\ \nonterm{term}]\texttt{ in }\oprocess\ [\texttt{else }\oprocess]\\
&\phop \ifchannels\texttt{out(}\nonterm{channel}\texttt{, }\nonterm{term}\texttt{)}\else \texttt{return(}\seq{term}\texttt{)}\fi[\texttt{; }\iprocess]\\
&\nonterm{findbranch} ::= \seq{identbound} \texttt{ suchthat }\nonterm{cond}\texttt{ then }\oprocess\\
&\iprocess ::= \nonterm{ident}[\texttt{(}\seq{term}\texttt{)}]\\
&\phip \texttt{(} \iprocess \texttt{)}\\
&\phip \texttt{0}\\
&\phip \iprocess \texttt{ | } \iprocess\\
&\phip \texttt{!} [\nonterm{ident}\texttt{ <=}]\ \nonterm{ident}\ \iprocess\\
&\phip \texttt{foreach }\nonterm{ident}\texttt{ <= } \nonterm{ident}\texttt{ do }\iprocess\\
&\phip \texttt{in(}\nonterm{channel}\texttt{,}\nonterm{pattern}\texttt{)}[\texttt{; }\oprocess]
\end{align*}
\caption{Grammar for processes (\texttt{channels} front-end)}
\label{fig:syntax3ch}
\end{figure}

The input file may consist of a list of declarations followed by  
a process:
\[\nonterm{declaration}^*\ {\tt process}\ \iprocess\]
The process
describes the considered security 
protocol; the declarations specify in particular hypotheses on the 
cryptographic primitives and the security properties to prove.

Alternatively, the input may also consist of a list of declarations followed
by an equivalence query:
\[\nonterm{declaration}^*\ {\tt equivalence}\ \iprocess\ \iprocess\ [\texttt{public\_vars}\ \seq{ident}]\]
The query ${\tt equivalence}\ Q_1\ Q_2$ tells CryptoVerif to show that
the processes (games) $Q_1$ and $Q_2$ are computationally indistinguishable.
When it is present, the indication $\texttt{public\_vars}\ x_1, \dots, x_n$
means that the adversary has read access to the variables $x_1, \dots, x_n$.

Finally, the input may also be:
\begin{align*}
&\nonterm{declaration}^*\ {\tt query\_equiv}[\texttt{(}\nonterm{ident}[\texttt{(}\nonterm{ident}\texttt{)}]\texttt{)}]\\
&\qquad \nonterm{omode}\ [\texttt{|}\ \ldots\ \texttt{|}\nonterm{omode}]\texttt{ <=(?)=> }
[\texttt{[}n\texttt{]}]\ [\texttt{[}\neseq{option}\texttt{]}]\ \nonterm{ogroup}\ [\texttt{|}\ \ldots\ \texttt{|}\nonterm{ogroup}]
\end{align*}
The keyword {\tt query\_equiv} is followed by an indistinguishability property
specified in the same syntax as assumptions on security primitives
(see the declaration \texttt{equiv}), except that the probability of
distinguishing the two sides is replaced with \texttt{?}. CryptoVerif
is going to bound this probability, so we do not need to give it.
\begin{itemize}

\item When the option \texttt{[computational]} is absent,
CryptoVerif then converts this assumption
into an \texttt{equivalence} between two processes
and tries to prove it.

\item When the option \texttt{[computational]} is present,
CryptoVerif then converts this assumption into the unreachability of
an event triggered when the oracles on the two sides return different
results. The unreachability of this event implies that both sides are
indistinguishable.  In this case, the random values
marked \texttt{[unchanged]} are shared between both sides, while the
others are considered independent. In principle, any mapping from the
random values of the left-hand side to the random values of the
right-hand side could allow us to prove the desired
indistinguishability property, as long as it preserves the probability
distributions; however, CryptoVerif only supports the case in which
some random values are equal on both sides and others are independent.

\end{itemize}
The goal of this query is to build modular proofs: we can prove
a property using this query, and then use it as assumption
in a subsequent proof by just copy-pasting it.

A library file (specified on the command-line by the
{\tt -lib} option) consists of a list of declarations.
Notations are summarized in Figure~\ref{fig:syntax0}
and various syntactic elements are described in 
Figures~\ref{fig:syntax1}, \ref{fig:syntax2}, \ref{fig:syntax3},
and~\ref{fig:syntax3ch}.

Processes are described in a process calculus.
In this calculus, terms represent computations on bitstrings. 
Simple terms consist
of the following constructs:
\begin{itemize}

\item A term between parentheses $\texttt{(}M\texttt{)}$
allows to disambiguate syntactic expressions.

\item An identifier can be either a constant symbol $f$
(declared by \texttt{const} or \texttt{fun} without argument)
or a variable identifier.

\item The function application $f\texttt{(}M_1, \ldots, M_n\texttt{)}$
applies function $f$ to the result of $M_1, \ldots, M_n$.

\item The tuple application $\texttt{(}M_1, \ldots, M_n\texttt{)}$
builds a tuple from $M_1, \ldots, M_n$ (corresponds to the concatenation
of $M_1, \ldots, M_n$ with length and type indications so that 
$M_1, \ldots, M_n$ can be recovered without ambiguity).
This is allowed only for $n \neq 1$, so that it is distinguished
from parenthesing.

\item The array access $x\texttt{[}M_1, \ldots, M_n\texttt{]}$
returns the cell of indices $M_1, \ldots, M_n$ of array $x$.

\item \texttt{=}, \texttt{<>}, \texttt{||}, \texttt{\&\&}
are function symbols that represent equality and inequality tests, 
disjunction and conjunction. They use the infix notation, but
are otherwise considered as ordinary function symbols.

\end{itemize}
Terms contain further constructs \texttt{<-R}, \texttt{<-}, \texttt{event}, \texttt{event\string_abort}, \texttt{if}, \texttt{find},
\texttt{let}, \texttt{new}, \texttt{insert}, and \texttt{get} which are similar to the corresponding
constructs of output processes but return a bitstring instead of
executing a process. 
They are not allowed to occur in \texttt{defined} conditions of \texttt{find}.
% and in input channels.
The constructs \texttt{event} and \texttt{insert} are not allowed to 
occur in conditions of \texttt{find} or {\tt get}.
We refer the reader to the description of 
processes below for a fully detailed explanation.
\begin{itemize}

\item $\Resa{x}{T}\texttt{;}M$ chooses a new
random number in type $T$, stores it in $x$, and returns the result 
of $M$.

$\Resb{x}{T}\texttt{;}M$ is equivalent to $\Resa{x}{T}\texttt{;}M$.

\item $\texttt{let }p \texttt{ = }M\texttt{ in }M'\texttt{ else }M''$
tries to decompose the term $M$ according to pattern $p$.
In case of success, returns the result of $M'$, otherwise 
the result of $M''$. 

The pattern $p$ can be:
\begin{itemize}

\item $x[\texttt{:}T]$ variable, possibly with its type. Matches any bitstring
(in type $T$), and stores it in $x$.

\item $f\texttt{(}p_1, \ldots, p_n\texttt{)}$ 
where the function symbol $f$ is declared 
\texttt{[data]}. Matches bitstrings $M$ equal to $f(M_1, \ldots, M_n)$
for some $M_1, \ldots, M_n$ that match $p_1, \ldots, p_n$.
(The poly-injectivity of $f$ allows us to compute possible
values $M_1, \ldots, M_n$ of its arguments from the value of $M$, and to check
whether $M$ is equal to the resulting value of $f(M_1, \ldots, M_n)$.) 

\item $\texttt{(}p_1, \ldots, p_n\texttt{)}$ tuples, which are particular \texttt{[data]}
functions encoding unambiguously the values of $p_1, \ldots, p_n$
and their type.

\item $\texttt{=}M'$ matches a bitstring equal to $M'$.

\end{itemize}
When $p$ is a variable, the \texttt{else}
branch can be omitted (it cannot be executed).

\item $x[:T] \texttt{ <- }M\texttt{;}M'$
stores the result of $M$  in $x$
and returns the result of $M'$. This is equivalent
to the construct 
$\texttt{let }x[:T] \texttt{ = }M\texttt{ in }M'$.

\item \texttt{if} $\mathit{cond}$ \texttt{then} $M$ \texttt{else} $M'$ is 
syntactic sugar for $\texttt{find suchthat }\mathit{cond}$ \texttt{then} $M$ \texttt{else} $M'$.
It returns the result of $M$ if the condition $\mathit{cond}$ evaluates to \texttt{true} and of $M'$ if $\mathit{cond}$ evaluates to \texttt{false}.

\item 
$\texttt{find}\ FB_1 \texttt{ orfind }\ldots\texttt{ orfind }FB_m \texttt{ else } M$ where $FB_j = u_{j1} \eqt  i_{j1} \leqt  n_{j1}, \ldots, u_{jm_j} \eqt  i_{jm_j} \leqt  n_{jm_j}$ $\texttt{suchthat}$ $cond_j$ $\texttt{then}$ $M_j$
evaluates the conditions
$cond_j$ for each $j$ and
each value of $i_{j1}, \ldots, i_{jm_j}$ in $[1, n_{j1}] 
\times \ldots \times [1, n_{jm_j}]$.
If none of these conditions is \texttt{true}, it returns the result of $M$.
Otherwise, it chooses randomly with (almost) uniform probability
one $j$ and one value of $i_{j1}, \ldots, i_{jm_j}$
such that the corresponding condition is \texttt{true},
stores it in $u_{j1}, \ldots, u_{jm_j}$ and returns 
the result of $M_j$.
See the explanation of the {\tt find} process below for more details.

\item $\texttt{event }e\texttt{(}M_1, \ldots, M_n\texttt{);}P$ executes the
event $e\texttt{(}M_1, \ldots, M_n\texttt{)}$, then executes $P$.
Events serve in recording the execution of certain parts of the program
for using them in queries. The symbol $e$ must have been declared
by an \texttt{event} declaration.

\item \texttt{event\string_abort $e$} executes event $e$ and aborts the game.
It is intended to be used in the right-hand side
of the definitions of some cryptographic primitives. (See also
the \texttt{equiv} declaration; events in the right-hand side can be
used when the simulation of left-hand side by the right-hand side
fails. CryptoVerif is going to find a bound for the probability that the event is
executed and include it in the probability of success of an attack.)

\item \texttt{insert} $\mathit{tbl}\texttt{(}M_1, \ldots, M_n\texttt{)}; M$
inserts the tuples $(M_1, \ldots, M_n)$ in the table $\mathit{tbl}$, 
then returns the result of $M$.
The table $\mathit{tbl}$ must have been declared with the appropriate
types using the $\texttt{table}$ declaration.

\item \texttt{get} $\mathit{tbl}\texttt{(}p_1, \ldots, p_n\texttt{)}$ \texttt{suchthat} $M$ \texttt{in} $M'$ \texttt{else} $M''$ tries to find an element of the table $\mathit{tbl}$ that matches the patterns $p_1, \ldots, p_n$ and such that $M$ is true. If it succeeds, it returns the result of $M'$ with the variables of $p_1, \ldots, p_n$ bound to that element of the table. If several elements match, one of them is chosen randomly with (almost) uniform probability. If no element matches, it returns the result of $M''$. 

When \texttt{suchthat} $M$ is omitted, it is equivalent to \texttt{suchthat} $\mathit{true}$. 

A variant of {\tt get} is {\tt get[unique]}, which guarantees that at most one element of the table satisfies the condition, except in cases of negligible probability.

Internally, \texttt{get} is converted into \texttt{find} by CryptoVerif.

\end{itemize}

\ifchannels
The calculus distinguishes two kinds of processes: input processes
$\iprocess$ are ready to receive a message on a channel; 
output processes $\oprocess$ 
output a message on a channel after executing some internal computations.
When an input or output process is an identifier, it is substituted with 
its value defined by a \texttt{let} declaration.
\else
The calculus distinguishes two kinds of processes: oracle definitions
$\iprocess$ define new oracles; oracle bodies $\oprocess$ return a
result after executing some internal computations.  When a process
(oracle definition or oracle body) is an identifier, it is substituted
with its value defined by a \texttt{let} declaration.
\fi
Processes allow parenthesing for disambiguation.

Let us first describe \ifchannels input processes\else oracle definitions\fi:
\begin{itemize}

\item $\mathit{proc}(M_1, \dots, M_n)$ is replaced with $P\{M_1/x_1, \dots, M_n/x_n\}$ when $\mathit{proc}$ is declared by $\texttt{let}\ \mathit{proc}(x_1:T_1, \dots, x_n:T_n)\texttt{ = }P\texttt{.}$
where $P$ is an input process.
The terms $M_1, \dots, M_n$ must contain only variables, replication indices, and function applications.

\item \texttt{0} does nothing.

\item $Q \texttt{ | } Q'$ is the parallel composition of $Q$ and $Q'$.

\item $\texttt{!}i\leqt N\ Q$ represents $N$ copies of $Q$ in
parallel each with a different value of $i \in [1,N]$.  The identifier
$N$ must have been declared by $\texttt{param }N$.  The identifier $i$
cannot be referred to explicitly in the process; it is used only
implicitly as array index of variables defined under the replication
$\texttt{!}i\leqt N$. The replication $\texttt{!}i\leqt N$ can be 
abbreviated $\texttt{!} N$.

When a program point is under replications $\texttt{!}i_1\leqt N_1$,
\ldots, $\texttt{!}i_n\leqt N_n$, the \emph{current replication
indices} at that point are $i_1, \ldots, i_n$.

$\texttt{foreach }i\leqt N\texttt{ do }Q$ is equivalent to
$\texttt{!}i\leqt N\ Q$.

\ifchannels
\item The semantics of the input 
%$\cinput{c[M_1, \ldots, M_l]}{x_1[i_1, \ldots, i_m]:T_1, \ldots, x_k[i_1, \ldots, i_m]:T_k};P$ 
$\texttt{in(}\nonterm{channel}\texttt{,}\nonterm{pattern}\texttt{);}\oprocess$
will be explained below together with the
semantics of the output. 

\else
\item $O\texttt{(}p_1, \ldots, p_n\texttt{) := }P$ defines an oracle
$O$ taking arguments $p_1, \ldots, p_n$, and returning the result of
the oracle body $P$. The patterns $p_1, \ldots, p_n$ are as in the
\texttt{let} construct above, except that variables in $p$ that are
not under a function symbol $f(\ldots)$ must be declared with their
type.

\fi

\end{itemize}
Note that the construct \ifchannels $\textbf{newChannel }c;Q$ \else
$\textbf{newOracle }c;Q$ \fi used in research papers
is absent from the implementation: this construct is useful in the proof
of soundness of CryptoVerif, but not essential for encoding games
that CryptoVerif manipulates.

Let us now describe output processes:
\begin{itemize}

\item $\mathit{proc}(M_1, \dots, M_n)$ is replaced with $\texttt{let}$ $x_1 = M_1$ $\texttt{in}$ \dots $\texttt{let}$ $x_n = M_n$ $\texttt{in}$ $P$ when $\mathit{proc}$ is declared by $\texttt{let}\ \mathit{proc}(x_1:T_1, \dots, x_n:T_n)\texttt{ = }P\texttt{.}$ where $P$ is an output process.

\ifchannels
\item {\yield} yields control to another process, by outputting
an empty message on channel \textit{yield}. It can be understood
as an abbreviation for $\texttt{out(}\textit{yield}\texttt{,());0}$.
\else
\item {\yield} terminates the oracle, returning control to the caller.
\fi

\item $\texttt{event }e\texttt{(}M_1, \ldots, M_n\texttt{);}P$ executes the
event $e\texttt{(}M_1, \ldots, M_n\texttt{)}$, then executes $P$.
Events serve in recording the execution of certain parts of the program
for using them in queries. The symbol $e$ must have been declared
by an \texttt{event} declaration.

\item {\tt event\string_abort} $e$ executes event $e$ and terminates the game. 
(Nothing can be executed after
this instruction, neither by the protocol nor by the adversary.)
The symbol $e$ must have been declared
by an \texttt{event} declaration, without any argument.

\item $\Resa{x}{T}\texttt{;}P$ or $\Resb{x}{T}\texttt{;}P$ chooses a new
random number in type $T$, stores it in $x$, and executes $P$.
$T$ must be declared with option {\tt fixed}, {\tt bounded}, or {\tt nonuniform}.
Each such type $T$ comes with an associated default probability distribution $D_T$;
the random number is chosen according to that distribution.
The time for generated random numbers in that distribution
is bounded by $\texttt{time(new }T\texttt{)}$ or equivalently
$\texttt{time(<-R }T\texttt{)}$.
\begin{itemize}

\item When the type $T$ is {\tt nonuniform}, the default probability 
distribution $D_T$ for type $T$ may be non-uniform. It is left unspecified.
(Notice that random bitstrings with non-uniform distributions can also
be obtained by applying a function to a random bitstring choosen 
uniformly among a finite set of bitstrings, chosen in another type.)

\item When the type $T$ is {\tt fixed}, it consists of the set of all 
bitstrings of a certain length $n$.  Probabilistic Turing machines can
return uniformly distributed random numbers in such types, in bounded
time.
If $T$ is not marked {\tt nonuniform}, the default probability 
distribution $D_T$ for $T$ is the uniform distribution.

\item For other {\tt bounded} types $T$, probabilistic bounded-time Turing 
machines can choose random numbers with a distribution as close as we
wish to uniform, but may not be able to produce exactly a uniform
distribution. If $T$ is not marked {\tt nonuniform}, 
the default probability distribution $D_T$ is such that its distance
to the uniform distribution is at most $\epsrand{T}$. The distance between
two probability distributions $D_1$ and $D_2$ for type $T$ is
\[d(D_1, D_2) = \sum_{a \in T} | \Pr[X_1 = a] - \Pr[X_2 = a] |\]
where $X_i$ ($i = 1, 2$) is a random variable of distribution $D_i$.

For example, a possible algorithm to obtain a random integer in $[0,
m-1]$ is to choose a random integer $x'$ uniformly among $[0, 2^k-1]$
for a certain $k$ large enough and return $x' \bmod m$. 
By euclidian division, we have $2^k = qm+r$ with $r \in [0,m-1]$.
With this algorithm
\[\Pr[x = a] = \begin{cases}
\frac{q+1}{2^k} &\text{if }a \in [0,r-1]\\
\frac{q}{2^k} &\text{if }a \in [r,m-1]
\end{cases}\]
so
\[\left|\Pr[x = a] -\frac{1}{m}\right| = \begin{cases}
\frac{q+1}{2^k} - \frac{1}{m}&\text{if }a \in [0,r-1]\\
\frac{1}{m} - \frac{q}{2^k}&\text{if }a \in [r,m-1]
\end{cases}\]
Therefore
\[\begin{split}
d(D_T, \mathit{uniform}) 
&=  \sum_{a \in T} \left| \Pr[x = a] - \frac{1}{m} \right|
= r\left(\frac{q+1}{2^k} - \frac{1}{m}\right) - (m-r)\left(\frac{1}{m} - \frac{q}{2^k}\right)\\
&=\frac{2r(m-r)}{m.2^k} \leq \frac{m}{2^k}
\end{split}\]
%If r <= m/2, we upper bound r(m-r) by m/2.m
%If r >= m/2, m-r <= m/2, so we upper bound r(m-r) by m.m/2 
so we can take $\epsrand{T} = \frac{m}{2^k}$. A given precision of $\epsrand{T} = \frac{1}{2^{k'}}$ can be obtained by choosing $k = k' + \text{number of bits of }m$ random bits.

When \texttt{ignoreSmallTimes} is set to a value greater than 0
(which is the default),
the time for random number generations and the probability
$\epsrand{T}$ are ignored, to make probability formulas 
more readable.

\end{itemize}

\item $\texttt{let }p \texttt{ = }M\texttt{ in }P\texttt{ else }P'$
tries to decompose the term $M$ according to pattern $p$.
In case of success, executes $P$, otherwise executes $P'$.

The pattern $p$ can be:
\begin{itemize}

\item $x[\texttt{:}T]$ variable, possibly with its type. Matches any bitstring
(in type $T$), and stores it in $x$.

\item $f\texttt{(}p_1, \ldots, p_n\texttt{)}$ 
where the function symbol $f$ is declared 
\texttt{[data]}. Matches bitstrings $M$ equal to $f(M_1, \ldots, M_n)$
for some $M_1, \ldots, M_n$ that match $p_1, \ldots, p_n$.
(The poly-injectivity of $f$ allows us to compute possible
values $M_1, \ldots, M_n$ of its arguments from the value of $M$, and to check
whether $M$ is equal to the resulting value of $f(M_1, \ldots, M_n)$.) 

\item $\texttt{(}p_1, \ldots, p_n\texttt{)}$ tuples, which are particular \texttt{[data]}
functions encoding unambiguously the values of $p_1, \ldots, p_n$
and their type.

\item $\texttt{=}M'$ matches a bitstring equal to $M'$.

\end{itemize}
The \texttt{else} clause is never executed when the pattern
is simply a variable.
When $\texttt{else }P'$ is omitted, it is equivalent to \texttt{else} \yield.
Similarly, when $\texttt{in }P$ is omitted, it is equivalent to 
\texttt{in} \yield.

\item $x[\texttt{:}T] \texttt{ <- }M\texttt{;}P$
stores the result of term $M$ in $x$ and executes $P$.
$M$ must be of type $T$ when $T$ is mentioned.
This is equivalent to the construct $\texttt{let } x[\texttt{:}T] 
\texttt{ = }M\texttt{ in }P$.

\item \texttt{if} $\mathit{cond}$ \texttt{then} $P$ \texttt{else} $P'$ is 
syntactic sugar for $\texttt{find suchthat }\mathit{cond}$ \texttt{then} $P$ \texttt{else} $P'$.
It executes $P$ if the condition $\mathit{cond}$ evaluates to \texttt{true} and $P'$ if $\mathit{cond}$ evaluates to \texttt{false}.
When the \texttt{else} clause is omitted, it is implicitly \texttt{else} \yield.
(\texttt{else 0} would not be syntactically correct.) 

\item 
Next, we explain the process 
$\texttt{find}\ FB_1 \texttt{ orfind }\ldots\texttt{ orfind }FB_m \texttt{ else } P$ where each branch $FB_j$ is $FB_j = u_{j1} \eqt  i_{j1} \leqt  n_{j1}, \ldots, u_{jm_j} \eqt  i_{jm_j} \leqt  n_{jm_j}$ $\texttt{suchthat}$ $cond_j$ $\texttt{then}$ $P_j$.

A simple example is the following:
$\texttt{find }u \eqt  i \leqt  n$ \texttt{suchthat} $\texttt{defined(}x[i]\texttt{) \&\& }x[i] = a$ \texttt{then} $P'$ \texttt{else} $P$
tries to find an index $i$ such that $x[i]$ is defined and
$x[i] = a$, and when such an $i$ is found,
it stores that $i$ in $u$ and executes $P'$;
otherwise, it executes $P$.
In other words, this $\texttt{find}$ construct looks for the value
$a$ in the array $x$, and when $a$ is found, it stores in
$u$ an index such that $x[u] = a$. Therefore, the $\texttt{find}$ construct
allows us to access arrays, which is key for our purpose.

More generally, $\texttt{find}\ u_{1} \eqt  i_1 \leqt  n_{1}, \ldots, u_{m} \eqt  i_m \leqt  n_{m}$ $\texttt{suchthat}$ $\texttt{defined(}M_{1}, \ldots, M_{l}\texttt{) \&\& } M$ \texttt{then} $P'$ \texttt{else} $P$ tries to find values of $i_1, \ldots, i_m$ for which
$M_1, \ldots, M_l$ are defined and $M$ is true. In case of success, it
stores the values of $i_1, \ldots, i_m$ in $u_1, \ldots, u_m$
executes $P'$. In case of failure, it executes $P$.

This is further generalized to $m$ branches: 
$\texttt{find}\ FB_1 \texttt{ orfind }\ldots\texttt{ orfind }FB_m \texttt{ else } P$
where $FB_j = u_{j1} \eqt  i_{j1} \leqt  n_{j1}, \ldots, u_{jm_j} \eqt  i_{jm_j} \leqt  n_{jm_j}$ $\texttt{suchthat}$ $\texttt{defined(}M_{j1}, \ldots, M_{jl_j}\texttt{) \&\& }M_j$ $\texttt{then}$ $P_j$
tries to find a branch $j$ in $[1,m]$ such that there are 
values of $i_{j1}, \ldots, i_{jm_j}$ for which 
$M_{j1}, \ldots, M_{jl_j}$ are defined and $M_j$ is true. In case of 
success, it stores the value of $i_{j1}, \ldots, i_{jm_j}$ in $u_{j1}, \ldots, u_{jm_j}$ and executes $P_j$.
In case of failure for all branches, it executes $P$. 
More formally, it evaluates the conditions
$cond_j = \texttt{defined(}M_{j1}, \ldots, M_{jl_j}\texttt{) \&\& }M_j$ for each $j$ and
each value of $i_{j1}, \ldots, i_{jm_j}$ in $[1, n_{j1}] 
\times \ldots \times [1, n_{jm_j}]$.
If none of these conditions is \texttt{true}, it executes $P$.
Otherwise, it chooses randomly with almost uniform 
probability\footnote{Precisely, the distance between the distribution actually
used for choosing $j, i_{j1}, \ldots, i_{jm_j}$ and the uniform
distribution is at most $\texttt{eps\_find}/2$. See the explanation of $\Resa{x}{T}$
for details on how to achieve this.}
one $j$ and one value of $i_{j1}, \ldots, i_{jm_j}$
such that the corresponding condition is \texttt{true},
stores that value in $u_{j1}, \ldots, iu_{jm_j}$ and executes $P_j$.

In the general case, the conditions $cond_j$ are of the form
$\texttt{defined(}M_1, \ldots, M_l\texttt{)}\ [\texttt{\&\& }M]$ or simply $M$.
The condition $\texttt{defined(}M_1, \ldots, M_l\texttt{)}$ means that
$M_1, \ldots, M_l$ are defined.
At least one of the two conditions $\texttt{defined}$ or $M$ must be
present. Omitted $\texttt{defined}$ conditions are considered empty;
when $M$ is omitted, it is considered \texttt{true}. 

The variables $i_{j1}, \ldots, i_{jm_j}$ are considered as replication indices, and are used in the $\texttt{defined}$ condition and in $M_j$: they are temporary variables that are used as loop indices to look for indices that satisfy the desired conditions. Once suitable indices are found, their value is stored in $u_{j1}, \ldots, u_{jm_j}$ and the \texttt{then} branch is executed using these variables. It is possible to make array accesses to $u_{j1}, \ldots, u_{jm_j}$ (such as $u_{j1}[M_1, \ldots, M_k]$) elsewhere in the game, which is not possible for $i_{j1}, \ldots, i_{jm_j}$.

As an abbreviation, one may write $FB_j = u_{j1} \leqt  n_{j1}, \ldots, u_{jm_j} \leqt  n_{jm_j}$ $\texttt{suchthat}$ $\texttt{defined(}M_{j1}, \allowbreak\ldots,\allowbreak M_{jl_j}\texttt{)}$ $\&\&$ $M_j$ $\texttt{then}$ $P_j$. In this case, the same identifier $u_{jk}$ is used for both the variable and the associated replication index $i_{jk}$.


A variant of {\tt find} is {\tt find[unique]}. 
Consider the process 
$\texttt{find[unique]}$ $FB_1$ \texttt{orfind} \ldots \texttt{orfind} $FB_m$ \texttt{else} $P$
where $FB_j = u_{j1} \eqt  i_{j1} \leqt  n_{j1}, \ldots, u_{jm_j} \eqt  i_{jm_j} \leqt  n_{jm_j}$ $\texttt{suchthat}$ $\texttt{defined(}M_{j1},\allowbreak \ldots, \allowbreak M_{jl_j}\texttt{)}$ $\texttt{\&\&}$ $M_j$ \texttt{then} $P_j$.
When there are several values of $j, i_{j1}, \ldots, i_{jm_j}$ for which 
$M_{j1}, \ldots, M_{jl_j}$ are defined and $M_j$ is true, this process executes an event $\mathsf{NonUnique}$ and aborts the game. In all other cases, it behaves as {\tt find}.
Intuitively, {\tt find[unique]} should be used when there is a negligible probability of finding several suitable values of $j, i_{j1}, \ldots, i_{jm_j}$. The construct {\tt find[unique]} is typically not used in the initial game. (One would have to prove manually that there is indeed a negligible probabibility of finding several suitable values of $j, i_{j1}, \ldots, i_{jm_j}$. CryptoVerif displays a warning if {\tt find[unique]} occurs in the initial game.) However, {\tt find[unique]} is used in the specification of cryptographic primitives, in the right-hand of equivalences specified by \texttt{equiv}. 

\item \texttt{insert} $\mathit{tbl}\texttt{(}M_1, \ldots, M_n\texttt{)}; P$
inserts the tuples $(M_1, \ldots, M_n)$ in the table $\mathit{tbl}$, 
then executes $P$.
The table $\mathit{tbl}$ must have been declared with the appropriate
types using the $\texttt{table}$ declaration.

\item \texttt{get} $\mathit{tbl}\texttt{(}p_1, \ldots, p_n\texttt{)}$ \texttt{suchthat} $M$ \texttt{in} $P$ \texttt{else} $P'$ tries to find an element of the table $\mathit{tbl}$ that matches the patterns $p_1, \ldots, p_n$ and such that $M$ is true. If it succeeds, it executes $P$ with the variables of $p_1, \ldots, p_n$ bound to that element of the table. If several elements match, one of them is chosen randomly with (almost) uniform probability. If no element matches, it executes $P'$. 

When \texttt{else} $P'$ is omitted, it is equivalent to \texttt{else} \yield. When \texttt{suchthat} $M$ is omitted, it is equivalent to \texttt{suchthat} $\mathit{true}$. 

A variant of {\tt get} is {\tt get[unique]}, which guarantees that at most one element of the table satisfies the condition, except in cases of negligible probability.

Internally, \texttt{get} is converted into \texttt{find} by CryptoVerif.

\ifchannels
\item
Finally, let us explain the output $\texttt{out(}c\texttt{[}M_1,
\ldots, M_l\texttt{],}N\texttt{);}Q$.  A channel $c\texttt{[}M_1,
\ldots, M_l\texttt{]}$ consists of both a channel name $c$ (declared
by $\texttt{channel }c$) and a tuple of terms $M_1, \ldots, M_l$.  Terms
$M_1, \ldots, M_l$ are intuitively analogous to IP addresses and ports
which are numbers that the adversary may guess.  Two channels are
equal when they have the same channel name and terms that evaluate to
the same bitstrings.
%
A semantic configuration always consists of a single output process
(the process currently being executed) and several input processes.
When the output process executes $\texttt{out(}c\texttt{[}M_1, \ldots,
M_l\texttt{],}N\texttt{);}Q$, one looks for an input on the same
channel in the available input processes. If no such input process is
found, the process blocks.  Otherwise, one such input process
$\texttt{in(}c\texttt{[}M'_1, \ldots, M'_l\texttt{],}p\texttt{);}P $
is chosen randomly with (almost) uniform probability. The communication is then
executed: the output message $N$ is evaluated, its result is truncated
to the maximum length of bitstrings on channel $c$, the obtained
bitstring is matched against pattern $p$.  Finally, the output process
$P$ that follows the input is executed. The input process $Q$ that
follows the output is stored in the available input processes for
future execution.

Patterns $p$ are as in the \texttt{let} process, except that
variables in $p$ that are not under a function symbol $f(\ldots)$
must be declared with their type.

In the game as given to CryptoVerif, the channel can be either
$c[i_1, \ldots, i_n]$ where $i_1, \ldots, i_n$ are the current
replication indices at the considered input or output, or just a
channel name $c$, as an abbreviation for $c[i_1, \ldots, i_n]$.  It is
recommended to use as channel a different channel name for each input
and output. Then the adversary has full control over the network: it
can decide precisely from which copy of which input it receives a
message and to which copy of which output it sends a message, by using
the appropriate channel name and value of the replication indices.

Note that the syntax requires an output
to be followed by an input process, as in~\cite{Laud05}. If one
needs to output several messages consecutively, one can simply
insert fictitious inputs between the outputs. The adversary can
then schedule the outputs by sending messages to these inputs.

\else

\item 
$\texttt{return(}N_1, \ldots, N_l\texttt{);}Q$ terminates the oracle,
returning the result of the terms $N_1, \ldots, N_l$. Then, it makes
available the oracles defined in $Q$.

\fi

\end{itemize}

In this calculus, all variables are implicitly arrays.  When a
variable $x$ is defined (by \texttt{new}, 
\texttt{<-R}, \texttt{<-}, 
\texttt{let}, \texttt{find},
\ifchannels \texttt{in} \else and oracle definitions\fi) 
under replications 
$\texttt{!}i_1\leqt N_1$, \ldots, $\texttt{!}i_n\leqt N_n$, 
$x$ has implicitly indices $i_1,
\ldots, i_n$: $x$ stands for $x[i_1, \ldots, i_n]$. Arrays allow us to
have full access to the state of the process. Arrays can be read using
\texttt{find}.
%
Similarly, when $x$ is used with $k < n$ indices the missing $n-k$ indices are
implicit: $x[u_1, \ldots, u_k]$ stands for $x[i_1, \ldots, i_{n-k},
u_1, \ldots, u_k]$ where $i_1, \ldots, i_{n-k}$ must be the $n-k$
first replication indices both at the creation of $x$ and at the usage
$x[u_1, \ldots, u_k]$. (So the usage and creation of $x$ must
be under the same $n-k$ top-most replications.)
%
\ifchannels\else
When an oracle $O$ is defined under $\texttt{foreach }i_1\leqt N_1$, 
\ldots, $\texttt{foreach }i_n\leqt N_n$, it also implicitly
defines $O[i_1, \ldots, i_n]$.
\fi

In the initial game, several variables may be defined with the same
name, but they are immediately renamed to different names, so that
after renaming, each variable is defined once.  When several variables
are defined with the same name, they can be referenced only under
their definition without explicit array indices, because for other
references, we would not know which variable to reference after
renaming.

In subsequent games created by CryptoVerif, a variable may be defined
at several occurrences, but these occurrences must be in different
branches of \texttt{if}, \texttt{find}, or \texttt{let}, so that
they cannot be executed with the same value of the array indices.
This constraint guarantees that each array cell is defined at most once.

Each usage of $x$ must be either:
\begin{itemize}

\item $x$ without array index syntactically under its definition.
(Then $x$ is implicitly considered to have as indices the current
replication indices at its definition.)

\item $x$ possibly with array indices inside the \texttt{defined}
condition of a find.

\item $x[M_1, \ldots, M_n]$ in $M$  in a find branch
$\ldots\texttt{ suchthat defined(}M'_1, \ldots, M'_l\texttt{) \&\& }M
\texttt{ then }\ldots$, such that $x[M_1, \ldots, M_n]$
is a subterm of $M'_1, \ldots, M'_l$. 

\item $x[M_1, \ldots, M_n]$ in $P$  in a find branch
$u_1 \eqt  i_1 \leqt  n_1, \ldots, u_m \eqt  i_m \leqt  n_m$ $\texttt{suchthat}$ $\texttt{defined(}M'_1, \allowbreak \ldots, \allowbreak M'_l\texttt{)}$ $\texttt{\&\&}$ \ldots
$\texttt{then }P$, such that $x[M_1, \ldots, M_n] = M\{u_1/i_1,\dots,u_m/i_m\}$
and $M$ is a subterm of $M'_1, \ldots, M'_l$. 

\item $x[M_1, \ldots, M_n]$ in $M''$ in a find branch
$u_1 \eqt  i_1 \leqt  n_1, \ldots, u_m \eqt  i_m \leqt  n_m$ $\texttt{suchthat}$ $\texttt{defined(}M'_1, \allowbreak \ldots, \allowbreak M'_l\texttt{)}$ $\texttt{\&\&}$ \ldots
$\texttt{then }M''$, such that $x[M_1, \ldots, M_n] = M\{u_1/i_1,\dots,u_m/i_m\}$
and $M$ is a subterm of $M'_1, \ldots, M'_l$. 

\end{itemize}
These syntactic constraints guarantee that a variable is accessed
only when it is defined. Moreover, the variables defined in
conditions of {\tt find} or in patterns or conditions of {\tt get}
must not have array accesses (that is, accesses corresponding to the
last four cases above).

Finally, the calculus is equipped with a type system.
To be able to use variables outside their scope (by \texttt{find}),
the type checking algorithm works in two passes. 

In the first pass, 
it collects the type of each variable: when a variable $x$ is
defined with type $T$ under 
replications $\texttt{!}N_1$, \ldots, $\texttt{!}N_n$,
$x$ has type $[1, N_1] \times \ldots \times [1, N_n] \rightarrow T$.
When the type of $x$ is not explicitly given in its declaration
(\ifchannels in \texttt{<-} or in patterns in \texttt{let} or \texttt{in}\else in \texttt{<-} or 
in patterns in \texttt{let} or oracle definitions\fi), its type is left undefined
in this pass, and $x$ cannot be used outside its scope.

In the second pass, the type system checks the following requirements:
%
In $x\texttt{[}M_1, \ldots, M_m\texttt{]}$, $M_1, \ldots, M_m$ must be of the suitable 
interval type, that is, a suffix of the types of replication indices
at the definition of $x$.
%
In $f\texttt{(}M_1, \ldots, M_m\texttt{)}$, if $f$ has been declared
by $\texttt{fun }f\texttt{(}T_1, \ldots, T_m\texttt{):}T$, $M_j$ must
be of type $T_j$, and $f(M_1, \ldots, M_m)$ is then of type $T$.
%$T_1, \ldots, T_m$ must be bitstring types.
%
In $\texttt{(}M_1, \ldots, M_n\texttt{)}$, $M_j$ can be of any
bitstring type (that is, not an index type $[1, N]$), 
and the result is of type \texttt{bitstring}.
%
In $M_1 \texttt{ = }M_2$ and $M_1 \texttt{ <> }M_2$, $M_1$ and $M_2$
must be of the same type, and the result is of type
$\texttt{bool}$. In $M_1 \texttt{ || }M_2$ and $M_1 \texttt{ \&\&
}M_2$, $M_1$ and $M_2$ must be of type $\texttt{bool}$ and the result
is of type $\texttt{bool}$.
%
The type system requires each subterm to be well-typed. 
%Event
Furthermore, in $\texttt{event }e\texttt{(}M_1, \ldots,
M_n\texttt{)}$, if $e$ has been declared by $\texttt{event
}e\texttt{(}T_1, \ldots, T_n\texttt{)}$, $M_j$ must be of type $T_j$.
%New
In $\Resa{x}{T}$ or $\Resb{x}{T}$, $T$ must be declared with option {\tt bounded} (or {\tt fixed}).
%If
In $\texttt{if }M\texttt{ then }\ldots\texttt{ else }\ldots$, 
$M$ must be of type $\texttt{bool}$.
%Find
Similarly, for
\[\texttt{find }\ldots\texttt{ orfind } \ldots \texttt{ suchthat defined(}
\ldots\texttt{) \&\& }M\texttt{ then }\ldots\]
$M$ must be of type $\texttt{bool}$.
%Let
In $\texttt{let }p\texttt{ = }M\texttt{ in }\ldots$, $M$ and $p$ must
be of the same type. For function application and tuple patterns, the
typing rule is the same as for the corresponding terms.  The pattern
$x:T$ is of type $T$; the pattern $x$ can be of any bitstring type,
determined by the usage of $x$ (when the pattern $x$ is used as
argument of a tuple pattern, its type is \texttt{bitstring}); the
pattern $\texttt{=}M$ is of the type of $M$.
%Out
\ifchannels
In $\texttt{out(}c\texttt{[}M_1, \ldots, M_n\texttt{],}M\texttt{)}$,
$M$ must be of a bitstring type.
\else
In $\texttt{return(}M_1, \ldots, M_n\texttt{)}$,
$M_j$ must be of a bitstring type $T_j$ for all $j \leq n$
and that return instruction is said to be of type $T_1 \times \ldots 
\times T_n$.
All return instructions in an oracle body $P$ (excluding return
instructions that occur in oracle definitions $Q$ in processes of the form 
$\texttt{return(}M_1, \ldots, M_n\texttt{);}Q$) must be of the same
type, and that type is said to be the type of the oracle body $P$.
%
For each oracle definition $O\texttt{(}p_1, \ldots, p_m\texttt{) :=
}P$ under $\texttt{foreach }i_1\leqt N_1$, \ldots,
$\texttt{foreach }i_n\leqt N_n$, the oracle $O$ is said to be of
type $[1, N_1] \times \ldots \times [1, N_n] \rightarrow T'_1 \times
\ldots \times T'_m \rightarrow T_1 \times \ldots \times T_n$ where
$p_j$ is of type $T'_j$ for all $j \leq m$ and $P$ is of type $T_1
\times \ldots \times T_n$. When an oracle has several definitions,
it must be of the same type for all its definitions. Furthermore,
definitions of the same oracle $O$ must not occur on both sides
of a parallel composition $Q \texttt{|} Q'$ (so that several definitions
of the same oracle cannot be simultaneously available).
\fi


A declaration can be:
\begin{itemize}

\item ${\tt set\ } \nonterm{parameter} \texttt{ = } \nonterm{value}{\tt .}$

This declaration sets the value of configuration parameters.
The following parameters and values are supported:

\begin{itemize}

\item \texttt{set allowUndefinedVar = false.}\\
\texttt{set allowUndefinedVar = true.}

By default (\texttt{allowUndefinedVar = false}), variables in
\texttt{defined} conditions must be defined somewhere in the game.
The setting \texttt{allowUndefinedVar = true} allows
\texttt{defined} conditions with variables that are defined
nowhere. The corresponding branch of \texttt{find} is then
removed immediately, since the \texttt{defined} condition does not hold.
This setting is useful to parse intermediate games generated by
CryptoVerif, because such impossible \texttt{defined} conditions
may occur in these games.

\item \texttt{set allowUnprovedUnique = false.}\\
\texttt{set allowUndefinedVar = true.}

By default (\texttt{allowUnprovedUnique = false}), CryptoVerif tries to
prove that the \texttt{[unique]} annotations of \texttt{find} and \texttt{get}
in the initial game are correct, and it is an error if it fails.
With \texttt{allowUndefinedVar = true}, a warning is emitted if this proof
fails, and CryptoVerif continues trying to prove the protocol. It is then
the responsability of the user to make sure that the \texttt{[unique]} annotations
are correct. This setting is not recommended as it may lead to unsound
results in case the \texttt{[unique]} annotations are incorrect.

\item \texttt{set diffConstants = true.}\\
\texttt{set diffConstants = false.}

When {\tt true}, different constant symbols are assumed to have a
different value. When {\tt false}, CryptoVerif does not make this
assumption.

\item \texttt{set constantsNotTuple = true.}\\
\texttt{set constantsNotTuple = false.}

When {\tt true}, constant symbols are assumed to be different from the
result of applying a tuple function to any argument. When {\tt false},
CryptoVerif does not make this assumption.

\item \texttt{set expandAssignXY = true.}\\
\texttt{set expandAssignXY = false.}

When {\tt true}, CryptoVerif automatically removes assignments 
{\tt let x = y} or {\tt x <- y}
where {\tt x} and {\tt y} are variables by substituting {\tt y} for {\tt x}
(in the transformation {\tt remove\string_assign useless})
When {\tt false}, this transformation is not performed as part of
{\tt remove\string_assign useless}.

\item \texttt{set minimalSimplifications = true.}\\
\texttt{set minimalSimplifications = false.}

When {\tt true}, simplification replaces a term with a rewritten term
only when the rewriting has used at least one rewriting rule given
by the user, not when only equalities that come from {\tt let} definitions
and other instructions in the game have been used.
When {\tt false}, a term is replaced with its rewritten form in
all cases. The latter configuration often leads to replacing
a term with a more complex one, in particular expanding {\tt let}
definitions, thus duplicating their contents.

\item \texttt{set autoMergeBranches = true.}\\
\texttt{set autoMergeBranches = false.}

When {\tt true}, the transformation {\tt merge\_branches} is applied
after simplification, to merge branches of {\tt if}, {\tt let},
and {\tt find} when all branches execute the same code.
This is useful in order to remove the test, which can remove
a use of a secret.
When {\tt false}, this transformation is not performed. 
This is useful in particular when the test has been
manually introduced in order to force CryptoVerif to
distinguish cases.

\item \texttt{set autoMergeArrays = true.}\\
\texttt{set autoMergeArrays = false.}

When {\tt true}, {\tt merge\_branches} advises {\tt merge\_arrays} commands
to make the merging of branches of {\tt if}, {\tt find}, {\tt let}
succeed more often. When {\tt false}, this advice is not
automatically given and the user should use the manual command
{\tt merge\_arrays} (defined in 
Section~\ref{sec:interact}) to perform the merging.

\item \texttt{set uniqueBranch = true.}\\
\texttt{set uniqueBranch = false.}

% We say that a {\tt find} is \emph{unique} when there is at most
% one branch and one value of the indices that we look up,
% for which the conditions are true.
When {\tt uniqueBranch = true}, the following transformation is 
enabled as part of {\tt simplify}:
if a branch of a {\tt find[unique]} is proved to succeed, 
then simplification removes all other branches of that {\tt find}.
When {\tt uniqueBranch = false}, this transformation is not performed. 

\item \texttt{set uniqueBranchReorganize = true.}\\
\texttt{set uniqueBranchReorganize = false.}

When {\tt uniqueBranchReorganize = true}, the following transformations are 
enabled as part of {\tt simplify}:
\begin{itemize}
\item
If a {\tt find[unique]} occurs in the {\tt then} branch 
of a {\tt find[unique]}, we reorganize them.

\item 
If a {\tt find[unique]} occurs in the condition of a {\tt find}, 
we reorganize them.

\end{itemize}
When {\tt uniqueBranchReorganize = false}, these transformations are not performed. 

\item \texttt{set inferUnique = false.}\\
\texttt{set inferUnique = true.}

When \texttt{inferUnique = true}, CryptoVerif tries to infer
that a \texttt{find} that is not explicitly tagged \texttt{[unique]}
is in fact unique, by showing that having several solutions
for this \texttt{find} leads to a contradiction.
When this proof succeeds, the \texttt{find} becomes {\tt find[unique]}.

When \texttt{inferUnique = false}, CryptoVerif does not try to
make such proofs and just exploits explicit \texttt{[unique]} tags.

\item \texttt{set autoSARename = true.}\\
\texttt{set autoSARename = false.}

When {\tt true}, and a variable is defined several times and
used only in the scope of its definition with the current
replication indices at that definition, each definition of
this variable is renamed to a different name, and the uses
are renamed accordingly, by the transformation {\tt remove\string_assign}.
When {\tt false}, such a renaming is not done automatically,
but in manual proofs, it can be requested specifically for each 
variable by {\tt SArename x}, where {\tt x} is the name of the variable.

\item \texttt{set autoRemoveAssignFindCond = true.}\\
\texttt{set autoRemoveAssignFindCond = false.}

When {\tt true}, the default removal of assignments performed by
CryptoVerif removes assignments on variables $x$ defined by
${\tt let}\ x = M\ {\tt in}\ ...$ inside a condition of {\tt find}.
When {\tt false}, the removal of this assignments is not
performed automatically, but in manual proofs, it can be requested 
by the command {\tt remove\string_assign\ findcond}.

\item \texttt{set autoRemoveIfFindCond = true.}\\
\texttt{set autoRemoveIfFindCond = false.}

When {\tt true}, simplification removes {\tt if} in defined conditions
of {\tt find} by transforming them into logical formulae.
When {\tt false}, this removal is not performed.

\item \texttt{set autoMove = true.}\\
\texttt{set autoMove = false.}

When {\tt true}, the transformation {\tt move all} is automatically
executed after each cryptographic transformation. This transformation
moves random number generations ({\tt new} or {\tt
<-R}) downwards as much as possible, duplicating them when crossing
a {\tt if}, {\tt let}, or {\tt find}.  (A future {\tt SArename}
transformation may then enable us to distinguish cases depending on
which of the duplicated random number generations a variable comes
from.)  It also moves assignments down in the syntax tree but without
duplicating them, when the assignment can be moved under a {\tt if},
{\tt let}, or {\tt find}, in which the assigned variable is used
only in one branch. (In this case, the assigned term is computed in
fewer cases thanks to this transformation.)

When {\tt false}, the transformation {\tt move all} is never
automatically executed.

\item \texttt{set autoExpand = true.}\\
\texttt{set autoExpand = false.}

When {\tt true}, the transformation \texttt{expand} is automatically
executed after transformations that result in a game containing
\texttt{if}, \texttt{let}, \texttt{find},
\texttt{event}, \texttt{event\_abort}, or \texttt{new} terms.
The transformation \texttt{expand} expands these terms into processes.
That leads to distinguishing the branches until the end of the process,
which may help the proof by distinguishing more cases, but
may lead to very large games.
This is also needed because some game transformations of CryptoVerif
do not support non-expanded games (\texttt{global\_dep\_anal},
\texttt{insert}, \texttt{merge\_arrays}, \texttt{merge\_branches}, \texttt{move};
furthermore, \texttt{simplify} is weaker when it is applied
to a non-expanded game, and \texttt{success} fails to prove
equivalence queries in non-expanded games and correspondence
queries when the arguments of the considered events contain
\texttt{if}, \texttt{let}, \texttt{find},
\texttt{event}, \texttt{event\_abort}, or \texttt{new}).

When {\tt false}, the transformation \texttt{expand} is never
automatically executed.

\item \texttt{set optimizeVars = false.}\\
\texttt{set optimizeVars = true.}

When {\tt true}, CryptoVerif tries to reduce the number of different
intermediate variables introduced in cryptographic
transformations. This can lead to distinguishing fewer cases,
which unfortunately often leads to a failure of the proof.
When {\tt false}, different intermediate varaibles are used for
each occurrence of the transformed expression.

\item \texttt{set interactiveMode = false.}\\
\texttt{set interactiveMode = true.}

When {\tt false}, CryptoVerif runs automatically.
When {\tt true}, CryptoVerif waits for instructions of the user
on how to perform the proof. (See Section~\ref{sec:interact}
for details on these instructions.)
%
This setting is ignored when proof instructions are included
in the input file using the \texttt{proof} command.
In this case, the instructions given in the \texttt{proof} command
are executed, without user interaction.

\item \texttt{set autoAdvice = true.}\\
\texttt{set autoAdvice = false.}

In interactive mode, when \texttt{autoAdvice = true}, execute the
advised transformations automatically. When \texttt{autoAdvice = false},
display the advised transformations, but do not execute them.
The user may then give them as instructions if he wishes.

\item \texttt{set noAdviceCrypto = false.}\\
\texttt{set noAdviceCrypto = true.}

When \texttt{noAdviceCrypto = true}, prevents the cryptographic 
transformations from generating advice. Useful mainly for debugging
the proof strategy.

\item \texttt{set noAdviceGlobalDepAnal = false.}\\
\texttt{set noAdviceGlobalDepAnal = true.}

When \texttt{noAdviceGlobalDepAnal = true}, prevents the global
dependency analysis from generating advice. Useful when the global
dependency analysis generates bad advice.

\item \texttt{set simplifyAfterSARename = true.}\\
\texttt{set simplifyAfterSARename = false.}

When \texttt{simplifyAfterSARename = true}, apply simplification after
each execution of the SArename transformation. This slows down
the system, but enables it to succeed more often.

\item \texttt{set backtrackOnCrypto = false.}\\
\texttt{set backtrackOnCrypto = true.}

When \texttt{backtrackOnCrypto = true}, use backtracking when the proof
fails, to try other cryptographic transformations. This slows down
the system considerably (so it is false by default), but enables
it to succeed more often, in particular for public-key protocols
that mix several primitives. One usage is to try first with the default
setting and, if the proof fails although the property
is believed to hold, try again with backtracking.

\item \texttt{set useKnownEqualitiesInCryptoTransform = true.}\\
\texttt{set useKnownEqualitiesInCryptoTransform = false.}

When \texttt{useKnownEqualitiesInCryptoTransform = true}, CryptoVerif
relies on known equalities between terms to replace variables with
their values in the cryptographic transformations.
When it is false, CryptoVerif just uses the variables as their
appear in the game, and relies only on advice to replace variables
with their values. 

\item \texttt{set priorityEventUnchangedRand = $n$.} (default: 5)

During the cryptographic transformation, variables that occur in event
and are mapped to random variables marked \texttt{[unchanged]} in the
equivalence can be left unchanged.

Sometimes, it is also possible to transform the term that contains
them using one of the oracles of the equivalence.

This settings determines which option is chosen: CryptoVerif prefers
leaving the variable unchanged rather than using an oracle with
priority at least $n$. It prefers using an oracle with priority
less than $n$ rather than leaving the variable unchanged.

\item \texttt{set casesInCorresp = true.}\\
\texttt{set casesInCorresp = false.}

When \texttt{casesInCorresp = true}, CryptoVerif distinguishes
cases depending on the definition point of variables, to infer
more facts in order to prove correspondence properties.
However, this can be slow in complex cases. Using
\texttt{set casesInCorresp = false} disables this case
distinction and speeds up the proof of correspondences.

% \item \texttt{set detectIncompatibleDefined = true.}\\
% \texttt{set detectIncompatibleDefined = false.}
% 
% When true, the simplification detects when two \texttt{defined}
% conditions of \texttt{find} are incompatible because they require two
% variables to be simultaneously defined at the same indices, while
% this is in fact impossible in the considered game. Detecting this
% is rather costly, so it can be turned off.
%BB: made this setting undocumented, since now checking that is not so costly.

\item \texttt{set elsefindFactsInReplace = true.}\\
\texttt{set elsefindFactsInReplace = false.}

When \texttt{elsefindFactsInReplace = true}, CryptoVerif will try to
infer more facts when doing a \texttt{replace} operation: when it
encounters a \texttt{find} branch in the process, it considers a
variable $x[M_1, \ldots, M_l]$, which is guaranteed to be defined by this \texttt{find}.
If $x$ is defined in the \texttt{else} part of another \texttt{find}
construct, then at the definition of $x$, we know that the conditions
of the \texttt{then} branches of this \texttt{find} are not satisfied:
\[\forall u_1, \ldots, u_k, \texttt{not}(\texttt{defined}(y_1[M_{11}, \ldots, M_{1l_1}], \ldots, y_k[M_{k1}, \ldots, M_{kl_k}]) \wedge t)\]
We try to infer $\texttt{not}(t)$ from this fact.
\begin{itemize}
\item if each variable $y_j[M_{j1}, \ldots, M_{jl_j}]$ is defined before $x[M_1, \ldots, M_l]$,
then $\texttt{not}(t)$ indeed holds by the fact above;
\item for each $y_j[M_{j1}, \ldots, M_{jl_j}]$, 
we assume that $y_j[M_{j1}, \ldots, M_{jl_j}]$ is defined after or at the same time as $x[M_1, \ldots, M_l]$
and try to prove $\texttt{not}(t)$.

It this proof succeeds, we can infer that $\texttt{not}(t)$ holds
at the current program point.
\end{itemize}

\item \texttt{set elsefindFactsInSimplify = true.}\\
\texttt{set elsefindFactsInSimplify = false.}

Similar to \texttt{elsefindFactsInReplace}, but applies in
\texttt{simplify} operations. 

\item \texttt{set elsefindFactsInSuccess = true.}\\
\texttt{set elsefindFactsInSuccess = false.}

Similar to \texttt{elsefindFactsInReplace}, but applies in
\texttt{success} operations. 

\item \texttt{set elsefindFactsInSuccessSimplify = true.}\\
\texttt{set elsefindFactsInSuccessSimplify = false.}

Similar to \texttt{elsefindFactsInReplace}, but applies in
the elimination of useless code in \texttt{success simplify} operations. 


\item \texttt{set elsefindAdditionalDisjunct = true.}\\
\texttt{set elsefindAdditionalDisjunct = false.}

When \texttt{elsefindAdditionalDisjunct = true}, the procedure that infers facts
from false conditions of \texttt{find} (see \texttt{set elsefindFactsInReplace})
is enriched: in case $y_j[M_{j1}, \ldots, M_{jl_j}]$ may be defined
at the same time as $x[M_1, \ldots, M_l]$, we additionally assume
that they have different indices, that is, $(M_{j1}, \ldots, M_{jl_j}) \neq (M_1, \ldots, M_l)$
to eliminate this case. Therefore, we infer 
$(M_{j1}, \ldots, M_{jl_j}) \neq (M_1, \ldots, M_l) \Rightarrow \texttt{not}(t)$
or equivalently $(M_{j1}, \ldots, M_{jl_j}) = (M_1, \ldots, M_l) \vee \texttt{not}(t)$.
This is typically more costly and more precise than the basic 
procedure that just infers $\texttt{not}(t)$ when possible.

\item \texttt{set improvedFactCollection = false.}\\
\texttt{set improvedFactCollection = true.}

When \texttt{improvedFactCollection = true}, and CryptoVerif collects
the facts that hold at each program point, it also takes into account
variables that cannot be defined at a certain program point, variables
that cannot be simultaneously defined, and elsefind facts, in order to
prove more facts.

It is a bit costly, so it is disabled by default
(\texttt{improvedFactCollection = false}).

\item \texttt{set useEqualitiesInSimplifyingFacts = false.}\\
\texttt{set useEqualitiesInSimplifyingFacts = true.}

When \texttt{useEqualitiesInSimplifyingFacts = true}, CryptoVerif
uses known equalities between terms to determine whether a fact
is equal to another fact.

It is a bit costly, so it is disabled by default
(\texttt{useEqualitiesInSimplifyingFacts = false}).

\item \texttt{set useKnownEqualitiesWithFunctionsInMatching = false.}\\
\texttt{set useKnownEqualitiesWithFunctionsInMatching = true.}

When \texttt{useKnownEqualitiesWithFunctionsInMatching = true}, CryptoVerif
uses known equalities $M_1 = M_2$ where the root of $M_1$ is a function 
application to normalize terms before testing whether they match
an equation or collision statement or an oracle in a cryptographic
transformation. That can allow to apply these statements or transformations
more often.

It is a bit costly, so it is disabled by default
(\texttt{useKnownEqualitiesWithFunctionsInMatching = false}).

\item $\texttt{set ignoreSmallTimes = }\nonterm{n}{\tt .}$ (default 3)

When {\tt 0}, the evaluation of the runtime is very precise,
but the formulas are often too complicated to read.

When {\tt 1}, the system ignores many small values when computing
the runtime of the games. It considers only function applications
and pattern matching.

When {\tt 2}, the system ignores even more details, including
application of boolean operations (\texttt{\&\&},
\texttt{\string|\string|}, \texttt{not}), constants generated by the
system, \texttt{()} and matching on \texttt{()}. It ignores the
creation and decomposition of tuples in \ifchannels
inputs and outputs\else oracle calls and returns\fi.

When {\tt 3}, the system additionally ignores the time of equality
tests between values of bounded length, as well as the time of
all constants.

\item $\texttt{set maxIterSimplif = }\nonterm{n}{\tt .}$ (default 2)

Sets the maximum number of repetitions of the simplification transformation
for each {\tt simplify} instruction.
A greater value slows down the system but may enable it to obtain
simpler games, and therefore increase its chances of success.
When $n \leq 0$, repeats simplification until a fixpoint is reached.

\item $\texttt{set maxAddFactDepth = }\nonterm{n}{\tt .}$ (default 1000)

Sets the maximum depth of recursion in the addition and simplification
of known facts. 
When $n \leq 0$, puts no limit on this depth of recursion.
Putting a limit avoids an infinite loop in some rare cases.

\item $\texttt{set maxTryNoVarDepth = }\nonterm{n}{\tt .}$ (default 20)

Sets the maximum depth of recursion in the replacement of
variables with their values.
When $n \leq 0$, puts no limit on this depth of recursion.
Putting a limit avoids an infinite loop in some rare cases.

\item \texttt{set maxReplaceDepth = $n$.} (default 20)

Sets the maximum number of rewriting steps that are allowed 
to prove that the new term is equal to the old one in a 
\texttt{replace} transformation. 

\item $\texttt{set maxIterRemoveUselessAssign = }\nonterm{n}{\tt .}$ (default 10)

Sets the maximum number of repetitions of the removal of useless assignments 
for each {\tt remove\string_assign useless} instruction.
A greater value slows down the system but may enable it to obtain
simpler games, and therefore increase its chances of success.
When $n \leq 0$, repeats removal of useless assignments until a fixpoint 
is reached.

\item $\texttt{set maxAdvicePossibilitiesBeginning = }n_1{\tt .}$ (default \texttt{50})\\
$\texttt{set maxAdvicePossibilitiesEnd = }n_2{\tt .}$ (default \texttt{10})

In cryptographic transformations, when CryptoVerif can transform many terms in several ways of different priority, these various ways combine, yielding a very large number of advice possibilities. These two options allow to limit the number of considered advice possibilities by keeping the $n_1$ first possibilities (with highest priority) and the $n_2$ last possibilities (with lowest priority but fewer advised transformations). When $n_1$ or $n_2$ are not positive, all advice possibilities are kept, but that may yield a very slow execution. 

\item $\texttt{set minAutoCollElim = }\nonterm{s}{\tt .}$ (default \texttt{pest80})

  Sets the maximum probability for which elimination of collisions is
  possible automatically (which corresponds to a minimum cardinal for
  the type, when the probability distribution is uniform).  The argument
  $\nonterm{s}$ can be \texttt{large} (probability $2^{-160}$),
  \texttt{password} (probability $2^{-20}$), or \texttt{pest}$n$
  (probability $2^{-n}$; see also the \texttt{type} declaration).

\item \texttt{set forgetOldGames = false.}\\
\texttt{set forgetOldGames = true.}

When \texttt{forgetOldGames = true}, old games are removed from memory after each
cryptographic transformation or each interactive command.  
That allows to save some memory, but prevents \texttt{undo}.
The display of the games is saved into a temporary file to allow
displaying the games at the end of the proof.

\end{itemize}
The default value is the first mentioned, except when explicitly specified.
In most cases, the default values should be left as they are, except
for {\tt interactiveMode}, which allows to perform 
interactive proofs.

\item $\texttt{param}\ \neseq{ident}\ [\texttt{[noninteractive]}\mid \texttt{[passive]} \mid \texttt{[default]} \mid \texttt{[small]} \mid \texttt{[size$n$]}]\texttt{.}$

$\texttt{param}\ n_1, \ldots, n_m\texttt{.}$ declares parameters $n_1, \ldots, n_m$.
Parameters are used to represent the number of copies of replicated processes
(that is, the maximum number of calls to each query).
In asymptotic analyses, they are polynomial in the security parameter.
In exact security analyses, they appear in the formulas that express the
probability of an attack.

The options \texttt{[noninteractive]}, \texttt{[passive]}, \texttt{[default]}, \texttt{[small]}, or \texttt{[size$n$]}
indicate to CryptoVerif an order of magnitude of the parameter.
%
The option \texttt{[size$n$]} (where $n$ is a constant integer) indicates
the parameter is at most $2^n$.
CryptoVerif uses this
information to optimize the computed probability bounds: when several
bounds are correct, it chooses the smallest one.
It also uses it to estimate the probability of collisions,
and decide whether to eliminate the collision or not.

The option \texttt{[noninteractive]} means that
the queries bounded by the considered parameters can be made by the
adversary without interacting with the tested protocol, so the number
of such queries is likely to be large.
Parameters with option \texttt{[noninteractive]} are typically 
used for bounding the number of calls to random oracles.
\texttt{[noninteractive]} is equivalent to \texttt{[size80]}.

The absence of option, the option \texttt{[default]}, and the option \texttt{[passive]} correspond
to adversary interacting with the tested protocol without any limitation
on the number of sessions. This can correspond to two situations:
\begin{itemize}
\item The protocol can start new sessions without limit even 
  if it could detect that an active attack happened in previous sessions.
\item The adversary listens passively to sessions of the protocol that
  run as expected (hence the word \texttt{[passive]}). Therefore, for
  such runs, the adversary is undetected.
\end{itemize}
No option, \texttt{[default]}, and \texttt{[passive]} are equivalent to \texttt{[size30]}.

The option \texttt{[small]} should be used for sessions in which the
adversary actively interacts with the honest participants and mounts
detectable attacks, when these participants stop after a certain
number of failed attempts (e.g. credit cards are blocked after 3 incorrect
PINs).
\texttt{[small]} is equivalent to \texttt{[size2]}.

\item $\texttt{proba}\ \nonterm{ident}[\texttt{(}\seq{dim}\texttt{)}]\ [\texttt{[$\nonterm{pest}$]}].$

$\texttt{proba}\ p\texttt{(}d_1, \dots, d_n\texttt{).}$ declares a probability function $p$
taking $n$ arguments of dimensions $d_1, \dots, d_n$ respectively. The syntax of dimensions is given in Figure~\ref{fig:syntaxdim}, where \texttt{*}, \texttt{/}, and \texttt{\^{ }} are the usual product, division, and exponentiation. After reduction, dimensions are of the form $\texttt{time}^t \times \texttt{length}^l$, where $t$ and $l$ are integers. The dimension \texttt{number} corresponds to $\texttt{time}^0 \times \texttt{length}^0$.

$\texttt{proba}\ p\texttt{.}$ declares a probability function $p$ taking any arguments. In this case, CryptoVerif checks that the number and dimensions of the arguments of $p$ are compatible across calls to $p$.

When \texttt{[$\nonterm{pest}$]} ({\sc p}robability {\sc est}imate) is present, 
it gives an estimate of the value of the probability:
\texttt{pest$n$}, where $n$ is an integer, means that the probability is at most $2^{-n}$;
\texttt{password} is equivalent to \texttt{pest20}, i.e. probability at most $2^{-20}$;
\texttt{large} is equivalent to \texttt{pest160}, i.e. probability at most $2^{-160}$.
When \texttt{[$\nonterm{pest}$]} is absent, \texttt{large} is the default.
%
When the probability $p$ appears in a \texttt{collision} statement and
the command \texttt{allowed\_collisions pest$n'$} has been issued,
CryptoVerif applies the \texttt{collision} statement only when the
probability of collision (taking into account how many times it is
applied) is less than $2^{-n'}$.
The estimate is only used to decide whether to eliminate collisions or not.
The probability formula output by CryptoVerif at the end of the proof
remains correct even if the estimates are incorrect. However,
incorrect estimates may have the consequence that, when evaluating
this probability, its value is larger than desired.

\item $\texttt{letproba}\ \nonterm{ident}[\texttt{(}\neseq{vardim}\texttt{)}] \texttt{ = }\nonterm{proba}\texttt{.}$

$\texttt{letproba}\ p\texttt{(}x_1:d_1, \dots, x_n:d_n\texttt{) = }\mathit{prob}\texttt{.}$ declares a probability function $p$ with $n$ arguments $x_i$ of dimention $d_i$, equal to the probability formula $\mathit{prob}$. See \texttt{proba} above for an explanation of dimensions. The formula $\mathit{prob}$ must represent a probability. It may refer to $x_1, \dots, x_n$. It is instantiated with the appropriate values of $x_1, \dots, x_n$ every time the probability function $p$ is applied.

\item $\texttt{type}\ \nonterm{ident}\ [\texttt{[}\neseq{option}\texttt{]}]\texttt{.}$

$\texttt{type}\ T\texttt{.}$ declares a type $T$. Types correspond to sets
of bitstrings or a special symbol $\bot$ (used for failed decryptions, 
for instance). Optionally, the declaration of a type may be followed by options
between brackets. These options can be:
\begin{itemize}

\item \texttt{bounded} means that the type is a set of bitstrings of
bounded length or perhaps $\bot$. In other words, the type is a finite
subset of bitstrings plus $\bot$.

\item \texttt{fixed} means that the type is the set of all bitstrings of 
a certain length $n$. In particular, the type is a finite set,
so \texttt{fixed} implies \texttt{bounded}. 

\item \texttt{nonuniform} means that random numbers may be chosen in
  the type with a non-uniform distribution. (When \texttt{nonuniform}
  is absent, random numbers are chosen using a uniform distribution
  for {\tt fixed} types, an almost uniform distribution for
  \texttt{bounded} types, and random values cannot be chosen among
  other types. Note that \texttt{fixed, nonuniform} and
  \texttt{bounded, nonuniform} are also allowed to have a non-uniform
  distribution on a \texttt{fixed} or \texttt{bounded} type.)

\item \texttt{size$n$} indicates
the order of magnitude of the cardinal of the type:
\texttt{size$n$} means that its cardinal is $|T| = 2^n$,
where $n$ is an integer
(like the set of bitstrings of length $n$).

\texttt{size$\mathit{min}$\_$\mathit{max}$} means that  $2^{\mathit{min}} \leq |T| \leq 2^{\mathit{max}}$, where $\mathit{min}$ and $\mathit{max}$ are integers.

\item \texttt{pcoll$n$} ({\sc p}robability of {\sc coll}ision) means that $\texttt{Pcoll1rand}(T) \leq 2^{-n}$, where $n$ is an integer.  ($\texttt{Pcoll1rand}(T)$ is the
  probability of collision between a random element chosen according
  to the default probability distribution $D_T$ for the considered
  type $T$, and an independent element of type $T$.)

  When the default distribution is uniform or almost uniform ({\tt
    fixed} and {\tt bounded} types),
  $\texttt{Pcoll1rand}(T) = \frac{1}{|T|}$, so CryptoVerif estimates
  the probability of collision from the cardinal of the type and
  conversely, so mentioning one of \texttt{size$n$} or
  \texttt{pcoll$n$} is sufficient.

CryptoVerif uses this information to determine whether collisions 
with random elements of the considered type $T$ should be eliminated.
For collisions to be eliminated, two conditions must be satisfied:
\begin{enumerate}

\item $\texttt{Pcoll1rand}(T) \leq 2^{-n'}$, that is, $T$ has option
  \texttt{pcoll$n$} with $n \geq n'$, where
$n'$ is set by $\texttt{set minAutoCollElim = pest}n'$ 
(the default is $n' = 80$),
or elimination of collisions
on this data has been manually requested by the command 
$\texttt{simplify coll\string_elim(}\ldots\texttt{)}$
or $\texttt{global\_dep\_anal}\ x\ \texttt{coll\_elim(}\ldots\texttt{)}$.

\sloppy

\item the probability of collision satisfies the conditions specified
  by the command $\texttt{allowed\_collisions}$ (used inside a {\tt
    proof} environment).  By default, collisions are eliminated when
  \begin{itemize}
\item either $\texttt{Pcoll1rand}(T) \leq 2^{-160}$ ($T$ has option
  \texttt{pcoll$n$} with $n \geq 160$ or option \texttt{large})
\item or
  $\texttt{Pcoll1rand}(T) \leq 2^{-20}$ ($T$ has option
  \texttt{pcoll$n$} with $n \geq 20$ or option \texttt{password}), the
  collision is repeated at most $N$ times, and $N$ is a parameter of
  size at most 2.  
\end{itemize}
See the command $\texttt{allowed\_collisions}$ for more details.
\fussy

\end{enumerate}

\item \texttt{large} is equivalent to \texttt{size160\_1000000000}, \texttt{pcoll160}, that is, $|T| \geq 2^{160}$ and $\texttt{Pcoll1rand}(T) \leq 2^{-160}$.
By default, \texttt{large} means that the type $T$ is large enough so that
all collisions with random elements of $T$ can be eliminated. 
(In asymptotic analyses, $\texttt{Pcoll1rand}(T)$ is negligible. 
In exact security analyses, the
probability of a collision is correctly expressed by the system.)

\item \texttt{password} is equivalent to \texttt{size20\_40}, \texttt{pcoll20},
that is, $2^{20} \leq |T| \leq 2^{40}$ and $\texttt{Pcoll1rand}(T) \leq 2^{-20}$.
\texttt{password} is intended for passwords in password-based
security protocols. These passwords are taken in a dictionary whose
size is much smaller than the size of a nonce for instance,
so the probability of collisions among passwords is larger 
than among data of \texttt{large} types. CryptoVerif
assumes that passwords are taken in a dictionary of between about one 
million ($2^{20}$) and about one trillion ($2^{40}$) elements.

\end{itemize}

\item $\texttt{fun}\ \nonterm{ident}\texttt{(}\seq{ident}\texttt{):}\nonterm{ident}\ [\texttt{[}\neseq{option}\texttt{]}]\texttt{.}$

$\texttt{fun}\ f\texttt{(}T_1, \ldots, T_n\texttt{):}T\texttt{.}$ 
declares a function that takes $n$ arguments, of types $T_1, \ldots, T_n$, 
and returns a result of type $T$.
Optionally, the declaration of a function may be followed by options
between brackets. These options can be:
\begin{itemize}

\item \texttt{[data]} means that $f$ is injective and that its
inverses can be computed in polynomial time: $f(x_1, \ldots, x_m) = y$
implies for $i \in \{1, \ldots, m\}$, $x_i = f_i^{-1}(y)$ for some 
functions $f_i^{-1}$. (In the vocabulary of~\cite{BlanchetEPrint05},
$f$ is poly-injective.) $f$ can then be used for pattern matching.

\item \texttt{[projection]} means that $f$ is an inverse of a poly-injective
function. $f$ must be unary. (Thanks to the pattern matching construct, one can
in general avoid completely the declaration of \texttt{projection} functions,
by just declaring the corresponding poly-injective function \texttt{data}.)

\item \texttt{[uniform]} means that $f$ maps the default distribution
of its argument into the default distribution of its result. $f$ must be unary;
the argument and the result of $f$ must be of types marked 
{\tt fixed}, {\tt bounded}, or {\tt nonuniform}.

\end{itemize}

\item $\texttt{letfun}\
  \nonterm{ident}[\texttt{(}\seq{vartypeb}\texttt{)}]\texttt{=}\nonterm{term}\texttt{.}$ 

  $\texttt{letfun}\ f\texttt{(} x_1\texttt{:} T_1, \ldots,
  x_n\texttt{:}T_n\texttt{)=} M\texttt{.}$
  declares a function $f$ that takes $n$ arguments named
  $x_1, \ldots, x_n$ of types $T_1, \ldots, T_n$, respectively. The
  subsequent calls to this function are replaced by the term $M$ in
  which we replace $x_1, \ldots, x_n$ with the arguments given by the
  caller. (We use $x_i \texttt{<=} N_i$ instead of
  $x_i\texttt{:} T_i$ when $x_i$ is of type $[1,N_i]$, where $N_i$ is
  a parameter, declared by $\texttt{param }N_i$.)

  Variables defined inside $\texttt{letfun}$ can be used in array references
  and in queries, provided the process after expansion of $\texttt{letfun}$
  satisfies the required conditions for that.

\item $\texttt{const}\ \neseq{ident}\texttt{:}\nonterm{ident}\texttt{.}$

$\texttt{const}\ c_1, \ldots, c_n\texttt{:}T\texttt{.}$ declares constants
$c_1, \ldots, c_n$ of type $T$.
Different constants are assumed to correspond to different bitstrings
(except when the instruction \texttt{set diffConstants = false.} is
given).

\item $\texttt{table}\ \nonterm{ident}\texttt{(}\neseq{ident}\texttt{).}$

$\texttt{table}\ \mathit{tbl}\texttt{(}T_1, \ldots, T_n\texttt{).}$
declares the table $\mathit{tbl}$, whose elements are tuples of
type $T_1, \ldots, T_n$. Types $T_i$ may be replaced with
parameters $N_i$, to declare a table that contains
a replication index of type $[1,N_i]$. Elements can be inserted in the table
by $\texttt{insert}\ \mathit{tbl}\texttt{(}M_1, \ldots, M_n\texttt{)}$
and the table can be read using $\texttt{get}$.

\ifchannels
\item $\texttt{channel}\ \neseq{ident}\texttt{.}$

$\texttt{channel}\ c_1, \ldots, c_n\texttt{.}$ declares communication channels
$c_1, \ldots, c_n$.
\fi

\item $\texttt{event}\ \nonterm{ident}[\texttt{(}\seq{ident}\texttt{)}]\texttt{.}$

$\texttt{event}\ e\texttt{(}T_1, \ldots, T_n\texttt{)}\texttt{.}$
declares an event $e$ that takes arguments of types $T_1, \ldots, T_n$.
When there are no arguments, we can simply declare 
$\texttt{event}\ e\texttt{.}$ Types $T_i$ may be replaced with
parameters $N_i$, to declare an event that takes as argument
a replication index of type $[1,N_i]$.


\item $\texttt{let}\ \nonterm{ident}[\texttt{(}\seq{vartypeb}\texttt{)}]\texttt{ = }\oprocess\texttt{.}$\\
$\texttt{let}\ \nonterm{ident}[\texttt{(}\seq{vartypeb}\texttt{)}]\texttt{ = }\iprocess\texttt{.}$

$\texttt{let}\ \mathit{proc}(x_1:T_1, \dots, x_n:T_n)\texttt{ = }P\texttt{.}$ says that $\mathit{proc}$ takes $n$ arguments, $x_1$ of type $T_1$, \dots, $x_n$ of type $T_n$, and is equal to the process
$P$. (We use $x_i \texttt{<=} N_i$ instead of
  $x_i\texttt{:} T_i$ when $x_i$ is of type $[1,N_i]$, where $N_i$ is
  a parameter, declared by $\texttt{param }N_i$.)
When parsing a process, $\mathit{proc}(M_1, \dots, M_n)$ will be replaced with $P\{M_1/x_1, \dots, M_n/x_n\}$ when $P$ is an input process. In this case, the terms $M_1, \dots, M_n$ must contain only variables, replication indices, and function applications and the variables $x_1, \dots, x_n$ cannot have array accesses.
The process $\mathit{proc}(M_1, \dots, M_n)$ will be replaced with $\texttt{let}$ $x_1 = M_1$ $\texttt{in}$ \dots $\texttt{let}$ $x_n = M_n$ $\texttt{in}$ $P$ when $P$ is an output process.

\item $\texttt{equation }[\texttt{forall }\seq{vartype}\texttt{;}]\nonterm{simpleterm}\ [\texttt{if}\ \nonterm{simpleterm}]\texttt{.}$

$\texttt{equation forall }x_1:T_1, \ldots, x_n:T_n\texttt{;}M\texttt{.}$ says
that for all values of $x_1, \ldots, x_n$ in types $T_1, \ldots, T_n$
respectively,
$M$ is true. The term $M$ must be a simple term without array accesses.
All bound variables $x_1, \dots, x_n$ must occur in $M$.
%
When $M$ is an equality $M_1 \eqt  M_2$, CryptoVerif uses this information
for rewriting $M_1$ into $M_2$, so one must be careful of the orientation
of the equality, in particular for termination. 
In this case, all bound variables $x_1, \dots, x_n$ must occur in $M_1$,
so that the target term $M_2$ is entirely determined knowing the instance of $M_1$.
%
When $M$ is an inequality, $M_1 \texttt{<>} M_2$, CryptoVerif rewrites
$M_1 \eqt  M_2$ to false and $M_1 \texttt{<>} M_2$ to true.
%
Otherwise, it rewrites $M$ to true.

$\texttt{equation forall }x_1:T_1, \ldots, x_n:T_n\texttt{;}M\texttt{ if }M'\texttt{.}$ says
that for all values of $x_1, \ldots, x_n$ in types $T_1, \ldots, T_n$
respectively such that $M'$ is true, we have that
$M$ is true. The terms $M$ and $M'$ must be simple terms without array accesses.
CryptoVerif tries to prove the precondition $M'$, and in case of success,
rewrites terms as explained above. 


\item $\texttt{equation builtin }\nonterm{eq\_name}\texttt{(}\neseq{ident}\texttt{).}$

This declaration declares the equational theories satisfied by function symbols.
The following equational theories are supported:
\begin{itemize}

\item \texttt{equation builtin commut($f$).} indicates that the function $f$ is commutative,
that is, $f(x,y) = f(y,x)$ for all $x,y$. In this case, the function
$f$ must be a binary function with both arguments of the same type.
(The equation $f(x,y) = f(y,x)$ cannot be given by the {\tt forall}
declaration because CryptoVerif interprets such declarations as rewrite rules,
and the rewrite rule $f(x,y) \rightarrow f(y,x)$ does not terminate.)

\item \texttt{equation builtin assoc($f$).} indicates that the function $f$ is associative, that is, $f(x,f(y,z)) = f(f(x,y),z)$ for all $x,y,z$. In this case, the function $f$ must be a binary function with both arguments and the result of
the same type.

\item \texttt{equation builtin AC($f$).} indicates that the function $f$ is associative and commutative. In this case, the function $f$ must be a binary function with both arguments and the result of
the same type.

\item \texttt{equation builtin assocU($f$, $n$).} indicates that the function $f$ is associative, and that $n$ is a neutral element for $f$, that $f(x,n) = f(n,x) = x$ for all $x$. In this case, the function $f$ must be a binary function with both arguments and the result of the same type as the type of the constant $n$.

\item \texttt{equation builtin ACU($f$, $n$).} indicates that the function $f$ is associative and commutative, and that $n$ is a neutral element for $f$. In this case, the function $f$ must be a binary function with both arguments and the result of the same type as the type of the constant $n$.

\item \texttt{equation builtin ACUN($f$, $n$).} indicates that the function $f$ is associative and commutative, that $n$ is a neutral element for $f$, and that $f$ satisfies the cancellation equation $f(x,x) = n$. In this case, the function $f$ must be a binary function with both arguments and the result of the same type as the type of the constant $n$.

\item \texttt{equation builtin group($f$, $inv$, $n$).} indicates that $f$ forms group with inverse $inv$ and neutral element $n$, that is, the function $f$ is associative, $n$ is a neutral element for $f$, and $inv(x)$ is the inverse of $x$, that is, $f(inv(x),x) = f(x,inv(x)) = n$. In this case, the function $f$ must be a binary function with both arguments and the result of the same type $T$, $inv$ must be a unary function that takes and returns a value of type $T$, and $n$ must be a constant of type $T$.

\item \texttt{equation builtin commut\_group($f$, $inv$, $n$).} indicates that $f$ forma commutative group with inverse $inv$ and neutral element $n$, that is, the function $f$ is associative and commutative, $n$ is a neutral element for $f$, and $inv(x)$ is the inverse of $x$. In this case, the function $f$ must be a binary function with both arguments and the result of the same type $T$, $inv$ must be a unary function that takes and returns a value of type $T$, and $n$ must be a constant of type $T$.

\end{itemize}

\item 
$\texttt{collision }\nonterm{res}^*
[\texttt{[random\_choices\_may\_be\_equal]}]
[\texttt{forall }\seq{vartype}\texttt{;}]$\\
\null\qquad $\texttt{return(}\nonterm{simpleterm}\texttt{) <=(}\nonterm{proba}\texttt{)=> return(}\nonterm{simpleterm}\texttt{)}
[\ \texttt{if}\ \nonterm{cond}]\texttt{.}$\\
where
\begin{align*}
\nonterm{cond} &::= \nonterm{simpleterm}\\
&\ \ \mid \ \nonterm{ident}\texttt{ independent-of }\nonterm{ident}\\
&\ \ \mid \ \nonterm{cond}\texttt{ \&\& }\nonterm{cond}\\
&\ \ \mid \ \nonterm{cond}\texttt{ || }\nonterm{cond}
\end{align*}

$\texttt{collision }\Resa{x_1}{T_1}\texttt{;}\ldots 
\Resa{x_n}{T_n}\texttt{;}
\texttt{forall }y_1:T'_1, \ldots, y_m:T'_m\texttt{;}$\\
$\null\qquad \texttt{return(}M_1\texttt{) <=(}p\texttt{)=> return(}M_2\texttt{).}$\\
means that when
$x_1, \ldots, x_n$ are chosen randomly 
and independently in $T_1, \ldots, T_n$ respectively (with the default probability distributions for these types), a Turing machine running in
time $\texttt{time}$ has probability at most $p$ of finding
$y_1, \ldots, y_m$ in $T'_1, \ldots, T'_m$ such that $M_1 \neq M_2$.
%
The terms $M_1$ and $M_2$ must be simple terms without array accesses.
See below for the syntax of probability formulas.

This allows CryptoVerif to rewrite $M_1$ into $M_2$ with probability
loss $p$, when $x_1, \ldots, x_n$ are created by independent random
number generations of types $T_1, \ldots, T_n$ respectively. One
should be careful of the orientation of the equivalence, in particular
for termination.

$\texttt{collision }\Resa{x_1}{T_1}\texttt{;}\ldots 
\Resa{x_n}{T_n}\texttt{;}
\texttt{forall }y_1:T'_1, \ldots, y_m:T'_m\texttt{;}$\\
$\null\qquad \texttt{return(}M_1\texttt{) <=(}p\texttt{)=> return(}M_2\texttt{) if $c$.}$\\
means that the previous property holds when the condition $c$ is true, where
$c$ is built by conjunctions or disjunctions of simple terms and independence conditions ``$y_i$ \texttt{independent-of} $x_j$'',
where $y_i$ is bound by  \texttt{forall} and $x_j$ is bound by \texttt{new}. (However, disjunctions cannot
mix terms and independence conditions.)

The option \texttt{[random\_choices\_may\_be\_equal]}, when it is present, allows several 
random number generations among $x_1, \ldots, x_n$ to be the same, instead of being
independent. One can then group, in a single \texttt{collision} statement, 
situations in which $x_1, \ldots, x_n$ are the same or they are independent.
The indices of the variables corresponding to $x_1, \ldots, x_n$
in the game are still made independent of $x_1, \ldots, x_n$. 
Hence, there are two cases: either $x_i$ is the same as $x_j$, or
$x_i$ and $x_j$ are independent of each other. With the option \texttt{[random\_choices\_may\_be\_equal]}, 
the independence conditions can also be ``$x_i$ \texttt{independent-of} $x_j$'',
where $x_i$ and $x_j$ are both bound by \texttt{new}. This condition
then means $x_i$ and $x_j$ are different random choices, so
$x_j$ is also independent of $x_i$. 

\ifchannels
\item $\texttt{equiv}[\texttt{(}\nonterm{ident}[\texttt{(}\nonterm{ident}\texttt{)}]\texttt{)}]$\\
$\nonterm{omode}\ [\texttt{|}\ \ldots\ \texttt{|}\nonterm{omode}]\texttt{ <=(}\nonterm{proba}\texttt{)=> }
[\texttt{[}n\texttt{]}]\ [\texttt{[}\neseq{option}\texttt{]}]\ \nonterm{ogroup}\ [\texttt{|}\ \ldots\ \texttt{|}\nonterm{ogroup}]\texttt{.}$

$\texttt{equiv(}\mathit{name}\texttt{)}\ L\texttt{ <=(}p\texttt{)=> }R\texttt{.}$ means that the
probability that a probabilistic Turing machine that runs in time
{\tt time} distinguishes $L$ from $R$ is at most $p$. The name $\mathit{name}$
is used to designate the equivalence in the \texttt{crypto} command used in manual proofs (see Section~\ref{sec:interact}). This name can be either an identifier $\mathit{id}$, or $\mathit{id}(f)$, where $\mathit{id}$ is an identifier and $f$ a second identifier. Names of the form $\mathit{id}(f)$ are most useful when the equivalence is defined inside a macro definition ($\texttt{def}$). In this case, the identifier $\mathit{id}$ is kept unchanged and the identifier $f$ is renamed during macro expansion; if $f$ is a parameter of the macro, it is then replaced with its value at macro expansion, so that one can always designate precisely the desired equivalence even when a macro is expanded several times.
The name may be omitted.

$L$ and $R$ define sets of oracles. (They can be translated into
processes as explained in~\cite{BlanchetEPrint05}.)
\begin{itemize}

\item $O\texttt{(}x_1:T_1, \ldots, x_n:T_n\texttt{) := }\mathit{FP}$ represents
an oracle $O$ that takes arguments $x_1, \ldots, x_n$ of types
$T_1, \ldots, T_n$ respectively, and returns the result computed by $\mathit{FP}$.
The oracle body $\mathit{FP}$ is similar to term, but terminates with
a $\texttt{return}$ as shown in the grammar of $\funbody$ 
(Figure~\ref{fig:syntax3}).

\item Optionally, in the left-hand side,
an integer between brackets $\texttt{[}n\texttt{]}$ ($n \geq 0$)
can be added in the definition of an oracle, which becomes 
$O\texttt{(}x_1:T_1, \ldots, x_n:T_n\texttt{) [}n\texttt{] := }\mathit{FP}$.
This integer does not change the semantics of the oracle, but is
used for the proof strategy: CryptoVerif uses preferably the oracles
with the smallest integers $n$ when several oracles can be used
for representing the same expression. When no integer is mentioned,
$n = 0$ is assumed, so the oracle has the highest priority.

\item Optionally, in the left-hand side, 
the indication \texttt{[useful\_change]} can also
be added in the definition of an oracle, which becomes 
$O\texttt{(}x_1:T_1, \ldots, x_n:T_n\texttt{) [useful\_change] := }\mathit{FP}$.
This indication is also used for the proof strategy: 
if at least one \texttt{[useful\_change]} indication is present,
CryptoVerif applies the transformation defined by the equivalence
only when at least one \texttt{[useful\_change]} function is called in the game.

\item $\texttt{!} i \leqt  N\ \Resa{y_1}{T'_1}\texttt{;}
\ldots \Resa{y_m}{T'_m} \texttt{;} \texttt{(}FG_1\texttt{|} \ldots \texttt{|}
FG_n\texttt{)}$ represents $N$ copies of a process that picks fresh
random numbers $y_1$, \ldots, $y_m$ of types $T'_1, \ldots, T'_m$
respectively, and makes available the functions described in $FG_1,
\ldots, FG_n$. Each copy has a different value of $i \in [1, N]$. The
identifier $i$ cannot be referred to explicitly in the process; it is
used only implicitly as array index of variables defined under
$\texttt{!} i \leqt  N$.  The replication $\texttt{!} i
\leqt  N$ can be abbreviated $\texttt{!} N$.

The replication $\texttt{!} i \leqt  N$ can be omitted 
only at the root of the equivalence, when it contains
a single $\funmode$ on the left-hand side, and a single $\fungroup$
on the right-hand side. CryptoVerif then automatically
adds a replication internally, and adjusts the probability
accordingly.

\end{itemize}
CryptoVerif uses such equivalences to transform processes that call
oracles of $L$ into processes that call oracles of $R$.

$L$ may contain mode indications to guide the rewriting: the mode
\texttt{[all]} means that all occurrences of the root function symbol
of oracles in the considered group must be transformed;
the mode \texttt{[exist]} means that at least one occurrence of an
oracle in this group must be transformed. (\texttt{[exist]} is the default;
there must be at most one oracle group with mode \texttt{[exist]};
when an oracle group contains no random number generation, it must be in mode 
\texttt{[all]}.)

Optionally, 
an integer between brackets $\texttt{[}n\texttt{]}$ ($n \geq 0$)
can be added in an equivalence.
This integer does not change the semantics of the equivalence, but is
used for the proof strategy: CryptoVerif uses preferably the equivalences
with the smallest integers $n$ when several equivalences can be used.
When no integer is mentioned,
$n = 0$ is assumed, so the equivalence has the highest priority.

Two options can specified for an equivalence, in
$\texttt{[}\neseq{option}\texttt{]}$:
\begin{itemize}

\item The \texttt{manual} option, when it is present in the equivalence,
prevents the automatic application of the transformation. The transformation
is then applied only using the manual \texttt{crypto} command.

\item The \texttt{computational} option, when it is present in the 
equivalence, means that the transformation relies on a computational assumption
(by opposition to decisional assumptions). This indication allows one to mark
some random number generations of the right-hand side of the equivalence with
\texttt{[unchanged]}, which means that the random value is preserved by 
the transformation. The transformation is then allowed even if the random 
value occurs as argument of events. (This argument will be unchanged.)
The mark \texttt{[unchanged]} is forbidden when the equivalence is
not marked \texttt{[computational]}. Indeed, decisional assumptions may
alter any random values.

\end{itemize}

$L$ and $R$ must satisfy certain syntactic constraints:
\begin{itemize}

\item %H0
$L$ and $R$ must be well-typed, satisfy the constraints on
array accesses (see the description of processes above), 
and the type of the results of 
corresponding oracles in $L$ and $R$ must be the same.

\item All oracle definitions in $L$ are of the form 
$O\texttt{(}\ldots\texttt{) := return(}M\texttt{)}$
where $M$ is a simple term. % without explicit array accesses.
Oracle definitions in $R$ are of the form 
$O\texttt{(}\ldots\texttt{) := }\funbody$.

\item $L$ and $R$ must have the same structure: same replications,
same number of oracles, same oracle names in the same order,
same number of arguments with the same types for each oracle.

\item Under a replication with no random number generation in $L$, 
one can have only a single oracle.

\item Replications in $L$ (resp. $R$) must have pairwise distinct
bounds. Oracles in $L$ (resp. $R$) must have pairwise distinct names.

\item %H7

Finds in $R$ are of the form
\[\begin{split}
&\texttt{find}[\texttt{[unique]}]\ \ldots\\
&\texttt{orfind }u_1 \texttt{ <= } N_1, \ldots, u_m \texttt{ <= }N_m
\texttt{ suchthat defined(}z_1[\tup{u_1}], \ldots, z_l[\tup{u_l}]\texttt{) \&\& }M\texttt{ then }\mathit{FP}\\
&\ldots \texttt{ else }\mathit{FP}'
\end{split}\]
where $\tup{u_k}$ is a non-empty
suffix of $u_1, \ldots, u_m$, at least one $\tup{u_k}$ for $1 \leq
k \leq l$ is the whole sequence $u_1, \ldots, u_m$ optionally followed by a sequence of indices $\tup{u_0}$,
and the implicit suffix of the current array indices is the same
for all $z_1, \ldots, z_l$.
%
(When $z$ is defined under replications $\texttt{!}N_1$, \ldots,
$\texttt{!}N_n$, $z$ is always an array with $n$ dimensions, so it
expects $n$ indices, but the first $n'<n$ indices are left implicit
when they are equal to the current indices of the top-most $n'$ replications
above the usage of $z$---which must also be the top-most $n'$
replications above the definition of $z$. We require the implicit
indices to be the same for all variables $z_1, \ldots, z_l$.)
%TO DO is that clear?
Furthermore, there must exist $k \in \{ 1, \ldots, l_j\}$ such that
for all $k' \neq k$, $z_{k'}$ is defined syntactically above all
definitions of $z_k$ and $\tup{u_{k'}}$ is a suffix of $\tup{u_k}$.

When $\tup{u_0}$ is not empty, the $\texttt{find}$ is automatically
transformed into
\[\begin{split}
&\texttt{find}[\texttt{[unique]}]\ \ldots\\
&\texttt{orfind }u_1 \texttt{ <= } N_1, \ldots, u_m \texttt{ <= }N_m, \tup{u'_0}\texttt{ <= }\tup{N'_0}
\texttt{ suchthat defined(}z_1[\tup{u'_1}], \ldots, z_l[\tup{u'_l}]\texttt{) \&\&}\\
&\qquad\qquad (\tup{u'_0},\tup{i}) = (\tup{u_0},\tup{i})\texttt{ \&\& }M'\texttt{ then }\mathit{FP}''\\
&\ldots \texttt{ else }\mathit{FP}'
\end{split}\]
where $\tup{u'_0}$ are fresh variables of the same type as $\tup{u_0}$ and $\tup{N'_0}$ are their bounds
($\tup{u'_0}\texttt{ <= }\tup{N'_0}$ abbreviates a sequence of possibly several inequalities),
$u'_k = u_k \{\tup{u'_0}/\tup{u_0}\}$,
$M' = M\{\tup{u'_0}/\tup{u_0}\}$,
$\mathit{FP}'' = \mathit{FP}\{\tup{u'_0}/\tup{u_0}\}$,
and $\tup{i}$ is the implicit suffix of the current array indices.
After this transformation, we are in the situation above with an
empty $\tup{u_0}$.

In case a variable $z_k$ is defined by a $\texttt{find}$ in $R$,
$z_k$ is automatically renamed into a fresh variable $z'_k$ at its definition,
and $z_k$ is defined by $\texttt{let}\ z_k = z'_k$ in the $\texttt{then}$
branch of the $\texttt{find}$ that defines $z'_k$. The array accesses
to $z_k$ are left unchanged. After this transformation, the
variables $z_k$ on which array accesses are performed are never
defined by a $\texttt{find}$ in $R$.

\item In addition to making array accesses, a limited usage of indices is allowed in $R$.
Precisely, the following sequences of indices are allowed:
\begin{enumerate}
\item the current array indices, and any suffix thereof;
\item the sequence of indices $u_1, \ldots, u_m$ defined by a $\texttt{find}$ followed by the associated
implicit suffix of the current array indices (see above), and any suffix thereof;
\item indices received as argument by the oracle, when a variable in $L$ has these indices.
\end{enumerate}
When such a sequence of indices contains a single element, it is represented
by the index itself. When it contains several elements, it is represented
as a tuple $\texttt{(}\dots\texttt{)}$ containing the indices.
Such sequences of indices can be stored in variables (using $\texttt{let}$),
and the sequences or variables containing them can be compared using
equality \texttt{=} or disequality \texttt{<>}. In such comparisons,
the types of the indices inside the sequences must be the same on
both sides of the comparison.
No other operation on indices is allowed, to make sure that the result is
independent of the numbering of the oracle calls.

\end{itemize}
\else
\item $\texttt{equiv }\nonterm{omode}\ [\texttt{|}\ \ldots\ \texttt{|}\nonterm{omode}]\texttt{ <=(}\nonterm{proba}\texttt{)=> }
[\texttt{[manual]}| \texttt{[computational]}]\ \nonterm{ogroup}\ [\texttt{|}\ \ldots\ \texttt{|}\nonterm{ogroup}]\texttt{.}$

$\texttt{equiv }\mathit{name}\ L\texttt{ <=(}p\texttt{)=> }R\texttt{.}$ means that the
probability that a probabilistic Turing machine that runs in time
{\tt time} distinguishes $L$ from $R$ is at most $p$. The name $\mathit{name}$
is used to designate the equivalence in the \texttt{crypto} command used in manual proofs (see Section~\ref{sec:interact}). This name can be either an identifier $\mathit{id}$, or $\mathit{id}(f)$, where $\mathit{id}$ is an identifier and $f$ a second identifier. Names of the form $\mathit{id}(f)$ are most useful when the equivalence is defined inside a macro definition ($\texttt{def}$). In this case, the identifier $\mathit{id}$ is kept unchanged and the identifier $f$ is renamed during macro expansion; if $f$ is a parameter of the macro, it is then replaced with its value at macro expansion, so that one can always designate precisely the desired equivalence even when a macro is expanded several times.

$L$ and $R$ define sets of oracles. (In these definitions, 
$\texttt{foreach }i\leqt N\texttt{ do }\Resb{x_1}{T_1}; \ldots
\Resb{x_m}{T_m};Q$ in fact stands for $\texttt{foreach }i\leqt N
\texttt{ do }O\texttt{() := } \Resb{x_1}{T_1}; \ldots \Resb{x_m}{T_m};
\texttt{return}; Q$, where $O$ is a fresh oracle name. The same oracle
names are used in both sides of the equivalence.)

In the left-hand side, an optional integer between brackets
$\texttt{[}n\texttt{]}$ ($n \geq 0$) can be added in the
definition of an oracle, which becomes 
$O\texttt{(}x_1:T_1, \ldots, x_n:T_n\texttt{) [}n\texttt{] := }P$.
This integer does not change the semantics of the oracle, but is
used for the proof strategy: CryptoVerif uses preferably the oracles
with the smallest integers $n$ when several oracles can be used
for representing the same expression. When no integer is mentioned,
$n = 0$ is assumed, so the oracle has the highest priority.

In the left-hand side, the optional indication \texttt{[useful\_change]} can also
be added in the definition of an oracle, which becomes 
$O\texttt{(}x_1:T_1, \ldots, x_n:T_n\texttt{) [useful\_change] := }P$.
This indication is also used for the proof strategy: 
if at least one \texttt{[useful\_change]} indication is present,
CryptoVerif applies the transformation defined by the equivalence
only when at least one \texttt{[useful\_change]} function is called in the game.

CryptoVerif uses such equivalences to transform processes that call
oracles of $L$ into processes that call oracles of $R$.

$L$ may contain mode indications to guide the rewriting: the mode
\texttt{[all]} means that all occurrences of the root function symbol
of oracles in the considered group must be transformed;
the mode \texttt{[exist]} means that at least one occurrence of an
oracle in this group must be transformed. (\texttt{[exist]} is the default;
there must be at most one oracle group with mode \texttt{[exist]};
when an oracle group contains no random number generation, it must be in mode 
\texttt{[all]}.)

The \texttt{[manual]} indication, when it is present in the equivalence,
prevents the automatic application of the transformation. The transformation
is then applied only using the manual \texttt{crypto} command.

The \texttt{[computational]} indication, when it is present in the 
equivalence, means that the transformation relies on a computational assumption
(by opposition to decisional assumptions). This indication allows one to mark
some random number generations of the right-hand side of the equivalence with
\texttt{[unchanged]}, which means that the random value is preserved by 
the transformation. The transformation is then allowed even if the random 
value occurs as argument of events. (This argument will be unchanged.)
The mark \texttt{[unchanged]} is forbidden when the equivalence is
not marked \texttt{[computational]}. Indeed, decisional assumptions may
alter any random values.

$L$ and $R$ must satisfy certain syntactic constraints:
\begin{itemize}

\item %H0
$L$ and $R$ must be well-typed, satisfy the constraints on
array accesses (see the description of processes above), 
and the type of the results of 
corresponding oracles in $L$ and $R$ must be the same.

\item All oracle definitions in $L$ are of the form 
$O\texttt{(}\ldots\texttt{) := return(}M\texttt{)}$
where $M$ is a simple term. % without explicit array accesses.
Oracle definitions in $R$ are of the form 
$O\texttt{(}\ldots\texttt{) := }\funbody$.

\item $L$ and $R$ must have the same structure: same replications,
same number of oracles, same oracle names in the same order,
same number of arguments with the same types for each oracle.

\item Under a replication with no random number generation in $L$, 
one can have only a single oracle.

\item Replications in $L$ (resp. $R$) must have pairwise distinct
bounds. Oracles in $L$ (resp. $R$) must have pairwise distinct names.

\item %H7
\newcommand{\tup}[1]{\widetilde{#1}}

Finds in $R$ are of the form
\[\begin{split}
&\texttt{find}[\texttt{[unique]}]\ \ldots\\
&\texttt{orfind }u_1 \texttt{ <= } N_1, \ldots, u_m \texttt{ <= }N_m
\texttt{ suchthat defined(}z_1[\tup{u_1}], \ldots, z_l[\tup{u_l}]\texttt{) \&\& }M\texttt{ then }\mathit{FP}\\
&\ldots \texttt{ else }\mathit{FP}'
\end{split}\]
where $\tup{u_k}$ is a non-empty
suffix of $u_1, \ldots, u_m$, at least one $\tup{u_k}$ for $1 \leq
k \leq l$ is the whole sequence $u_1, \ldots, u_m$,
and the implicit suffix of the current array indices is the same
for all $z_1, \ldots, z_l$.
%
(When $z$ is defined under replications $\texttt{!}N_1$, \ldots,
$\texttt{!}N_n$, $z$ is always an array with $n$ dimensions, so it
expects $n$ indices, but the first $n'<n$ indices are left implicit
when they are equal to the current indices of the top-most $n'$ replications
above the usage of $z$---which must also be the top-most $n'$
replications above the definition of $z$. We require the implicit
indices to be the same for all variables $z_1, \ldots, z_l$.)
%TO DO is that clear?
Furthermore, there must exist $k \in \{ 1, \ldots, l_j\}$ such that
for all $k' \neq k$, $z_{k'}$ is defined syntactically above all
definitions of $z_k$ and $\tup{u_{k'}}$ is a suffix of $\tup{u_k}$. 
%
Finally, variables $z_k$ must not be defined by a $\texttt{find}$ in $R$.


\end{itemize}


\fi


This is the key declaration for defining the security properties of
cryptographic primitives. Since such declarations are delicate to
design, we recommend using predefined primitives listed in
Section~\ref{sect:prim}, or copy-pasting declarations from examples.

\item $\texttt{equiv}[\texttt{(}\nonterm{ident}[\texttt{(}\nonterm{ident}\texttt{)}]\texttt{)}]\texttt{ special }\nonterm{ident}\texttt{(}\seq{specialarg}\texttt{)}\ [\texttt{[manual]} \mid \texttt{[}n\texttt{]}]\texttt{.}$

  $\texttt{equiv(}\mathit{name}\texttt{) special
  }\mathit{specialname}\texttt{(}a_1, \dots a_n\texttt{)}$
  declares an equivalence (that is, indistinguishability) between two
  games, like the previous version of \texttt{equiv}. However, instead
  of using games given explicitly, CryptoVerif generates the games
  from $\mathit{specialname}\texttt{(}a_1, \dots a_n\texttt{)}$.

  The following values of $\mathit{specialname}$ are supported:
  \texttt{rom} and \texttt{rom\_partial} for random oracles,
  \texttt{prf} and \texttt{prf\_partial} for pseudo-random functions, 
  \texttt{prp} and \texttt{prp\_partial} for pseudo-random permutations,
  \texttt{sprp} and \texttt{sprp\_partial} for super pseudo-random permutations (pseudo-random permutations whose inverse is also a pseudo-random permutations),
  \texttt{icm} and \texttt{icm\_partial} for the ideal cipher model.

  Let us first explain the cases
  \texttt{rom}, \texttt{rom\_partial},
  \texttt{prf}, \texttt{prf\_partial}, 
  \texttt{prp}, and \texttt{prp\_partial}.
  They take the following arguments $\seq{specialarg} = a_1, \dots a_n$:
  \begin{enumerate}

\item A string $\mathit{key\_pos}$, which can be \texttt{"key\_first"}
when the key is the first argument of the considered function,
\texttt{"key\_last"} when it is the last argument, or
\texttt{"key $n$"} when it is its $n$-th argument. ($n$ is an integer
between 1 and the number of arguments of $f$.)

\item An identifier $f$, the considered function. The function $f$
must be declared before the $\texttt{equiv}$
declaration. For \texttt{prp} and \texttt{prp\_partial}, the function
$f$ must take one argument in addition to the key, the type $T$ of
this argument must be the same as the type of the result of $f$, and
it must be large enough so that collisions between a random element of
the domain and an independent value can be eliminated (because we
model a PRF and apply the PRF/PRP switching lemma), that is,
$\texttt{Pcoll1rand}(T) \leq 2^{-n'}$, that is, $T$ has
option \texttt{pcoll$n$} with $n \geq n'$ where $n'$ is set by
$\texttt{set minAutoCollElim} = \texttt{pest}n'$; the default is $n' =
80$.  For other values of $\mathit{specialname}$, the function $f$
must take at least one argument in addition to the key.  In all cases,
we must be able to choose an element randomly in the type of the key
and in the type of the result of $f$, that is, these types must be
declared \texttt{fixed}, \texttt{bounded}, or \texttt{nonuniform}.

\item When $\mathit{specialname}$ is not \texttt{rom} nor
  \texttt{rom\_partial}, an identifier $\mathit{p}$ such that
  $\mathit{p}(t, N, l_1, \dots, l_m)$ is the probability that an
  adversary breaks the PRF (resp. PRP) assumption in time $t$, with at
  most $N$ queries to the function $f$, with arguments of lengths at
  most $l_1, \dots, l_m$. The length is omitted when the corresponding
  type is bounded. The identifier $\mathit{p}$ must be declared with
  $\texttt{proba}\ p$. This argument is omitted for random oracles
  because the probability is always 0.

\item A tuple of identifiers $(k, r, x, y, z, u)$ for ROM and PRF,
$(k, r, x, u)$ for PRP, which are used to
determine identifiers of variables in the generated equivalence:
\begin{itemize}
\item $k$ is the identifier of the key;
\item $r$ is the identifier of the random result of $f$ after game transformation;
\item $x$ is the identifier used for arguments of $f$ in most oracles;
\item $y$ and $z$ are the identifiers used for arguments of the two calls to $f$
in oracles generated by the collision LHS
$\texttt{"}\mathit{Ocoll}: \Resa{r_1}{T}; \Resa{r_2}{T}; \texttt{forall}\ a_1:T_1, \dots, a_n:T_n; M\texttt{"}$
(see the next argument);
\item $u$ is the identifier used for indices of {\tt find}.
\end{itemize}
The identifiers $x$, $y$, $z$, $u$ are suffixed by \texttt{\_} and the name of the oracle
in which they are used. The identifier $r$ is suffixed by \texttt{\_} and the suffix of
the name of the oracle in which it is used.
Moreover, if needed to avoid name clashes or to generate several variables,
a suffix $\_n$ may be added to these identifiers or modified if they
already have one. Using identifiers not used elsewhere allows
the user to have stable identifiers in the generated equivalence.

\item A tuple of strings $\mathit{collisions\_LHS}$, which can be either \texttt{("large")} or a tuple of strings of the following forms:
  \begin{itemize}
  \item $\texttt{"}\mathit{Ocoll}: \texttt{forall}\ a_1:T_1, \dots, a_n:T_n; \Resa{r_1}{T}; M\texttt{"}$
    where $T$ is the type of the result of $f$ and the simple term $M$ uses the variables $a_1, \dots a_n, r_1$. In this case, CryptoVerif tries
    to simplify $M$ assuming $r_1$ is a random value and $a_1, \dots, a_n$ do not depend on $r_1$. If it
    rewrites $M$ into a term $N$ that does not contain $r_1$, then it uses this information to transform
    terms $M\{f(\dots)/r_1\}$ into $N$ when the result of $f(\dots)$ is a fresh random value, in the generated
    cryptographic transformation. (See files \nolinkurl{examples/obasic/undeniable-sig.ocv} and
    \nolinkurl{examples/obasic/undeniable-sig2.ocv} for examples.)
  \item $\texttt{"}\mathit{Ocoll}: \Resa{r_1}{T}; \texttt{forall}\ a_1:T_1, \dots, a_n:T_n; M\texttt{"}$
    where $T$ is the type of the result of $f$ and the simple term $M$ that uses the variables $a_1, \dots a_n, r_1$. In this case, CryptoVerif tries
    to simplify $M$ assuming $r_1$ is a random value ($a_1, \dots, a_n$ may depend on $r_1$). 
    If it rewrites $M$ into a term $N$ that does not contain $r_1$, then it uses this information to transform
    terms $M\{f(\dots)/r_1\}$ into $N$, in the generated cryptographic transformation.
  \item $\texttt{"}\mathit{Ocoll}: \Resa{r_1}{T}; \Resa{r_2}{T}; \texttt{forall}\ a_1:T_1, \dots, a_n:T_n; M\texttt{"}$
    where $T$ is the type of the result of $f$ and the simple term $M$ that uses the variables $a_1, \dots a_n, r_1, r_2$. In this case, CryptoVerif tries
to rewrite $M$ assuming $r_1$ and $r_2$ are independent random values into a term $N_2$ that does not contain $r_1$ nor $r_2$,
and to rewrite $M$ assuming $r_1 = r_2$ is a random value into a term $N_1$ that does not contain $r_1$ nor $r_2$.
If it succeeds, then it uses this information to transform terms $M\{f(args_1)/r_1, f(args_2)/r_2\}$ into
$\texttt{if }args_1 = args_2 \texttt{ then }N_1\texttt{ else }N_2$, in the generated cryptographic transformation.
  \end{itemize}
  Obviously, when $n = 0$, $\texttt{forall}\ a_1:T_1, \dots, a_n:T_n; $ is omitted.
  The identifiers $\mathit{Ocoll}$ are used to form the oracle names in the generated equivalence (see below); they must not contain \texttt{\_}, and must be different from \texttt{O} and pairwise distinct.
  Only the first form is allowed for \texttt{prp} and \texttt{prp\_partial}.
  Even when a single string is present, the argument must be a tuple of strings,
  so this string must be between parentheses.

  When $\mathit{collisions\_LHS}$ is \texttt{("large")}, the type $T$ of the result of $f$
must be large enough so that collisions between a random element of
the domain and an independent value can be eliminated, that is,
$\texttt{Pcoll1rand}(T) \leq 2^{-n'}$, that is, $T$ has
option \texttt{pcoll$n$} with $n \geq n'$ where $n'$ is set by
$\texttt{set minAutoCollElim} = \texttt{pest}n'$; the default is $n' =
80$. In this case, this is equivalent to $\mathit{collisions\_LHS}$ containing:
\begin{itemize}
\item
  $\texttt{"Oeq:}\ \texttt{forall}\ a_1:T; \Resa{r_1}{T}; r_1 = a_1\texttt{"}$.
  Assuming $a_1$ does not depend on $r_1$, $r_1 = a_1$ simplifies into
  $\texttt{false}$, so $f(\dots) = a_1$ is transformed into
  $\texttt{false}$ in the generated cryptographic transformation, when
  the result of $f(\dots)$ is a fresh random value.

\item When $\mathit{specialname}$ is not \texttt{prp} nor
  \texttt{prp\_partial},
  $\texttt{"Ocoll:}\ \Resa{r_1}{T}; \Resa{r_2}{T}; r_1 = r_2\texttt{"}$.
  The term $r_1 = r_2$ simplifies into $\texttt{false}$ when $r_1$ and
  $r_2$ are independent random values and into $\texttt{true}$ when
  $r_1 = r_2$, so $f(args_1) = f(args_2)$ is transformed into
  $args_1 = args_2$ in the generated cryptographic transformation.

\end{itemize}
The argument $\mathit{collisions\_LHS}$ can be overriden when the 
equivalence is used in a \texttt{crypto} command, by passing
the desired $\mathit{collisions\_LHS}$ as special argument to the 
\texttt{crypto} command. 

\end{enumerate}
The last or the last two arguments may be omitted.

When $\mathit{specialname}$ is \texttt{rom}, \texttt{prf}, or \texttt{prp},
the generated equivalence provides the following oracles:
\begin{itemize}
\item Oracle \texttt{O} evaluates $f$ on its arguments in the
  left-hand side, and performs a lookup into previous arguments of
  \texttt{O} in the right-hand side: it returns the previous result
  when the current arguments are equal to previous arguments and
  otherwise it returns a fresh random value.

\item For each element of $\mathit{collisions\_LHS}$, oracle
  $\mathit{Ocoll}$ evaluates $M$ with $r_i$ replaced with a call to
  $f$ in left-hand side and uses the simplified form of $M$ in the
  right-hand side.

\end{itemize}
When $\mathit{specialname}$ is \texttt{rom\_partial}, \texttt{prf\_partial}, or \texttt{prp\_partial},
the generated equivalence provides oracles named $\mathit{Ocoll}$ for each
element $\texttt{"}\mathit{Ocoll}: \Resa{r_1}{T}; \texttt{forall}\ a_1:T_1, \dots, a_n:T_n; M\texttt{"}$
or $\texttt{"}\mathit{Ocoll}: \Resa{r_1}{T}; \Resa{r_2}{T}; \texttt{forall}\ a_1:T_1, \dots, a_n:T_n; M\texttt{"}$
of $\mathit{collisions\_LHS}$.
These oracles act like the oracle of the same name when $\mathit{specialname}$ is \texttt{rom} (resp. \texttt{prf}---there are no such oracles for \texttt{prp\_partial}).

It also provides oracles named $\mathit{prefix}\texttt{\_}\mathit{suffix}$
where $\mathit{prefix}$ is \texttt{O} or an identifier $\mathit{Ocoll}$ from a collision $\texttt{"}\mathit{Ocoll}: \texttt{forall}\ a_1:T_1, \dots, a_n:T_n; \Resa{r_1}{T}; M\texttt{"}$ in $\mathit{collisions\_LHS}$
and $\mathit{suffix}$ is arbitrary. These oracles act like the oracle $\mathit{prefix}$ when $\mathit{specialname}$ is \texttt{rom} (resp. \texttt{prf} or \texttt{prp}), except that:
\begin{itemize}
\item When $\mathit{suffix}$ starts with \texttt{leave}, 
  the right-hand side still uses calls to $f$; it does not replace them
  with fresh random values. However, it still performs look ups as needed
  to make sure that the returned result is coherent with results
  previously returned by other oracles.
\item It uses a collision matrix to determine whether arguments
  of oracles with various suffixes are allowed to collide with non-negligible
  probability. By default, this collision matrix says that 
  the arguments of two oracles are allowed to collide when they have the same suffix
  or when one of the suffixes starts with \texttt{leave}.
  A different collision matrix can be specified by passing
  a string as a special argument to the \texttt{crypto} command, which can be
either \texttt{"no collisions"} or statements $\neseq{suffix}$ {\tt may collide with previous} $\neseq{suffix}$
separated by semi-colons (\texttt{;}). \texttt{"no collisions"} says that the arguments of two oracle calls are never allowed to collide.
$\mathit{suffix}_1, \dots, \mathit{suffix}_n$ {\tt may collide with previous} $\mathit{suffix}'_1, \dots, \mathit{suffix}'_{n'}$ says that the arguments of oracles with suffix $\mathit{suffix}_i$ ($i \in \{1, \dots, n\}$) are allowed to collide with arguments of previous calls to oracles with suffix $\mathit{suffix}'_j$ ($j \in \{1, \dots, n'\}$).
In the right-hand side, the oracles execute \texttt{event\_abort ev\_coll} when a disallowed collision happens. That avoids generating further code in this case, and thus may considerably reduce the size of the generated game after applying the cryptographic transformation. However, in case a disallowed collision actually happens with non-negligible probability, CryptoVerif will be unable to prove that event \texttt{ev\_coll} does not happen, so the proof will fail.

\end{itemize}
The oracles $\mathit{prefix}\texttt{\_leave}$ are generated
by default. The other oracles are generated on demand when
they are present in the \texttt{terms:} information of the
\texttt{crypto} command. Therefore, you must explicitly
mention in the \texttt{terms:} information all occurrences
of terms that should be transformed by an oracle different
from $\mathit{prefix}\texttt{\_leave}$.
(See file \nolinkurl{examples/arinc823/sharedkey/lemmaEnc_equiv_v2_optim.ocv}
for an example with \texttt{prf\_partial}.)

Let us now explain the cases \texttt{sprp}, \texttt{sprp\_partial},
  \texttt{icm}, and \texttt{icm\_partial}.
  They take the following arguments $\seq{specialarg} = a_1, \dots a_n$:
\begin{enumerate}

\item A tuple of strings $\mathit{arg\_order}$, which contains the strings
\texttt{"msg"}, \texttt{"key"}, and for \texttt{icm} and \texttt{icm\_partial}, \texttt{"local\_key"}, in the order in which the encryption and decryption functions take their arguments. (For the ideal cipher model, the \texttt{"key"} is the key that models the choice of the encryption scheme, and the \texttt{"local\_key"} is the key passed to each encryption and decryption.)

\item A identifier $\mathit{enc}$ and an identifier $\mathit{dec}$, which are respectively the encryption and decryption functions. These functions must have the same type, and take arguments as specified by $\mathit{arg\_order}$.
We must be able to choose an element randomly in the type of the argument \texttt{"key"}
and in the type of the result of $\mathit{enc}$ and $\mathit{dec}$, that is, these types must be
declared \texttt{fixed}, \texttt{bounded}, or \texttt{nonuniform}.
The type of the argument \texttt{"msg"} must be the same as the type of the result of $\mathit{enc}$ and $\mathit{dec}$. ($\mathit{enc}$ and $\mathit{dec}$ are permutations of this type.)
This type $T$ must be large enough so that collisions between a random element of
this type and an independent value can be eliminated (because we
model a PRF and apply the PRF/PRP switching lemma), that is,
$\texttt{Pcoll1rand}(T) \leq 2^{-n'}$, that is, $T$ has
option \texttt{pcoll$n$} with $n \geq n'$ where $n'$ is set by
$\texttt{set minAutoCollElim} = \texttt{pest}n'$; the default is $n' =
80$.

\item When $\mathit{specialname}$ is \texttt{sprp} or
  \texttt{sprp\_partial}, an identifier $\mathit{p}$ such that
  $\mathit{p}(t, N, N', l, l')$ is the probability that an adversary
  breaks the SPRP assumption in time $t$, with at most $N$ queries to
  the function $\mathit{enc}$, with messages of length at most $l$,
  and at most $N'$ queries to the function $\mathit{dec}$, with
  ciphertexts of length at most $l'$. The lengths are omitted when the
  type is bounded. The identifier $\mathit{p}$ must be declared with
  $\texttt{proba}\ p$. This argument is omitted for the ideal cipher
  model because the probability is always 0.

\item A tuple of identifiers $(k, lk, m, c, u)$ for ICM,
$(k, m, c, u)$ for SPRP, which are used to
determine identifiers of variables in the generated equivalence:
\begin{itemize}
\item $k$ is the identifier of the key;
\item $lk$ is the identifier of the local key;
\item $m$ is the identifier of cleartext messages;
\item $c$ is the identifier of ciphertexts;
\item $u$ is the identifier used for indices of {\tt find}.
\end{itemize}
The identifiers $lk$, $m$, $c$, $u$ are suffixed by \texttt{\_} and the
name of the oracle in which they are used. 
Moreover, if needed to avoid name clashes or to generate several variables,
a suffix $\_n$ may be added to these identifiers or modified if they
already have one. Using identifiers not used elsewhere allows
the user to have stable identifiers in the generated equivalence.

\item A tuple of strings $\mathit{collisions\_LHS}$, which can be either \texttt{("large")} or a tuple of strings of the following form:
\[\texttt{"}\mathit{Ocoll}: \texttt{forall}\ a_1:T_1, \dots, a_n:T_n; \Resa{r_1}{T}; M\texttt{"}\]
    where $T$ is the type of the result of $\mathit{enc}$ and the simple term $M$ uses the variables $a_1, \dots a_n, r_1$. In this case, CryptoVerif tries
    to simplify $M$ assuming $r_1$ is a random value and $a_1, \dots, a_n$ do not depend on $r_1$. If it
    rewrites $M$ into a term $N$ that does not contain $r_1$, then it uses this information to transform
    terms $M\{f(\dots)/r_1\}$ into $N$ when the result of $f(\dots)$ is a fresh random value, in the generated
    cryptographic transformation.
  Obviously, when $n = 0$, $\texttt{forall}\ a_1:T_1, \dots, a_n:T_n; $ is omitted.
  The identifiers $\mathit{Ocoll}$ are used to form the oracle names in the generated equivalence (see below); they must not contain \texttt{\_}, and must be different from \texttt{O} and pairwise distinct.
  Even when a single string is present, the argument must be a tuple of strings,
  so this string must be between parentheses.

  When $\mathit{collisions\_LHS}$ is \texttt{("large")}, this is equivalent to $\mathit{collisions\_LHS}$ containing:
  \[\texttt{"Oeq:}\ \texttt{forall}\ a_1:T; \Resa{r_1}{T}; r_1 = a_1\texttt{"}\]
  Assuming $a_1$ does not depend on $r_1$, $r_1 = a_1$ simplifies into
  $\texttt{false}$, so $f(\dots) = a_1$ is transformed into
  $\texttt{false}$ in the generated cryptographic transformation, when
  the result of $f(\dots)$ is a fresh random value.

The argument $\mathit{collisions\_LHS}$ can be overriden when the 
equivalence is used in a \texttt{crypto} command, by passing
the desired $\mathit{collisions\_LHS}$ as special argument to the 
\texttt{crypto} command. 

\end{enumerate}
The last or the last two arguments may be omitted.

When $\mathit{specialname}$ is \texttt{icm} or \texttt{sprp},
the generated equivalence provides the following oracles:
\begin{itemize}
\item Oracle \texttt{O\_enc} evaluates $\mathit{enc}$ on its arguments in the
  left-hand side, and performs a lookup into previous cleartexts (and
  local keys for \texttt{icm}) of calls to \texttt{O\_enc}
  and \texttt{O\_dec} in the right-hand side: it returns the previous
  ciphertext when the current arguments are equal to previous cleartexts
  (and local keys for \texttt{icm}) and otherwise it returns a fresh
  random value.

\item Oracle \texttt{O\_dec} evaluates $\mathit{dec}$ on its arguments in the
  left-hand side, and performs a lookup into previous ciphertexts (and
  local keys for \texttt{icm}) of calls to \texttt{O\_enc}
  and \texttt{O\_dec} in the right-hand side: it returns the previous
  cleartext when the current arguments are equal to previous ciphertexts
  (and local keys for \texttt{icm}) and otherwise it returns a fresh
  random value.

\item For each element of $\mathit{collisions\_LHS}$, oracles
  $\mathit{Ocoll\_enc}$ and $\mathit{Ocoll\_dec}$ evaluate $M$ with
  $r_i$ replaced with a call to $\mathit{enc}$ (resp. $\mathit{dec}$)
  in left-hand side and uses the simplified form of $M$ in the
  right-hand side.

\end{itemize}
When $\mathit{specialname}$ is \texttt{icm\_partial} or \texttt{sprp\_partial},
the generated equivalence provides oracles named
$\mathit{prefix}\texttt{\_}\mathit{middle}\texttt{\_}\mathit{suffix}$
where $\mathit{prefix}$ is \texttt{O} or an identifier $\mathit{Ocoll}$ from a collision in $\mathit{collisions\_LHS}$,
$\mathit{middle}$ is \texttt{enc} or \texttt{dec},
and $\mathit{suffix}$ is arbitrary. These oracles act like the oracle $\mathit{prefix}\texttt{\_}\mathit{middle}$ when $\mathit{specialname}$ is \texttt{icm} (resp. \texttt{sprp}), except that it uses a collision matrix to determine whether arguments
  of oracles with various suffixes are allowed to collide with non-negligible
  probability. By default, this collision matrix says that 
  the arguments of two oracles are allowed to collide when they have the same suffix
  or when one of the suffixes is \texttt{default}.
  A different collision matrix can be specified by passing
  a string as a special argument to the \texttt{crypto} command, which can be
either \texttt{"no collisions"} or statements $\neseq{suffix}$ {\tt may collide with previous} $\neseq{suffix}$
separated by semi-colons (\texttt{;}). \texttt{"no collisions"} says that the arguments of two oracle calls are never allowed to collide.
$\mathit{suffix}_1, \dots, \mathit{suffix}_n$ {\tt may collide with previous} $\mathit{suffix}'_1, \dots, \mathit{suffix}'_{n'}$ says that the arguments of oracles with suffix $\mathit{suffix}_i$ ($i \in \{1, \dots, n\}$) are allowed to collide with arguments of previous calls to oracles with suffix $\mathit{suffix}'_j$ ($j \in \{1, \dots, n'\}$).
In the right-hand side, the oracles execute \texttt{event\_abort ev\_coll} when a disallowed collision happens. That avoids generating further code in this case, and thus may considerably reduce the size of the generated game after applying the cryptographic transformation. However, in case a disallowed collision actually happens with non-negligible probability, CryptoVerif will be unable to prove that event \texttt{ev\_coll} does not happen, so the proof will fail.

The oracles with suffix \texttt{default} are generated
by default. The other oracles are generated on demand when
they are present in the \texttt{terms:} information of the
\texttt{crypto} command. Therefore, you must explicitly
mention in the \texttt{terms:} information all occurrences
of terms that should be transformed by an oracle with a suffix different
from \texttt{default}.


The \texttt{[manual]} option, when it is present in the declaration,
prevents the automatic application of the transformation. The transformation
is then applied only using the manual \texttt{crypto} command.
Alternatively, an integer between brackets $\texttt{[}n\texttt{]}$ ($n \geq 0$)
can also be added to the declaration.
This integer does not change the semantics of the equivalence, but is
used for the proof strategy: CryptoVerif uses preferably the equivalences
with the smallest integers $n$ when several equivalences can be used.
When no integer is mentioned,
$n = 0$ is assumed, so the equivalence has the highest priority.


\item $\texttt{query }[\seq{vartypeb}\texttt{;}]\nonterm{query}(\texttt{;}\nonterm{query})^* \texttt{.}$

The {\tt query} declaration indicates which security properties we 
would like to prove. It is of the form $\texttt{query }x_1\texttt{:}T_1\texttt{,} \ldots\texttt{,} x_n\texttt{:}T_n\texttt{;} Q_1\texttt{;} \dots\texttt{;}Q_n$. First, we declare the types of all variables $x_1, \ldots, x_n$
that occur in correspondence queries that follow. (We use $x_i \texttt{<=} N_i$ instead of
  $x_i\texttt{:} T_i$ when $x_i$ is of type $[1,N_i]$, where $N_i$ is
  a parameter, declared by $\texttt{param }N_i$.) Second, we give the queries themselves. The available queries $Q_i$ are as follows:
\begin{itemize}

\item ${\tt secret}\ x\ [\texttt{public\_vars}\ l]$: show that the array $x$ is indistinguishable
from an array of independent random numbers (by several test queries),
even when the variables in $l$ are public. The list $l$ is considered empty when it is omitted.
In the vocabulary of~\cite{BlanchetEPrint05}, this is secrecy.

\item ${\tt secret}\ x\ [\texttt{public\_vars}\ l]\ \texttt{[cv\_onesession]}$: 
show that any element of the array $x$ 
cannot be distinguished from a random number (by a single test query),
even when the variables in $l$ are public. The list $l$ is considered empty when it is omitted.
In the vocabulary of~\cite{BlanchetEPrint05}, this is one-session
secrecy.

In addition to the option \texttt{cv\_onesession}, the options \texttt{real\_or\_random},
\texttt{cv\_real\_or\_random} and all options starting with \texttt{pv\_} are also allowed,
but ignored. Real-or-random secrecy is the default for CryptoVerif and
the options starting with \texttt{pv\_} are for ProVerif.

\item $M \texttt{ ==> } M'$.
The system shows that, for all values of variables that occur in $M$,
if $M$ is true then there exist values of variables of $M'$ that do not
occur in $M$ such that $M'$ is true.

$M$ must be a conjunction of terms $\texttt{event(}e\texttt{)}$, $\texttt{inj-event(}e\texttt{)}$, 
$\texttt{event(}e\texttt{(}M_1, \ldots, M_n\texttt{))}$, or 
$\texttt{inj-event(}e\texttt{(}M_1, \ldots, M_n\texttt{))}$
where $e$ is an event declared by ${\tt event}$ and
the $M_i$ are simple terms without array accesses (not containing
events). 

$M'$ must be formed by conjunctions and disjunctions of terms 
$\texttt{event(}e\texttt{)}$, $\texttt{inj-event(}e\texttt{)}$, 
$\texttt{event(}e\texttt{(}M_1, \ldots, M_n\texttt{))}$, 
$\texttt{inj-event(}e\texttt{(}M_1, \ldots, M_n\texttt{))}$, or
simple terms without array accesses
(not containing events).

When $\texttt{inj-event}$ is present, the system proves an injective
correspondence, that is, it shows that several different events marked
$\texttt{inj-event}$ before $\texttt{==>}$ imply the execution of several
different events marked $\texttt{inj-event}$ after $\texttt{==>}$.
%
More precisely, $\texttt{inj-event(}e_1\texttt{(}M_{11}, \ldots, M_{1m_1}\texttt{))}$
$\texttt{\&\&}$ $\ldots$ $\texttt{\&\&}$ $\texttt{inj-event(}e_n\texttt{(}M_{n1}, \ldots, \allowbreak
M_{nm_n} \texttt{))}$ $\texttt{\&\&}$ $\ldots$ $\texttt{==>}$ $M'$ means that for each
tuple of executed events $e_1(M_{11}, \allowbreak \ldots, M_{1m_1})$
(executed $N_1$ times), \ldots, $e_n(M_{n1}, \ldots, M_{nm_n})$
(executed $N_n$ times), $M'$ holds, considering that an event
$\texttt{inj-event(}e'\texttt{(}M_1, \ldots, M_m\texttt{))}$ in $M'$ holds when it has been
executed at least $N_1 \times \ldots \times N_n$ times.
%
The $\texttt{inj-event}$ marker must
occur either both before and after $\texttt{==>}$ or not at all. (Otherwise,
the query would be equivalent to a non-injective correspondence.)

\item $M$. This query is an abbreviation for $M \texttt{ ==> false}$.

\end{itemize}

\item $\texttt{proof \{}\nonterm{command}\texttt{;}\ldots \texttt{;}\nonterm{command} \texttt{\}}$

Allows the user to include in the CryptoVerif input file the commands
that must be executed by CryptoVerif in order to prove the protocol.
The allowed commands are those described in Section~\ref{sec:interact},
except that \texttt{help} and \texttt{?} are not allowed and that
the \texttt{crypto} command must be fully specified (so that no user 
interaction is required). If the command contains a string that
is not a valid identifier, \texttt{*}, or \texttt{.}, then this string
must be put between quotes \texttt{"}. This is useful in particular for
variable names introduced internally by CryptoVerif and that contain
\texttt{\string@} (so that they cannot be confused with variables introduced
by the user), for example \texttt{"\string@2\_r1"}.

\item ${\tt def\ }\nonterm{ident}\texttt{(}\seq{ident}\texttt{) \{}
\seq{decl}\texttt{\}}$ 

${\tt def\ }m\texttt{(}x_1, \ldots, x_n\texttt{) \{}
d_1, \ldots, d_k\texttt{\}}$ defines a macro named $m$, with arguments
$x_1, \ldots, x_n$. This macro expands to the declarations
$d_1, \ldots, d_k$, which can be any of the declarations listed in
this manual, except $\texttt{def}$ itself.
The macro is expanded by the \texttt{expand} declaration described below.
When the \texttt{expand} declaration appears inside a \texttt{def}
declaration, the expanded macro must have been defined before the
\texttt{def} declaration (which prevents recursive macros, whose
expansion would loop).
Macros are used in particular to define a library of standard
cryptographic primitives that can be reused by the user without
entering their full definition. These primitives are presented
in Section~\ref{sect:prim}.

\item ${\tt expand\ }\nonterm{ident}\texttt{(}\seq{ident}\texttt{).}$

${\tt expand\ }m\texttt{(}y_1, \ldots, y_n\texttt{).}$ expands the macro
$m$ by applying it to the arguments $y_1, \ldots, y_n$. If the definition
of the macro $m$ is ${\tt def\ }m\texttt{(}x_1, \ldots, x_n\texttt{) \{}
d_1, \ldots, d_k\texttt{\}}$, then it generates $d_1, \ldots, d_k$ in which
$y_1, \ldots, y_n$ are substituted for $x_1, \ldots, x_n$ and the other
identifiers that were not already defined at the $\texttt{def}$ declaration
are renamed to fresh identifiers.

\end{itemize}

The following identifiers are predefined:
\begin{itemize}

\item The type {\tt bitstring} is the type of all bitstrings.

\item The type {\tt bitstringbot} is the type that contains
all bitstrings and $\bot$.

\item The type {\tt bool} is the type of boolean values, which consists
of two constant bitstrings {\tt true} and {\tt false}.
It is declared {\tt fixed}.

\item The function {\tt not} is the boolean negation, from
{\tt bool} to {\tt bool}.

\item The constant {\tt bottom} represents $\bot$. (The special
element of {\tt bitstringbot} that is not a bitstring.)

\end{itemize}

The syntax of probability formulas allows parenthesing and the usual
algebraic operations \texttt{+}, \texttt{-}, \texttt{*}, \texttt{/}, \texttt{\^{ }}.
(\texttt{\^{ }} is the exponentiation, its second argument must be an integer; 
\texttt{\^{ }} has higher priority than \texttt{*} and \texttt{/}, 
which have higher priority than \texttt{+} and
\texttt{-}, as usual), as well as the maximum, denoted 
$\texttt{max(}p_1\texttt{,}\ldots\texttt{,}p_n\texttt{)}$, and
minimum, denoted $\texttt{min(}p_1\texttt{,}\ldots\texttt{,}p_n\texttt{)}$. 
They may also contain 
\begin{itemize}

\item $P$ or $P(p_1, \ldots,
p_n)$ where $P$ has been declared by $\texttt{proba }P$ and $p_1,
\ldots, p_n$ are probability formulas; this formula represents an
unspecified probability depending on $p_1, \ldots, p_n$. 

\item $N$, where $N$ has been declared by $\texttt{param }N$,
designates the number of copies of a replication.

\item $\#O$, where $O$ is an oracle,
designates the number of different calls to the oracle $O$. 

\item $|T|$, where
$T$ has been declared by $\texttt{type }T$ and is \texttt{fixed}
or \texttt{bounded}, designates the cardinal of $T$.

\item $\texttt{maxlength(}M\texttt{)}$ is the maximum
length of term $M$ ($M$ must be a simple term without array access, 
and must be of a non-bounded type).

\item $\texttt{length(}f, p_1, \ldots, p_n\texttt{)}$ designates the maximal
length of the result of a call to $f$, where $p_1, \ldots, p_n$
represent the maximum length of the non-bounded arguments of $f$
($p_i$ must be built from $\texttt{max}$,
$\texttt{maxlength(}M\texttt{)}$, and $\texttt{length(}f', \ldots
\texttt{)}$, where $M$ is a term of the type of the corresponding
argument of $f$ and the result of $f'$ is of the type of the
corresponding argument of $f$).

\item $\texttt{length}(T)$ designates the maximal 
length of a bitstring of type $T$, where $T$ is a bounded type.

\item $\texttt{length((}T_1, \ldots, T_n\texttt{)}, p_1, \ldots,
p_n\texttt{)}$ designates the maximal length of the result of the
tuple function from $T_1 \times \ldots \times T_m$ to
\texttt{bitstring}, where $p_1, \ldots, p_n$ represent the maximum
length of the non-bounded arguments of this function.

\item $n$ is an integer constant.

\item \texttt{eps\_find} is 2 times the maximum distance between the uniform probability
distribution and the probability distribution used for choosing elements
in {\tt find}.

\item $\texttt{eps\_rand(}T\texttt{)}$ is the maximum distance between the 
uniform probability distribution and the default probability distribution 
$D_T$ for type $T$ (when $T$ is \texttt{bounded}).

\item 
$\texttt{Pcoll1rand(}T\texttt{)}$ is the maximum probability of
collision between a random value $X$ of type $T$ chosen according
to the default distribution $D_T$ for type $T$ and an element of type $T$
that does not depend on it (when $T$ is \texttt{nonuniform}).
This is also the maximum probability of choosing any given element of 
$T$ in the default distribution for that type:
\[\texttt{Pcoll1rand(}T\texttt{)} = \max_{a \in T} \Pr[X = a]\]
where $X$ is chosen according to distribution $D_T$.

\item $\texttt{Pcoll2rand(}T\texttt{)}$ is the maximum probability of
collision between two independent random values of type $T$  
chosen according to the default distribution $D_T$ for type $T$
(when $T$ is \texttt{nonuniform}). We have
\[\frac{1}{|T|} \leq \texttt{Pcoll2rand(}T\texttt{)} = \sum_{a \in T} \Pr[X = a]^2 \leq \texttt{Pcoll1rand(}T\texttt{)}\]
where $X$ is chosen according to the default distribution $D_T$.

\item $\texttt{optim-if}\ \mathit{condition}\ \texttt{then}\ p_1\ \texttt{else}\ p_2$
evaluates to $p_1$ when the condition $\mathit{condition}$ is proved to be
true and to $p_2$ otherwise. Hence, the formula $p_2$ must always be a
sound estimate, whether the condition is true or not (because it may happen that
the condition is true and CryptoVerif does not manage to prove it). The formula
$p_1$ is typically a better estimate valid when the condition holds.
The grammar for the condition $\mathit{condition}$ is defined in Figure~\ref{fig:syntax2}.
The condition $\texttt{is-cst(}p\texttt{)}$ is true when $p$ is a constant.
The other conditions have their usual meaning.

\item $\texttt{time}$ designates the runtime of the environment (attacker).

\end{itemize}
Finally, $\texttt{time(}\ldots\texttt{)}$ designates the runtime time of each
elementary action of a game:
\begin{itemize}
\item
$\texttt{time(}f, p_1, \ldots, p_n\texttt{)}$ designates the maximal runtime of
one call to function symbol $f$, where $p_1, \ldots, \allowbreak p_n$ represent
the maximum length of the non-bounded arguments of $f$.
\item
$\texttt{time(let }f, p_1, \ldots, p_n\texttt{)}$ designates the
maximal runtime of one pattern matching operation with function symbol
$f$, where $p_1, \ldots, p_n$ represent the maximum length of the
non-bounded arguments of $f$.
\item
$\texttt{time((}T_1, \ldots, T_m\texttt{)}, p_1, \ldots, p_n\texttt{)}$ designates the
maximal runtime of one call to the tuple function from $T_1 \times
\ldots \times T_m$ to \texttt{bitstring}, where $p_1, \ldots, p_n$
represent the maximum length of the non-bounded arguments of this
function.
\item
$\texttt{time(let(}T_1, \ldots, T_m\texttt{)}, p_1, \ldots, p_n\texttt{)}$ designates the
maximal runtime of one pattern matching with the tuple function from
$T_1 \times \ldots \times T_m$ to \texttt{bitstring}, where $p_1, \ldots, p_n$
represent the maximum length of the non-bounded arguments of this function.
\item
$\texttt{time(=}T[, p_1, p_2]\texttt{)}$ designates the
maximal runtime of one call to bitstring comparison function
for bitstrings of type $T$, where $p_1, p_2$ represent the
maximum length of the arguments of this function when $T$ is non-bounded.
\item
$\texttt{time(!)}$ or $\texttt{time(foreach)}$ is the maximum time of an access to a replication index.
\item
$\texttt{time([}n\texttt{])}$ is the maximum time of an array access 
with $n$ indices.
\item
$\texttt{time(\&\&)}$ is the maximum time of a boolean and.
\item
$\texttt{time(\string|\string|)}$ is the maximum time of a boolean or.
\item
$\texttt{time(new }T\texttt{)}$ or $\texttt{time(<-R }T\texttt{)}$ 
is the maximum time needed to choose
a random number of type $T$ according to the default distribution for type $T$.
\item
\ifchannels
$\texttt{time(newChannel)}$ is the maximum time to create a new
private channel.
\else
$\texttt{time(newOracle)}$ is the maximum time to create a new
private oracle.
\fi
\item
$\texttt{time(if)}$ is the maximum time to perform a boolean test.
\item
$\texttt{time(find }n\texttt{)}$ is the maximum time to perform 
one condition test of a find with $n$ indices to choose.
(Essentially, the time to store the values of the indices in a 
list and part of the time needed to randomly choose an element
of that list.)
\ifchannels
\item
$\texttt{time(out [}T_1, \ldots, T_m\texttt{]}T, p_1, \ldots, p_n\texttt{)}$
represents the time of an output in which the channel indices are
of types $T_1, \ldots, T_m$, the output bitstring is of type $T$,
and the maximum length of the channel indices and the output bitstring
is represented by $p_1, \ldots, p_n$ when they are non-bounded.
\item
$\texttt{time(in }n\texttt{)}$ is the maximum time to store an
input in which the channel has $n$ indices in the list of
available inputs.
\fi
\end{itemize}
CryptoVerif checks the dimension of probability formulas.
%Actually, the check is stricter than usual dimension checking,
%since it distinguishes probabilities from other data without dimension.

